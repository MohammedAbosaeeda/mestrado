\documentclass[twoside,12pt]{scrartcl}
%
\areaset[1.8cm]{16cm}{25cm}
\sloppy
\usepackage{makeidx}  % allows for indexgeneration
\usepackage{psfig}
\usepackage{url}
\usepackage{epsf}
%

% useful command:
\newcommand{\uk}{$\mu$kernel}

\bibliographystyle{alpha}

\begin{document}
\title{Something Strikingly New \\
{\small\textsc{(A Sample ECOOP-OOOSWS Paper)}}}

\author{Francisco J. Ballesteros$^1$, Ashish Singhai$^2$,\\ 
        Lutz Wohlrab$^3$, Frank Schubert$^3$,\\ 
        and Henning Schmidt$^4$\\[0.2cm]
        {\small $^1$ Univ. Carlos III de Madrid 
        \hspace{1cm} \emph{nemo@gsyc.inf.uc3m.es}} \\
        {\small $^2$ Univ. of Illinois at Urbana-Champaign 
        \hspace{1cm}\emph{ashish.singhai@acm.org}} \\
        {\small $^3$ Chemnitz University of Technology
        \hspace{1cm}\emph{\{lwo, fsc\}@informatik.tu-chemnitz.de}} \\
        {\small $^4$ Potsdam University 
        \hspace{1cm}\emph{hesch@haiti.cs.uni-potsdam.de}}}

\date{} % no date

\maketitle              % typeset the title of the contribution

%\begin{center}
%{\bf Abstract}
%\end{center}
\begin{abstract}
  \centerline{\bf Abstract}
  This paper serves as a sample document for an article to be
  published in the proceedings of the 2nd ECOOP Workshop on
  Object-Orientation and Operating Systems (ECOOP-OOOSWS�99).
  The layout is important.
\end{abstract}
%

\section{Introduction}

The proceedings of ECOOP-OOOSWS�99 will appear as a part of the series 
``Chemnitzer Informatik-Berichte'', ISSN 0947-5125.
Please use this file as a template to edit your paper.
These are cite examples: a proceedings paper \cite{engler95:_exoker}, a book
\cite{gamma95:_patterns}, something miscellaneous \cite{offppw3site}.
The $\mu$ often needed when talking about a certain kind of kernel is coded
as follows: \$mu\$

A reference to a session: \ref{section1}. Mind you have to define the
corresponding label there. 

\section{Pictures and Code Scraps}\label{section1}

Mind the capitalization of the headings.
Figure~\ref{fig:sample-pic} demonstrates how to include a PostScript picture
with the psfig macro, turned by 270 degrees. 
Figure~\ref{fig:sample-pic2} is the same picture
included and scaled with the macros of the epsfig package.
Figure~\ref{fig:osample-code} demonstrates how to write code scraps.

% htb stands for the preference where to place the picture if possible:
% here, top, bottom. Change the sequence if necessary.
\begin{figure}[htb]
\centerline{\psfig{file=ecoop-ooosws-samplepic.ps,angle=270}}
\caption{A Sample postscript figure}
\label{fig:sample-pic}
\end{figure}

%
% How to write code
%
\begin{figure}
\begin{verbatim}
void
SomeFunction() {
 ...
 AnotherType var2 = functionB(args);
 //Some random comment
 AnotherType anotherVar = obj1->getAnother();
 Pointer *ptr = nil;
 if (var2 == anotherVar) {
    doSomething();
    var1 = obj2->function(args);
    someMoreCode();
 }
}
\end{verbatim}
\caption{\label{fig:osample-code} Sample piece of code; the formatting is important,
                                  the code is not.}
\end{figure}

% an example table: |l|r|... means vertical line,left justified col., vertical
% line, right justified col....
% \hline is a horizontal line

\begin{table}[hb]
{\small
\begin{tabular}{|l||r|r|r|r|r|r|}
\hline
(cycles) & {\tt sys\_open} & {\tt close\_fp} & {\tt dupfd} & {\tt do\_unlink} & {\tt do\_symlink} & {\tt sys\_link}\\
\hline \hline
add.Code & 645             & 376             & 65          & 326              & 457               & -140\\
\hline
Orig.    & 108~013         & 3~770           & 137         & 10~760           & 11~691            & 22~420\\
\hline \hline
 \%      & 42,36           & 55,87           & 1,33        & 0,41             & 0,00              & 0,00\\
\hline
\end{tabular}
}
\caption{\label{extable} example table}
\end{table}
%

\begin{figure}[htb]
\begin{center}
\epsfxsize=6cm
\epsfysize=7.5cm
\epsffile{ecoop-ooosws-samplepic.ps}
%
\caption{Sample postscript figure included via epsfig package and scaled}
\label{fig:sample-pic2}
\end{center}
\end{figure}


\subsection{A Subsection}
A nice little subsection.

\section{Conclusions}
The formatting instructions are sparse, but we hope that the papers will not
be too differently formatted.

\bibliography{ecoop-ooosws-sample}
\end{document}