\section{Introduction}
% Reprogramao, estrutura + protocolo
% O que  um protocolo de disseminao
% Importncia de um prot. de diss.
% Importncia de uma boa estrutura
% Ex. de uso: reprogramao RSSF
% Contribuio: infra-estrutura p/ disseminao e reprogramao
% Organizao do artigo

%Reprogramming the software of a program in execution is a feature present in most computer environments.
%A wide range of applications make use of some reprogramming method: from internet browsers to dedicated systems, as controllers in vehicles for instance.
%Due to the limited resources of embedded systems, the software reprogramming infrastructure is different from that implemented in conventional computer environments.
%Moreover, some of these dedicated systems, as Wireless Sensor Networks (WSNs), are formed by a big amount of nodes, in which collecting and reprogramming all nodes is impractical. 

%Wireless Sensor Networks (WSN), are typically formed by a big amount of nodes, in which collecting and reprogramming all nodes is impractical.
A software reprogramming infrastructure for a Wireless Sensor Network (WSN) is composed of a data dissemination protocol and a structure capable of organizing the data in the system's memory. By using an embedded Operating System (OS) it is possible to provide for embedded applications an infrastructure to hide this data organization. Usually, the reprogramming structures present in OSs are composed of updatable modules. These modules are memory position independent and are replaced at runtime~\cite{sos}~\cite{contiki}. 

In addition, it is essential that all new data of one or more modules is correctly received by all nodes involved in the reprogramming process. In order to provide safe data transfer, a data dissemination protocol is used together with the OS infrastructure.
%In general, a dissemination protocol works as follows: the dissemination begins from a base station responsible to transfer the new data to its neighbor nodes. Once a node receives the new data, it is capable of retransmitting it to its own neighbors.  The process repeats until the entire network is up to date~\cite{moap}~\cite{deluge}.
%Thus, if a node A is neighbor of a node B, and the node B has A and C as neighbors, the node B after receiving the new data from the node A will retransmit it to the node C. The process repeats for all nodes until the entire network is up to date~\cite{moap, deluge}.

Epos Live Update System (\ELUS{}) is an OS infrastructure for software updating that has better performance in terms of memory consumption, method invocation time, and reconfiguration time when compared to related works~\cite{Gracioli:2010}.
%Although this favorable result, we identifed that memory consumption of \ELUS{} could be improved.
Although this favorable result, \ELUS{} memory consumption could still be improved.
Furthermore, \ELUS{} does not have any support for data dissemination. 

%In summary, in this paper, we make the following contributions:
%\begin{itemize}
%	\item We improved the memory consumption of \ELUS{} by using C++ templates specialization techniques (more than 50\% of improvement). %\ELUS{} is implemented around the EPOS metaprogrammed framework~\cite{Frohlich:2001}. Thus, some code regions are duplicated due to the use of templates. We identified some of these regions and applied a template specialization technique~\cite{stroustrup:2000}.
%	\item We provide a domain engineering analysis considering the data dissemination protocols characteristics. The protocols characteristics are decomposed into a feature diagram that shows common and variable features present in different protocols.
%	\item We integrate a data dissemination protocol to \ELUS{} and evaluate the new infrastructure in terms of memory consumption and dissemination and reprogramming times. %As result, we provide a lightweight software reprogramming infrastructure for resource-constrained embedded systems.
%\end{itemize}

%In this paper we make the following contributions: (i) we have improved the memory consumption of \ELUS{} by using C++ templates specialization techniques; (ii) we provide a data dissemination protocol domain engineering analysis considering protocols characteristics, and integrate our developed protocol to \ELUS{}; and (iii) we evaluate the new infrastructure in terms of memory consumption, and dissemination and reprogramming times.
In this paper we make the following contributions: (i) we develop and integrate a data dissemination protocol to \ELUS{}; (ii) we improve the memory consumption of \ELUS{} by using C++ templates specialization techniques; and (iii) we evaluate the new infrastructure in terms of memory consumption, and dissemination and reprogramming times.

%The rest of this paper is organized as follows.
%Section~2 presents the related work.
%Section~\ref{sec:ddp} shows the designed dissemination protocol and compares its characteristics to other proposed protocols.
Section~\ref{sec:ddp} presents the design of the developed dissemination protocol.
Section~\ref{sec:integration} presents the integration between the dissemination protocol and \ELUS{}.
The evaluation of the infrastructure is carried out in Section~\ref{sec:evaluation}.
Finally, Section~\ref{sec:conc} concludes the paper.
