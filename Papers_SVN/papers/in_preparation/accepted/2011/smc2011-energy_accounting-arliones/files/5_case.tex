\section{Evaluation}
\label{sec:case}

% The proposed approach was implemented as an extension to the power management
% mechanism of \epos~\cite{Frohlich:2011}. \epos~is a component-based operating
% system for embedded applications. The approach was modeled as an scenario
% adapter~\cite{ScnAdpt}, thus being implemented in a way that is not intrusive to
% system components. Also, we ran this implementation in the
% \emote~platform~\cite{emote}. \emote~is a module for the development of
% low-power wireless sensor network applications. Due to space limitations, we
% recommend reading the cited references in order to go deeper in the
% implementation details.

In this section we present the deployment of the approach described in this
paper in a mobility-enabled wireless sensor network running the Ant-based
Dynamic Hop Optimization Protocol (\adhop) over an IP network using IEEE
802.15.4. \adhop~is a self-configuring, reactive routing protocol inspired by
the HOPNET protocol for \emph{Mobile Ad Hoc Networks}~(MANETs) and designed with
the typical limitations of sensor nodes in mind, energy in
particular~\cite{Okazaki:2011}. \adhop's reactive component relies on an
\emph{Ant Colony Optimization} algorithm to discover and maintain routes. Ants
are sent out to track routes, leaving a trail of pheromone on their way back.
Routes with a higher pheromone deposit are preferred for data exchange.

\tab{adzrp-taskset}{\adhop~case-study tasks'
parameters\protect\footnotemark[3].}

\footnotetext[3]{T: task; P: period in $ms$; WCET: worst-case execution time in
$ms$; WCEC: worst-case energy consumption in $\eta Ah$; 25-days: energy
consumption for the targeted lifetime (25 days) in $mAh$.}
\footnotetext[4]{This is a worst-case scenario as values of ``P'' and, as
consequence, ``25-days'', are to be defined by the optimization.}  
\footnotetext[5]{``Route'' is an sporadic task. Once it is a best-effort task in
the system, we consider a hypothetic frequency of $2$ Hz (period of 500 $ms$) to
show the impact of routing in the node energy consumption.}

With the purpose of corroborating the approach presented in this paper we
modified \adhop~in order to make it energy-aware. We classified \adhop's tasks
as mandatory (hard real-time) or optional (best-effort). The main idea behind
this setup was to homogenize the battery discharge for every node in the network
to enhance the lifetime of the network as a whole. Considering the radio the
most energy-hungry component in a wireless sensing node, we toke the design
decision of modeling the routing activity of \adhop~as best-effort tasks, as
shown by the task set at Tab.~\ref{tab:adzrp-taskset}. The basic node
functionality of sensing a value (task $Sense$) and forwarding it through the
radio to the next node (task $Transmit$) where modeled as hard real-time tasks.
The functionality of forwarding other nodes' packets (and ants) when acting as a
``router'' was modeled as two best-effort tasks, one for monitoring the channel
for arriving messages ($LPL$ - Low Power Listen), and another to effectively
receive the message and route it to another node ($Route$).

We set the lifetime objective for this system to 25 days. By analyzing the task
set it is possible to compute the total energy consumption of hard real-time
tasks for the desired lifetime to be of $601.88 mAh$, thus, the initial battery
charge for the system has to be greater than that. We then defined the system
battery as an of-the-shelf CR-2/$3V$ battery with a total capacity of $850 mAh$.

We simulated several network scenarios (with up to 200 nodes) using the
\emph{Global Mobile Information System Simulator} (\glmsim). In this setup,
nodes were programmed to communicate intensively and move randomly within a grid
of 700 x 400 meters for 25 days, thus stimulating both the routing protocol and
the power management mechanisms. \glmsim~was integrated to the same
\nsga~optimizer described in Section~\ref{sec:frequency}, and the optimization
process ran with the same parameters.
The results of this optimization are shown in
%Fig.~\ref{fig:adzrp-solutions} (simulated solutions),
Tab.~\ref{tab:adzrp-solutions}.
% , and Fig.~\ref{fig:adzrp-bet_freq} and Fig.~\ref{fig:adzrp-batt_freq}
% (observed parameters plotted against the collector task's frequency).

% \fig{adzrp-solutions}{All solutions for the \adhop~case-study.}{width=\columnwidth}

\tab{adzrp-solutions}{Solutions for the frequency of the collector task for
the \adhop~case-study.}

% \figtwo{adzrp-bet_freq}{Execution rate of best-effort tasks.}
% {adzrp-batt_freq}{Residual energy after projected lifetime.}
% {Optimization objectives plotted against variations on the frequency of the
% collector task for the \adhop~case-study.}

Also, we analyze the impact on the network performance by comparing the
obtained results with the data originally published by
Okazaki~\cite{Okazaki:2011}.
Fig.~\ref{figtwo:adzrp-avg_node_energy-adzrp-avg_node_lifetime} (above) shows a
reduction on the average energy consumed by each node on the network while
Fig.~\ref{figtwo:adzrp-avg_node_energy-adzrp-avg_node_lifetime} (bellow) shows
the expected enhancement on the average battery lifetime of nodes. It is
important to note that all nodes in the network lived for, at least, 25 days as
expected, being that the reason why the average lifetime stayed well above it,
around 30.

\figtwo{adzrp-avg_node_energy}{}{adzrp-avg_node_lifetime}{}{Average
energy-related parameters for the simulated \adhop~setup.}

Besides the good results from the energy consumption perspective, we observed an
important decrease on the overall network quality, as shown in
Fig.~\ref{figtwo:adzrp-network_original-adzrp-network_ea} for the ``Broken
routes'' and ``Delivery ratio'' parameters. These contrast, however, with the
obtained results on ``Link failures'', indicating that \adhop~deals well with
the broken routes, allowing undelivered packets to be re-routed and finally
delivered.
% Future work on
% energy-aware scheduling, which were not focus of the present work, will rely on
% the presently proposed accounting mechanism to enable fairer scheduling of such
% tasks and handle the impact on network performance.

\figtwo{adzrp-network_original}{}{adzrp-network_ea}{}{Impact on network quality
for the \adhop~case-study.}
