\documentclass[12pt]{article}

\usepackage{sbc-template}

\usepackage{graphicx,url}

\usepackage[brazil]{babel}   
\usepackage[utf8]{inputenc}  

     
\sloppy

\title{Um estudo de caso do desenvolvimento de um produto utilizando a plataforma EPOS}

\author{João G. Zeni and Arliones Hoeller Jr.}


\address{Laboratório de Integração Software/Hardware\\
Universidade Federal de Santa Catarina\\
Florianopolis -- SC -- Brasil
\{jgzeni,arliones\}@lisha.ufsc.br
}

\newcommand{\epos}{\textsc{Epos}}
\newcommand{\emote}{\textsc{EposMote}}
\newcommand{\tsg}{\textsc{TSG}}

\newcommand{\tab}[3][ph]{
  \begin{table}[#1]
    {\centering\small\textsf{\input{fig/#2.tab}}\par}
    \caption{#3}
  \label{tab:#2}
  \end{table}
}

\newcommand{\fig}[4][ht!]{
  \begin{figure}[#1]
    {\centering{\includegraphics[#4]{fig/#2}}\par}
    \caption{#3}
    \label{fig:#2}
  \end{figure}
}


\begin{document}

\maketitle

\begin{abstract}
This paper presents a case study of the development of software and hardware of a bracelet for fall detection using the free platform called \epos. \epos~is a free platform (software + hardware + tools) which includes operating system and hardware for embedded systems. In this case study we go from a specification of application requirements to the rapid prototiping of the system using the free hardware modules \emote~and \tsg, to the final product.
\end{abstract}

\begin{resumo} 
Este artigo apresenta um estudo de caso de desenvolvimento do software e hardware para uma pulseira para detecção de quedas utilizando a plataforma livre \epos. O \epos~é uma plataforma (software + hardware + ferramentas) livre que inclui sistema operacional e hardware para sistemas embarcados. Neste estudo de caso, partimos de uma especificação de requisitos da aplicação, à prototipagem rápida utilizando os hardware livres \emote~e \tsg, e à confecção do produto final.
\end{resumo}

\section{Introdução}

O projeto se inicio a partir da necessidade apresentada de um dispositivo capaz de detectar quedas de seres humanos. Sabendo disso foi escolhido para o desenvolvimento desse produto a plataforma \epos~visto que a plataforma oferece suporte tanto a software quanto a hardware e é livre.

Este texto apresenta ao leitor a plataforma e relata o processo de desenvolvimento das aplicações necessárias ao produto e o redesenho do hardware feito para que esse apresenta-se as características funcionais e físicas requisitadas pelo produto.
\section{A plataforma livre \protect\epos} 

O \epos~(Embedded Parallel Operating System)~\cite{Marcondes:2006} é um arcabouço baseado em componentes para a geração de ambientes dedicados de suporte de tempo de execução. O arcabouço do \epos~permite que programadores desenvolvam aplicações independentes de plataforma, e ferramentas de análise permitem que componentes sejam adaptados automaticamente para atender a requisitos destas aplicações particulares. Por definição, uma instância do sistema agrega todo suporte necessário para a sua aplicação dedicada, e nada mais.

O projeto modular do \epos~foi guiado pela metodologia Projeto de Sistemas Embarcados Guiado pela Aplicação (ADESD - Application-Driven Embedded System Design)~\cite{Froehlich:2001}. A ADESD baseia-se nas conhecidas estratégias de decomposição de domínio por trás do Projeto Baseado em Famílias (FBD - Family-Based Design) e Orientação a Objetos (OO) como, por exemplo, análise de atributos comuns e variabilidade, para adicionar o conceito de identificação e separação de aspectos ainda nos estágios iniciais do projeto. Deste modo, a ADESD guia a engenharia de domínio para famílias de componentes, das quais dependências de cenários de execução são fatoradas na forma de aspectos e relacionamentos externos são capturados em um arcabouço de componentes. Esta estratégia de engenharia de domínio trata consistentemente algumas das questões mais relevantes do desenvolvimento de software baseado em componentes: reuso, gerenciamento da complexidade e composição.

Aplicações de redes de sensores sem-fio apresentam requisitos específicos que vão além dos atendidos pelos serviços tradicionais de sistemas operacionais. Estes incluem gerenciamento de energia eficiente, reprogramação em campo, abstração uniforme de sensores e serviços de comunicação configuráveis. O \epos~foi estendido de modo a atender estes requisitos extras~\cite{Wanner:2008}.

O \epos~provê serviços de gerência de energia dirigida pela aplicação que permite um consumo consciente de energia em sistemas profundamente embarcados sem comprometer a portabilidade da aplicação e sem gerar sobrecustos excessivos. O objetivo do subsistema de gerenciamento de energia do \epos~é permitir que aplicações expressem quando determinados componentes de software não estão sendo utilizados, permitindo que o sistema migre dispositivos (hardware) associados a estes componentes de software para modos de operação que consumam menos energia. Disto emergiram várias questões que dizem respeito a diferenças arquiteturais entre dispositivos distintos e ao acesso concorrente aos recursos de hardware por diferentes componentes de software. Para tratar destas questões, foram concebidas (1) uma interface genérica para gerenciamento de energia, (2) um mecanismo de propagação de mensagens e (3) um modelo de formalização das trocas entre modos de operação~\cite{Froehlich:2011}.

A infraestrutura de comunicação do \epos~para redes de sensores sem-fio é implementada pelo protocolo C-MAC (Configurable MAC) que provê suporte a comunicação de baixo nível~\cite{Steiner:2010}, protocolos de roteamento como o ADHOP (???)~\cite{Okazaki:2011}, e uma pilha leve TCP/IP.

\subsection{\protect\emote}

O projeto \emote~tem por objetivo desenvolver uma família de módulos de sensoriamento que permita ampla configurabilidade tanto da plataforma quanto do ambiente de software (sistema operacional). O \emote, apresentado na Figura 4, foi concebido com um projeto modular, sendo previstos três módulos, entre os quais foram estabelecidas interfaces padrão, permitindo o uso intercambiável de diferentes versões dos módulos. A Figura 5 apresenta os três tipos de módulos, que são os seguintes:

\begin{itemize}
	\item \textbf{Módulo de Base:} o módulo base incorpora as funcionalidades de processamento e de comunicação. O projeto \emote~desenvolveu duas versões deste módulo, uma utilizando o single-package ZigBitTM e outra utilizando o SoC (single-die) MC13224V, da Freescale. Características específicas de cada versão do módulo base são apresentadas abaixo. Este módulo deve implementar detalhes de do su- porte a estes dispositivos, como a regulação de tensão e o dimensionamento da antena, além rotear os pinos dos dispositivos de modo a manter o padrão das interfaces de alimentação e de entrada e saída.

	\item \textbf{Módulo de Entrada e Saída:} no módulo de entrada e saída devem ser implementadas as interfaces necessárias de entrada e saída, podendo um novo módulo destes ser desenvolvido para cada aplicação que se pretende desenvolver, permitindo o emprego dos sensores ou atuadores desejados para uma aplicação específica. O projeto \emote~desenvolveu um módulo de entrada e saída ao qual deu o nome de start-up board. Esta placa incorpora uma interface USB, sensor de temperatura, acelerômetro de 3 eixos, alguns LEDs e botões~\cite{LISHA:Projetos:EMOTE}.

	\item \textbf{Módulo de Alimentação:} de modo a permitir o emprego de diferentes fontes de alimentação, uma interface de alimentação foi implementada. Módulos que se conectam a esta interface podem ser tão simples como uma bateria alcalina AA, ou tão complexas quanto um sistema com bateria de lítio recarregável ou com painéis solares. A interface de alimentação ainda disponibiliza uma interface I2C, permitindo que o módulo de alimentação se comunique com o de processamento. O projeto \emote~ainda não desenvolveu nenhum módulo de alimentação específico, mas trabalhos em andamento estão explorando tecnologias de captação de energia utilizando esta interface.

\end{itemize}

\fig{emote2-block_diagram.png}{\protect\emote~block diagram}{width=\columnwidth}

As interfaces padrão definidas pelo projeto \emote~para interconexão dos módulos desenvolvidos são as seguintes:
\begin{itemize}
	\item \textbf{Interface de Entrada e Saída:} 32 pinos, sendo 2 para alimentação e outros 30 que podem ser utilizados ou como GPIO, ou com funções especificas que inclui ADC, UART e SPI.
	\item \textbf{Interface de Alimentação:} disponibiliza pinos aos quais devem ser conectados o terra e alimentação do modulo de alimentação. O modulo de processamento devolve ao modulo de alimentação o sinal com tensão regulada. Também existem os 2 pinos empregados na comunicação $I^2C$.
\end{itemize}

\section{A aplicação de detecção de queda}

A Figura~\ref{fig:Planta_baixa} apresenta uma visão geral da aplicação proposta.
Em um determinado local, diversas pessoas (usuários) utilizam pulseiras capazes
de detectar uma eventual queda. Ao detectar uma determinada queda, a pulseira
deve emitir um sinal sonoro e aguardar alguns segundos (e.g., 5 segundos) para
que o usuário, eventualmente, sinalize, pressionando um botão, a ocorrência de
um falso positivo, ou seja, de que a detecção foi indevida. Não ocorrendo a
sinalização no tempo esperado, a pulseira deve emitir um sinal (pacote de dados)
que deve chegar até um gateway capaz de remeter esta informação a um servidor,
onde um sistema de monitoramento se encarregaria de providenciar atendimento
adequado à vítima da queda. Neste trabalho, focaremos no fluxo da informação
sobre a queda da pulseira até o sistema de monitoramento. Não aboradamos, ainda,
integração com serviços de emergência ou de comunicação extra para providenciar
a prestação de socorro.

\fig{Planta_baixa}{Planta baixa mostrando o funcionamento da
aplicação}{width=0.8\columnwidth}

Os requisitos da aplicação são apresentados na Tabela~\ref{tab:requisitos}.
Neste ponto ressalta-se a importância do uso de uma plataforma completa. Muitos
dos requisitos da aplicação são resolvidos praticamente de imediato com a
seleção do hardware a ser utilizado e com a configuração adequada do sistema
operacional.

\tab{requisitos}{Tabela de requisitos da aplicação de detecção de quedas.}

O hardware de sensoriamento precisava possuir um tamanho reduzido e um formato
que pudesse ser usado por uma pessoa, optou-se por uma pulseira, alem dos
requisitos de forma precisava da funçao de monitor de queda, a qual foi
realizada usando um acelerômetro de três eixos, uma interface com o usuário que
consistia de um botão e de um LED que seriam ativado em caso de falso positivo,
e por final a capacidade de enviar os dados sem fio a outros dispositivos.

O hardware de recepção precisava ser capas de receber o sinal de queda enviado e
enviar esse sinal para a internet de modo a ser visualizado pela pessoa
responável por monitorar os dispositivos.

Com base nas informações supra citadas inicio-se o projeto utilizando o hardware
do \emote~para o desenvolvimento das aplicações de software, após a validação do
software em cima do \emote~houve a necessidade do desenho de um hardware
reduzido visto que o hardware de \emote~possui características que não eram
úteis ao projeto e e demasiado grande para a finalidade desejada do projeto,
sendo assim se projetou um hardware derivado do antigo de modo a não ser
necessário refazer o software ja validado e para que esse hardware atende-se as
necessidades de projeto tanto de tamanho quanto de funcionalidade.

\input{files/4-sw.tex}
\input{files/5-hw.tex}
\section{Conclusao} 

Este artigo apresenta uma plataforma de hardware/software livre, o epos, e como apartir da plataforma e dos requisitos de um projeto se desenvlveo um produto. O principal foco do artigo é no processo de desenvolvimento.

A partir desse texto é perceptivel não apenas a viabilidade do desenvolvimento de produtos com base em plataformas livre como tambem uma serie de vangentagem na utilização de plataformas livres para o desenvolvimento de produtos.



\bibliographystyle{sbc}
\bibliography{sbc-template}

\end{document}
