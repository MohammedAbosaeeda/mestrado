\documentclass[english,10pt,twocolumn]{article}
\usepackage[T1]{fontenc}
\usepackage[latin1]{inputenc}
\usepackage{babel}
\usepackage{latex8}
\usepackage{times}
\usepackage{epsfig}

\newcommand{\fig}[2]{
  \begin{figure}[htb]
  {\centering\scalebox{.8}{\includegraphics{fig/#1.pdf}}\par}
  \caption{#2\label{fig:#1}}
  \end{figure}
}

\newcommand{\wfig}[2]{
  \begin{figure}[htb]
  {\centering\resizebox*{\columnwidth}{!}{\includegraphics{fig/#1.pdf}}\par}
  \caption{#2\label{fig:#1}}
  \end{figure}
}

\pagestyle{empty}

\begin{document}

\title{On Component-Based Communication Systems\\for Clusters of
  Workstations}

\author{
  Ant�nio Augusto Fr�hlich\\
  GMD-FIRST\\
  Kekul�stra�e 7\\
  12489 Berlin, Germany\\
  guto@first.gmd.de \and
  Wolfgang Schr�der-Preikschat\\
  University of Magdeburg\\
  Universit�tsplatz 2\\
  39106 Magdeburg, Germany\\
  wosch@ivs.cs.uni-magdeburg.de }

\maketitle
\thispagestyle{empty}

\begin{abstract}
  Most of the communication systems used to support high performance
  computing in clusters of workstations have been designed focusing on
  ``the best'' solution for a certain network architecture. However, a
  definitive best solution, independently of how well tuned to the
  underlying hardware it is, cannot exist, for parallel applications
  communicate in quite different ways. In this paper, we describe a
  novel design method that supports the construction of run-time systems
  as an assemblage of components that can be configured to closely match
  the demands of any given application. We also describe how this method
  has been deployed in the development of a communication system in the
  realm of \textsc{Epos}, a project that aims at delivering
  automatically generated application-oriented run-time support systems.
  The communication system in question has been implemented for a
  cluster of PCs interconnected with Myrinet, and corroborates the
  effectiveness of the proposed design method.
\end{abstract}

\Section{Introduction}

% - Cluster computing X communication

The parallel computing community has been using clusters of commodity
workstations as an alternative to expensive parallel machines for
several years by now. The results obtained meanwhile, both positive and
negative, often lead to the same point: inter-node communication.
Consequently, much effort has been dedicated to improve communication
performance in these clusters: from the hardware point of view,
high-speed networks and fast buses provide for low-latency and
high-bandwidth; while from the software point of view, \emph{user-level
  communication}~\cite{Bhoedjang:1998} enables applications to access
the network without operating system intervention, significantly
reducing the software overhead on communication. Combined, these
advances enabled applications to break the giga-bit-per-second bandwidth
barrier.

% - Criticize rigid design
%   - AM, FM, RC are target at specific communication patterns
%   - Fail to deliver comparable performance with real applications
% - Suggest application-orientation
%   - Generic X Optimum {Preikschat:1994}

Nevertheless, good communication performance is hard to obtain when
dealing with anything but the test applications supplied by the
developers of the communication package. Real applications, not seldom,
present disappointing performance figures. We believe the origin of this
shortcoming to be in the attempt of delivering generic communication
solutions.  Most high performance communication systems are engaged in a
``the best'' solution for a certain architecture. However, a definitive
best solution, independently of how well tuned to the underlying
architecture it is, cannot exist, since parallel applications
communicate in quite different ways. Aware of this, many communication
packages claim to be ``minimal basis'', upon which application-oriented
abstractions can (have to) be implemented. Once more, there cannot be a
best minimal basis for all possible communication strategies. This
contradiction between generic and optimal is consequently discussed
in~\cite{Preikschat:1994}.

If applications communicate in distinct ways, we have to deliver each
one a tailored communication system that satisfies its requirements (and
nothing but its requirements). Of course we cannot implement a new
communication system for each application, what we can do is to design
the communication system in such a way that it is possible to tailor it
to any given application. In the \emph{Embedded Parallel Operating
  System} (\textsc{Epos}) project~\cite{Froehlich:ehpc:1999}, we
developed a novel design method that is able to accomplish this duty.
\textsc{Epos} consists of a collection of components, a component
framework, and tools to support the automatic construction of a variety
of run-time systems, including complete operating systems.

The particular focus of this paper is on \textsc{Epos} communication
system, which has been implemented for a cluster of PCs interconnect by
a Myrinet high-speed network. In the next sections, the
\emph{Application-Oriented System Design} method will be introduced,
followed by a case study of its applicability to design a communication
system.  The implementation of this communication system is later
discussed, including a preliminary performance evaluation. The paper is
closed with authors' conclusions.


\Section{Application-Oriented System Design}

\emph{Application-Oriented System Design} (AOSD) is a novel operating
system\footnote{The term ``operating system'' is used here in its
  broadest meaning, encompassing all kinds of run-time support systems.}
design method that, as the name suggests, has a strong compromise with
applications. Its main goal is to produce run-time support systems that
can be tailored to fulfill the requirements of particular applications.
Accomplishing this task begins with the decomposition of the operating
system domain in abstractions that are natural to application
programmers. This is exactly the decomposition strategy promoted by
\emph{Object-Oriented Design}~\cite{Booch:1994} and may sound obvious to
application designers, but most system designers simply neglect the
problem domain and let implementation details, such as target
architecture, programming languages, and standardized interfaces, guide
the design process~\cite{Pike:2000}. Application programmers, not
seldom, get run-time systems that barely resemble the corresponding
domain.

The next step is to model software components that properly represent
the abstractions from the decomposed domain. Extensive components, that
encapsulate all perspectives of an abstraction in a single entity, are
not an alternative, since we want components to closely match the
requirements of particular applications. A more adequate approach would
be to apply the commonality and variability analysis of
\emph{Family-Based Design}~\cite{Parnas:1976} to yield a family of
abstractions, with each member capturing a significant variation and
shaping a component.  Nevertheless, this approach has the inconvenient
of generating a high number of components, thus increasing the
complexity of the composition process.  We handle this drawback by
exporting all members of a family through a single \emph{inflated
  interface}.  In a system designed accordingly, adequate members of
each required family could be automatically select by a tool that
performs syntactical analysis of the corresponding application's source
code.

Another important factor to be considered while modeling abstractions is
scenario independence. When a designer realizes, for instance, that a
communication mechanism may have to be specialized in order to join a
multithreaded scenario, he has to choose between modeling a new family
member and capturing this scenario dependency in a separate construct.
Allowing abstractions to incorporate scenario dependencies reduces their
degree of reusability and produces an explosion of scenario-dependent
components. Therefore, an application-oriented design should try to
avoid it, only allowing those variations that are inherent to the family
to shape new members. The resulting \emph{scenario-independent
  abstractions} shall be reusable in a larger variety of scenarios, some
of them unknown at the time they were modeled.

Scenario specificities, in turn, can be captured in constructs like the
\emph{scenario adapters} described in~\cite{Froehlich:sci:2000}. Because
scenario adapters share the semantics of collaborations in
\emph{Collaboration-Based Design}~\cite{Reenskaug:1992}, one could say
that an abstraction collaborates in a scenario. This separation of
abstractions and scenarios is also pursued by \emph{Aspect-Oriented
  Programming}~\cite{Kiczales:1997}, nevertheless, though it provides
means to support this separation, it does not yet feature a design
method.

\fig{aosd}{An overview of Application-Oriented System Design.}

After decomposing the problem domain in scenario-independent
abstractions and scenario-adapters, organizing the solution domain
accordingly becomes straightforward. \emph{Inflated interfaces} hide
most details of the solution domain by exporting all members of a family
of abstractions, as well as the corresponding scenario adapters, through
a single interface. Since these interfaces emanate directly from the
problem domain, application programmers should feel comfortable to use
them. What is missing to deliver a true application-oriented run-time
system is a way to assemble components together correctly and
efficiently. By correct assembly we mean preserving the individual
semantics of each component in the presence of others and under the
constraints of an execution scenario.  By efficient assembly we mean
preserving their individual efficiency in the resulting composite.

One possibility to produce the desired compositions is to capture a
reusable system architecture in a \emph{component framework}. A
framework enables system designers to predefine the relationships
between abstractions and therefore can prevent misbehaved compositions.
Furthermore, a framework defined in terms of scenario adapters can
achieve a high degree of adaptability. Efficient composition can be
accomplished if the framework uses \emph{Generative Programming}
techniques~\cite{Czarnecki:2000}, such as \emph{static metaprogramming}.
Since static metaprograms are executed at compile-time, a statically
metaprogrammed framework can avoid most of the overhead typical of
traditional object-oriented frameworks.  It is also important to notice
that, though component composition would take place at compile-time,
nothing would prevent components from using dynamic reconfiguration
mechanisms to internally adapt themselves.

In brief, \emph{Application-Oriented System Design}
(figure~\ref{fig:aosd}) is a multiparadigm design method that supports
the construction of customizable run-time support systems by decomposing
the system domain in families of reusable, scenario-independent
abstractions and the corresponding scenario adapters. Reusable system
architectures are modeled as component frameworks that can guide the
compilation of the target system. Application programmers interact with
the system through inflated interfaces, without having to know details
about the organization of families or scenarios.


\Section{The Design of an Application-Oriented Communication System}

We applied Application-Oriented System Design to develop a communication
system for clusters of workstations in the realm of project
\textsc{Epos}. By decomposing the domain of high-performance cluster
communication, we obtained two families of abstractions:
\texttt{Network} and \texttt{Communicator}.  The first family abstracts
the physical network as a logical device able to handle one of the
following strategies: \emph{datagram}, \emph{stream}, \emph{active
  message} (AM), \emph{asynchronous remote copy} (ARC), or
\emph{distributed shared memory} (DSM). Since system abstractions are to
be independent from execution scenarios, aspects such as access control
and sharing are not modeled as properties of \texttt{Network}, but as
``decorations'' that can be added by scenario adapters. \textsc{Epos}
family of \texttt{Networks} is depicted in figure~\ref{fig:network}.

\fig{network}{The Network family.}

For most of \textsc{Epos} abstractions, architectural aspects are also
modeled as part of the execution scenario, however, network
architectures vary drastically, and implementing portable abstractions
would certainly push performance bellow acceptable levels. Consider, for
instance, the architectural differences between Myrinet and SCI: a
portable active message abstraction would underestimate Myrinet, while a
portable asynchronous remote copy abstraction would misuse SCI.
Therefore, the family of \texttt{Network} abstractions will be
individually designed and implemented for each desired network
architecture. Some family members that are not directly supported by the
architecture will be emulated, because we believe that, if the
application really needs (or wants) them, it is better to emulate them
close to the hardware.

The second family of abstractions deals with communication end-points.
These are the abstractions effectively used by applications to
communicate with each other. \textsc{Epos} family of
\texttt{Communicators} is shown in figure~\ref{fig:communicator} and has
the following members: \emph{connection}, \emph{port}, \emph{mailbox},
\emph{active message handle}, \emph{asynchronous remote copy segment},
and \emph{distributed shared memory segment}. Again, scenario
dependencies such as multitasking and multithreading are modeled as
scenario adapters.

\wfig{communicator}{The Communicator family.}

These two families, when completely implemented for a variety of network
architectures, will yield a large number of components that will be
stored in a repository together with several other subsystems. With such
a large number of components, selecting and configuring the right ones
in order to produce an application-oriented system may become a defying
activity, even when assisted by visual tools. Hence,
Application-Oriented System Design proposes all members of a family to
be exported through a single, inflated interface.  In this way,
application programmers can design and implement their applications
referring to fewer interfaces and ignoring the particular properties of
each component.  Actually, the programmer catches a comprehensive
perspective of the family, as though a super-component was available,
and uses the operations that better fulfills his
requirements\footnote{In case an application programmer with enough
  expertise about the system wishes to extend a component, or bypass
  automatic configuration, the individual interfaces of each family
  member are also made available.}. As an example, the inflated
interface of the \texttt{Communicator} family is depicted in
figure~\ref{fig:communicator_int}.

\fig{communicator_int}{The Communicator inflated interface and its
  realizations.}

The process of binding an inflated interface to one of its realizations
can be automated if we are able to clearly distinguish one realization
from another.  In \textsc{Epos}, we identify realizations through the
signatures of their methods, so that syntactical analysis of application
source code can identify which of the realizations are needed. If two
realizations present the same set of signatures, as with \texttt{Port}
and \texttt{Mailbox} in figure~\ref{fig:communicator_int}, syntactical
analysis may not be enough to decide for one of them, and user
intervention may be required. Nevertheless, although \texttt{Port} and
\texttt{Mailbox} differ only semantically\footnote{Both \texttt{Port}
  and \texttt{Mailbox} support multiple senders, but the first supports
  a single receiver, while the second support multiple receivers too.},
the syntactical analysis of other components may render one possibility
invalid. For example, if the application is known to execute on a
single-task-per-node basis, a scenario with multiple receivers is not
possible, breaking the tie in favor of \texttt{Port}.

The set of selected family members, in addition to information obtained
from the user, defines the execution scenario for the application. As
proposed by Application-Oriented System Design, scenario peculiarities
are applied to abstractions by means of scenario adapters. In
\textsc{Epos}, a scenario adapter wraps an abstraction as to enclose
invocations of its operations between the \texttt{enter} and
\texttt{leave} scenario primitives (see
figure~\ref{fig:scenario_adapter}). Besides enforcing scenario specific
semantics, a scenario adapter can extend the state and behavior of an
abstraction, for it inherits from both scenario and abstraction. For
example, all abstractions in a scenario could be tagged with a
capability, without internal modifications, by associating the
capability with the corresponding scenario.

\wfig{scenario_adapter}{The structure of a scenario adapter.}

\textsc{Epos} statically metaprogrammed framework is defined around a
collection of interrelated scenario adapters. As shown in
figure~\ref{fig:scenario_adapter}, scenario adapters are designed as
parametrized classes that take a component (abstraction) as parameter.
Hence they can act as placeholder for components in the framework. In
order to generate a system, information about the mapping of inflated
interfaces to realizations, and also about system-wide properties such
as target architecture and protection, are passed as input to the
metaprogram.  The resulting system would include only the components
needed to support the corresponding application in the respective
execution scenario.


\Section{The Implementation of the Communication System}
 
Following the design described earlier, \textsc{Epos} is being
implemented as a collection of components, a framework, and a set of
tools that support automatic generation of application-oriented run-time
systems. Currently the system can run in two modes: native on
\textsc{ix86} computers, and guest on \textsc{Linux} systems.  The
\textsc{ix86}-native version can be configured either to be embedded in
the application (single-task), or as $\mu$-kernel (multi-task), while
the \textsc{Linux}-guest version comprises a library and kernel loadable
modules.

Components are being implemented in C++ and described in XML. The XML
description is used by the tools that support automatic system
generation. The framework is also implemented in C++, but mainly with
its built-in static metalanguage.  The tools to proceed syntactical
analysis of applications, to configure the target system, and to check
configuration dependencies are made available to users through a
compiler wrapper similar to \texttt{mpicc}. This enables users to
implicitly generate the run-time system during the compilation of
applications.  Nevertheless, if these tools fail to configure the
system, user intervention is requested via an interactive graphical tool
that supports configuration adjustments by feature
selection\footnote{The same tool can be used to tailor the system
  manually.}.

\textsc{Epos} family of communication abstractions is currently being
implemented for the Myrinet high-speed network~\cite{Boden:1995}. So
far, we concluded the implementation of the \texttt{Datagram}
\texttt{Network}, the \texttt{Port} and the \texttt{Mailbox}
\texttt{Communicators}, and a mechanism to support \emph{remote object
  invocation} (ROI). These components can be adapted to the following
scenarios: \texttt{Protected}, \texttt{Multitask}, \texttt{Multithread},
and \texttt{Global}. The \texttt{Protected} scenario ensures that only
authorized agents gain access to abstractions. The \texttt{Multitask}
and \texttt{Multithread} scenarios adapt abstractions to execute in the
presence of multiple tasks and threads. The \texttt{Global} scenario
adapts abstractions to interact in a cluster-wide environment of active
objects, hiding communication behind ordinary method invocations.

With these components we generated a couple of application-oriented
run-time systems. One of them supports two simple applications that
communicate intensively in a producer/consumer fashion. For this
purpose, they use the \texttt{Datagram Network} and the \texttt{Port
  Communicator}. Figures~\ref{fig:latency} and~\ref{fig:bandwidth} show
respectively the latency and the bandwidth available to these
applications in both \textsc{ix86}-native and \textsc{Linux}-guest
modes. The hardware test-bed for this measurements consisted of two PCs
connected to the same Myrinet switch. Each PC has a 266 MHz Pentium II
processor, 128 MBytes of memory (10 ns DRAM) on a 66 MHz bus, and a
32-bits Myrinet NIC on a 33 MHz PCI bus.

\wfig{latency}{Datagram/Port one-way latency.}
\wfig{bandwidth}{Datagram/Port one-way bandwidth.}

The difference in favor of the \textsc{ix86}-native version arises from
the contiguous memory allocation method adopted, which allows the DMA
engines on the Myrinet card to be programmed with logical addresses and
eliminates an additional message copy into a system DMA buffer. This
difference could have been even more expressive if the applications were
multithreaded, since the extra copy would have concurred with
application threads for processor time and especially for memory
bandwidth.  Nevertheless, most parallel applications execute on a
single-task-per-node basis and will benefit from the single-task
versions of \textsc{Epos}. Other communication systems, such as Active
Messages~\cite{Lumetta:1997}, Fast Messages~\cite{Pakin:1997},
PM~\cite{Tezuka:1997}, and BIP~\cite{Prylli:1998}, run exclusively on
top of ordinary operating systems, such as \textsc{Unix} or
\textsc{Windows NT}, and have no alternative to escape the extra copy
than making a system call to translate logical addresses into physical
ones, what is usually even more time consuming\footnote{Sharing system
  DMA buffers with applications is not really an alternative, because
  they are usually restricted in size and will not be able accommodate
  the large data structures typical of parallel applications. It would
  most likely result in the application performing the additional
  copy.}.

Furthermore, \textsc{Epos} quality evaluation should not be restricted
to performance. Because only the components effectively required by the
application are included, the resulting system is usually extremely
compact. The system in the example above, which in addition to
communication also includes process and memory management, has a size of
11 KBytes. This means less resource consumption and less space for bugs.
Usability is also improved, since \textsc{Epos} visible interfaces are
defined in the context of applications.


\Section{Conclusion}

In this paper we introduced \emph{Application-Oriented System Design}, a
novel design method that prevents the monolithic conception of generic
solutions that fail to scale along with application demands. We also
described how this method has been deployed to construct a communication
system for the Myrinet high-speed network. This communication system,
implemented in the realm of project \textsc{Epos}, consists of a
collection of \emph{application-ready, scenario-independent
  abstractions} (components) that can be adapted to specific execution
scenarios by means of \emph{scenario-adapters} and can be arranged in a
\emph{statically metaprogrammed framework} to produce an
application-oriented communication system. The system is presented to
application programmers through \emph{inflated interfaces} that gather
all variations of an abstraction (family members) under a single
comprehensive interface. By programming based on these interfaces,
programmers enable \textsc{Epos} tools to automatically generate an
adequate system for their applications.

The results obtained so far are highly positive and help to corroborate
the guidelines of \emph{Application-Oriented System Design}, as well as
\textsc{Epos} design decisions. The evaluation of \textsc{Epos}
communication system revealed performance figures that, as far as we are
concerned, have no precedents in the history of PC clusters
interconnected with 32-bits Myrinet. Nevertheless, \textsc{Epos} is a
long term project that aims at delivering application-oriented run-time
systems to a large universe of applications. Therefore, several system
abstractions, scenario adapters, and tools are still to be implemented
or improved.


\bibliographystyle{latex8}
\bibliography{se,os,network,cluster,guto}

\end{document}

%%% Local Variables: 
%%% mode: latex
%%% TeX-master: t
%%% End: 
