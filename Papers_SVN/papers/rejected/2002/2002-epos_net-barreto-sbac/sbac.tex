\documentclass[english]{article}
\usepackage[T1]{fontenc}
\usepackage[latin1]{inputenc}
\usepackage{babel}
\usepackage{times}
\usepackage{latex8}
\usepackage{graphics}

%\twocolumn
%\pagestyle{plain}
%\setlength{\textheight}{8.875in}
%\setlength{\textwidth}{6.875in}
%\setlength{\columnsep}{0.3125in}
%\setlength{\oddsidemargin}{-0.1in}
%\setlength{\evensidemargin}{-0.1in}
%\setlength{\topmargin}{1in}
%\setlength{\headheight}{-0.1in}
%\setlength{\headsep}{-0.1in}
%\setlength{\parindent}{1pc}

\onecolumn
\pagestyle{plain}
\linespread{1.4}
\setlength{\textheight}{8.8in}
\setlength{\textwidth}{6.5in}
\setlength{\columnsep}{0.3125in}
\setlength{\oddsidemargin}{-0.1in}
\setlength{\evensidemargin}{-0.1in}
\setlength{\topmargin}{1in}
\setlength{\headheight}{-0.1in}
\setlength{\headsep}{-0.1in}
\setlength{\footskip}{.5in}
\setlength{\parindent}{1pc}
\fontsize{14}{16pt}

\font\tenhv  = phvb at 10pt
\font\elvbf  = ptmb scaled 1100

\newcommand{\fig}[3][htbp]{
  \begin{figure}[#1] {\centering\scalebox{0.8}{\includegraphics{#2}}\par}
    \caption{#3\label{fig:#2}}
  \end{figure}
}

\title{Design and Implementation of \textsc{Epos} Communication System\\
  for Fast Ethernet}
%  \thanks{The research work described in this article has been partially
%  supported by Fraunhofer FIRST.}}

\author{
  Fernando Barreto and Ant�nio Augusto Fr�hlich\\
  UFSC/CTC/LISHA\\
  PO Box 476\\
  88049-900 Florian�polis - SC, Brazil\\
  \texttt{\{fbarreto|guto\}@lisha.ufsc.br}\\
  \texttt{http://www.lisha.ufsc.br/$\sim$\{fbarreto|guto\}}
}

\date{}

\bibliographystyle{latex8}

\begin{document}


\maketitle

\begin{abstract}
  
  This paper presents the \textsc{Epos} approach to deliver parallel
  applications a high performance communication system. \textsc{Epos} is
  not an operating system, but a collection of components that can be
  arranged together to yield a variety of run-time systems, including
  complete operating systems. This paper focuses on the communication
  subsystem of \textsc{Epos}, which is comprised by the \emph{network
    adapter} and \emph{communicator} scenario-independent system
  abstractions. Like other \textsc{Epos} abstractions, they are adapted
  to specific execution scenarios by means of scenario adapters and are
  exported to application programmers via inflated interfaces. The paper
  also covers the implementation of the \emph{network adapter} system
  abstraction for ordinary Fast Ethernet networks, completing a previous
  article that described the implementation for the Myrinet high-speed
  network.

\end{abstract}


\section{Introduction}

The parallel computing community has been using clusters of commodity
workstations as an alternative to expensive parallel machines for
several years by now. The results obtained meanwhile, both positive and
negative, often lead to the same point: inter-node communication.
Consequently, much effort has been dedicated to improve communication
performance in these clusters: from the hardware point of view,
high-speed networks and fast buses provide for low-latency and
high-bandwidth; while from the software point of view, \emph{user-level
  communication}~\cite{Bhoedjang:1998} enables applications to access
the network without operating system intervention, significantly
reducing the software overhead on communication. 

Nevertheless, good communication performance is hard to obtain when
dealing with anything but the test applications supplied by the
developers of the communication package. Real applications, not seldom,
present disappointing performance figures~\cite{npb}. We believe the
origin of this shortcoming to be in the attempt of delivering generic
communication solutions.  Most high-performance communication systems
are engaged in a ``the best'' solution for a certain architecture.
However, a definitive best solution, independently of how well tuned to
the underlying architecture it is, cannot exist, since parallel
applications communicate in quite different ways. Aware of this, many
communication packages claim to be ``minimal basis'', upon which
application-oriented abstractions can (have to) be implemented. Once
more, there cannot be a best minimal basis for all possible
communication strategies. This contradiction between generic and optimal
is consequently discussed in~\cite{Preikschat:1994}.

If applications communicate in distinct ways, we have to deliver each
one a tailored communication system that satisfies its requirements (and
nothing but its requirements). Of course we cannot implement a new
communication system for each application, what we can do is to design
the communication system in such a way that it becomes possible to
tailor it to any given application. In the \emph{Embedded Parallel
  Operating System} (\textsc{Epos}) project~\cite{Froehlich:ehpc:1999},
we developed a novel design method that is able to accomplish this duty.
\textsc{Epos} consists of a collection of components, a component
framework, and tools to support the automatic construction of a variety
of run-time systems, including complete operating systems.

The particular focus of this paper is the implementation of
\textsc{Epos} communication system for ordinary \textsc{Fast Ethernet}
networks.  In the next sections, \textsc{Epos} communication system
design will be discussed.  The implementation of this communication
system will be discussed later, including a preliminary performance
evaluation.  The paper is closed with authors' conclusions.


\section{Communication System Design}

\textsc{Epos} has been conceived following the guidelines of traditional
object-oriented design. However, scalability and performance constrains
impelled us to define some \textsc{Epos} specific design elements. These
design elements will be described next in the realm of the communication
system.


\subsection{Scenario-independent System Abstractions}

Granularity plays a decisive role in any component-based system, since
the decision about how fine or coarse components should be have serious
implications. A system made up of a large amount of fine components will
certainly achieve better performance than one made up of a couple of
coarse components, since less unneeded functionality incurs less
overhead. Nevertheless, a large set of fine components is more complex
to configure and maintain.
  
In \textsc{Epos}, visible components have their granularity defined by
the smallest-yet-application-ready rule. That is, each component made
available to application programmers implements an abstract data type
that is plausible in the application's run-time system domain. Each of
these visible components, called \emph{system abstractions}, may in turn
be implemented by simpler, non application-ready components.

In any run-time system, there are several aspects that are orthogonal to
abstractions. For instance, a set of abstractions made SMP safe will
very likely show a common pattern of synchronization primitives. In this
way, we propose \textsc{Epos} system abstractions to be implemented as
independent from execution scenario aspects as possible. These
adaptable, scenario-independent system abstractions can then be put
together with the aid of a \emph{scenario adapter}.

Communication is handled in \textsc{Epos} by two sets of system
abstractions: \emph{network adapters} and \emph{communicators}. The
first set regards the abstraction of the physical network as a logical
device able to handle one of the following strategies: datagram, stream,
active message, or asynchronous remote copy. The second set of system
abstractions deals with communication end-points, such as links, ports,
mailboxes, distributed shared memory segments and remote object
invocations. Since system abstractions are to be independent from
execution scenarios, aspects such as reliability, sharing, and access
control do not take part in their realizations; they are ``decorations''
that can be added by scenario adapters.

For most of \textsc{Epos} system abstractions, architectural aspects are
also seen as part of the execution scenario, however, network
architectures vary drastically, and implementing unique portable
abstractions would compromise performance.  As an example, consider the
architectural differences between Myrinet and SCI: a portable active
message abstraction would waste Myrinet resources, while a portable
asynchronous remote copy would waste SCI resources. Therefore,
realizations for the \emph{network adapter} system abstraction shall
exist for several network architectures. Some abstractions that are not
directly supported by the network will be emulated, because we believe
that, if the application really needs (or wants) them, it is better to
emulate them close to the hardware.


\subsection{Scenario Adapters}

\textsc{Epos} system abstractions are adapted to specific execution
scenarios by means of \emph{scenario adapters}. Currently, \textsc{Epos}
scenario adapters are classes that wrap system abstractions, so that
invocations of their methods are enclosed by the \texttt{enter} and
\texttt{leave} pair of scenario primitives. These primitives are usually
inlined, so that nested calls are not generated. Besides enforcing
scenario specific semantics, scenario adapters can also be used to
``decorate'' system abstractions, i.e., to extend their state and
behavior.  For instance, all abstractions in a scenario may be tagged
with a capability to accomplish access control.

In general, aspects such as application/operating system boundary
crossing, synchronization, remote object invocation, debugging and
profiling can easily be modeled with the aid of scenario adapters, thus
making system abstractions, even if not completely, independent from
execution scenarios.

The approach of writing pieces of software that are independent from
certain aspects and later adapting them to a given scenario is usually
referred to as \emph{Aspect-Oriented
  Programming}~\cite{Kiczales:1997}. We refrain from using this
expression, however, because much of AOP regards the development of
languages to describe aspects and tools to automatically adapt
components (\emph{weavers}). If ever used in \textsc{Epos}, AOP will
give means but not goals.


\subsection{Inflated Interfaces}

Another important decision in a component-based system is how to export
the component repository to application programmers. Every system with a
reasonable number of components is challenged to answer this question.
Visual and feature-based selection tools are helpless if the number of
components exceeds a certain limit ---depending on the user expertise
about the system, in our case the parallel application programmer
expertise on operating systems. Tools can make the selection process
user-friendlier, but certainly do not solve the user doubt about which
selections to make. Moreover, users can usually point out what they
want, but not how it should be implemented. That is, it is perhaps
straightforward for a programmer to choose a mailbox as a communication
end-point of a datagram oriented network, but perhaps not to decide
whether features like multiplexing and dynamic buffer management should
be added to the system.
  
The approach of \textsc{Epos} to export the component (system
abstraction) repository is to present the user a restricted set of
components. The adoption of scenario adapters already hides many
components, since instead of a set of scenario specific realizations of
an abstraction, only one abstraction and one scenario adapter are
exported.  Nevertheless, \textsc{Epos} goes further on hiding components
during the system configuration process. Instead of exporting individual
interfaces for each flavor of an abstraction, \textsc{Epos} exports all
of its flavors with a single \emph{inflated interface}. For example, the
datagram, stream, active message, and asynchronous remote copy
\emph{network adapters} are exported by a
single~\texttt{Network\_Adapter} inflated interface as depicted in
figure \ref{fig:network_adapter}.

\fig{network_adapter}{The Network\_Adapter inflated interface
  and its partial realizations.}

An inflated interface is associated to the classes that realize it
through the \emph{selective, partial realize} relationship. This
relationship is partial because only part of the inflated interface is
realized, and it is selective because only one of the realizations can
be bound to the inflated interface at a time. Each selective realize
relationship is tagged with a key, so that defining a value for this key
selects a realization for the corresponding interface. The way this
relationship is implemented enables \textsc{Epos} to be configured by
editing a single key table, and makes conditional compilations and
``makefile'' customizations unnecessary.

The process of binding an inflated interface to one of its realizations
can be automated if we are able to clearly distinguish one realization
from another.  In \textsc{Epos}, we identify abstraction realizations by
the signatures of their methods. In this way, an automatic tool can
collect signatures from the application and select adequate realizations
for the corresponding inflated interfaces. Nevertheless, if two
realizations have the same set of signatures, they must be exported by
different interfaces.

The combination of \emph{system abstractions}, \emph{scenario adapters}
and \emph{inflated interfaces}, effectively reduces the number of
decisions the user has to take, since the visual selection tool will
present a very restricted number of components, of which most have been
preconfigured by the automatic binding tool. Besides, they enable
application programmers to express their expectations concerning the
run-time system simply by writing down well-known system object
invocations.


\Section{Communication System Implementation for Fast Ethernet}

\textsc{Epos} is coded in C++ and is currently being developed to run
either as a native ix86 system, or at guest-level on Linux. The
ix86-native system can be configured either to be embedded in the
application, or as $\mu$-kernel. The Linux-guest system is implemented
by a library and a kernel loadable module. \textsc{Epos} communication
system implementation is detailed next.

To provide a Fast Ethernet communication suport to the \textsc{Epos}
system in a parallel cluster, came the ideia of create a communication
mechanism. This mechanism may be considered a new protocol that should
use the network speed technology at its maximum to execute the message
exchanging. To reach this objective, the protocol must implement only
the minimum requirements so that the system can transmit and receive a
network package.

The architecture of this communication protocol, comparing with the OSI
protocol stack (ISO, 1981), is located only at physic, data link, and
application layers. This protocol is located only at these layers
because parallel aplications in a dedicated cluster architecture do not
need of the resources of the other layers. With these characteristics,
the implementation of these layers is avoided and also the resulting
overhead caused by them.

In the context of the \textsc{Epos} object oriented system, these layers
would be abstracted among members of three families. One member is the
interface with user application, representing in the form of a mailbox
variation from Communicators family, and its called cluster\_com. The
other member stores the methods and essential attributes in the
communication mechanism, represented as a variation of Network family,
named cluster\_net. There is a member represented as a Interrupt Handler
family's member named cluster\_int. This way the cluster\_net and
cluster\_int members abstract the physic, data link layer and the
cluster\_com member with the application layer. The Communicator's
attribute channel is a "neutral" attibute, it is only used to directly
reference the network family from Communicators Family.

The cluster\_com member acts in the handling of two modules calls
(Send,Receive) that directly reference the cluster\_net's methods by
means of channel attribute. In other words, it acts as an interface to
obtain the passed parameters, make the analisys, fill attributes end
call the cluster\_net's methods.

So that the user's application can use the resources of this
communication protocol, the system \textsc{Epos} supplies the inflated
interface of Communicator's family. That allows to user's application to
know how to use the parameters passed and how to execute the
transmission or the reception. The same happens with the Network family,
that supplies an inflated interface, allowing the Communicators family
to reference Network's methods.

The cluster\_net member adds, as variations of Network family, control
attributes, an object queue and trasmission and reception methods that
work directly with the methods of object attribute Device. These control
attributes are filled by the information obtained from the parameters in
the methods called from cluster\_com, as much in transmission as in
reception, being used to identify the destination or origin of a
message, and execute it's handling. The members are showned in figure
\ref{fig:family}.

In this paper, the cluster\_net members were projected to make use only
of Device attribute, inherited from network family, of Fast Ethernet
type to communication with the cluster.

The member cluster\_int is responsable by handle all the packages
incomming from NIC Ethernet in the system. The handling is made in the
analisys of the package and its contents, taking measures according with
the result of the analisys. The measures taken by the cluster\_int
member are signaling and filling the package cluster\_net reception
queue. The picture \ref{fig:family} presents the main responsabilities
of cluster\_int in the \textsc{Epos} system.

\fig{family}{\textsc{Epos} implementation}

The routines of transmission and reception of the cluster\_net's
communication mechanism are described in section 3.1 and 3.2

\SubSection{Transmission}

In the transmission, the transmiting method called from cluster\_com
creates Ethernet Frames based in the information contained in the
control attributes. The message is placed in the Ethernet package,
beggining at memory region of the user process. In the case that the
message is bigger then the data area of the Ethernet package including
their headers, there is the need of the fragmentation. The number of
fragments are generated only a quantity necessary to transmit a given
message of any size.

The fragments will be sent as fast as possible by polling in the object
attribute's method instanciated from family Device. This way , the
implementation try to leave the NIC as used as possible by using the
polling technic. This technic does not generate problems, because only
one process is being executed in the host.

The transmission process is influenced by the cluster\_int member, that
identify the control packages comming from the receiever, signalizes
cluster\_net, reporting a broken sequence of packages or a final ACK.
Both permiting that the Send method performs the measures according with
the contents of the control package.


\SubSection{Reception}

In the reception it's composed by a method of cluster\_net called by
cluster\_com and by a cluster\_int routine. The method called by
cluster\_com depends almost entirely from cluster\_int routine.

The method in cluster\_net, effects the filling of the control
attributes of cluster\_net, also puts the process in wait state until
the message is ready to be trasfered from the kernel to the user space.

The cluster\_int routines, represented as interruption handler, analises
the header of the package that arrived with the existing control
attributes in the object instantiated from cluster\_net. The analisys
tries to identify wich application is waiting for a message. After
identify the application, the package is enqueued in the cluster\_net's
reception queue, that is an attribute instantiated from queue family of
a fifo type when instantiating a cluster\_net object.

The feature Order is used to check if there was a sequence break in the
arriving packages. If was, a package is sent to the sender host
reporting a sequence break. As the sender checks the break, it sends
back to the receiver from the lost package.

The method in cluster\_net called from cluster\_com transfers the data
from the receiver queue to the user space as soon as cluster\_int
signals cluster\_net that the entire message was received. A control
acknowledgement package is sent to the sender reporting that the message
was successfully received.

\Section{\textsc{Epos} guest level in Linux}

To validate this communication protocol, a Linux module was created to
provide support of this protocol to applications. The interface created
by cluster\_com is simulated in Linux as a additional code in IPC
syscall. The cluster\_net methods and attributes are declared in the
module as a control structure to each application and function calls in
the syscall. The member cluster\_int is represented by a softirq and the
standard routines of a NIC device driver. The figures \ref{fig:softirq}
and \ref{fig:syscall} describes in details the kernel Linux data flow.

\fig{softirq}{Package receiver softirq}

\fig{syscall}{Syscall user interface}

This protocol architecture in Linux is described in the following subsections.

\SubSection{Transmission}

In transmission, the system call fills the control structure data with
the syscall parameters. With this informations, the transmission
function makes Ethernet frames filling the frame data region with the
process user memory. With a created package, its transmission is
realized enqueueing it in the transmission queue of Fast Ethernet device
driver. This queue packages are processed by the network driver until
the queue get empty or the network driver overloaded. If a overload
occur, the package transmission is interrupted and reactivated when a
network hardware became ready to process the queue again. This
reactivation occurs by a network hardware interruption that release the
queue to be processed again and set the transmission softirq in Linux.

The softirqs exist in Linux since kernel 2.3.43. They can be considered
system routines that are executed at interruption level. Their main
purpose is execute most of the tasks of the interruption handler. There
are two network softirqs, the transmissor (NET\_TX\_SOFTIRQ) and the
receiver (NET\_RX\_SOFTIRQ).

\SubSection{Reception}

In the reception process, the module establishes itself like a system
call and as a receiver softirq routine.

Besides this system call fills the control structure data, it puts the
process in wait state until the message become ready to be transfered
from kernel to user space.

The softirq routine become responsable for context analisys of the
Ethernet packages that arrive from the network driver.

When a package of this protocol arrives, it's handled by the driver
interruption routine, then the receive softirq routine analises the
package header and execute the receive routine related to the header
identification. This routine locate among control structures the one
that fits the package identification and puts it in the reception queue
of this structure. When the routine checks that the entire message was
already received, the softirq change the process state from waiting to
ready.

When the system call runs again, it'll process the reception queue of
its control structure transfering data from kernel to the process
designated memory region referenced by the control structure. As soon as
this task is completed, it returns to the user context.

As the package receive softirq routine realize that it's the last
package being analised, it sends a control package to the sender host
reporting that all packages were successfully received. This kind of
sequence control simulate the feature order's use in network\_adapter
family.

The figure \ref{fig:fluxo} ilustrate how the common flow of this
communication mechanism works.

\fig{fluxo}{Normal communication flow}

As the receiver is always waiting for the message before the sender
starts to transmit, the sincronization between them is dispensable. It
enables the sender to begin the message transmission as soon as possible
by polling technic.

Like the project in \textsc {EPOS}, this implementation has the
following otimizations: less number of layers and consequently less
header per package, routing and checksum inexistence and there is a
simple control sequence.

The ideal communication,it means, without package loose, it'll be sent
as many packages as it needs to mount the message. Just one control
acknowledge package from receiver to sender is used to report that all
fragments successfully arrived, as show in figure \ref{fig:fluxo}.

\Section{Results}

In effects of comparission and a better look of this protocol, protocols
GAMMA (CHIOLA, 2002) and TCP/IP were used in a cluster environment with
Linux kernel 2.4.17.

The experiments realized with this protocol ran at AMDk6 64MBytes of RAM
memory, Intel EtherExpress 100 network cards interconnected by a 3Com
switch 3C16734B 100Mbits/s.

The GAMMA protocol, out of the context of this paper, has as its main
advantage an use of the network bandwidth almost at maximum, because its
data transfer occurs at device driver interruption level. This kind of
implementation breaks the rules of Linux network system. This
architecture receiver buffers are directly allocated in user space
without Linux network system usage. However, important changes must be
made in the network driver to manipulate these buffers. This protocol is
completely dependent from the network hardwares that it can operate, it
means it doesn't support any network hardware. The user process that
uses this protocol should have its memory region forbbiden to be
paginated, this avoid that a page fault at interruption level occurs. To
avoid a memory region to be paginated, privileged permissions from the
operating system are needed.

The figure \ref{fig:compara} shows a comparative chart of the maximum
throughputs reached in the experiments with the protocols GAMMA, TCP/IP
and this one. All of them are compared to the ideal maximum allowed by
physical environment of 12,5MB/s (BUYYA, 1999).

\fig{compara}{Communication performance}

The chart data was obtained using tools of each one of the protocols to
get maximum throughput. The GAMMA protocol used the "ping-pong" tool.
TCP/IP used the "ttcp" tool. This protocol used a measuration in
receiver system at system call level that evaluate the time to transfer
data from sender host to memory user in receiver host. Both tools
display the communication maximum throughput. Many tests battery were
made and the higher throughput average were collected.

Over the chart analisys shown in figure \ref{fig:fluxo}, packages with
small size (0 to 100KBytes) can be observed, this protocol showed itself
more efficiently than TCP/IP. As most of the messages exchanged among in
paralell cluster are of small sizes, this protocol can efficiently
approach these types of message reaching its objective.

\nocite{Rubini:2001} \nocite{Bar:2000} \nocite{osi} \nocite{Buyya:1999}
\nocite{Chiola:1997}

\bibliography{os,se,network,cluster,parallel,guto}

\end{document}
