\documentclass[conference]{IEEEtran}

\usepackage[utf8]{inputenc}	% for Latin languages
\usepackage[T1]{fontenc}	% for ISO and UTF characters
\usepackage[english]{babel}	% for multilingual support
\usepackage{graphicx}
\usepackage{subfig}

\newcommand{\fig}[4][htbp]{
  \begin{figure}[#1] {\centering\scalebox{#2}{\includegraphics{fig/#3}}\par}
    \caption{#4\label{#3}}
  \end{figure}
}

\begin{document}

\title{On the Way to a Trustful Embedded Infrastructure for Future Internet}

\author{
    \IEEEauthorblockN{Rodrigo Vieira Steiner, Alexandre Massayuki Okazaki, Alex de Magalhães Machado, \\Giovani Gracioli, and Antônio Augusto Fröhlich}\\
    \IEEEauthorblockA{
  Software/Hardware Integration Lab\\
  Federal University of Santa Catarina\\
  PO Box 476, 88040-900 - Florianópolis, SC, Brazil \\
            \{rodrigo,alexandre,alex,giovani,guto\}@lisha.ufsc.br
    }
}

\maketitle

%Summary: FP7 - IoT
%Conferência alvo: IWCMC2011
%Deadline: 15/12/2010

%Nature: Trustworthiness on Future Internet
%Scope: Named Data Networking executing on EPOSMote II
%Motivation: It is a mistake to assume that each embedded object of Future Internet will implement the TCP/IP stack, especially considering aspects such as power consumption, memory requirement, size, and cost. Furthermore, the IP protocol itself, the driving force of technological revolution which we live, seems to have reached its limit when it comes to Internet and the Future Internet of Things.
%Contribution: This work proposes a Trustful Infrastructure for Future Internet, consisting of an embedded platform (EPOSMote II) and a stack of communication protocols (UDP@NDN@C-MAC) focusing on trustworthiness.

%Title: On the Way for a Trustful Infrastructure for Future Internet

\begin{abstract}
%+ Motivation
%todo
It is a mistake to assume that each embedded object of Future Internet will implement the TCP/IP stack, especially considering aspects such as power consumption, memory requirement, size, and cost.
Furthermore, the IP protocol itself, the driving force of technological revolution which we live, seems to have reached its limit when it comes to the Future Internet.
%+ Problem
%todo
One of the challenges in designing Future Internet applications, however, is the very distinct basic software (e.g OS, communication protocols) that may be present in different devices, and how to combine them in order to build a coherent final application.
The current research and development are focusing on Internet based communication protocols, and on top of that, new protocols are defined to address the target service.
Safety and security capabilities in this domain are open issues that need to be addressed.
Nevertheless, implementing traditional secure communication approaches in resource-constrained platforms, such as those found in Future Internet, is not a straightforward task.
%+ Contribution
In this article, we discuss the limitations of the TCP/IP stack and present an alternative that might be used for Future Internet.
Our solution is composed by an embedded platform, EPOSMote II, and a stack of communication protocols (UDP@NDN@C-MAC) designed specifically to guarantee a trustful communication while still compromised with the low utilization of resources.
\end{abstract}

\section{Introduction}
% Reprogramao, estrutura + protocolo
% O que  um protocolo de disseminao
% Importncia de um prot. de diss.
% Importncia de uma boa estrutura
% Ex. de uso: reprogramao RSSF
% Contribuio: infra-estrutura p/ disseminao e reprogramao
% Organizao do artigo

%Reprogramming the software of a program in execution is a feature present in most computer environments.
%A wide range of applications make use of some reprogramming method: from internet browsers to dedicated systems, as controllers in vehicles for instance.
%Due to the limited resources of embedded systems, the software reprogramming infrastructure is different from that implemented in conventional computer environments.
%Moreover, some of these dedicated systems, as Wireless Sensor Networks (WSNs), are formed by a big amount of nodes, in which collecting and reprogramming all nodes is impractical. 

%Wireless Sensor Networks (WSN), are typically formed by a big amount of nodes, in which collecting and reprogramming all nodes is impractical.
A software reprogramming infrastructure for a Wireless Sensor Network (WSN) is composed of a data dissemination protocol and a structure capable of organizing the data in the system's memory. By using an embedded Operating System (OS) it is possible to provide for embedded applications an infrastructure to hide this data organization. Usually, the reprogramming structures present in OSs are composed of updatable modules. These modules are memory position independent and are replaced at runtime~\cite{sos}~\cite{contiki}. 

In addition, it is essential that all new data of one or more modules is correctly received by all nodes involved in the reprogramming process. In order to provide safe data transfer, a data dissemination protocol is used together with the OS infrastructure.
%In general, a dissemination protocol works as follows: the dissemination begins from a base station responsible to transfer the new data to its neighbor nodes. Once a node receives the new data, it is capable of retransmitting it to its own neighbors.  The process repeats until the entire network is up to date~\cite{moap}~\cite{deluge}.
%Thus, if a node A is neighbor of a node B, and the node B has A and C as neighbors, the node B after receiving the new data from the node A will retransmit it to the node C. The process repeats for all nodes until the entire network is up to date~\cite{moap, deluge}.

Epos Live Update System (\ELUS{}) is an OS infrastructure for software updating that has better performance in terms of memory consumption, method invocation time, and reconfiguration time when compared to related works~\cite{Gracioli:2010}.
%Although this favorable result, we identifed that memory consumption of \ELUS{} could be improved.
Although this favorable result, \ELUS{} memory consumption could still be improved.
Furthermore, \ELUS{} does not have any support for data dissemination. 

%In summary, in this paper, we make the following contributions:
%\begin{itemize}
%	\item We improved the memory consumption of \ELUS{} by using C++ templates specialization techniques (more than 50\% of improvement). %\ELUS{} is implemented around the EPOS metaprogrammed framework~\cite{Frohlich:2001}. Thus, some code regions are duplicated due to the use of templates. We identified some of these regions and applied a template specialization technique~\cite{stroustrup:2000}.
%	\item We provide a domain engineering analysis considering the data dissemination protocols characteristics. The protocols characteristics are decomposed into a feature diagram that shows common and variable features present in different protocols.
%	\item We integrate a data dissemination protocol to \ELUS{} and evaluate the new infrastructure in terms of memory consumption and dissemination and reprogramming times. %As result, we provide a lightweight software reprogramming infrastructure for resource-constrained embedded systems.
%\end{itemize}

%In this paper we make the following contributions: (i) we have improved the memory consumption of \ELUS{} by using C++ templates specialization techniques; (ii) we provide a data dissemination protocol domain engineering analysis considering protocols characteristics, and integrate our developed protocol to \ELUS{}; and (iii) we evaluate the new infrastructure in terms of memory consumption, and dissemination and reprogramming times.
In this paper we make the following contributions: (i) we develop and integrate a data dissemination protocol to \ELUS{}; (ii) we improve the memory consumption of \ELUS{} by using C++ templates specialization techniques; and (iii) we evaluate the new infrastructure in terms of memory consumption, and dissemination and reprogramming times.

%The rest of this paper is organized as follows.
%Section~2 presents the related work.
%Section~\ref{sec:ddp} shows the designed dissemination protocol and compares its characteristics to other proposed protocols.
Section~\ref{sec:ddp} presents the design of the developed dissemination protocol.
Section~\ref{sec:integration} presents the integration between the dissemination protocol and \ELUS{}.
The evaluation of the infrastructure is carried out in Section~\ref{sec:evaluation}.
Finally, Section~\ref{sec:conc} concludes the paper.


\section{Related Work}
\label{sec:related_work}



\section{Building a Trustful Infrastructure for Future Internet}
\label{sec:solution}
The Internet architecture demonstrate inefficiency and problems in several and large areas, such as mobility, real-time applications,
failures (e.g. equipment, software bugs, and configuration mistakes), and especially in pervasive security problems \cite{Rexford:2010}.
Moreover, the Internet lacks effective solutions in terms of scalability and sustainability, 
consuming much more energy and hindering the management of countless sensor devices that are so important for several applications in the Future Internet.
Hence, we propose the use of a stack of communication protocols (UDP@NDN@C-MAC), in the scope of the EPOSMote project,
designed specifically to guarantee a trustful communication
%Our solution also includes EPOSMote II, an embedded platform. Thus, 
while still compromised with the low utilization of resources (processing, memory, power and communication bandwidth).
%and the use of EPOSMote II which is an embedded platform and represents a typical Future Internet device.

\subsection{EPOSMote}
The EPOSMote is an open hardware project~\cite{eposmote}. Initially it aimed at 
the development of a wireless sensor network module, and focused on environment 
monitoring. Its first version, the EPOSMote I, features an 8-bit AVR microcontroller, 
IEEE 802.15.4 communication capability and a small set of sensors.

As the project evolved a second version arose, with the objective of delivering a 
hardware platform to allow research on energy harvesting, biointegration, and 
MEMS-based sensors. The EPOSMote II focus on modularization, and thus is composed 
by interchangeable modules for each function.

Figure \ref{emote2-block_diagram} shows an overview of the EPOSMote II architecture.
Its hardware is designed as a layer architecture composed by a main module,
a sensoring module, and a power module. The main module is responsible for processing
and communication. It is based on the Freescale MC13224V microcontroller~\cite{mc13224v}, which possess 
a 32-bit ARM7 core, an IEEE 802.15.4-compliant transceiver, 128kB of flash memory, 80kB of ROM memory
and 96kB of RAM memory. We have developed a startup sensoring module, which contains some sensors  
(temperature and accelerometer), leds, switches, and a micro USB (that can also be used as power supply). 
Figure \ref{emote2-mc13224v-pictures-real_white_background} shows the development kit which is slightly 
larger than a R\$1 coin, on the left the sensoring module, and on the right the main module.

\fig{.45}{emote2-block_diagram}{Architectural overview of EPOSMote II.}

\fig{.07}{emote2-mc13224v-pictures-real_white_background}{EPOSMote II SDK side-by-side with a R\$1 coin.}

\subsection{C-MAC}
C-MAC is a highly configurable MAC protocol for WSNs realized as a framework of
medium access control strategies that can be combined to produce
application-specific protocols~\cite{steiner:2010}. It enables application
programmers to configure several communication parameters (e.g.  synchronization,
contention, error detection, acknowledgment, packing, etc) to adjust the protocol
to the specific needs of their applications. The framework was implemented in C++ 
using static metaprogramming techniques (e.g. templates, inline functions, and 
inline assembly), thus ensuring that configurability does not come at expense of 
performance or code size. The main C-MAC configuration points include:

\textbf{Physical layer configuration:} These are the configuration points defined
by the underlying transceiver (e.g. frequency, transmit power, date rate).

\textbf{Synchronization and organization:} Provides mechanisms to send or receive
synchronization data to organize the network and synchronize the nodes duty
cycle.

\textbf{Collision-avoidance mechanism:} Defines the contention mechanisms used to
avoid collisions. May be comprised of a carrier sense algorithm (e.g. CSMA-CA),
the exchange of contention packets (\emph{Request to Send} and \emph{Clear to
Send}), or a combination of both.

\textbf{Acknowledgment mechanism:} The exchange of \emph{ack} packets to
determine if the transmission was successful, including preamble acknowledgements.

\textbf{Error handling and security:} Determine which mechanisms will be used to
ensure the consistency of data (e.g. CRC check) and the data security.

The Future Internet will be composed by a wide range of both applications and devices, 
each with its own requirements and available resources. Through C-MAC configurability we
can provide the most adequate MAC functionalities for each case, instead of providing a 
general non-optimal solution for all of them.

\subsection{NDN}
Communication in NDN is impelled by the data consumers.
Nodes that are interested in a content transmit \emph{Interest} packets, which contains the name of the requested data. %selector, nonce
Every node that receives the \emph{Interest} and have the requested data can respond with a \emph{Data} packet that follows back the path from which the \emph{Interest} came. %content name, signature, signed info, data
It is important to notice that one \emph{Data} satisfies one \emph{Interest}, thus ensuring flow balance in the network.
Since the content being exchanged is identified by its name, all nodes interested in the same content can share transmissions (considering a broadcast medium, which is the case for most Future Internet devices).

NDN packet forwarding engine has three main data structures: the FIB (Forwarding Information Base), which is used to forward \emph{Interest} packets to potential sources; 
the ContentStore, which is a buffer memory used to maximize the sharing of packets; 
and the PIT (Pending Interest Table), which is used to keep track of \emph{Interest} packets so that \emph{Data} packets can be sent to its requester(s).

When a node receives an \emph{Interest} packet it searches for its content name, looking for a match primarily at the ContentStore, then the PIT, and ultimately at the FIB.
If there is a match at the ContentStore, it is sent and the \emph{Interest} discarded.
Otherwise, if there is a match at the PIT, the set of requesting interfaces for that data is updated, and the \emph{Interest} discarded (at this point an \emph{Interest} in this data has already been sent).
Otherwise, if there is a match at the FIB, the \emph{Interest} is sent towards the data, and a new PIT entry is created. 
In case there is no match for the \emph{Interest} then it is discarded.

As for the \emph{Data} packet they simply follow the chain of PIT entries back to the original requester(s).
When a node receives a \emph{Data} packet it searches for its content name. 
If there is a ContentStore match, then the \emph{Data} is a duplicate and is discarded.
%A FIB match means there are no matching PIT entries, so the \emph{Data} is unsolicited and it is discarded.
In case of a PIT match, the data is validated, added to the ContentStore, and sent to the set of requesting interfaces from the corresponding PIT entry.

In NDN the name in every packet is bound to its content with a signature.
This enables data integrity and provenance, allowing consumers to trust the data they receive regardless of how the data came to them.
To provide content protection and access control NDN uses encryption.
The encryption of content or names is transparent to the network, since to NDN it is all just named binary data.
%The signature algorithm used may be selected by the content publisher, 
%and chosen to meet performance requirements such as latency or computational cost of signature generation or verification.
Nevertheless, NDN does not mandate any particular key distribution scheme, signature, or encryption algorithm.

\subsection{UDP}
The User Datagram Protocol has been chosen for its simplicity. Its simple transmission 
model avoids unnecessary overhead, since it does not handle reliability, ordering, 
and data integrity, leaving these characteristics to be treated in other layers if necessary, which is a 
perfect blend with the rest of our protocol stack.


% ------------------------------------------------------------------------------
\section{Conclusões}
\label{sec:discussion}
% Aplicações para sistemas embarcados usualmente necessitam interagir com
% diversos tipos de dispositivos de hardware como sensores, atuadores, 
% transmissores, receptores e \emph{timers}.
%
% Interface de função estrangeira é o mecanismo adotado por Java para superar as
% limitações da linguagem e permitir acesso direto a memória e a dispositivos de
% hardware. 
% Entretanto, como mostrado na seção \ref{sec:related_work}, as principais FFIs 
% Java não são eficientes em termos de consumo de recursos, ou possuem limitações
% de projeto que dificultam o desenvolvimento de novas interfaces entre Java e os
% dispositivos de hardware.
%
Neste artigo, apresentou-se um meio de realizar a interface entre componentes 
de hardware e aplicações Java para sistemas embarcados. 
Isto foi obtido utilizando-se a interface de função estrangeira da JVM KESO e o
EPOS.

% O EPOS permite o desenvolvimento de aplicações portáveis, independentes de
% especificidades de máquina. 
% Isto é conseguido utilizando-se o conceito de mediadores de hardware, os quais
% sustentam um contrato de interface entre abstrações de sistemas e a máquina.

% O JVM KESO compila o bytecode de uma aplicação Java em código C e gera as partes
% da JVM necessárias pela aplicação. 
% A FFI da KESO também utiliza esta abordagem estática, gerando o código C 
% especificado nas classes \emph{Weavelet}. 
% Então o código C gerado pelo compilador KESO e pela FFI da KESO são compilados
% em conjunto em código nativo, utilizando-se um compilador C padrão.

Nós avaliamos nossa abordagem em termos de desempenho, consumo de memória e 
portabilidade.
Para a aplicação utilizando o mediador de hardware da UART o sobrecusto 
de tempo obtido foi menos de 0.04 \% do tempo total de execução da aplicação e
nossa solução é 38 vezes mais rápido do que a JNI da Sun.
O consumo de memória para tal aplicação foi de 33KB, incluindo todo o suporte
de ambiente de execução, o qual é adequado para diversos sistemas embarcados.
Utilizando o EPOS nós obtivemos portabilidade para várias plataformas e 
utilizando o conceito de componentes híbridos podemos utilizar os mesmos 
adaptadores de código nativo tanto para componentes implementados em 
hardware como implementados em software.

Visando avaliar nossa abordagem em uma aplicação real, nós escrevemos
adaptadores de código nativo para um componente o qual realizada estimativa de
movimento para codificação de vídeo H.264.

% ------------------------------------------------------------------------------



\bibliographystyle{IEEEtran}
\bibliography{references}

\end{document}
