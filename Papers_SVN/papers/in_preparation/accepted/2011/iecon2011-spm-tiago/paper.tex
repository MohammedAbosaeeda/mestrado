\documentclass[a4paper, 10pt, conference]{IEEEtran}

\usepackage[latin1]{inputenc}	% for Latin languages
\usepackage[T1]{fontenc}	% for ISO and UTF characters
\usepackage[english]{babel}	% for multilingual support
\usepackage{graphicx}
\usepackage[caption=false,font=footnotesize]{subfig}
\usepackage{verbatim}
\usepackage{fancyvrb}
\usepackage{color}



\newcommand{\fig}[4][htbp]{
  \begin{figure}[#1] {\centering\scalebox{#2}{\includegraphics{fig/#3}}\par}
    \caption{#4\label{#3}}
  \end{figure}
}

\newcommand{\multfigtwoh}[6][htbp]{
\begin{figure*}[#1]
  \centering
  \subfloat[]{\label{#3}\scalebox{#2}{\includegraphics{fig/#3}}}
  \subfloat[]{\label{#4}\scalebox{#2}{\includegraphics{fig/#4}}}
  \caption{#6}
  \label{#5}
\end{figure*}
}

\newcommand{\multfigtwov}[6][htbp]{
\begin{figure}[#1]
  \centering
  \subfloat[]{\label{#3}\scalebox{#2}{\includegraphics{fig/#3}}}\\
  \subfloat[]{\label{#4}\scalebox{#2}{\includegraphics{fig/#4}}}
  \caption{#6}
  \label{#5}
\end{figure}
}

\newcommand{\figR}[5][htbp]{
  \begin{figure}[#1]{\centering\scalebox{#2}{\includegraphics[angle=#5]{fig/#3}}\par}
    \caption{#4\label{#3}}
  \end{figure}
}

\newcommand{\figTC}[4][htbp]{
  \begin{figure*}[#1] {\centering\scalebox{#2}{\includegraphics{fig/#3}}\par}
    \caption{#4\label{#3}}
  \end{figure*}
}

\newcommand{\figEMPTY}[5][htbp]{
  \begin{figure}[#1]
    \fbox{\begin{minipage}{#2}\hfill\vspace{#3}\end{minipage}}
    \centering
     \label{#4}
    \caption{#5}
  \end{figure}
}

\newcommand{\tab}[3][htbp]{
  \begin{table}[#1]
    \footnotesize
    \centering
    \caption{#3}
    \include{tab/#2}
    \label{#2}
  \end{table}
}

\newcommand{\tabTC}[3][htbp]{
  \begin{table*}[#1]
    \footnotesize
    \centering
    \caption{#3}
    \include{tab/#2}
    \label{#2}
  \end{table*}
}



\let\VERBATIM\verbatim
\def\verbatim{%
\def\verbatim@font{\small\ttfamily}%
\VERBATIM}

\begin{document}

\title{Run-time Scratch-pad Memory Management for Embedded Systems}

%\author{
%\alignauthor
%Tiago Rog�rio M�ck and Ant�nio Augusto Fr�hlich\\
%\affaddr{Software/Hardware Integration Lab}\\
%\affaddr{Federal University of Santa Catarina}\\
%\affaddr{Florian�polis, Brazil}\\ 
%\email{\{tiago,guto\}@lisha.ufsc.br}
%}

\author{\IEEEauthorblockN{Tiago Rog�rio M�ck and Ant�nio Augusto Fr�hlich}
\IEEEauthorblockA{Software/Hardware Integration Lab\\
Federal University of Santa Catarina\\
Florian�polis, Brazil\\
\{tiago,guto\}@lisha.ufsc.br}
}

%\author{Authors omitted for blind review}

%TODO passar corretor do WORD.

\maketitle


\begin{abstract}
\textit{Scratch-pad memories} (SPM) are being increasingly used in embedded systems
due to their higher energy and silicon area efficiency in comparison to ordinary caches. However, in order
to exploit all of its advantages, efficient memory allocation mechanisms must be
provided. In this work we propose a run-time memory management approach for SPMs
at OS-level that can be combined with other compile-time approaches. The operating
system memory manager takes annotations inserted into the code by the programmer as
hints to choose the most appropriate memory (i.e. main memory or SPM) for each allocation.
Experimental results confirm the approach's efficiency when compared to a similar
compile-time technique.
\end{abstract}


\section{Introduction}

Contemporary embedded system applications are requiring faster processors and
larger memories. Previous studies have shown that the memory subsystem is responsible
for 50\%--75\% of the total system power consumption and occupies significant chip
area~\cite{Kandemir::2002}, thus becoming an important
optimization point. Most systems rely on cache-based memory hierarchies due to the
capacity of caches to exploit the spatial and temporal locality of memory access.
However, caches require additional memory for tag and address comparison logic, which may
significantly increase their power consumption and silicon area. These often render cache
inappropriate for embedded applications~\cite{Verma::2006}. In addition,
\textit{worst case execution time} (WCET) must be overestimated due to the lack
of predictability in many cache implementations~\cite{Wehmeyer::2005}, which may forbid their use on
real-time embedded systems.

\emph{Software-controlled caches}, often called \textit{scratch-pad memories} (SPM), 
are emerging as alternatives to traditional caches. SPMs do not need the extra logic
to map data and instructions  because memory allocation 
is controlled by software, which makes them more power and area efficient than
ordinary caches~\cite{Verma::2006}. Also, considering that the contents of the SPM are
known, tighter bounds on WCET prediction can be achieved~\cite{Wehmeyer::2005}.
Nevertheless, to exploit all the advantages of SPMs, an efficient allocation
mechanism must be provided in software. Several software allocation approaches that
have been proposed rely on profiling information to define the instructions and
data to be allocated to the SPM at compile-time. These approaches have a major drawback.
The use of profiling limits the scope of the mapping techniques not only because of the
difficulty in obtaining reasonable profiles but also due to high space and time requirements
for their generation~\cite{Shrivastava::2009}. This is especially true for dynamic
applications in which the memory access patterns depend on the input data~\cite{Cho::2009}.

%These compile-based approaches
%have two major drawbacks. First, the use of profiling limits the scope of the
%apping techniques not only because of the difficulty in obtaining reasonable
%profiles but also due to high space and time requirements to generate a
%profile~\cite{Shrivastava::2009}. This is especially true for dynamic
%applications were the memory access patterns depend on the application input data~\cite{Cho::2009}.
%Second, this kind of allocation scheme leads
%to a more complicated software development and reduced portability because it
%requires very specific toolchains with modified compilers.

In this work we propose a run-time memory management approach for SPMs
at OS-level which does not rely on compiler support, profiling or hardware support. We provide
a framework that abstracts the SPM and parts of the main memory 
as operating system heaps. When the application dynamically allocates data, 
the operating system uses annotations, inserted into the code by the programmer, 
as hints to choose the most appropriate level in the memory hierarchy to allocate 
the data (i.e. the main memory heap or the SPM heap). If an allocation request
on the preferred memory component fails, the operating system attempts to allocate
on a less optimal one, thus handling exhaustion of a particular memory component. 
With this approach, the OS can take advantage of the developer's knowledge about
the application to provide an efficient data allocation.

The remaining of this paper is organized as follows: section \ref{RELATED_WORK} presents
a discussion about related work; section \ref{PROPOSAL} presents our proposal;
and sections \ref{EVALUATION} and \ref{CONCLUSION} show our experimental
results and conclusions.


\section{Related work}
\label{RELATED_WORK}

%In the few last years, several SPM management approaches have been proposed.
%We have separated these approaches in the different categories shown in figure
%\ref{fig_spm_state_of_art_small}. The \emph{static} 
%approaches~\cite{Avissar::2002, Angiolini::2004, Hiser::2004} are those in
%which the contents of the SPM are fixed at compile-time and never change.
%In the \emph{dynamic} approaches, the SPM contents change during the program
%execution. If the code and/or data sets that will be moved to/from the SPM
%are defined at compile-time, we call it a 
%\emph{compile-time} approach~\cite{Verma::2006, Udayakumaran::2006, Ozturk::2007}.
%If this sets can change or are defied during the program execution, we call it a 
%\emph{run-time} approach~\cite{Cho::2009, Shrivastava::2009, Milidonis::2009, Dominguez::2005, McIlroy::2008, Nguyen::2007, Huneycutt::2002}.
%The run-time approaches generally fallow one or more of the following three
%basic methodologies: it is \emph{compiler-assisted}
%if actions are taken by the compiler to define the allocation; \emph{hardware-assisted} if it relies on special
%hardware support; and also there are the approaches which require
%\emph{run-time software support} provided by a standalone library or by
%an operating system.

%\fig{.40}{fig_spm_state_of_art_small}{Classification of SPM allocation approaches}

In the few last years, several SPM management approaches have been proposed.
We have separated these approaches in the different categories shown bellow.%in 
%figure \ref{fig_spm_state_of_art_small2}.

%\fig{.35}{fig_spm_state_of_art_small2}{Classification of SPM allocation approaches}

\textbf{Static approaches} are those in which the contents of the SPM are fixed prior to
system deployment and never change. They can also be subdivided into two categories.
In \textbf{compile-time techniques}, the allocation is defined during the compilation process
~\cite{Avissar::2002}\cite{Hiser::2004}. In \textbf{post-compile techniques},
algorithms are applied at the final binary code to change the memory
allocation~\cite{Angiolini::2004}. These static allocation approaches either use
greedy strategies to find an efficient solution, or model the problem as a
\emph{knapsack} problem or an \textit{integer-linear programming} problem (ILP)
to find an optimal solution. \textit{Avissar et al}~\cite{Avissar::2002} proposed
a memory allocation strategy that is optimal in relation to the profiling provided.
In \textit{Angiolini et al}~\cite{Angiolini::2004} another optimal solution was
proposed. This technique is applied in the application binary code and is optimal
in relation to a certain set of execution traces. Despite being optimal, all the
static methods have the disadvantage of being dependent of an efficient application
profile.

\textbf{Dynamic approaches} are those in which the SPM contents change during the program
execution. Dynamic methods based on \textbf{compile-time techniques} change the
SPM allocation based only on compile time decisions and profile information
\cite{Verma::2006}\cite{Udayakumaran::2006}\cite{Ozturk::2007}.
\textit{Verma and Marwedel}~\cite{Verma::2006} proposed an overlay-based memory
allocation approach for both code and data. It uses ILP to find the optimal memory
allocation and overlay points that minimizes energy consumption for a given profile. 
Compile-time dynamic methods are usually more efficient than static ones 
in exploiting the benefits of SPMs, but they are still dependent on good application profiles.
Another issue of these methods is that they do not provide an efficient allocation
when the accessed memory regions are correlated to the application's input data~\cite{Milidonis::2009}.
This issue is overcome by the \textbf{run-time techniques}. These approaches are generally composed by a 
run-time software or operating system which determines the SPM contents based on information 
inserted by the compiler and/or using hardware support.

In \textit{Milidonis et al}~\cite{Milidonis::2009} the global data structures
are sliced into \emph{tiles} which are allocated to the SPM by the compiler.
A hardware component called \textit{Data Type Unit} (DTU) works like a special cache
that keeps references to the most accessed tiles and sends commands to a DMA unit to
move the data between the SPM and main memory when necessary. This approach solves the
problem of dynamic access patterns, but it still require an initial profiling to determine
the possible tiles, and a very specific hardware support that forbids it application to fixed
hardware platforms.

In \textit{Shrivastava et al}~\cite{Shrivastava::2009} is proposed an approach that
does not require profiling or hardware support. They manage function stack frames,
allocating them to the SPM when they are in use. They proposed a software library which
provides functions to manage the stack. The calls to the library are
inserted by the compiler before and after function calls, using information
provided by \textit{Global Call Control Flow Graphs} (GCCFG).
This approach have the advantage of not requiring profiling
or hardware support, but it still require compiler modifications and focus
on only one class of applications (multimedia applications).
 
In \textit{Cho et al}~\cite{Cho::2009} was proposed an allocation scheme that
integrates compiler, OS, and hardware, which is very similar to
\textit{Milidonis et al}~\cite{Milidonis::2009}. First, through profiling, the
most accessed data sets are defined. Based on a cost analysis, the compiler
define the optimal points to insert the operating system calls the reallocate the SPM data.
When reallocating, the operating system verifies a hardware structure which keeps information about the most
accessed data sets. The OS uses this information to define the new memory allocation.
Since the data addresses change during run-time, they added a simplified MMU for
address translation. The authors show that the technique is very efficient for
multimedia applications, but it requires compiler, OS, and
hardware support, thus it comes up with all the drawbacks of the dynamic run-time
techniques discussed earlier.

%There are dynamic run-time techniques that require only run-time software support.
%A common technique is \textbf{software caching}. This approach emulates a cache in
%SRAM using software. The tag, data and valid bits are all managed by code inserted
%at each memory access. Significant software overhead is incurred to manage these
%fields, and all the improvements of the SPM in relation to caches disappear.
%Another run-time-only technique that follows a different approach was proposed in
%\textit{Nguyen et al}~\cite{Nguyen::2007}. In this approach the authors uses the
%same ideas discussed in the hardware-based approaches to implement SPM management
%for Java applications. The main difference is that the tracking of the accessed
%data structures and allocation code are implemented at JVM level. This technique
%does not require special compiler or hardware support, but the overhead of a JVM
%may be unacceptable on an embedded system.

All of the previous techniques handled only code and/or global/stack data. The
only known technique that allocates heap data to the SPM was proposed in
\textit{Dominguez et al}~\cite{Dominguez::2005}. The proposed technique divides
the application code in regions in which the beginning and end of each region is
defined by functions and loops bounds. A compile-time analysis
is performed in these regions to define the heap variables that will be allocated to
the SPM and where the code to make this allocation is to be inserted. Despite of being the
first technique that allocates heap data to the SPM, it follows the
\textbf{dynamic compile-time approach} discussed earlier, and suffers from the
same problems. Another work that deals with the management of heap data was
presented in \textit{McIlroy et al}~\cite{McIlroy::2008}. In this work, the authors
just suppose a run-time system to manage SPMs as dynamic heaps, 
and propose a heap memory allocation algorithm optimized for very small memories.


\section{Run-time SPM management}
\label{PROPOSAL}

%In our approach we assume that embedded systems target specific hardware platforms
%and specific applications, thus allowing the designer to provide an embedded operating system 
%with information about the underlying memory hierarchy and the kind of access to 
%the application data. The operating system can use this information to perform a more
%efficient memory allocation. 

We have implemented a framework for the C++ language which provides annotated
versions of the \emph{new} and \emph{delete} operators, allowing the programmer
to easily insert allocation hints into the program. The framework was implemented on the
\textit{Embedded Parallel Operating System} (EPOS)~\cite{EPOS::2011:Online}.
EPOS relies on the \textit{Application-Driven Embedded System Design}
(ADESD)~\cite{Frohlich::2001} methodology to design and implement both software and hardware components
that can be automatically adapted to fulfill the requirements of particular applications.
High reusability and low overhead are achieved by a careful implementation that makes use of
object-oriented programming and \emph{generative programming}~\cite{Czarnecki::2000} techniques,
including \emph{static metaprogramming}. A detailed description of the memory management
framework and its implementation on EPOS is presented bellow.

\subsection{Placement \emph{new} and \emph{delete} operators}

The annotated versions of the \emph{new} and \emph{delete} operators were implemented
using \emph{placement new expressions}, which are part of the ISO C++ standard and are supported by any 
standard C++ compiler. \emph{Placement new and delete expressions} provide a way to implement
custom allocation strategies. Figure \ref{fig_placement_new} shows the syntax of
these expressions. Invocations of the type \emph{new ( expression-list ) type\_name} 
will require its respective overloaded version of the \emph{operator new} function.
The ISO C++ standard already defines a default implementation for the type \emph{void*}.
Pointer placement new is necessary for hardware that expects a certain object
at a specific hardware address~\cite{Stroustrup::1995}.

\begin{figure}[htbp]
\small
\scriptsize
\begin{Verbatim}[frame=single,framerule=0.2pt,framesep=1pt]
(a)  new type_name (initializer-list);
(b)  void * operator new (size_t);

(c)  new ( expression-list ) type_name (initializer-list);
(d)  void * operator new (size_t, expression-list);

(e)  new ((void*)ADDRESS) type_name;
(f)  void * operator new (size_t, void *);
\end{Verbatim}
  \normalsize
  \caption{Default syntax (a) and function (b) of the \emph{new} operator. 
           Bellow, the generic syntax (c) and function (d) of the \emph{placement new}
           is shown with an example of the default implementation (e) (f).}
  \label{fig_placement_new}
\end{figure}

\subsection{Memory management on EPOS}

%EPOS can be configured using three different architectures: \emph{kernel}, \emph{built-in}, and
%\emph{library}. In the two former architectures, EPOS support multiple applications and 
%virtual memory, while in the latter one, EPOS supports a single application and is linked
%with it as a library at compile-time. Since most embedded system are designed to run a 
%single application, in this work we have focused on the \emph{library} configuration. 

The memory mapping in EPOS is shown in figure \ref{fig_epos_mem_framework}. The allocation
of code and global data is defined at compile-time for both operating system and application.
During system initialization, the initial stack used by operating system is allocated in
the \emph{system stack} region. All of the remaining free memory is distributed between the
\emph{system and application heaps}. These regions are used for dynamic memory allocation, 
and each one is managed by an instance of the \emph{Heap} C++ class.
The system heap (\emph{System::heap}) is used to allocate the stacks for the application's and
system's threads, while the application heap (\emph{Application::heap}) is used by the C++ \emph{new}
and \emph{delete} operators, which are used by the application programmer to
dynamic allocate memory.

%\fig{.39}{fig_epos_mem_framework}{EPOS memory mapping.}

%\fig{.39}{fig_epos_mem_framework_spm}{EPOS memory mapping when using the new framework.}

\multfigtwov{.35}{fig_epos_mem_framework}{fig_epos_mem_framework_spm}
{fig_epos_mem_framework_all}
{EPOS memory mapping before (a) and after (b) using the new framework.}

%\fig{.39}{fig_spm_epos_heap}
%{C++ class responsible for managing dynamic memory allocation requests.}

\subsection{The allocation framework}

Figure \ref{fig_epos_mem_framework_spm} shows the modified memory mapping when using
our SPM management approach. A new \emph{Heap} instance is created to manage
the SPM, which is memory-mapped. We overloaded the implementation of the \emph{new}
and \emph{delete} operators to include a decision algorithm
which decides if the data should be allocated using the \emph{main\_heap}
or the \emph{spm\_heap}, based on annotations given by the programmer.  

Three different types of annotations are supported: \textbf{ALLOC\_HIGH}, \textbf{ALLOC\_LOW},
and \textbf{ALLOC\_NORMAL}. When the \emph{ALLOC\_HIGH} annotation is used, it means that particular object
has a high memory access priority (i.e. performance/power consumption of read/write
operations on it have a major impact over the system efficiency) and should be allocated
on the more efficient level of the memory hierarchy (e.g. a SPM). The
\emph{ALLOC\_LOW} have the inverse meaning. It means that the object has a low
memory access priority and it can be allocated on the less efficient level of the
memory hierarchy without a significant impact on the system efficiency. Finally,
the \emph{ALLOC\_NORMAL} annotation is used when the programmer does not 
know or does not care about the physical location of the data. In this case, the OS 
decides the best place to allocate the object.

Figure \ref{fig_spm_code_alloc} shows how these annotations are used with
the \emph{new} operators. The annotations are defined as \emph{enum constants}
and a new \emph{operator new} implementation is defined using the \emph{placement new}
syntax. In this example, four arrays of type \emph{int}
are allocated with and without annotations. When annotations
are not used, the OS assumes \emph{ALLOC\_NORMAL}. The usage of the \emph{delete}
operator doesn't change.  

\begin{figure}[ht]
\small
\scriptsize
\begin{Verbatim}[frame=single,framerule=0.2pt,framesep=5pt]
//Placement type definitions
typedef enum {
    ALLOC_HIGH,
    ALLOC_LOW,
    ALLOC_NORMAL,
} alloc_priority;

//Annotations implementation using placement new
void * operator new(size_t bytes, alloc_priority p){
    ...
    //do memory allocation    
    ...
}

//Anotated memory allocation examples
int *at_spm = new (ALLOC_HIGH) int[10];
int *at_main_mem = new (ALLOC_LOW) int[10];
int *somewhere = new (ALLOC_NORMAL) int[10];
int *somewhere_else = new int[10];/*equivalent to the
                                   statement above/*

delete[] at_spm;
delete[] at_main_mem;
delete[] somewhere;
delete[] somewhere_else;
\end{Verbatim}
  \normalsize
  \caption{Memory allocation request and implementation using annotations.}
  \label{fig_spm_code_alloc}
\end{figure}
 
The pseudo-code in figure \ref{fig_spm_code_alloc_algo} describes the decision
algorithm used to define where the data will be allocated. If an allocation
request can't be accomplished on the preferred memory component, the operating system
attempts to allocate on a less optimal memory component. If it is left to the
operating system to decide where to allocate, it attempts to allocate on the heap
with the highest percentage of free space, thus handling exhaustion of a
particular memory component. The implementation of the memory deallocation is
straightforward. The OS just checks the pointer address to decide if the memory
is to be deallocated on the main memory or on the SPM.

\begin{figure}[ht]
\scriptsize
\centering
\begin{Verbatim}[frame=single,framerule=0.2pt,framesep=5pt]
do memory allocation (size, annotation)
    case: annotation = ALLOC_HIGH 
        if fits on spm (size)
            allocate on spm
        else
            allocate on main mem
    case: annotation = ALLOC_LOW
        if fits on main mem (size)
            allocate on main mem
        else
            allocate on spm
    case: annotation = ALLOC_NORMAL
        if spm free percentage > 
                  main mem free percentage
            if fits on spm (size)
                allocate on spm
            else
                allocate on main mem
        else
            if fits on main mem (size)
                allocate on main mem
            else
                allocate on spm
\end{Verbatim}
  \normalsize
  \caption{Algorithm which defines where the allocation will be done.}
  \label{fig_spm_code_alloc_algo}
\end{figure}


\section{Evaluation}
\label{EVALUATION}

We have used the \textit{Xilinx ML605 Evaluation Board} to build a platform for our
evaluation. The platform is based on open source hardware IP-cores from 
\textit{OpenCores}~\cite{OpenCores::2011:Online}. We have used the \emph{aeMB} processor
which is connected to a memory and several peripherals using a \emph{Wishbone} bus. 
Figure \ref{fig_eval_plaftform_no_cache} shows a block diagram of our platform 
(for simplicity, it shows only memory-related blocks). It features a 
cache for instructions. The 512 Mb DDR3 SDRAM available in the ML605
board is used as the main memory; while a 16 Kb SPM is created using the FPGA's
internal RAM blocks. We have implemented a \emph{bus sniffer}, which is 
connected to both instruction (\emph{IWB}) and data (\emph{DWB}) sides of the
Wishbone bus. This block contains performance counters, which are used to
collect statistics about memory operations. The final system
is generated using ISE 13.1 (for hardware synthesis) and GCC 4.1.1 (for software
compilation), both provided by Xilinx.

\fig{.35}{fig_eval_plaftform_no_cache}
{Block diagram of the evaluation platform. Only memory-related components are shown.} 

\subsection{Benchmarks}

In order to evaluate our approach, we have selected the following set of benchmarks from
\emph{MiBench}~\cite{MiBench::2011:Online}:

\textbf{Dijkstra: } calculates the shortest path between nodes using Dijkstra's algorithm.
%\item[Blowfish] a symmetric block cipher with a variable length key.
%\item[Rijndael] implements the AES's block cipher.

\textbf{SHA: } a secure hash algorithm.
%\item[JPEG] algorithm for image compression and decompression.

\textbf{Susan: } implements a set of image recognition algorithms. 
In the following evaluation, \emph{S-Smoothing} stands for the image
smoothing algorithm, and \emph{S-Corners} stands for the corner detection
algorithm.

\textbf{FFT: } Fast Fourier Transform on an array of data. 
\newline

These benchmarks were modified in
order to be correctly compiled as EPOS C++ applications. The main modifications
consisted of: replacing \emph{malloc} and \emph{free} C function calls by C++
\emph{new} and \emph{delete} operators, and changing the way the benchmark
reads the input data; instead of reading from a file, it now reads from a FLASH
memory (a file system is not available in the evaluation platform). Table
\ref{tab_mibench_mem2} shows the resulting memory footprint of the benchmarks.

In our approach, only data which is dynamic allocated at run-time can be
handed out to the SPM. However, this limitation can be softened by declaring
global and stack data as heap data. The last two columns of table \ref{tab_mibench_mem2}
(\emph{Modified benchmarks} columns) show the footprint of the benchmarks
when they are modified with this optimization. We have modified the benchmarks
by redefining \emph{only arrays} which were declared as global and stack data to heap data.
Former global data are allocated before the application executes (e.g. at the 
beginning of \emph{main} function). Former stack data are allocated and deallocated
at the beginning and the end of functions. The \emph{Susan} benchmarks did not require
any modification since they originally relied on dynamic allocation.

\tabTC{tab_mibench_mem2}
{Benchmarks memory footprint (all values in bytes). Stack depth and Heap depth
represents the maximum amount of data allocated in the stack and the heap,
respectively.}

\subsection{Results}

We evaluated both the original and the modified (global/stack arrays as heap arrays) 
benchmarks on three different configurations of our evaluation
platform:

\textbf{No SPM:} our allocation framework is not used and all heap data is
allocated in the external SDRAM.
%\item[Cache:] our allocation framework is not used and everything is
%allocated in the external SDRAM. All memory access are cached.

\textbf{SPM:} our framework is enabled but no annotations are given.
SPM allocation is fully handled by the operating system.
%\item[SPM:] the data cache is removed and we use our framework to
%dynamic allocate data to the SPM. No annotations are given, thus
%the SPM allocation is fully handled by the operating system.

\textbf{SPM-A:} the \textbf{SPM} configuration with \emph{ALLOC\_HIGH} and
\emph{ALLOC\_LOW} annotations added to \emph{new} invocations. Since one 
of the main ideas of this work is to avoid profiling, the annotations where 
intuitively added based only on the authors knowledge about the benchmarks.
\newline

Figures \ref{fig_plot_time_orig} and \ref{fig_plot_time_mod} 
show the execution time of the original and the modified
benchmarks, respectively. On the original benchmarks 
an average improvement of $5\%$ is achieved when the
SPM is fully managed by the operating system. Some benchmarks
didn't show any significant improvement. This can be explained
by the data access patterns shown in figure \ref{fig_plot_mem_access}.
Benchmarks on which most of the memory accesses were performed on
global and stack data didn't benefit from our approach. By adding
annotations we have improved the execution time in $7\%$ (average), since 
\emph{ALLOC\_HIGH} annotations forced additional data to the
SPM. The modified benchmarks showed an average improvement of $4\%$
and $13\%$ for the \emph{SPM} and the \emph{SPM-A} configurations,
respectively. Some benchmarks have had a significant improvement when more data
is allocated using our framework.

%\fig{.73}{fig_plot_time_orig}
%{Benchmarks execution time comparison between the system with no 
%SPM management, with automatic SPM management, and with annotated SPM management.} 

%\fig{.73}{fig_plot_time_mod}
%{Modified benchmarks execution time comparison between a system with cache, 
%with automatic SPM management, and with annotated SPM management.}

\multfigtwoh{.65}{fig_plot_time_orig}{fig_plot_time_mod}
{fig_plot_time_all}
{Original (a) and modified (b) benchmarks execution time comparison.}

\figTC{.65}{fig_plot_mem_access}
{Distribution of memory operations on the different data types. Each bar represents
a different benchmark configuration, from left to right: \emph{SPM (original benchmark)}, 
\emph{SPM-A (original benchmark)}, \emph{SPM (modified benchmark)}, and \emph{SPM-A (modified benchmark)}.}

The evaluation of energy consumption was performed using 
memory models generated by CACTI~\cite{CACTI::2011:Online}. The models
for the cache and the SPM were generated for a $40 nm$ technology, in
order to match the one used on the Virtex 6 fabrication. For the external
SDRAM, we extracted the parameters from
the datasheet of the \emph{Micron 512 MB MT4JSF6464HY-1G1} memory
module used in the evaluation platform. Figure \ref{fig_plot_energy_mod} shows the energy consumption of the
memory hierarchy during the execution of the modified benchmarks.
Average improvements of $20\%$ (SPM) $34\%$ (SPM-A) were achieved due
to the power efficiency of the SPM when compared with the external SRAM
memory.

%Table \ref{tab_mem_energy}
%summarizes the power and energy consumption data of the memory components.

%\tab{tab_mem_energy}
%{Static and dynamic energy consumption (for a single read/write operation) of the memory components.}

\fig{.65}{fig_plot_energy_mod}
{Modified benchmarks energy consumption comparison
of the memory hierarchy.} 

We also compare our framework with 
\textit{Dominguez et al}~\cite{Dominguez::2005}, in which the authors 
propose a compile-time approach for handling heap data. In this work,
the authors provide results showing the improvements of using their method for SPM placement
versus placing data in DRAM. They report an average reduction of  $39.9\%$ in
energy consumption. The comparison in figure \ref{fig_plot_energy_comp} shows that our
leveraging of the SPM benefits is comparable to a dynamic compile-time approach,
but without the requirement of special compiler support and compile-time analysis (like
the one proposed in \textit{Dominguez et al}). 

\fig{.65}{fig_plot_energy_comp}
{Average energy reduction. Comparison between our approach and \textit{Dominguez et al}.}




\section{Conclusions}
\label{CONCLUSION}
We have proposed a runtime operating system management approach for SPMs which
do not require compiler support, profiling or hardware support. On the proposed
approach, annotations, inserted into the code by the programmer, are used 
by the operating system to allocate the data in the most appropriate level
of the memory hierarchy. The results have shown that, even by only handling user declared
heap variables, we were able to improve the execution time and significantly decrease the
total energy consumption. Yet simple, our solution yielded results comparable to a 
dynamic compile-time approach.

We also have shown how some kinds of global and stack variables allocated at compile-time can be easily converted
to heap variables. However, this conversion was done by hand, and, depending on the application,
it may not be feasible or natural. Nevertheless, our flexible software solution is clearly
orthogonal to existing approaches, thus it can be easily used to handle heap data,
while a different compile-time solution can be used handle instructions, global, and stack data.
The efficient allocation of all instruction and data types in a heterogeneous memory hierarchy
is a topic which will be covered in future works.

\section*{Acknowledgments}
The authors deeply acknowledge the anonymous reviewers for their valuable comments 
and suggestions to improve the quality of the paper.
This work was partially supported by the 
\emph{Coordination for Improvement of Higher Level Personnel}~(CAPES) grant,
projects RH-TVD 006/2008 and 240/2008.

\bibliographystyle{abbrv}
\bibliography{references_spm.bib}

\end{document}
