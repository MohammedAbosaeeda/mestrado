\subsubsection{Em Tempo de Execu��o}
\label{sc:proposta:proposta:execucao}

%- Proposta para a estimativa do tempo restante de dura��o da bateria. Utilizo uma t�cnica para estimar o tempo restante da bateria. Estimativa do tempo restante na figura~\ref{fig:energiaTempoRestante}.

Com objetivo de prover \qos{} em termos de energia � necess�rio verificar
periodicamente, em tempo de execu��o, se o
tempo de dura��o do sistema requerido pela aplica��o $T_t$ pode ser alcan�ado.
Para isso, $T_t$ � atualizado pelo tempo decorrido, e depois, � comparado 
com a estimativa do tempo de dura��o do sistema,
$T_{\alpha \kappa}$, no instante $\kappa$ . $T_{\alpha \kappa}$ � obtido atrav�sdo c�lculo da carga da bateria, $E_{\alpha \kappa}$, no instante $\kappa$ e 
da estimativa do consumo de energia necess�ria para as tarefas futuras.
As plataformas dos sistemas embarcados, normalmente, provem mecanismos para
obter $E_{\alpha \kappa}$ .

Para a estimativa do consumo de energia futuro das tarefas � utilizado a an�lise
da velocidade m�dia de descarga da bateria. Isso corresponde ao c�lculo da m�dia
das energias consumidas, $E_{\lambda \kappa}$, em determinados intervalos de 
tempo, ou seja, 

\[ E_{\lambda \kappa} = \frac{E_{\lambda \kappa-1} + \left (
E_{\alpha \kappa} - E_{\alpha \kappa-1} \right ) }{2} \]



, e o c�lculo da m�dia desses intervalos, $T_{\lambda \kappa}$ � obtida

\[ T_{\lambda \kappa} = \frac{T_{\lambda \kappa-1} + \left (
T_\kappa - T_{\kappa-1} \right ) }{2} \]


Dessa forma, obtemos $T_{\alpha \kappa}$ atrav�s da equa��o

\[ T_{\alpha \kappa} = \left ( \frac{E_{\alpha \kappa}}{E_{\lambda \kappa}}
\right ) * T_{\lambda \kappa} \]

Existe energia suficiente para atender o tempo de dura��o do sistema 
especificado pelo usu�rio caso $T_t \le T_{\alpha \kappa}$. Caso contr�rio, as
partes opcionais ser�o descartadas e nesse tempo ocioso, um gerente do consumo
de energia ser� chamdado.



