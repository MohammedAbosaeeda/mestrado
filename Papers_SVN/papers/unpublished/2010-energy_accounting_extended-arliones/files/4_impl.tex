\section{Implementation for the EPOS Project}
\label{sec:impl_epos}

The \epos~Project (Embedded Parallel Operating System) aims at automating the
development of embedded systems so that developers can concentrate on the
applications. \epos~relies on the Application-Driven Embedded System Design
(\adesd)~\cite{Frohlich:2001} method to guide the development of both software
and hardware components that can be automatically adapted to fulfill the
requirements of particular applications. \epos~features a set of tools to
support developers in selecting, configuring, and plugging components into its
application-specific framework. The combination of methodology, components,
frameworks, and tools enable the automatic generation of application-specific
embedded system instances~\cite{Project:EPOS:2010}.

\subsection{The EPOSMote}

Besides run time support system and tools, the \epos~Pro\-ject has driven the
development of hardware platforms, being the \emote~among them. \emote~is a
modular platform for wireless sensor network applications. It has three
different modules as shown in Figure~\ref{fig:emote2-mc13224v-block_diagram}.
The \emph{Processing Module} incorporates the core processing and communication
components of the system. There are two different versions of this module: one
based on Atmel's ZigBit package and another, used in this work, based on
Freescales' System-on-Chip MC13224V. Both present RF transceivers compatible
with IEEE 802.15.4 standard. Power supply and I/O interfaces were factored out
on \emote's design to allow adaptation to specific applications. The power
interface has separated signals for the power source ($V_{cc}$, $V_{dd}$ and
$Gnd$) and an $I^2C$ interface for communication with the processing module. The
I/O interface make 32 pins available for custom designs including a bypass of
the power source, all $ADC$ channels, $SPI$, $UART$ and several $GPIO$ pins. The
\emote~Project developed a \emph{Start-Up} board to be connected to the I/O
interface featuring $USB$ converter, a thermistor, a 3-axis accelerometer, LEDs,
and buttons that was also used in this work.

\fig{emote2-mc13224v-block_diagram}{EPOSMote's block diagram.}{scale=.54}

To be able to account for energy consumption on the \emote~we first need to
analyze its power characteristics and build its energy model. These information
were collected from the devices' datasheets, when available, or measured using
an oscilloscope in current mode. Table~\ref{tab:emote-energy_currents} shows
values of current drains of system devices in different operating modes to be
used by time-based accounters, while Table~\ref{tab:emote-energy_consumptions}
show drained battery charge for monitored events to be used by event-based
accounters.

\tab{emote-energy_currents}{Current drawn by EPOSMote's components in different
operating modes for time-based accounting.}

\tab{emote-energy_consumptions}{Battery charge used by EPOSMote's components in
monitored events for event-based accounting.}


\subsection{The EPOS Power Manager}

Once power management is a non-functional property of computing
systems~\cite{Lohmann:2005} the \epos' power manager was modeled as an software
aspect~\cite{Mens:1997}, thus being its implementation orthogonal to the
implementation of other components in \epos' framework. In \epos, aspects are
implemented as constructs called \emph{Scenario
Adapters}~\cite{Frohlich:SCI:2000}, which relies on C++'s static metaprogramming
capability (templates) and doesn't imply in the use of extra tools such as
aspect weavers.

\wfig{pm-uml-classes}{UML class diagram of EPOS' power manager.}{scale=.99}

Figure~\ref{fig:pm-uml-classes} shows a class diagram for the \epos' power
manager modeled as a scenario adapter.

The base \texttt{Power\_Manager} wraps the target class (to which the aspect may
be applied) by inheritance and function overriding (the \emph{Adapter Design
Pattern}). Additional methods may them be easily included, as is the case of the
\texttt{power} methods in the base \texttt{Power\_Manager} class.
\texttt{Power\_Manager} is, in turn, a facade (the \emph{Facade Design Pattern})
to other functionalities related to power management being implemented by other
components. For instance, \texttt{Power\_Manager\_Shared} and
\texttt{Power\_Manager\_Instances} are responsible for, respectively,
controlling of operating modes for shared components, and keeping of object
references for system-wide power management actions~\cite{Hoeller:DIPES:2006}.

In this work, \epos' power manager was extended to include the energy
consumption accounting functionality. This was done by aggregating the
\texttt{Power\_Manager\_Accounter} class which, in turn, implements the energy
consumption profiles described at Section~\ref{sec:profiles} with the algorithms
listed in Figure~\ref{prg:batt_updates}. Accounting of events on the event-based
profile are performed through the \texttt{account(e:Event)} method, which may be
called in a \texttt{wrapped\_method()} if the event generated by such a method
is a monitored one. As concerning overhead issues, it is important to note that
the \texttt{account(e:Event)} method, which increments an event counter, is an
inline function, thus incurring in no overheads related to function calls at
runtime. Also, the branches that implement the facade at the \texttt{power(m:OP\_Mode)}
method of \texttt{Power\_Manager} use constant boolean values which, in \epos,
are defined at configuration time, before compilation. As such, these are
subject to compiler optimizations which are able to remove these branches in the
final system binary. Table~\ref{tab:pm-overhead} presents the impact of the
proposed accounter on \epos~in terms of code and data memory usage, showing that
the accounting mechanism aggregates 2,768 bytes of code (ROM) and 70 bytes of
data (RAM) over the original fully functional \texttt{Power\_Manager}.

\tab{pm-overhead}{Memory footprint overhead of the EPOS power manager.(?COMO
TIRA A HLINE DE BAIXO DO SETUP?)}

