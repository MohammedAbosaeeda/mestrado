\section{Módulos de sensoriamento}
\label{sec:modulos}

%introdução
Este capítulo aborda a arquitetura dos módulos de sensoriamento. Será discutida
a arquitetura básica, comumente composta de um conjunto de sensores, um
processador e um transceptor de rádio~\cite{Barr:2002,Pottie:2000}. Serão
apresentadas abordagens comerciais recentes de integração destes componentes
em \emph{single-package} ou em \emph{single-die}. Também serão discutidos os
requisitos buscados em um módulo de sensoriamento em termos de dimensões,
consumo de energia, modularidade e adaptabilidade do canal de comunicação.

%arquitetura dos módulos de sensoriamento (sensores + processador + transceptor)
\subsection{Arquitetura de módulos de sensoriamento}

Em\rssf, diversos nodos sensores, compostos por um conjunto de sensores
analógicos e digitais, um microcontrolador, um transceptor sem-fios e bateria,
coordenam-se e trocam informações de maneira a prover uma visão global de um
dado objeto de estudo. Cada nodo individual possui capacidade limitada, mas a
comunicação e processamento cooperativo na rede permitem a obtenção de dados
mais precisos. Com base na pesquisa e aplicações atuais, e possível definir que
a arquitetura básica de um nodo de sensor, composta por um microcontrolador e a
um transceptor sem-fios deve~\cite{Wanner:MSC:2006}:
\begin{itemize}
    \item Ter dimensões físicas reduzidas.\\
    Para poderem ser instalados de maneira não intrusiva, os nodos sensores
    devem ter dimensões reduzidas. Dado o constante avanço das técnicas de
    miniaturização de hardware, o tamanho dos componentes eletrônicos utilizados
    nos nodos tende a diminuir constantemente. Entretanto, a miniaturização dos
    nodos sensores pode estar limitada ao tamanho da fonte de energia (seja na
    forma de baterias ou dispositivos para captura de energia ambiente).

    \item Ser capaz de operar por um longo tempo com quantidade limitada de
    energia.\\
    A necessidade de operação autônoma de um nodo sensor, e a capacidade
    limitada de energia disponível ao mesmo, fazem com que o baixo consumo de
    energia seja um fator determinante no projeto de hardware. Sendo assim, o
    projeto de um nodo sensor priorizará componentes de baixa potência e com
    suporte a gerencia do consumo de energia (e.g., microcontroladores,
    transceptores de baixa potência) em detrimento de componentes direcionados a
    alta capacidade de processamento, desempenho ou potência.

    \item Ter um projeto modular, permitindo a conexão com sensores específicos
    para diferentes aplicações.\\
    Os serviços de uma rede de sensores tendem a ser específicos, e utilizar
    somente o hardware necessário aos requisitos de cada aplicação. Desta forma,
    é importante que o projeto seja modular, e permita a remoção e inclusão de
    sensores conforme as necessidades da aplicação.

    \item Permitir a mais ampla configuração possível do canal de transmissão de
    dados.\\
    O transceptor de dados sem-fios é, em geral, o componente com maior consumo
    de energia em um nodo sensor. Desta forma, é importante que este transceptor
    passe a maior parte do tempo desligado. Por outro lado, aplicações
    específicas terão padrões de comunicação específicos, e poderão se beneficiar
    de diferentes técnicas de modulação de dados e controle de acesso ao meio,
    que permitam o controle do consumo de energia sem comprometer a comunicação
    de dados. Desta forma, o transceptor deve permitir a maior configuração do
    canal de dados possível.

\end{itemize}

%abordagens de integração
Para atender a esta série de requisitos, módulos de sensoriamento focaram no uso
de componentes de baixa potência de pequeno tamanho. É o caso, por exemplo, da
família módulos de sensoriamento Mica da Crossbow, Inc.~\cite{Crossbow:MTS:2005}.
Estes \emph{motes} (como o da Figura~\ref{fig:mica_mote}) são baseados em
processadores \textsc{Avr}, da Atmel, Inc., que são processadores RISC de baixa
potência. Além dos processadores \textsc{Avr}, estes dispositivos apresentam
variados modelos de transceptores de rádio, como o caso do CC1000 no Mica2 e do
CC2400 no MicaZ, ambos transceptores do fabricante Chipcom, Inc. O CC1000 é um
dispositivo de rádio que opera com modulação FM (\textit{Frequency Modulation})
nas faixas de frequência 315, 433, 868 e 915 MHz. Este dispositivo implementa
apenas a camada física de um transceptor FM, sendo que a camada de enlace (MAC e
LLC) são implementadas em software. A popularização do uso deste tipo de
transceptor deu origem a uma série de protocolos de MAC criados para dar melhor
suporte a certas categorias de aplicação. Já o CC2400 implementa as camadas
física e de enlace (MAC) do padrão IEEE 802.15.4. Este padrão, embora diminua a
flexibilidade no uso da camada física, permitiu o desenvolvimento de abstrações
de mais alto nível estáveis, já que os desenvolvedores passaram a trabalhar
sobre versões estáveis das camadas 1 (física) e 2 (enlace) da pilha de
comunicação. É exemplo destes desenvolvimentos o consórcio $ZigBee^{TM}$, que
implementa uma série de protocolos das camadas de 3 (rede) e 4 (transporte) para
uso em dispositivos das chamadas PAN (\textit{Personal Area Network}). Embora
descrever a tecnologia $ZigBee^{TM}$ não seja objetivo deste minicurso, os
diferentes MACs utilizados em \rssf, incluindo o IEEE 802.15.4, são melhores
descritos na seção~\ref{sec:MAC}.

\fig{mica_mote}{Mica2 mote da Crossbow Inc.}{scale=.7}

\tab{micaz_x_zigbit_x_mc13224v}{Comparação de características do MicaZ, ZigBit e
MC13224V. Todos os três apresentam um rádio compatível com IEEE 802.15.4.}

Para atender as demandas por maior poder de processamento e maior potência de
rádio em plataformas com menores dimensões e menor consumo de energia, os
projetos modulares caminharam para a integração de microcontrolador, transceptor
e outros componentes como antena e reguladores de tensão em uma abordagem
\emph{single-package}. Como exemplo pode ser citado o projeto $ZigBit^{TM}$ da
MeshNetics~\cite{Meshnetics:ZigBit:2007}. Buscando um desempenho ainda maior,
abordagens \emph{single-die} permitem a integração de microcontrolador, rádio, e
diversos outros dispositivos em um único circuito integrado, como, por exemplo,
os dispositivos da família MC1322X da Freescale, ou os dispositivos STM32W da
STMicroelectronics. A Tabela~\ref{tab:micaz_x_zigbit_x_mc13224v} mostra como
dispositivos que representam cada uma das abordagens apresentam melhor
desempenho junto com melhores características de tamanho e consumo de energia.

%requisitos buscados em um módulo de sensoriamento
\subsection{A arquitetura do \emote}
\label{sec:emote_arch}

O projeto \emote tem por objetivo desenvolver uma família de módulos de
sensoriamento que permita ampla configurabilidade tanto da plataforma quanto do
ambiente de software (sistema operacional). O \emote, apresentado na
Figura~\ref{fig:emote-modulos}, foi concebido com um projeto modular, sendo
previstos três módulos, entre os quais foram estabelecidas interfaces padrão,
permitindo o uso intercambiável de diferentes versões dos módulos. A
Figura~\ref{fig:emote2-block_diagram} apresenta os três tipos de módulos, que
são os seguintes:
\fig{emote-modulos}{O \emote ao lado de uma moeda de uma libra.}{scale=.15}
\begin{itemize}
    \item \textbf{Módulo de Base:} o módulo base incorpora as funcionalidades de
    processamento e de comunicação. O projeto \emote desenvolveu duas versões
    deste módulo, uma utilizando o \emph{single-package} $ZigBit^{TM}$ e outra
    utilizando o \textsc{SoC} (\emph{single-die}) MC13224V, da Freescale.
    Características específicas de cada versão do módulo base são apresentadas
    abaixo. Este módulo deve implementar detalhes de do suporte a estes
    dispositivos, como a regulação de tensão e o dimensionamento da antena, além
    rotear os pinos dos dispositivos de modo a manter o padrão das interfaces de
    alimentação e de entrada e saída.
    \item \textbf{Módulo de Entrada e Saída:} no módulo de entrada e saída devem
    ser implementadas as interfaces necessárias de entrada e saída, podendo um
    novo módulo destes ser desenvolvido para cada aplicação que se pretende
    desenvolver, permitindo o emprego dos sensores ou atuadores desejados para
    uma aplicação específica. O projeto \emote desenvolveu um módulo de entrada
    e saída ao qual deu o nome de \textit{start-up board}. Esta placa incorpora
    uma interface USB, sensor de temperatura, acelerômetro de 3 eixos, alguns
    LEDs e botões~\cite{Project:Emote:2010}.
    \item \textbf{Módulo de Alimentação:} de modo a permitir o emprego de
    diferentes fontes de alimentação, uma interface de alimentação foi
    implementada. Módulos que se conectam a esta interface podem ser tão simples
    como uma bateria alcalina AA, ou tão complexas quanto um sistema com bateria
    de lítio recarregável ou com painéis solares. A interface de alimentação
    ainda disponibiliza uma interface I2C, permitindo que o módulo de
    alimentação se comunique com o de processamento. O projeto \emote ainda não
    desenvolveu nenhum módulo de alimentação específico, mas trabalhos em
    andamento estão explorando tecnologias de captação de energia utilizando
    esta interface.
\end{itemize}

\fig{emote2-block_diagram}{Arquitetura básica do \emote}{scale=.45}

As interfaces padrão definidas pelo projeto \emote para interconexão dos
módulos desenvolvidos são as seguintes:
\begin{itemize}
    \item \textbf{Interface de Entrada e Saída:} 32 pinos, sendo 2 para
    alimentação e outros 30 que podem ser utilizados ou como GPIO, ou com
    funções específicas que inclui ADC, UART e SPI.
    \item \textbf{Interface de Alimentação:} disponibiliza pinos aos quais devem
    ser conectados o terra e alimentação do módulo de alimentação. O módulo de
    processamento devolve ao módulo de alimentação o sinal com tensão regulada.
    Também existem os 2 pinos empregados na comunicação I2C.
\end{itemize}
