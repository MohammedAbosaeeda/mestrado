
\section{UML notations used in the paper}
The \textit{Unified Modeling Language}~(UML)~\cite{OMG:UML} is a standardized modeling language. UML
includes a set of diagrams to specify both the structural and the behavioral view of 
object-oriented systems. The former emphasizes the static structure of the system using objects,
attributes, operations and relationships. The structural view includes class diagrams and composite
structure diagrams. The latter emphasizes the dynamic behavior of the system by showing
collaborations among objects and changes to the internal states of objects. It includes sequence
diagrams, activity diagrams and state machine diagrams.

This appendix provides a summary of UML notations used througout the paper to represent
object-oriented concepts. The reader may refer to \cite{Larman:2005} for a more complete explanation
about the use of UML in OOP.

\subsection{Class diagram}
Figure showing a complete class with public and private attributtes.
Also mention here the naming convention class<->component<->object

\subsubsection{Relationships}
Use OO model of figure 2. Explain the 'short' class definition. Extend the diagram to contain all
associations.

\noindent\textbf{Association:}

\noindent\textbf{Aggregation:}

\noindent\textbf{Composition:}

\noindent\textbf{Generalization:}

\subsection{Other notations}
In the paper we use other notations that are not defined in the standard.

\subsubsection{Conditional inheritance}
Show a simplified version of the new figure 4

\subsubsection{Scenario diagram}
Show the old figure 4





