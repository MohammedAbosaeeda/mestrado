% ------------------------------------------------------------------------------
\section{Conclusão} \label{concl}

Após realizarmos os testes concluímos a viabilidade da implementação do protocolo PTP para realizar a sincronização dos tempos de relógio em um sistema operacional embarcado, pois conseguimos manter o {\it offset} próximo a 0 segundo. Isso nos fornece uma base para trabalharmos a implementação com o intuito de obter um {\it offset} na faixa de sub-milissegundos. Obtendo essa precisão conseguiríamos garantir a aplicação, por exemplo, da implementação para sistemas de sensoriamento oceânico, como é abordado em \cite{DelRio2012}. Onde é feita uma abordagem sobre a implementação do protocolo PTP para distribuir os tempos de relógio em uma rede Ethernet de Sensoriamento Marinho(MSN). O fato da necessidade de um protocolo deste tipo para tal escopo se dá pelo fato de sinais GPS não estarem disponíveis devido à atenuação da água no fundo do mar e à requisitos de sincronização de instrumentos marítimos, tais como sismógrafos.


