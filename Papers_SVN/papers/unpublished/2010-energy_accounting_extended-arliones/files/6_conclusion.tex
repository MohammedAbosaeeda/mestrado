\section{Conclusion}
\label{sec:concl}

In this paper we presented a software implementation of an energy consumption
accounter for battery-operated embedded systems. We modeled the accounter and
implemented it in a simulation environment and in a real
platform~\cite{Project:EPOS:2010}. In order to lower the processing overhead
imposed by our approach we extracted runtime parameters of energy consumption
and execution time of a given application and submitted it to a genetic
optimizer (\nsga) to look for good solutions for the period at which
battery-related information should be updated in order to maximize system utilization while
minimizing residual energy and guaranteeing a pre-defined system lifetime
(mission duration). A case study on an IP-based network running over IEEE
802.15.4 sensing nodes showed promising results of the application of the
proposed approach to real applications.

On going studies are focusing on deeper exploration of the proposed adjustable
energy accounter by deploying fairer scheduling mechanisms to reduce the impact
on system quality. This effort aims at (1) enhancing system quality by using
flexible task scheduling schemes, such as
\textit{(m,k)-firm}~\cite{Ramanathan:1997} or the elastic
model~\cite{Buttazzo:1998}, to be put in place of the current egoist approach of
preventing tasks' execution; and (2) alleviating the impact on the network
quality by using network-wide battery charge information as a parameter for the
pheromone generation function of \adhop.
