\section{Introduction}
\label{sec:intro}


% Energy is a major concern in CPS/WSN - Battery lifetime.
The energy consumption of devices in a wireless sensor network (WSN) is a serious concern for system designers.
These systems usually need to operate for long periods of time without having their energy sources replaced or recharged.
Also, \emph{long lime} is a relative concept, once required operation time will always depend on several variables, related to both the environment and the application itself.
% Failure - safe execution of critical parts.
Besides the operation time, complexity of WSN systems is increasing as deployments begin to spread to a variety of scenarios.
Such scenarios often present critical requirements where timing and correctness also become crucial issues.
At this point, energy and correctness become conflicting requirements, once traditional means to increase correctness imply on higher energy consumption and means to reduce energy consumption often degrade system performance.

% Harvesting is extending networks lifetime.
The idea of extracting energy from the environment has become an attractive alternative to deal with the energy deficit and extend the longevity of sensor nodes.
% Unpredictable energy input demands for intelligent scheduling of energy usage.
Although the energy harvesters bring more energy into the system, the extra energy needs to be intelligently stored and used to stabilize system performance.
This happens because the energy input of these harvesters varies according to the physical phenomena used to produce electric energy.
For instance, energy production in systems using solar or wind as energy sources is subject to weather conditions that vary seasonally and for which accurate prediction methods are not available~\cite{Fay:2010}.
Thus, the systems use energy-aware schedulers to adapt the system behavior in accordance to energy availability and consumption.
In the context of real-time systems, the successful operation of a system depends on the compliance to the requirements of its critical parts.
As a consequence, an energy-aware scheduler for real-time systems must somehow guarantee that there will be enough energy to execute the critical parts of the system.

% Summarize our approach (best-effort X hard RT)
% Summarize contributions in this paper
This work presents an adaptive real-time scheduling method for systems under oscillatory energy availability.
The approach includes mechanisms to ensure the scheduling of hard real-time tasks by  guaranteeing their time constraints and their energy demands for a previously defined period.
Guarantees for hard real-time tasks rely on the worst-case computation time and worst-case energy consumption of each task.
% The system design process must take this information into account when building the system.
At runtime, the system scheduler uses slack time and surplus energy to execute a set of best-effort tasks.
The scheduling of best-effort tasks uses heuristics to get the best benefit from the incoming energy while trying to stabilize the rate of execution of these tasks.

% Evaluation scenario and expected (achieved?) results
A case-study presents the use of the proposed approach in a mobile WSN application that presents both hard real-time and best-effort tasks.
In the application, the mandatory functionalities of sensing data and forwarding it through the network are hard real-time tasks.
Once the routing functionality often consumes a lot of energy in mobile sensor networks, a set of best-effort tasks implements the routing mechanism.
This means that routing only takes place if there is enough energy.
The only entity that manages power in the system is the energy scheduler proposed here.
In the experiments, one instance of the energy scheduler operates independently in each node of the network, making its energy management decisions solely based on energy-related information of the node.

The paper structure is as follows.
Section~\ref{sec:sys_models} details the task model, including the scheduling approach, and the energy model, including monitoring of energy consumption, production, and storage.
Section~\ref{sec:adaptation} presents the approach for adapting system energy consumption in order to deliver hard real-time guarantees and maximize the execution rate of best-effort tasks.
Section~\ref{sec:case} presents a WSN case study where routing functions are best-effort tasks and nodes harvest energy from photovoltaic panels.
Section~\ref{sec:related} presents related work.
Finally, Section~\ref{sec:conclusion} closes the paper.

% The ideas developed in this paper are built on the top of that approach.
% Now, the focus is on the scheduling mechanism to ensure battery lifetime under oscillatory energy availability.
% For this, the previously proposed energy model is extended to include the concepts of a rechargeable battery and an energy harvester.
% Although a photovoltaic model have been used, any energy harvesting mechanism may be used given the existence of a known model for the device's response to environmental parameters in terms of electric current and voltage.


% Moreover, the approach described here also takes into account systems' real-time constraints.
% The main objective is to build a framework for the deployment of real-time WSN applications that allows for application adaptation in order to satisfy system's time and energy constraints.
% Over this framework, the relation among observed events and system set-points are explored and characterized in order to enable adequate system behavior.

% In order to adequately explore the targeted problem, characterization of both application-related and environmental variables that affect the system operation is performed.
% We also address the issue that accurate quantification and/or prediction of such variables during system operation is not always possible.
% A classic example of such quantification issues is the extraction of WCET and WCEC of application tasks.
% Actual execution time and energy consumption will always depend on application behavior, which may, in turn, depend on another set of application-specific variables.
% Complete clairvoyance in this scenario is seldom possible~\cite{Wilhelm:2008}.
% Other unpredictable variables may also be present.
% For instance, energy availability in a system using solar panels to recharge its batteries is subject to weather conditions such as sunlight irradiation levels and temperature, which varies seasonally, and for which predictions are known to show large errors~\cite{Fay:2010}.
% For instance, Figure~\ref{figtwo:solar-plot_1day-solar-plot_1year} show sample irradiation data.
% Also, several sources of interference exist in WSN that may impact on system communication (e.g. reflection, EMI).
%In WSN, as on virtually any radio communication system, such interferences are hard to be directly observed by the system itself, being them only noticed due to the degradation they pose in communication quality.


% The framework proposed herein is comprised by a set of tools and system software components that enable for adequate adaptation of applications.
% The set of tools enable the offline analysis of applications, defining its WCET and WCEC values and verifying its feasibility.
% The adaptation approach described herein will adapt specific application parameters in order to ensure energy and timing requirements.
% Specification of adaptable application parameters and its operational ranges needs to be performed prior to system deployment.
% During runtime, besides traditional real-time scheduling support (i.e. Earliest Deadline First and Rate Monotonic), three system components are specified in order to allow for application adaptation: the Energy Monitor, the Quality Monitor, and the Adaptation Controller.
% The Energy Monitor is able to identify where in the system energy is being spent and at which rate.
% The Quality Monitor observes application-related parameters in order to exploit room for improvements and/or give feedback for the Adaptation Controller about the efficiency of the adaptations.
% Runtime adaptation is then performed by the Adaptation Controller based on data gathered by the monitors and on system's task and energy models.



% The models supporting the Adaptation Controller have been carefully revised in order to be expressed as linear equations.
% Once adaptation set-points assume discrete values instead of real ones, the problem is in fact an integer linear problem, for which solutions are, usually, np-complete~\cite{ILP-is-NPC}.
% However, after a simplification to the problem shown to be viable, it has been proven equivalent to the Knapsack Problem, for which an approximation in pseudo-polynomial time exists~\cite{dyn_prog-knapsack}.

% The remaining of the paper is organized as follows.
% Section~\ref{sec:related} summarizes related work.
% Section~\ref{sec:epospm} describes the power management infrastructure on which this work is building upon.
% Section~\ref{sec:sys_models} describes the system models employed in the proposed adaptation mechanisms.
% Section~\ref{sec:adaptation} presents the proposed adaptation mechanisms.
% Section~\ref{sec:cases} show case studies built to evaluate the proposed mechanisms.
% Section~\ref{sec:conclusion} closes the paper and give some insights about on-going and future work.
