\documentclass{acmtrans2m}
\acmVolume{1}
\acmNumber{1}
\acmYear{1}
\acmMonth{1}

\usepackage[latin1]{inputenc}  
\usepackage[T1]{fontenc}
\usepackage[english]{babel}   
\usepackage{url}

%\usepackage{amssymb}
%\usepackage{amsbsy,natbib}
%\usepackage{amsthm}
%\usepackage{amsfonts,amsmath,bm}

\usepackage[pdftex]{graphicx}

\usepackage{listings}
\lstset{keywordstyle=\bfseries, flexiblecolumns=true}
\lstloadlanguages{C,[ANSI]C++,HTML}
\lstdefinestyle{prg}{
  basicstyle=\small\sffamily,
%  lineskip=-0.2ex,
  showspaces=false
}
% \usepackage{cite}
% \usepackage{afterpage}
% \usepackage{units}

%\makeatletter
%\newif\if@restonecol
%\makeatother
%\let\algorithm\relax
%\let\endalgorithm\relax
%\usepackage[figure,linesnumbered]{algorithm2e}


\newcommand{\fig}[4][h]{
  \begin{figure}
    {\centering{\includegraphics[#3]{fig/#2}}\par}
    \caption{#4}\label{fig:#2}
  \end{figure}
}

\newcommand{\wfig}[4][tb]{
  \begin{figure*}
    {\centering{\includegraphics[#3]{fig/#2}}\par}
    \caption{#4}\label{fig:#2}
  \end{figure*}
}

\usepackage{multirow}
%\setlength{\tabcolsep}{1mm}
\newcommand{\tab}[4][h]{
  \begin{acmtable}{#2}
    {\centering\footnotesize\textsf{\input{fig/#3.tab}}\par}
    \caption{#4}\label{tab:#3}
  \end{acmtable}
}

\newcommand{\wtab}[4][h]{
  \begin{*acmtable}{#2}
    {\centering\footnotesize\textsf{\input{fig/#3.tab}}\par}
    \caption{#4}\label{tab:#3}
  \end{acmtable*}
}

\newcommand{\prg}[4][h]{
  \begin{figure}
    \begin{center}
      \makebox[\width]
 	{\centering\lstinputlisting[language=#2,style=prg]{fig/#3.prg}\par}
      \caption{#4}\label{prg:#3}
    \end{center}
  \end{figure}
}

\pdfpagewidth=8.5in
\pdfpageheight=11in
\sloppy

%%
\markboth{Ant�nio Augusto Fr�hlich}
	{A Comprehensive Approach for Power Management in Embedded Systems}

\title{A Comprehensive Approach for Power Management in Embedded Systems}

\author{ANT�NIO AUGUSTO FR�HLICH \\
	Federal University of Santa Catarina}

\begin{abstract}
% Version submitted on-line
%In this article, power management is addressed in the context of embedded systems from energy-aware design to energy-efficient implementation. A set of mechanisms specifically conceived for this scenario is proposed, including: a power management API defined at the level of user-visible system components, the infrastructure necessary to implement that API, an energy-event propagation mechanism based on Petry Nets and implemented with AOP techniques, and an autonomous power manager. The proposed mechanisms are illustrated and evaluated by realistic embedded systems, thus enabling comparisons with other proposals.

  In this article, power management is addressed in the context of
  embedded systems from energy-aware design to energy-efficient
  implementation. A set of mechanisms specifically conceived for this
  scenario is proposed, including: a power management API defined at the
  level of user-visible system components, the infrastructure necessary
  to implement that API (namely, battery monitoring, accounting,
  auto-suspend, and auto-resume), an energy-event propagation mechanism
  based on \emph{Petry Nets} and implemented with \emph{Aspect-Oriented
    Programming} techniques, and an autonomous power manager build upon
  the proposed API and infrastructure. These mechanisms are illustrated
  and evaluated by a realistic embedded system that is also used to
  sustain comparisons with other proposals at each of the considered
  levels. As a result, this article has its main contribution on the
  introduction of a comprehensive and systematic way to deal with power
  management issues in embedded systems.
\end{abstract}

\category{D.4.7}{Or\-gan\-i\-za\-tion and Design}{Real-time systems
  and embedded systems}

\terms{Design, Management, Experimentation}

\keywords{Embedded systems, power management, energy-aware scheduling}

\begin{document}

\setcounter{page}{111}

\begin{bottomstuff}
  Author's address: Ant�nio Augusto Fr�hlich, Software/Hardware
  Integration Lab, Federal University of Santa Catarina, 88040\-900
  Florian�polis, SC, Brazil, \texttt{guto@lisha.ufsc.br}.
\end{bottomstuff}

\maketitle

% ------------------------------------------------------------------------------
\section{Introduction} \label{intro}
% + Introduction
% 
% The very first letter is a 2 line initial drop letter followed
% by the rest of the first word in caps.
%
% form to use if the first word consists of a single letter:
% \IEEEPARstart{A}{demo} file is ....
%
% form to use if you need the single drop letter followed by
% normal text (unknown if ever used by IEEE):
% \IEEEPARstart{A}{}demo file is ....
%
% Some journals put the first two words in caps:
% \IEEEPARstart{T}{his demo} file is ....
%
% Here we have the typical use of a "T" for an initial drop letter
% and "HIS" in caps to complete the first word.
% \IEEEPARstart{T}{his} demo file is intended.

\IEEEPARstart{E}{nergy} consumption is a determining factor when designing wireless sensor networks.
As a consequence, battery lifetime is a limitation on the development of such systems.
Therefore, the idea of extracting energy from the environment has become attractive.
Looking to the energy consumption problem, the intelligent usage of the stored energy contributes to extend the sensor nodes' longevity.
Consequently, energy schedulers have been developed in order to adequately assess the energy consumption and adapt the system accordingly to the available amount of energy.
The purpose of this work is to adapt a solar energy harvesting circuit to supply energy to low power wireless platforms, i.e., those that operate under $50~mW$.
Simultaneously, we aim at improving the performance of the energy-aware task scheduler in wireless sensor network systems by providing fine-grained battery and environmental monitoring.

Among a number of energy sources that have been studied so far, solar has proved to be one of the most effective~\cite{Roundy:2003}.
The solar energy conversion through photovoltaic (PV) cells is better performed at an optimum operating voltage.
Operating a solar panel on this voltage results in transferring to the system the maximum amount of power available.
In this context, \emph{maximum power point tracker circuits} (MPPT) have been proposed.
The drawback is that MPPT circuitry may introduce losses to a solar harvesting system.
Concerning low-power applications, it may be more energy efficient to have a good matching between the solar panel and the energy storage unit~\cite{Raghunathan:2005}.
This well matched system is than able to work close to the maximum power point with less power loss.

In this work, an evaluation of the proposed harvesting circuit is performed in order to show improvements on an energy-aware task scheduler~\cite{Hoeller:SMC:2011}.
It is shown that the combination of the proposed circuit with the cited scheduler not only extended the longevity of the wireless sensor network, but also improved system quality.

The paper is organized as follows:
Section~\ref{fund} presents the fundamentals of solar energy harvesting and energy-aware task scheduler.
Section~\ref{design} discusses the design of the harvesting circuit under the perspective of low power wireless platforms.
Section~\ref{case} presents the evaluation of the harvesting circuit and a case study showing the improvements on system quality.
Finally, section~\ref{concl} closes the paper.

% ------------------------------------------------------------------------------


\section{Power Management API}\label{sec:api}

% PM at design-time
% PM API in other systems
% - Automatic (ignores designer's knowledge)
% - At the hardware interface level
% - Non portable
% Proposed PM API
% - User-visible abstractions
% - No distinction from software and hardware components
% - Operational modes
%   - Arbitrary x semantic modes
% - Process throttling
%   - DVS

% Design of Energy-aware Embedded Systems
In order to introduce a discussion about power management
\emph{Application Programming Interfaces}~(API), let us first recall how
energy consumption requirements arise during the design of an
energy-aware embedded system and how they are usually captured. In such
systems, designers look for available energy-efficient components and,
eventually, specify new components to be implemented. During this
process, they inherently acquire knowledge about the most adequate
operating strategy for each component and for the system as a whole.
Whenever the identified strategies are associated to modifications in
the energy level of a given component, this can be captured in
traditional design diagrams, such as sequence, activity and timing, or
by specific tools~\cite{Chou:2002}.

% Example
% - Summary
% - Block diagram
Now let us consider the design of a simple application, conceived
specifically to illustrate the translation of energy constraints from
design to implementation. This application realizes a kind of remote
monitoring system, capable of sensing a given property (e.g.
temperature), reporting it to a control center, and reacting by
activating an actuator (e.g. external cooler) whenever it exceeds a
certain limit.  Interaction with the control center is done via a
communicator (e.g.  radio). The system operates on batteries and must
run uninterruptedly for one year. A block diagram of the system is shown
in figure~\ref{fig:example-block_diagram}.

\fig{example-block_diagram}{scale=0.75}{Example monitoring system block
  diagram.}

% Example
% - Description
% - Diagrams
The application is modeled around four tasks whose behavior is depicted
in the sequence diagrams of
figures~\ref{fig:example-sequence_diagram-main} through
\ref{fig:example-sequence_diagram-recovery}: \texttt{Main} allocates
common resources and creates threads to execute the other three tasks;
\texttt{Monitor} is responsible for periodic temperature monitoring
(every second), for reporting the current temperature to the control
center (every 10 seconds), and, in case the temperature threshold is
exceeded, for triggering the emergency handling thread; \texttt{Trigger}
is responsible for triggering the emergency handling thread on command
of the control center; and \texttt{Recovery}, the emergency handling
thread, is in duty of firing an one-shot actuator intended at restoring
the temperature to its normal level. Coordination is ensured by properly
assigning priorities to threads and by the \texttt{emergency}
semaphore. % (see figure~\ref{fig:example-coordination}).

\fig{example-sequence_diagram-main}{scale=0.75}{Main thread sequence
  diagram with power management actions.}
\fig{example-sequence_diagram-monitor}{scale=0.75}{Monitor thread sequence
  diagram with power management actions.}
\fig{example-sequence_diagram-trigger}{scale=0.75}{Trigger thread sequence
  diagram with power management actions.}
\fig{example-sequence_diagram-recovery}{scale=0.75}{Recovery thread sequence
  diagram with power management actions.}

% PM at design-time
% - PM knowledge from designer
% - Expressible in UML
In the sequence diagrams of
figures~\ref{fig:example-sequence_diagram-main}
through~\ref{fig:example-sequence_diagram-recovery}, energy-related
actions captured during design are expressed by messages and remarks.
For instance, the knowledge that the \texttt{Thermometer} component uses
a low-cost thermistor and therefore must perform 64 measurements before
being able to return a temperature value within the desired precision is
expressed as a note associated with the method
(figure~\ref{fig:example-sequence_diagram-monitor}).  In order to be
energetically efficient, the circuitry behind \texttt{Thermometer} (i.e.
ADC and thermistor) should be kept active during all the measurement
cycle, thus avoiding repetitions of the electrical stabilization
phase\footnote{In a typical ADC/thermistor configuration, the electrical
  stabilization phase accounts up to 98\% of the measurement cycle, both
  in terms of time and in terms of energy~\cite{Panasonic:ERTJ:2004}.}.

The diagrams also show power management hints for the
\texttt{Communicator} component (figures
\ref{fig:example-sequence_diagram-monitor},
\ref{fig:example-sequence_diagram-trigger}, and
\ref{fig:example-sequence_diagram-recovery}), which is mostly used to
listen for a message from the control center and thus can be configured
on a listen-only state for most of the time. And, indirectly, also for
the \texttt{CPU} component, which must operate in the maximum frequency
while running the \texttt{Recovery} thread, but can operate in lower
frequencies for the other threads
(figure~\ref{fig:example-sequence_diagram-main}). With this information
in hand, the system can be implemented to be more efficient in terms of
energy, either by the programmer himself or by means of an automatic
power manager.

% Example
% - Energy estimates
% - Power management
Considering the functional properties described so far and the execution
period of each thread, it is possible to estimate duty cycles for each
of the major components in the example system.  This information can
then be combined with energy consumption estimates of individual
components to calculate the power supply required by the system. This
procedure is summarized in table~\ref{tab:example-energy_estimates},
which shows hypothetical energy consumption estimates, duty cycles and
energy consumption for the four major components in the
system\footnote{Estimates were based on the Mica2 sensor
  node~\cite{Hill:2004}}.

\tab{.9\textwidth}{example-energy_estimates}{Example monitoring system energy
  consumption estimates.}

In order to match the requirement of operating uninterruptedly for one
year, the system would demand a battery capable of delivering
approximately 3576~mAh (at 3~V). For comparison, the same system
operating with all components constantly active, that is, without any
power saving strategy, would require about 142 Ah, almost 40 times more.

\subsection{Current APIs}

% Current APIs
% - Low level -> breaks portability
% - OS-driven -> ignores designer's knowledge
Few systems targeting embedded computing can claim to deliver a real
Power Management API. Nevertheless, most systems do deliver mechanisms
that enable programmers to directly access the interface of some
hardware components. These mechanism, though not specifically designed
for power management, can be used for that purpose at the price of
binding the application to hardware details.

% uClinux
% - 
\textsc{$\mu$Clinux}, like many other \textsc{Unix}-like systems, does
not feature a real power management API. Some device drivers provide
power management functions inspired on ACPI. Usually these mechanisms
are intended to be used by the kernel itself, though a few device
drivers export them via the \texttt{/sys} or \texttt{/proc} file
systems, thus enabling applications to directly control the operating
modes of associated devices.

\prg{C}{example-source-uclinux}{\texttt{Thermometer::sample()} method
  implementation for \textsc{$\mu$Clinux}.}

The source code in figure~\ref{prg:example-source-uclinux} is a
user-level implementation of the \texttt{Thermometer::sample()} method
of our example monitoring system. In this implementation, programmers
must explicitly identify the driver responsible for the ADC to which
the thermistor is connected. Besides the overhead of interacting with
the device driver through the \texttt{/sys} file system,
\textsc{$\mu$Clinux} PM API creates undesirable dependencies and would
fail to preserve the application in case the thermistor gets connected
to another ADC channel or in case the ADC in the system gets replaced by
another model.


% TinyOS
% - Low-level, good performance, non-portable
\textsc{TinyOS}, a popular operating system in the wireless sensor
network scene, allows programmers to control the operation of hardware
components through a low-level, architecture-dependent API. Though not
specifically designed for power management purposes, this API ensures
direct access to the hardware and thus can be used in this sense. When
compared to \textsc{$\mu$Clinux}, \textsc{TinyOS} delivers a lighter
mechanism, more adequate for most embedded system, yet suffers from the
same limitations in respect to usability and portability. The use of
\textsc{TinyOS} hardware control API for power management is illustrated
in figure~\ref{prg:example-source-tinyos}, which depicts the
implementation of the \texttt{Trigger} thread of our example.

\prg{C}{example-source-tinyos}{\texttt{Trigger} thread implementation
  for \textsc{TinyOS}.}

% MantisOS
% - 
\textsc{Mantis}, features a \textsc{Posix}-inspired API that abstracts
hardware devices as \textsc{Unix} special files. Differently of
\textsc{$\mu$Clinux} and \textsc{TinyOS}, however, \textsc{Mantis} does
not propose that API to be used for power management purposes: internal
mechanisms automatically deactivate components that have not been used
for a given time, or perform an ``on-act-off'' scheme, thus implementing
a sort of OS-driven power manager. This strategy can be very efficient,
but makes it difficult for programmers to express the knowledge about
energy consumption acquired during the design process.  This is made
evident in the implementation of the \texttt{Thermometer::sample()}
depicted in figure~\ref{prg:example-source-mantis}. Unaware of the
precision required for the temperature variable, \textsc{Mantis} cannot
predict that the ADC is being used in a loop and misses the opportunity
to avoid the repetition of the expensive electrical stabilization phase
of the thermometer operation.

\prg{C}{example-source-mantis}{Thermometer::sample() method implementation
  for \textsc{Mantis}.}

%\clearpage

Some systems assume that architectural dependencies are intrinsic to the
limitations of typical embedded systems, however, this is exactly the
share of the computing systems market that could benefit from a large
diversity of suppliers~\cite{Tennenhouse:2000} and therefore would
profit from quick changes from one architecture to another.  This, in
addition to the fact that current APIs do not efficiently support the
expression of design knowledge during system implementation, led us to
propose a new PM API.


\subsection{Proposed API}

The Power Management API proposed here arose from the observation that
currently available APIs require application programmers to go down to
the hardware whenever they want to manage power, inducing unnecessary
and undesirable architectural dependencies between application and the
hardware platform. In order to overcome these limitations, we believe a
PM API for embedded systems should present the following
characteristics:

\begin{itemize}
\item Enable direct application control over energy-related issues, yet
  not excluding delegation or cooperation with an autonomous power
  manager;

\item Act also at the level of user-visible components, instead of being
  restricted to the level of hardware component interfaces, thus
  promoting portability and usability;

\item Be suitable for both application and system programming, thus
  unifying power management mechanisms and promoting reuse;

\item Include, but not be restricted to, semantic modes, thus enabling
  programmers to easily express power management operations while
  avoiding the limitations of a small, fixed number of operating modes
  (as is the case of ACPI).
\end{itemize}

With these guidelines in mind, we developed a very simple API, which
comprises only two methods, and an extension to the methods responsible
for process creation. They are:

\bigskip

\texttt{Power\_Mode power(void)}

\smallskip

\texttt{void power(Power\_Mode)}

\bigskip

The first method returns the current power mode of the associated object
(i.e., component), while the second allows for mode changes.  Aiming at
enhancing usability, four power modes have been defined with semantics
that must be respected for all components in the system: \emph{off},
\emph{stand-by}, \emph{light} and \emph{full}. Each component is still
free to define additional power modes with other semantics, as long as
the four basic modes are preserved. Enforcing universal semantics for
these power modes enables application programmers to control energy
consumption without having to understand the implementation details of
underlying components (e.g., hardware devices). Allowing for additional
modes, on the other hand, enables programmers to precisely control the
operation of special components, whose operation transcend the
predefined modes.

The introduction of these methods in user-visible components such as
files and sockets certainly requires some sort of propagation mechanism
and could itself introduce undesirable dependencies. We describe a
strategy to implement them using a combination of \emph{Aspect-Oriented
  Programming} techniques and \emph{Hierarchical Petry Nets} later in
section~\ref{sec:infra}. For now, lets concentrate on the
characterization of the API, not the mechanisms behind it.

Table~\ref{tab:api-modes} summarizes the semantics defined for the
four universal operating modes. A component operating in mode
\emph{full} provides all its services with maximum performance,
possibly consuming more energy than in any other mode. Contrarily, a
component in mode \emph{off} does not provide any service and also
does not consume any energy. Switching a component from \emph{off} to
any other power mode is usually an expensive operation, specially for
components with high initialization and/or stabilization times. The
mode \emph{stand-by} is an alternative to \emph{off}: a component in
\emph{stand-by} is also not able to perform any task, yet, bringing it
back to \emph{full} or \emph{light} is expected to be quicker than
from mode \emph{off}. This is usually accomplished by maintaining the
state of the component ``alive'' and thus implies in some energy
consumption.  A component that does not support this mode natively
must opt between remaining active or saving its state, perhaps with
aid from the operating system, and going off.

\tab{.75\textwidth}{api-modes}{Semantic power modes of the proposed PM API.}

Defining the semantics for mode \emph{light} is not so
straightforward. A component in this mode must deliver all its services,
but consuming the minimum amount of energy. This definition brings about
two important considerations. First, if there is a power mode in which
the component is able to deliver all its services with the same quality
as if it was in mode \emph{full}, then this should be mode \emph{full}
instead of \emph{light}, since it would make no sense to operate in a
higher consumption mode without arguable benefits.  Hence, mode
\emph{light} is often attained at the cost of performance (e.g., through
DVS). This, in turn, brings about a second consideration: for a
real-time embedded system, it would be wrong to state that a component
is able to deliver ``all its services'' if the added latency is let to
interfere with the time requirements of applications. Therefore, mode
\emph{light} shall not be implicitly propagated to the CPU
component. Programmers must explicitly state that they agree to slow
down the processor to save energy, or a energy-aware, real-time
scheduler must be deployed~\cite{Wiedenhoft:ETFA:2007}.

Besides the four operating modes with predefined, global semantics, a
component can export additional modes through the API. These modes are
privately defined by the component based on its own peculiarities, thus
requiring the client components to be aware of their semantics in order
to be deployed.  The room for extensions is fundamental for hardware
components with many operating modes, allowing for more refined energy
management policies. For instance, the \emph{listen-only} radio mode in
our example (see figure~\ref{fig:example-sequence_diagram-trigger})
relies on such an extension.

The proposed API also features the concept of a \texttt{System}
pseudo-component, which can be seen as a kind of aggregator for the
actual components selected for a given system instance. The goal of
the \texttt{System} component is to aid programmers to express global
power management actions, such as putting the whole system in a given
operating mode, perhaps after having defined specific modes for
particular components.


\fig{api-hierarchy}{width=\textwidth}{Power Management API
  utilization example.}

Figure~\ref{fig:api-hierarchy} presents all these interaction modes in a UML
communication diagram of a hypothetical system instance. The
application may access a global component (\texttt{System}) that has
knowledge of every other component in the system, triggering a
system-wide power mode change (execution flow 1). The \textsc{API} can
also be accessed to change the operating mode of a group of components
responsible for the implementation of a specific system functionality
(in this example, communication functionality through execution flow
2). The application may also access the hardware directly, using the
API available in the device drivers, such as \textit{Network Interface
Card} (\texttt{NIC}), \texttt{CPU}, \texttt{ADC} (in the figure,
application is accessing the CPU through the execution flow 3). The
\textsc{API} is also used between the system's components, as can be
seen in the figure.

\prg{C++}{example-source-epos}{Example monitoring system
  implementation using the proposed PM API.}

In a system that realizes the proposed API, the monitoring system
introduced earlier could be implemented as show in
figure~\ref{prg:example-source-epos}, a rather direct transcript of the
sequence diagrams of figures~\ref{fig:example-sequence_diagram-main}
through~\ref{fig:example-sequence_diagram-recovery}.


%%% Local Variables:
%%% mode: latex
%%% TeX-master: "pm"
%%% End:


\section{Power Management Infrastructure}\label{sec:infra}

% Traditional Infrastructure
% - Hardware -> sophisticate
% - Software -> low-level, primitive, misuse hardware
% - Implementation -> hard-coded
% - Energy efficient (otherwise, senseless)
From the discussion about traditional power management APIs for embedded
systems in the previous section, we can infer that the infrastructure
behind those APIs are mostly based on features directly exported by
hardware components and do not escape from the software/hardware
interface.  As a matter of fact, the power management infrastructure
available in modern hardware components is far more evolved than the
software counterpart, which not rarely is restricted to mimic the
underlying hardware. For example, \textsc{XScale} microprocessors
support a wide range of operating frequencies that allow for fine grain
DVS. They also feature a Power Management Unit that manages idle, sleep,
and a set of quick wake-up modes, and a Performance Monitoring Unit that
furnishes a set of event counters that may be configured to monitor
occurrence and duration of events.

Power management mechanisms can benefit from such hardware features to
implement context-aware power management strategies. Device drivers for
the operating systems discussed in the previous section, however, do not
make use of most of these features. In order to take advantage of them,
application programmers must often implement platform-specific
primitives by themselves. This, besides being out-of-scope for many
embedded application developers, will certainly hinder portability and
reuse.  The same can be observed with peripherals such as wireless
network cards, which often provide a large set of configurable
characteristics that are not well explored.

In order to support both application-directed and autonomous power
management strategies, the infrastructure necessary to implement the
proposed API must feature the following services:

% Traditional Services
% - Auto-resume
% - Accounting
%   - Event counters
%   - Consumption estimates
% - Auto-suspend

\begin{description}
\item[Battery monitoring:] monitoring battery charge at run-time is
  important to support power management decisions, including generalized
  operating mode transitions when certain thresholds are reached; some
  systems are equipped with intelligent batteries that inherently
  provide this service, others must tackle on techniques such as voltage
  variation along discharge measured via an ADC to estimate the energy
  still available to the system~\cite{Mohanty:2010}.

\item[Accounting:] tracking the usage of components is fundamental to
  any dynamic power management strategy; this can be accomplished by
  \emph{event counters} implemented either in software or in hardware;
  some hardware platforms feature event counters that are usually
  accessible from software, thus allowing for more precise
  tracking; in some systems, for which energy
  consumption measurements have been carried out on a per-component
  basis, it might even be possible to perform energy accounting based on
  these counters~\cite{Bellosa:2007}.

\item[Auto-resume:] a component that has been set to an energy-saving
  mode must be brought back to activity before it can deliver any
  service; in order to relieve programmers from this task, most
  infrastructures usually implement some sort of ``auto-resume''
  mechanism, either by inserting mode verification primitives in the
  method invocation mechanism of components or by a trap mechanism
  that automatically calls for operating system intervention whenever
  a inactive component is accessed.

\item[Auto-suspend:] with accounting capabilities in hand, a power
  management infrastructure can deliver ``auto-suspend'' mechanisms that
  automatically change the status of components that are not currently
  being used to energy-saving modes such as \emph{stand-by} or
  \emph{off}; however, suspending a component to short after resume it
  will probably spend more energy than letting it to continue in the
  original mode, therefore, the heuristics used to decide which and when
  components should be suspended is one of the most important issues in
  the field and is now subject to intense
  research~\cite{Ren:2005,Melhem:2006,Bang:2009,Dhiman:2009,Pettis:2009}.
\end{description}

% Proposed Infrastructure
% - Same traditional services
% - Mode change propagation mechanisms for high-level components
% - Realization that services are ``aspects''

Our proposed power management API allows interaction between the
application and the system, between system components and hardware
devices, and directly between application and hardware. Thus, in order
to realize this API, each software and hardware component in our system
must be adapted to provide the above listed services.

\subsection{Implementation through Aspect Programs}

% AOP
\emph{Aspect-Oriented Programming}~(AOP)~\cite{Kiczales:1997} allows
non-functional properties (e.g. identification, synchronization, sharing
control) to be modeled separately from the components they affect.
Associated implementation techniques enable the subsequent
implementation of such properties as aspect programs that are kept
isolated from components, thus preventing a generalized proliferation of
manual, error-prone modifications across the system. As a non-functional
property, power management fits well into this paradigm.

% AOP without weavers
% Scenario adapters
\textsc{Epos}~\cite{Frohlich:2001}, our testbed system, supports AOP
through a C++ construct called \emph{Scenario Adapter}. Scenario
adapters enable aspects to be implemented as ordinary C++ programs that
are subsequently applied to component code during system compilation,
thus eliminating the need for external tools such as aspect weavers.
Figure~\ref{fig:scenario_adapter} shows the general structure of a
scenario adapter. The aspect programs \texttt{Aspect} implement their
duties as the \texttt{Scenario\_Adapter} intercepts every invocation of
a component operation by its \texttt{Clients} and embraces it withing a
\texttt{enter}/\texttt{leave} pair. The \texttt{Scenario} construct
collects these aspect programs, each with its own definition for
\texttt{enter} and \texttt{leave}, and adjusts their activation for each
individual target component\footnote{Each component in the system is
  characterized by a \texttt{Trait} construct that is used by
  \texttt{Scenario} to decide which aspect programs must be applied to
  the component and in which order.}. C++ operators \texttt{new} and
\texttt{delete} can also be redefined to induce the invocation of static
versions of \texttt{enter} and \texttt{leave} respectively for the
instantiation and destruction of components.

\fig{scenario_adapter}{scale=0.75}{\textsc{Epos} Scenario Adapter.}

% Aspect programs for energy accounting and auto-resume
Following AOP principles, \textbf{energy accounting} can be implemented
as an aspect program that adds event counters to components and adapts
the corresponding methods to manipulate them as illustrated in
figure~\ref{fig:aspect_account}. When \texttt{power()} is invoked on a
component, the aspect program checks for mode changes while
\emph{entering} the corresponding \emph{scenario}, issuing the
accounting directives accordingly.  \textbf{Auto-resuming} a component
that has been put in an energy-saving mode can be accomplished by
testing and conditionally restoring the component's power mode on each
method invocation as illustrated by figure~\ref{fig:aspect_resume}.

\fig{aspect_account}{scale=0.75}{Energy accounting aspect.}
\fig{aspect_resume}{scale=0.75}{Auto-resume aspect.}

% Skip from Auto-suspend in favor of an scheduler-based approach
\textbf{Auto-suspend} mechanisms can also take advantage of AOP
techniques. Turning off components that are no longer being used could
be easily accomplished by an aspect program that maintains usage
counters associated to components. An automated suspend policy could
then be implemented in the corresponding \texttt{leave} method (that
would probably rely on heuristics to decide whether suspension should
really take place). Nonetheless, automating power management decisions
without taking scheduling concerns into consideration might compromise
the correctness of real-time embedded applications. A more consistent
strategy would consist in deploying aspect programs of this kind to
collect run-time information, while delegating actual power management
to an agent integrated with the scheduler.

% System-wide aspects
% Generalized transitions
One pitfall in using AOP techniques to implement a power management
infrastructure arises from the fact that individual software components
manipulate distinct hardware components in quite specific
ways. Implementing the proposed API, so that power mode transitions can
be issued at high-level abstractions such as files and processes, would
require the envisioned aspect program to consider a complex set of
rules. In this proposal, we tackle this problem by formalizing the
interaction between components through a set of
\emph{Hierarchical Petri Nets} that are automatically transformed in the component-specific rules that are used by our generic aspect programs.

\subsection{Operation Mode Transition Networks}\label{sec:petri_nets}

% Petri Nets, and why we need them
Petri Nets are a convenient tool to model operating mode transitions of
components, not only because of its hierarchical representation
capability, but also due to the availability of verification tools that
can automatically check the whole system for deadlocks and unreachable
states~\cite{Peterson:1977}. Figure~\ref{fig:transition_net-overview}
shows a simplified view of the operating mode transition networks used
in this proposal (only the transition from OFF to FULL is shown). The
complete network encompasses all valid transitions in a similar way,
with \emph{places} being associated to operating modes (FULL and OFF in
the figure), and \emph{resources} designating the component's current
operating mode.

\fig{transition_net-overview}{scale=0.5}{Generalized Operating Mode
  Transition Network.}

The \texttt{Atomic\_Execution} place is responsible for ensuring that
multiple mode change operations do not take place simultaneously. For
that, this place is always initialized with one resource. When a power
management API method is invoked, the corresponding transition is
triggered (in the figure, \texttt{power(FULL)}) and the resource in the
\texttt{Atomic\_Execution} place is consumed.  Additionally, a new
resource is inserted into the \texttt{Triggering\_FULL} place to enable
the transactions that remove the resources that marks the component's
current operating mode (\texttt{OFF}). Since the component in the
example is in the OFF state, only the \texttt{OFF\_TO\_FULL} transition
is enabled. When this transition is triggered, the resource that marked
the \texttt{OFF} place is consumed, and three resources are inserted
into the \texttt{FULL\_Enable} place. This enables the
\texttt{Enter\_FULL} transition, that is responsible for executing the
operations that actually change the component's power mode. After this
transition is triggered, two resources are inserted into the
\texttt{FULL} place, enabling the \texttt{FULL\_Entered} transition,
which finalizes the process, consuming the final resource in the
\texttt{FULL\_Enable} place, and inserting one resource back into the
\texttt{Atomic\_Execution} place. The entire process results in a
resource being removed from the \texttt{OFF} place and inserted into the
\texttt{FULL} place. In order to avoid deadlock when a component is
requested to switch to its current operating mode (i.e., a component in
\texttt{FULL} mode is requested to go into \texttt{FULL} mode), another
transition was added to the model: \texttt{Recurrence}.  This transition
returns the resource removed from the \texttt{Atomic\_Execution} place
in case of recurrence.

% Propagation/Transition(?) networks
The generalized network represents operating mode transitions from a
high-level perspective, without modeling the specific actions that must
be taken to put a component into a given power mode. Those actions are
subsequently modeled by specializations of mode transitions (such as
\texttt{Enter\_FULL} in figure~\ref{fig:transition_net-overview}) for each
individual component. At this refinement level, Petry Net tools can be
used to simulate the network, validating it while generating traces that
can be directly mapped to rules used in the aspect programs described
earlier.

\fig{transition_net-cmac-enter_full}{scale=0.5}{Communicator transition network
  to enter mode FULL.}

For instance, the communicator in our example propagates a
\texttt{power(FULL)} directive down to associated hardware components as
specified by the transition network shown in
figure~\ref{fig:transition_net-cmac-enter_full}. The simulation of this
transition network produces a trace that is automatically converted to
the following code:

\begin{lstlisting}[language=C++,style=prg]
void Communicator::power_full()
{
    _radio.power(Radio::FULL);
    _spi.power(SPI::FULL);
    _timer.power(Timer::FULL);
}
\end{lstlisting}

Note that each distinct communicator has its own transition network,
thus ensuring that an application issuing the power directive does not
need to be patched if the radio on the hardware platform changes or even
if it is replaced by a wired transceiver. Similar transition networks
are used for all modes, including the apparently more complex
\emph{stand-by} and \emph{light} modes. The role of transition networks
is solely to propagate power management invocations from high-level
abstraction down to hardware components in a consistent manner. The
implementation of method \texttt{power()} for hardware mediators (i.e.
device drivers) does not use the traces of Petry Net simulations. They
are entirely written by hand, taking in consideration the operating
peculiarities of each hardware device.

Furthermore, invocations of \texttt{power()} at the level of hardware
mediators cannot be simply propagated, since transitions initiated by
invocations on different high-level components (possibly by distinct
threads) might conflict as they reach the hardware. For instance, a
thread could issue a \texttt{power(OFF)} on a file that is stored in a
flash memory that also stores other files currently in use by other
threads. Therefore, each hardware mediator defines its own
\texttt{power()} method considering the operating modes available in
hardware but also considering its peculiarities in respect to higher
level access. Common duties, such as serialization and share control are
available as generic aspect programs, but the deployment of such
programs is carefully decided by the development team.


%%% Local Variables:
%%% mode: latex
%%% TeX-master: "pm"
%%% End:


\section{Autonomous Power Manager}\label{sec:auto}

A considerable fraction of the research effort around power management
at software-level has been dedicated to design and implement
\emph{autonomous power managers} for general-purpose operating systems,
such as \textsc{Windows} and \textsc{Unix}. Today battery-operated
portable computers, including notebooks, PDAs, and high-end cellphones,
can rely on sophisticated management strategies to dynamically control
how the available energy budget is spent by distinct application
processes. Although not directly applicable to the embedded system
realm, those power managers bear concepts that can be promptly reused in
this domain.

As a matter of fact, autonomous power managers grab to a periodically
activated operating system component (e.g. timer, scheduler, or an
specific thread) in order to trigger operation mode changes across
components and thus save energy. For instance, a primitive power manager
could be implemented by simply modifying the operating system scheduler
to put the CPU in standby whenever there are no more tasks to be
executed.  DVS capabilities of underlying hardware can also be easily
exploited by the operating system in order to extend the battery
lifetime at the expense of performance, while battery discharge alarms
can trigger mode changes for peripheral devices~\cite{Aydin:2008}.
Nevertheless, these basic guidelines of power management for personal
computers must be brought to context before they can be deployed in
embedded systems:

\begin{itemize}
\item Embedded systems are often engineered around hardware platforms
  with very limited resources, so the power manager must be designed
  to be as slim as possible, sometimes taking software engineering to
  its limits.

\item Many embedded systems run real-time tasks, therefore a power
  manager for this scenario must be designed in such a way that its own
  execution does not compromise the deadlines of such tasks.
  Furthermore, the decisions taken by an autonomous power manager must
  be in accordance with the requirements of such tasks, since the
  latency of operating mode changes (e.g. waking up a component) may
  impact their deadlines. For a real-time embedded system, having a
  power manager that runs unpredictably might be of consequences similar
  to the infamous garbage collection issues in \textsc{Java}
  systems~\cite{Bacon:2003}.

\item Embedded systems often pay a higher energy bill for peripheral
  devices than for the CPU. Therefore, CPU-centric strategies, such as
  DVS-aware scheduling, must be reviewed to include external devices.
  Thus an active power manager must keep track of peripheral device
  usage and apply some heuristics to change their operating mode along
  the system lifetime.  The decision of which devices will have their
  operating modes changed and when this will occur is mostly based on
  event counters maintained by the power management infrastructure,
  either in hardware or in software.

\item As a matter of fact, critical real-time systems are almost
  always designed considering energy sources that are compatible with
  system demands. Power saving decisions, such as voltage scaling and
  device hibernation, are also made at design-time and thus are also
  taken in consideration while defining the energy budget necessary to
  sustain the system. At first sight, autonomous power management
  might even seem out of scope for critical systems. Nonetheless,
  complex, battery-operated, real-time embedded system, such as
  satellites, autonomous vehicles, and even sensor networks, are often
  modeled around a set of tasks that include both, critical and
  non-critical tasks. A power manager for one such embedded system
  must respect design-time decisions for critical parts while trying
  to optimize energy consumption by non-critical parts.
\end{itemize}

With these premises in mind, the next section briefly surveys the
current scenario for power management in embedded systems.


\subsection{Current Power Managers}

Just like APIs and infrastructures, most of the currently available
embedded system power managers focus on features exported by the
underlying hardware.  \textsc{$\mu$Clinux} captures \textsc{APM},
\textsc{ACPI} or equivalent events to conduct mode transitions for the
CPU and also for devices whose drivers explicitly registered to the
power manager~\cite{Vaddagiri:2004}.

In \textsc{TinyOS}, OS-driven power management is implemented by the
task scheduler, which makes use of the \texttt{StdControl} interface
to start and stop components~\cite{Hill:2000}. When the scheduler
queue is empty, the main processor is put in \emph{sleep} mode.  In
this way, new tasks will only be enqueued during the execution of an
interrupt handler.  This method yields good results for the main
microcontroller, but leaves more aggressive methods, including
starting and stopping peripheral components up to the application.
When compared to \textsc{$\mu$Clinux}, \textsc{TinyOS} delivers a
lighter mechanism, more adequate to embedded systems, yet suffers from
the same limitations with regard to usability and portability.

\textsc{Mantis} uses an \emph{idle} thread as entry point for the
system's power management policies, which put the processor in
\emph{sleep} mode whenever there are no threads waiting to be
executed~\cite{Bhatti:2005}.

\textsc{Grace-OS} is an energy-efficient operating system for mobile
multimedia applications implemented on top of
\textsc{Linux}~\cite{Yuan:2004}. The system combines real-time
scheduling and DVS techniques to dynamically control energy consumption.
The scheduler configures the CPU speed for each task based on a
probabilistic estimation of how many cycles they will need to complete
their computations. Since the systems is targeted at soft real-time
multimedia applications, loosing deadlines due to estimation errors is
tolerated.  \textsc{Grub-PA} follows the same guidelines, but addresses
hard real-time requirements more consistently by imposing DVS
configuration restrictions for this kind of task~\cite{Scordino:2004}.

Niu also proposes an strategy to minimize energy consumption in soft
real-time systems through adjusts in the system QoS
level~\cite{Niu:2005}.  In this proposal, tasks specify CPU QoS
requirements through \texttt{(m,k)} pairs. These pairs are interpreted
by the scheduler as execution constraints, so that a task must meet at
least \texttt{m} deadlines for any \texttt{k} consecutive releases. The
possibility to lose some deadlines enables the scheduler to explore DVS
more efficiently at the cost of preventing its adoption in many (hard
real-time) embedded systems.

Yet in the line of energy savings through adaptive scheduling and QoS,
\textsc{Odyssey} takes the concept of soft real-time to the limit. The
system periodically monitors energy consumption by applications in order
to adjust the level of QoS.  Whenever energy consumption is too high,
the system decreases QoS by selecting lower performance and power
consumption modes.  In this way, system designers are able to specify a
minimum lifetime for the system, which might be achieved by severely
degrading performance~\cite{Flinn:2004}.

\textsc{ECOS} defines a currency, called \emph{currentcy}, that
applications use to \emph{to pay for} system resources~\cite{Zeng:2005}.
The system distributes \emph{currentcies} to tasks periodically
accordingly to an equation that tracks the battery discharge rate as to
ensure a minimum lifetime for the system.  Applications are thus forced
to adapt their execution pace according to their \emph{currentcy}
balances.  This strategy has one major advantage over others discussed
so far in this paper: the \emph{currentcy} concept encompasses not only
the energy spent by the CPU (to adjust DVS configuration), but the
energy spent by the system as a whole, including all peripheral devices.

Harada explores the trade-off between QoS maximization and energy
consumption minimization by allocating processor cycles and defining
operating frequencies with QoS guarantees for two classes of tasks:
real-time (mandatory) and best-effort (optional)~\cite{Harada:2006}.
The division of tasks in two parts, one \emph{mandatory}, that must
always be executed, and another \emph{optional} that is only executed
after ensuring that there are enough resources to execute the mandatory
parts of all tasks is the basic premise behind \emph{Imprecise
  Computation}~\cite{Liu:1994}, which is also one of the foundations of
the power manager proposed in this work.


\subsection{Proposed Power Manager}

From the above discussion about currently available power managers for
embedded system, one can conclude that no single manager consistently
addresses all the points identified earlier in this section: leanness,
real-time conformance, peripheral device control, and design-time
decision awareness. We follow these premises and build on the API
proposed in section~\ref{sec:api} and on the infrastructure presented
in section~\ref{sec:infra} to propose an effective autonomous power
manager for real-time embedded systems.

% - Prepara a proposta
For the envisioned scenarios of battery-operated, real-time, embedded
systems, energy budgets would be defined at design-time based on
critical tasks, while non-critical tasks would be executed on a
best-effort policy, considering not only the availability of time, but
also of energy.  Along with the assumption that an autonomous power
manager cannot interfere with the execution of hard real-time tasks
(i.e., cannot compromise their deadlines), the separation of critical
and non-critical tasks at design-time lead us to the following
scheduling strategy:

% - RT -> EDF, RM, etc, even DVS
% - BE -> only if all RT AND enough energy
% - PM -> run when a BE is canceled to power down devices
\begin{itemize}
\item Hard real-time tasks are handled by the system as mandatory tasks,
  executed independently of the energy available at the moment.  These
  tasks are scheduled according to traditional algorithms such as
  Earliest Deadline First~(EDF) and Rate Monotonic~(RM)~\cite{Liu:1973},
  either in their original shape or extended to support DVS.

\item Best-effort tasks, periodic or not, are assigned lower
  priorities than hard real-time ones and thus are only executed if no
  hard real-time tasks are ready to run.  Furthermore, the decision to
  dispatch a best-effort task must also take in consideration whether
  the remaining energy will be enough to schedule all hard-real time
  tasks.

\item Whenever a best-effort task is prevented from executing due to
  energy limitations, a speculative power manager is activated in order
  to try to change components, including peripheral devices, to less
  energy-demanding operating modes, thus promoting energy savings.
\end{itemize}

% Corolario
With this strategy, the autonomous power manager will only be executed
if energy consumption is detected excessive (i.e. a best-effort task
has been denied execution) and time is available (i.e. a best-effort
task would be executed). Non-interference between power manager and
hard real-time tasks is ensured, in terms of scheduling, by having the
power manager to run in preemptive mode, so that a hard real-time task
would interrupt its execution as soon as it gets ready to run (e.g.
after waiting for the next cycle).

% Needed infrastructure
This scheduling strategy has only small implications in terms of
process management at the operating system level, but require a
comprehensive power management infrastructure, like the one presented
in section~\ref{sec:infra}, in order to be implemented. In particular,
battery monitoring services are needed to support the scheduling
decisions around best-effort tasks and component dependency maps are
needed to avoid power management decisions that could impact the
execution of hard real-time tasks.

% Energy estimation combining battery monitoring and accounting
The battery monitoring service provided by the PM infrastructure can be
combined with the energy accounting service to reduce the costs of
gauging the amount of energy still available to the system. With updated
statistics from the energy accounting infrastructure in hand, the
scheduler can predict battery discharge without having to physically
interact with it, thus sparing the corresponding energy. In this way,
battery monitoring is programmed to take place sporadically based on the
lifetime specified for the system. An additional trigger is bound to the
prediction counter kept by the scheduler, so monitoring also takes place
when power consumption reaches specified thresholds.

% - SO keeps a list of active components
% - PM takes on propagation networks to shut down devices
The operating mode transition networks introduced in
section~\ref{sec:infra} as means to control the propagation of power
management actions from high-level components down to the hardware can
be used by the autonomous power manager to keep track of dependencies
among components. Along with a list of currently active components
maintained by the operating system, these transition networks build
the basis on which peripheral control can be done by the power
manager. For instance, if a task has an open file that is no longer
being used, the power manager could track that component down to a
flash memory and change its operating mode to standby or off.

% - But PM must respect API hints of RT as orders
Nevertheless, the compromise with real-time systems requires our power
manager to take API calls made by hard real-time tasks as ``orders''
instead of ``hints''. We assume that, if a hard real-time task calls the
\texttt{power()} API method on a component to set its operating mode to
\emph{full}, then that component must be kept in that mode even if the
collected statistics indicate that it is no longer being used and thus
would be a good candidate to be shutdown.  Otherwise, the corresponding
task could miss its deadline due to the delay in reactivating that
component.


%%% Local Variables:
%%% mode: latex
%%% TeX-master: "pm"
%%% End:


\section{Implementation in \textsc{EPOS}: a Case Study}\label{sec:epos}

In order to validate the power management strategy for embedded systems
proposed in this paper, which includes an API specification, guidelines
for power management infrastructure implementation through
aspect-programs, and design constraints for the development of
autonomous power management agents, these mechanisms have been
implemented in \textsc{Epos} along with the hypothetical remote
monitoring application described in section~\ref{sec:api}. 

\subsection{\textsc{Epos} Overview}

\textsc{Epos}, the Embedded Parallel Operating System, aims at building
tailor-made execution platforms for specific
applications~\cite{Frohlich:JCS:2008}. It follows the principles of
\emph{Application-driven Embedded System Design}~\cite{Frohlich:2001} to
engineer families of software and hardware components that can be
automatically selected, configured, adapted, and arranged in a component
framework according with the requirements of particular applications.

An application written based on \textsc{Epos} published interfaces can
be submitted to a tool that performs source code analysis to identify
which components are needed to support the application and how these
components are being deployed, thus building an execution scenario for
the application. Alternatively, users can specify execution scenarios
by hand or also review an automatically generated one. A build-up
database, with component descriptions, dependencies, and composition
rules, is subsequently accessed by the tool to proceed component
selection and configuration, as well as software/hardware partitioning
based on the availability of chosen components in each domain. If
multiple components match the selection criteria, then a cost model is
used, along with user specifications for non-functional properties,
such as performance and energy consumption, to choose one of
them\footnote{Design-space exploration is currently being pursued in
  \textsc{Epos} by making the cost model used by the building tool
  dynamic.}.

After being chosen and configured, software components can still
undergo application-specific adaptations while being plugged into a
statically metaprogrammed framework that is subsequently compiled to
yield a run-time support system. This application-specific system can
assume several shapes, from simple libraries to operating system
micro-kernels. On the hardware side, component selection and
configuration yields an architecture description that can be either
realized by discrete components (e.g. microcontrollers) or submitted
to external tools for \textsc{IP} synthesis. An overview of the whole
process can be seen in figure~\ref{fig:epos_tools}.

\wfig{epos_tools}{scale=.75}{Overview of tools involved in
  \textsc{Epos} automatic generation.}


\subsection{Example Application}

The remote sensing application described in section~\ref{sec:api} was
implemented in \textsc{Epos} as excerpted in
figure~\ref{prg:example-source-epos}. When submitted to \textsc{Epos}
tools, the remote sensing program yielded a run-time library that
realizes the required interfaces and a hardware description that could
be matched by virtually any hardware platform in the system build-up
database. We forced the selection of a well-known platform, the Mica2
sensor node~\cite{Hill:2004} by manually binding \texttt{Communicator}
to the \textsc{CC1000} radio on the Mica2 platform and \texttt{Actuator}
to a led. 

In the experiment, energy for the system was delivered by two
high-performance alkaline AA batteries with a total capacity of 58320~J
(5400~mAh at 3~V), in excess of table~\ref{tab:example-energy_estimates}
estimates of what would be necessary to match the intended life-time of
one year (3576~mAh at 3~V). The system was configured with a scheduling
quantum of 15 ms and a battery monitoring period of one day.  Energy
accounting was enabled and produced statistics that were used by the
scheduler on every thread dispatching.

The experiment was profiled during approximately one week using a
digital oscilloscope. From the collected data, we determined the average
energy consumption per hour to be of 5.07 J. We then extrapolated the
total energy consumption for one year to be of 4112~mAh. This
extrapolation projects a system lifetime of 479 days, confirming that
the system will match the expectations in this respect. This experiment
also shows that the energy overhead caused by the implemented power
management mechanisms is largely compensated by the power it saves (as
calculated in section~\ref{sec:api}, running the example application
without any power management would demand 40 times more energy than the
predicted 3576~mAh). It is also important to notice that the additional
536~mAh cannot be entirely accounted to power management. A fraction of
it arises from the additional circuitry needed to couple the key
components considered at design-time, another fraction from the
non-linear discharge nature of the chosen batteries, and yet another can
be accounted to misleading estimates published by manufacturers.

This experiment also allowed us to asses the strategy overhead in terms
of memory and CPU utilization.
Table~\ref{tab:example-overhead-footprint} shows the memory increase
caused by the proposed mechanisms.  The reference system was stripped of
any PM capabilities and than enriched with the PM API, power accounting,
and finally the autonomous manager integrated into the scheduler. The
considerable increase in size for every step is justified by the fact
that they affect all components in the system. The PM API required
versions of mediators that are able to control the power mode of
associated devices plus a global \texttt{System} object and access
control to handle event propagation conflicts as described in
section~\ref{sec:petri_nets}. Accounting enriched components with
counters and the associated maintenance code. The autonomous manager
required battery monitoring plus statistics handling and decision making
support.

\tab{.75\textwidth}{example-overhead-footprint}{Power management memory
  overhead (sizes in bytes).}

In respect to performance, the proposed mechanisms only substantially
affect hardware mediators and the scheduler. Other system components,
although adorned with \texttt{power()} methods and event counters, do
not have their original behavior altered by the aspect programs
responsible for power management and therefore show no performance
loss\footnote{Some event counters are initialized at object construction
  time through modified versions of operator \texttt{new}, but this
  small overhead is usually restricted to start up phases.}.  In order
to precisely asses the overhead caused by the proposed power management
mechanisms, we inserted simple primitives to switch a led on and off
around target methods and obtained the average active period with an
oscilloscope. Table~\ref{tab:example-overhead-performance} shows the
increase in execution time for context switch and I/O operations.�The
calculations performed by the scheduler to keep statistics and decide
whether a best-effort task can be dispatched extended context switch
time by 9$\mu$s (the target platform features an 8-bit AVR running at
8MHz). The increment in the path to I/O devices caused by auto-resume
and accounting was measured to be 60$\mu$s and reveals weakness of the
target architecture to handle the associated arithmetics. Platforms
capable of keeping event counters associated to I/O ports in hardware
could eliminate a reasonable fraction of this overhead.

\tab{.75\textwidth}{example-overhead-performance}{Power management
  performance overhead (times in $\mu$s).}

\subsection{Autonomous Power Manager}

The example application discussed along this article has been conceived
to support the explanation of the proposed power management strategy.
Its implementation described in the previous section also allowed us to
confirm most of the claimed benefits. Nonetheless, it does not feature a
best-effort task that could corroborate the proposed autonomous power
manager design.  Therefore, we extended it with two additional
best-effort threads on a second experiment: thread \texttt{Calibrator}
periodically calibrates the temperature sensor, and thread
\texttt{Display} shows the current temperature on a led display.  Both
threads make use of hardware components available in the original
platform, but with a significant difference: \texttt{Calibrator} uses
the thermometer, which it shares with \texttt{Monitor} and
\texttt{Recovery}, while \texttt{Display} uses dedicated leds. This
distinction is important to drive the power manager through a situation
in which the decision about suspending resources (ADC in the
thermometer) used by a frozen best-effort task (\texttt{Calibrator})
must consider hard real-time tasks demands (\texttt{Monitor} and
\texttt{Recovery}). The threads were created with periods of 100 seconds
and 100 ms, respectively.

% \begin{lstlisting}[language=C++,style=prg]
% Periodic_Thread(&calibrator, BEST_EFFORT, 100000000);
% Periodic_Thread(&display, BEST_EFFORT, 100000);
% \end{lstlisting}

Just like the example application, this second experiment was profiled
during approximately one week using a digital oscilloscope and a new
battery set. The results of the experiment are summarized in
table~\ref{tab:example-energy_measurements}, which presents the system
average energy consumption for five different setups: (a) executing
without the additional threads; (b) executing the \texttt{Calibrator}
thread with hard real-time priority and \texttt{Display} as best-effort;
(c) the reverse situation, \texttt{Display} as hard real-time and
\texttt{Calibrator} as best-effort; (d) with both threads running with
hard-real time priority; and (e) with both threads running in
best-effort priority.

\tab{.75\textwidth}{example-energy_measurements}{Energy consumption
  under different autonomous power manager setups (RT = hard
  real-time / BE = best-effort).}

Setup (a) is equivalent to the example application evaluated in the
previous section and produced equivalent results. Setup (b) is impacted
by the periodic ADC operations performed by \texttt{Calibrator} (100
samplings every 100 seconds). The best-effort thread \texttt{Display}
has virtually no chance to execute in this setup, since the accounting
system quickly feeds the scheduler with information indicating that the
desired lifetime of 365 days cannot be matched under the observed energy
consumption rate. Setup (c) features a similar situation, but with
switched roles. The led display used has a high tool on energy, so the
system reached a lifetime of only 84 days. In this setup,
\texttt{Calibrator} is the thread that is prevented from executing,
however, the ADC it uses (via the thermometer abstraction) is not
implicitly put to sleep, since it is also used by the \texttt{Monitor}
and \texttt{Recovery} hard real-time threads.

For setup (d), both threads have been configured as hard real-time, so
they are always executed. This reduces the system's lifetime to about 76
days.  Setup (e) is the one that best characterizes the proposed
autonomous power manager. Running the additional threads in best-effort
priority enables the scheduler to suppress their execution whenever the
energy budget needed to achieve the specified lifetime is threatened.
This smoothly drives the system toward the desired lifetime, enabling
both threads to run just sporadically when the battery monitor indicates
that there is enough energy.

%%% Local Variables:
%%% mode: latex
%%% TeX-master: "pm"
%%% End:


% Summary 

% Contributions: (1) common and simple interface (minor), (2)
% Power-management on embedded systems without using any complex
% high-cost methodology.
In this paper we presented an strategy to enable application-driven
power management in deeply embedded systems. In order to achieve this
goal we allowed application programmers to express when certain
components are not being used. This is expressed through a simple
power management interface which allows power mode switching of system
components, subsystems or the system as a whole, making all
combinations of components operating modes feasible. By using the
hierarchical architecture by which system components are organized in
our system, effective power management was achieved for deeply
embedded systems without the need for costly techniques or strategies,
thus incurring in no unnecessary processing or memory overheads.

A case study using a 8-bit microcontroller to monitor temperature in
an indoor ambient showed that almost 40\% of energy could be saved
when using this strategy. % and with minimal application intervention.

% Problems: concurrence. Describe the Thread problem.

% The paper also listed some identified problems on the path for
% power-aware software and hardware components, discussing and
% explaining how some of these problems have been solved in this work
% and how some of them can be solved, and will be, in future work.

% Even so, it still have its usability.



\begin{acks}
I would like to thank and acknowledge former LISHA members Arliones
S. Hoeller Jr, Geovani R. Wiedenhoft, Giovani Gracioli and Lucas Wanner
for implementing many of the concepts and ideas presented in this
article.
\end{acks}

\bibliographystyle{acmtrans}
\bibliography{pm,lisha}

\begin{received}
Received November 2009;
Revised May 2010;
Accepted ???.
\end{received}

\end{document}

