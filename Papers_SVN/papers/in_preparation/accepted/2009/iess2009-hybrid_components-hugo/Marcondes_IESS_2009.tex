%%%%%%%%%%%%%%%%%%%%%%%%%%%%%%%%%%%%%%%%%%%%%%%%%%%%%%%%%%%%%%%%
%% Kluwer Proceedings Sample, ProcSamp.tex
%%
%% Kluwer Academic Press
%%
%% Prepared by Amy Hendrickson, TeXnology Inc., July 1999.
%%%%%%%%%%%%%%%%%%%%%%%%%%%%%%%%%%%%%%%%%%%%%%%%%%%%%%%%%%%%%%%%

\documentclass{kapproc}

\usepackage{procps} 
\usepackage[dvips]{graphicx}


\upperandlowercase
\setcounter{secnumdepth}{1}
\setcounter{tocdepth}{1}

\kluwerbib 

%\draft

\begin{document}

\articletitle[]{A Hybrid Hardware and Software Component Architecture for Embedded System Design}

%% optional, to supply a shorter version of the title for the running head:
\chaptitlerunninghead{A Hybrid Hw/Sw Component Architecture for ES Design}

\author{Hugo Marcondes\\ 
and Ant�nio Augusto Fr�hlich}

\affil{Laboratory for Hardware and Software Integration\\
         Federal University of Santa Catarina\\
	PO Box 476 - Florian�polis - Brazil\\
	88040-900}

\email{\{hugom,guto\}@lisha.ufsc.br}


%------ prologue, abstract, keywords ----------->>
% optional prologue
%\prologue{O diabo � o pai do rock, enquanto Freud explica, o diabo d� uns toques }{Raul Seixas, 1985}

%\begin{abstract}
%Embedded systems are increasing in complexity, while several metrics as time-to-market, reliability, safety and performance should the considering during the design of such systems. Frequently, the design of such systems imposes an integrated hardware/software design to cope with such metrics. In this sense, a component-based design methodology with components that can freely migrate through hardware and software domain benefits the process of design such systems. Moreover, a design based on higher-level abstraction enable a better design space exploration between several hardware and software compositions. We define hybrid hardware and software components as a development artifact that can be deployed by different combinations of hardware and software elements. In this paper, we present an architecture for developing such components in order to construct a repository of components that can migrate between the hardware and software domains to meet the design system requirements.
%\end{abstract}

% optional keywords
\begin{keywords}
 Hardware/software co-design; re-configurable architectures and applications
\end{keywords}

% - for blind review cover
-- \\
\textbf{Contact Author:}\\
Hugo Marcondes\\
Phone number: +55 (48) 8425-9911\\
Fax number: +55 (48) 3721-9516 ext: 16\\
hugom@lisha.ufsc.br
% - for blind review cover

%------------ body of article ------------------->>



% Include article on final version
%\input{Contents}



%------------ end of article ------------------->>

%% optional
%\section{Summary}

%% optional
%\begin{acknowledgments}
%...
%\end{acknowledgments}


%% appendix optional
%\appendix{This is the Appendix Title}
%This is an appendix with a title.

%\appendix{}
%This is an appendix without a title.

%
% Bibliography made with BibTeX:
%% kapalike is preferred if you have used \kluwerbib, above.
%% Otherwise you may use any .bst style your editor approves.

%This will allow many Bib\TeX\ bibliographies in one book.
%See the documentation, edbk.doc, for more information.

\bibliographystyle{kapalike}
%\chapbblname{<name of .bbl file>}
\chapbibliography{dl.bib}

\end{document}






