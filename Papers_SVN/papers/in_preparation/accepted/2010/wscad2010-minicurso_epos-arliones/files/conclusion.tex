\section{Conclusão}
\label{sec:conclusao}

Este texto foi desenvolvido para dar suporte a um curso sobre conceitos básicos
de \rssf em que se relata as experiências dos autores no desenvolvimento dos
projetos \epos e \emote~\cite{Project:EPOS:2010}. O principal foco do curso está
no suporte de sistema operacional para o hardware de \rssf, incluindo os
subsistemas de processamento e memória, comunicação e sensoriamento.

O texto descreve em detalhes os resultados alcançados com o desenvolvimento no
\epos do C-MAC, um MAC configurável para \rssf, e de um conjunto de abstrações
para o hardware de sensoriamento. Estes desenvolvimentos, junto com os serviços
básicos de sistema operacional do \epos (escalonamento, gerenciamento de memória
e gerenciamento de energia) possibilitaram a criação de um ambiente para
desenvolvimento de aplicações de \rssf independentes de plataforma, enquanto
ainda mantendo bons resultados em termos de tamanho de código e dados,
desempenho e consumo de energia.

