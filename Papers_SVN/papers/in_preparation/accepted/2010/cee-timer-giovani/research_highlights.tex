\documentclass[11pt]{article}
\usepackage{graphicx,url}
\usepackage[english]{babel}   
\usepackage[latin1]{inputenc}
\usepackage{listings}
\usepackage{subfigure}
\usepackage{multirow}
\usepackage{pictexwd}
\usepackage[absolute]{textpos}
\usepackage{subfigure}
\usepackage{fullpage}
\usepackage{rotating}

\begin{document}

\title{\textbf{Periodic Timers Revisited: the Real-time Embedded System Perspective}}
\author{Ant\^{o}nio Augusto Fr\"{o}hlich, Giovani Gracioli, and Jo\~{a}o Felipe Santos\\
\small Software/Hardware Integratio Lab\\[-0.8ex]
\small Federal University of Santa Catarina\\[-0.8ex]
\small 88040-900, Florian\'{o}polis, Brazil\\
\small \texttt{\{guto,giovani,jfsantos\}@lisha.ufsc.br}
}

\maketitle

\section*{Research Highlights}

\begin{itemize}
\item A properly configured periodic timer in combination with a smart
  designed event queue can match the single-shot approach in terms of
  performance and interference.

\item The overhead of reprogramming the hardware timer in a single-shot
  interrupt event handler is 5 times higher than a single event handler
  in the periodic approach.

\item A periodic timer can outperform an equivalent single-shot
  mechanism when the requested period exceeds the maximum hardware
  period and the single-shot timer falls back to software tick counting.

\item In a multi-threaded environment, the periodic interrupt handler
  presented better execution time (up to 5 threads) in comparison to the
  single-shot interrupt handler using an 8-bit microcontroller.

\item Periodic timer mechanisms proved to be a concrete alternative for
  real-time embedded systems, which are essentially periodic.
\end{itemize}

\end{document}
