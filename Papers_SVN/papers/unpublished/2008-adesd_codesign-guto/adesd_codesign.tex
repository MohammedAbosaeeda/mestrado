Hardware/Software Co-design methodologies, such as IP-Based Design~(IPD) and Platforma-Based Design~(PBD), despite the name, do not address software design issues comparably to modern software engineering methodologies. The main focus of current co-design methodologies seems to be on the design of a hardware platform for a given application domain, on top of which some software will be running. The software itself, is mostly regarded as application programmer's duties. Furthermore, the guidelines to decompose an application domain in IPs that will eventually build up a platform are mostly empirical. 

Application-Oriented System Design~(AOSD) is a domain engineering methodology that elaborates on the well-known domain decomposition strategies behind Family-Based Design~(FBD) and Object-Orientation~(OO), i.e. commonality and variability analysis, to add the concept of aspect identification and separation yet at the early stages of design. In this way, AOSD guides a domain engineering towards families of components, of which execution scenario dependencies are factored out as "aspects" and external relationships are captured in a component framework. This domain engineering strategy consistently addresses some of the most relevant issues in component-based software development:

1 - Reusability: components tend to be highly reusable, for they are modeled as abstractions of real elements of a given domain and not as parts of a target system. More over, by factoring out execution scenario dependencies as aspects, components can be reused unmodified in a variety of scenarios simply by defining new aspect programs.

2 - Complexity management: the identification and separation of execution scenario dependencies implicitly reduces the number of components in each family, since those components that would have been modeled to express a variation in the domain that originates from a scenario dependency are suppressed whenever the dependency can be modelled as an aspect. Simply stated, a set of 100 components could be modeled as a set of 10 components plus a set of 10 aspects and a mechanism to apply aspects to components. The overall complexity (and functionality) in the new set of 100 generated components is the same, but it is now split in fewer constructs. These directly improves on maintainability.

3 - Composability: by capturing component relationships in a component framework, AOSD enable components to more easily combined while a system is being generated, besides putting limits to the syntactical and semantical misbehaviors that can arise from applying aspects to pre-validated components. Feature-based Models are of great value at this point to capture configuration knowledge and thus make system generation a more predictable procedure. 

While trying to combine AOSD with Platform-Based Design to produce a comprehensive embedded systems design methodology, we concluded that the main concepts behind Platform-Based Design, i.e. an Application Programming Interface and an Architecture, were already promptly addressed by AOSD and, therefore, concluded that AOSD is indeed a methodology able to guide the complete development of embedded systems. In 2004, a project funded by FINEP was setup to investigate exactly that. Supporting tools were extended to cover hardware components demands, the concept of "Hardware Mediator" evolved into a powerful software/hardware interface mechanism, and several embedded systems were developed. In this paper, we discuss one of those case studies, an MPEG encoder developed in the scope of the Brazilian Digital Television Project (SBTVD), presenting its design and implementation and also identifying the limitations of AOSD, or more precisely, the limitations of ordinary domain decomposition observed as the project approached the implicitly parallel realm of hardware components.


