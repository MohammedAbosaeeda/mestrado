%% bare_conf.tex
%% V1.2
%% 2002/11/18
%% by Michael Shell
%% mshell@ece.gatech.edu
%%
%% NOTE: This text file uses MS Windows line feed conventions. When (human)
%% reading this file on other platforms, you may have to use a text
%% editor that can handle lines terminated by the MS Windows line feed
%% characters (0x0D 0x0A).
%%
%% This is a skeleton file demonstrating the use of IEEEtran.cls
%% (requires IEEEtran.cls version 1.6b or later) with an IEEE conference paper.
%%
%% Support sites:
%% http://www.ieee.org
%% and/or
%% http://www.ctan.org/tex-archive/macros/latex/contrib/supported/IEEEtran/
%%
%% This code is offered as-is - no warranty - user assumes all risk.
%% Free to use, distribute and modify.

% *** Authors should verify (and, if needed, correct) their LaTeX system  ***
% *** with the testflow diagnostic prior to trusting their LaTeX platform ***
% *** with production work. IEEE's font choices can trigger bugs that do  ***
% *** not appear when using other class files.                            ***
% Testflow can be obtained at:
% http://www.ctan.org/tex-archive/macros/latex/contrib/supported/IEEEtran/testflow


% Note that the a4paper option is mainly intended so that authors in
% countries using A4 can easily print to A4 and see how their papers will
% look in print. Authors are encouraged to use U.S. letter paper when
% submitting to IEEE. Use the testflow package mentioned above to verify
% correct handling of both paper sizes by the author's LaTeX system.
%
% Also note that the "draftcls" or "draftclsnofoot", not "draft", option
% should be used if it is desired that the figures are to be displayed in
% draft mode.
%
% This paper can be formatted using the peerreviewca
% (instead of conference) mode.
% If the IEEEtran.cls has not been installed into the LaTeX system files,
% manually specify the path to it:
% \documentclass[conference]{IEEEtran}
% To use A4 paper, add a4paper option as in
% \documentclass[conference,a4paper]{IEEEtran}


% setup page to suit conference specification using fancyhdr
\documentclass[conference,letterpaper]{IEEEtran}
\usepackage[utf8]{inputenc}
\usepackage{paralist}
\usepackage{listings}
\usepackage{subfigure}
\lstset{keywordstyle=\bfseries, flexiblecolumns=true}
\lstloadlanguages{C,[ANSI]C++,HTML}
\lstdefinestyle{prg}{
  basicstyle=\small\sffamily,
%  lineskip=-0.2ex,
  showspaces=false
}
\usepackage{fancyhdr}
\setlength{\paperwidth}{215.9mm}
\setlength{\hoffset}{-9.7mm}
\setlength{\oddsidemargin}{0mm}
\setlength{\textwidth}{184.3mm}
\setlength{\columnsep}{6.3mm}
\setlength{\marginparsep}{0mm}
\setlength{\marginparwidth}{0mm}

\setlength{\paperheight}{279.4mm}
\setlength{\voffset}{-7.4mm}
\setlength{\topmargin}{0mm}
\setlength{\headheight}{0mm}
\setlength{\headsep}{0mm}
\setlength{\topskip}{0mm}
\setlength{\textheight}{235.2mm}
\setlength{\footskip}{12.4mm}

\setlength{\parindent}{1pc}


% some very useful LaTeX packages include:

%\usepackage{cite}      % Written by Donald Arseneau
                        % V1.6 and later of IEEEtran pre-defines the format
                        % of the cite.sty package \cite{} output to follow
                        % that of IEEE. Loading the cite package will
                        % result in citation numbers being automatically
                        % sorted and properly "ranged". i.e.,
                        % [1], [9], [2], [7], [5], [6]
                        % (without using cite.sty)
                        % will become:
                        % [1], [2], [5]--[7], [9] (using cite.sty)
                        % cite.sty's \cite will automatically add leading
                        % space, if needed. Use cite.sty's noadjust option
                        % (cite.sty V3.8 and later) if you want to turn this
                        % off. cite.sty is already installed on most LaTeX
                        % systems. The latest version can be obtained at:
                        % http://www.ctan.org/tex-archive/macros/latex/contrib/supported/cite/

%\usepackage{graphicx}  % Written by David Carlisle and Sebastian Rahtz
                        % Required if you want graphics, photos, etc.
                        % graphicx.sty is already installed on most LaTeX
                        % systems. The latest version and documentation can
                        % be obtained at:
                        % http://www.ctan.org/tex-archive/macros/latex/required/graphics/
                        % Another good source of documentation is "Using
                        % Imported Graphics in LaTeX2e" by Keith Reckdahl
                        % which can be found as esplatex.ps and epslatex.pdf
                        % at: http://www.ctan.org/tex-archive/info/
                        % NOTE: for dual use with latex and pdflatex, instead load graphicx like:
                        %\ifx\pdfoutput\undefined
                        %\usepackage{graphicx}
                        %\else
                        %\usepackage[pdftex]{graphicx}
                        %\fi
                        %
                        % However, be warned that pdflatex will require graphics to be in PDF
                        % (not EPS) format and will preclude the use of PostScript based LaTeX
                        % packages such as psfrag.sty and pstricks.sty. IEEE conferences typically
                        % allow PDF graphics (and hence pdfLaTeX). However, IEEE journals do not
                        % (yet) allow image formats other than EPS or TIFF. Therefore, authors of
                        % journal papers should use traditional LaTeX with EPS graphics.
                        %
                        % The path(s) to the graphics files can also be declared: e.g.,
                        % \graphicspath{{../eps/}{../ps/}}
                        % if the graphics files are not located in the same directory as the
                        % .tex file. This can be done in each branch of the conditional above
                        % (after graphicx is loaded) to handle the EPS and PDF cases separately.
                        % In this way, full path information will not have to be specified in
                        % each \includegraphics command.
                        %
                        % Note that, when switching from latex to pdflatex and vice-versa, the new
                        % compiler will have to be run twice to clear some warnings.

%\usepackage{psfrag}    % Written by Craig Barratt, Michael C. Grant,
                        % and David Carlisle
                        % This package allows you to substitute LaTeX
                        % commands for text in imported EPS graphic files.
                        % In this way, LaTeX symbols can be placed into
                        % graphics that have been generated by other
                        % applications. You must use latex->dvips->ps2pdf
                        % workflow (not direct pdf output from pdflatex) if
                        % you wish to use this capability because it works
                        % via some PostScript tricks. Alternatively, the
                        % graphics could be processed as separate files via
                        % psfrag and dvips, then converted to PDF for
                        % inclusion in the main file which uses pdflatex.
                        % Docs are in "The PSfrag System" by Michael C. Grant
                        % and David Carlisle. There is also some information
                        % about using psfrag in "Using Imported Graphics in
                        % LaTeX2e" by Keith Reckdahl which documents the
                        % graphicx package (see above). The psfrag package
                        % and documentation can be obtained at:
                        % http://www.ctan.org/tex-archive/macros/latex/contrib/supported/psfrag/

%\usepackage{subfigure} % Written by Steven Douglas Cochran
                        % This package makes it easy to put subfigures
                        % in your figures. i.e., "figure 1a and 1b"
                        % Docs are in "Using Imported Graphics in LaTeX2e"
                        % by Keith Reckdahl which also documents the graphicx
                        % package (see above). subfigure.sty is already
                        % installed on most LaTeX systems. The latest version
                        % and documentation can be obtained at:
                        % http://www.ctan.org/tex-archive/macros/latex/contrib/supported/subfigure/

%\usepackage{url}       % Written by Donald Arseneau
                        % Provides better support for handling and breaking
                        % URLs. url.sty is already installed on most LaTeX
                        % systems. The latest version can be obtained at:
                        % http://www.ctan.org/tex-archive/macros/latex/contrib/other/misc/
                        % Read the url.sty source comments for usage information.

%\usepackage{stfloats}  % Written by Sigitas Tolusis
                        % Gives LaTeX2e the ability to do double column
                        % floats at the bottom of the page as well as the top.
                        % (e.g., "\begin{figure*}[!b]" is not normally
                        % possible in LaTeX2e). This is an invasive package
                        % which rewrites many portions of the LaTeX2e output
                        % routines. It may not work with other packages that
                        % modify the LaTeX2e output routine and/or with other
                        % versions of LaTeX. The latest version and
                        % documentation can be obtained at:
                        % http://www.ctan.org/tex-archive/macros/latex/contrib/supported/sttools/
                        % Documentation is contained in the stfloats.sty
                        % comments as well as in the presfull.pdf file.
                        % Do not use the stfloats baselinefloat ability as
                        % IEEE does not allow \baselineskip to stretch.
                        % Authors submitting work to the IEEE should note
                        % that IEEE rarely uses double column equations and
                        % that authors should try to avoid such use.
                        % Do not be tempted to use the cuted.sty or
                        % midfloat.sty package (by the same author) as IEEE
                        % does not format its papers in such ways.

%\usepackage{amsmath}   % From the American Mathematical Society
                        % A popular package that provides many helpful commands
                        % for dealing with mathematics. Note that the AMSmath
                        % package sets \interdisplaylinepenalty to 10000 thus
                        % preventing page breaks from occurring within multiline
                        % equations. Use:
                        %\interdisplaylinepenalty=2500
                        % after loading amsmath to restore such page breaks
                        % as IEEEtran.cls normally does. amsmath.sty is already
                        % installed on most LaTeX systems. The latest version
                        % and documentation can be obtained at:
                        % http://www.ctan.org/tex-archive/macros/latex/required/amslatex/math/


% Other popular packages for formatting tables and equations include:

%\usepackage{array}
% Frank Mittelbach's and David Carlisle's array.sty which improves the
% LaTeX2e array and tabular environments to provide better appearances and
% additional user controls. array.sty is already installed on most systems.
% The latest version and documentation can be obtained at:
% http://www.ctan.org/tex-archive/macros/latex/required/tools/

% Mark Wooding's extremely powerful MDW tools, especially mdwmath.sty and
% mdwtab.sty which are used to format equations and tables, respectively.
% The MDWtools set is already installed on most LaTeX systems. The lastest
% version and documentation is available at:
% http://www.ctan.org/tex-archive/macros/latex/contrib/supported/mdwtools/


% V1.6 of IEEEtran contains the IEEEeqnarray family of commands that can
% be used to generate multiline equations as well as matrices, tables, etc.


% Also of notable interest:

% Scott Pakin's eqparbox package for creating (automatically sized) equal
% width boxes. Available:
% http://www.ctan.org/tex-archive/macros/latex/contrib/supported/eqparbox/


% Notes on hyperref:
% IEEEtran.cls attempts to be compliant with the hyperref package, written
% by Heiko Oberdiek and Sebastian Rahtz, which provides hyperlinks within
% a document as well as an index for PDF files (produced via pdflatex).
% However, it is a tad difficult to properly interface LaTeX classes and
% packages with this (necessarily) complex and invasive package. It is
% recommended that hyperref not be used for work that is to be submitted
% to the IEEE. Users who wish to use hyperref *must* ensure that their
% hyperref version is 6.72u or later *and* IEEEtran.cls is version 1.6b
% or later. The latest version of hyperref can be obtained at:
%
% http://www.ctan.org/tex-archive/macros/latex/contrib/supported/hyperref/
%
% Also, be aware that cite.sty (as of version 3.9, 11/2001) and hyperref.sty
% (as of version 6.72t, 2002/07/25) do not work optimally together.
% To mediate the differences between these two packages, IEEEtran.cls, as
% of v1.6b, predefines a command that fools hyperref into thinking that
% the natbib package is being used - causing it not to modify the existing
% citation commands, and allowing cite.sty to operate as normal. However,
% as a result, citation numbers will not be hyperlinked. Another side effect
% of this approach is that the natbib.sty package will not properly load
% under IEEEtran.cls. However, current versions of natbib are not capable
% of compressing and sorting citation numbers in IEEE's style - so this
% should not be an issue. If, for some strange reason, the user wants to
% load natbib.sty under IEEEtran.cls, the following code must be placed
% before natbib.sty can be loaded:
%
% \makeatletter
% \let\NAT@parse\undefined
% \makeatother
%
% Hyperref should be loaded differently depending on whether pdflatex
% or traditional latex is being used:
%
%\ifx\pdfoutput\undefined
%\usepackage[hypertex]{hyperref}
%\else
%\usepackage[pdftex,hypertexnames=false]{hyperref}
%\fi
%
% Pdflatex produces superior hyperref results and is the recommended
% compiler for such use.

\usepackage{fancyhdr}


% *** Do not adjust lengths that control margins, column widths, etc. ***
% *** Do not use packages that alter fonts (such as pslatex).         ***
% There should be no need to do such things with IEEEtran.cls V1.6 and later.


% correct bad hyphenation here
\hyphenation{op-tical net-works semi-conduc-tor IEEEtran}

\usepackage{graphicx,url}

\newcommand{\fig}[4][tb]{
  \begin{figure}[#1]
    {\centering{\includegraphics[#4]{fig/#2}}\par}
    \caption{#3}
    \label{fig:#2}
  \end{figure}
}

\newcommand{\wfig}[4][tb]{
  \begin{figure*}[#1]
    {\centering{\includegraphics[#4]{fig/#2}}\par}
    \caption{#3}
    \label{fig:#2}
  \end{figure*}
}

\newcommand{\figtwo}[6][tb]{
\begin{figure}[tb!]
    \label{figtwo:#2#4}
    \begin{center}
        \subfigure{#3}{
            \label{fig:#2}
            \includegraphics[width=\columnwidth]{fig/#2}
        }\\
        \subfigure{#5}{
            \label{fig:#4}
            \includegraphics[width=\columnwidth]{fig/#4}
            }
    \end{center}
    \caption{#6}
\end{figure}
}

\newcommand{\figthree}[8][tb]{
\begin{figure}[tb!]
    \label{figthree:#2#4#6}
    \begin{center}
        \subfigure{#3}{
            \label{fig:#2}
            \includegraphics[width=\columnwidth]{fig/#2}
        }\\
        \subfigure{#5}{
            \label{fig:#4}
            \includegraphics[width=\columnwidth]{fig/#4}
        }\\
        \subfigure{#7}{
            \label{fig:#6}
            \includegraphics[width=\columnwidth]{fig/#6}
        }
    \end{center}
    \caption{#8}
\end{figure}
}


\usepackage{multirow}
%\setlength{\tabcolsep}{1mm}
\newcommand{\tab}[4][tb]{
  \begin{table}[tb]
    \caption{#3}\label{tab:#2}
    {\centering\footnotesize\textsf{\input{fig/#2.tab}}\par}
  \end{table}
}

\newcommand{\wtab}[4][tb]{
  \begin{table*}[tb]
    \caption{#3}\label{tab:#2}
    {\centering\footnotesize\textsf{\input{fig/#2.tab}}\par}
  \end{table*}
}

\newcommand{\prg}[4][tb]{
  \begin{figure}
    \begin{center}
      \makebox[\width]
    {\centering\lstinputlisting[language=#2,style=prg]{fig/#3.prg}\par}
      \caption{#4}\label{prg:#3}
    \end{center}
  \end{figure}
}

\newcommand{\promille}{
  \relax\ifmmode\promillezeichen
        \else\leavevmode\(\mathsurround=0pt\promillezeichen\)\fi}
\newcommand{\promillezeichen}{
  \kern-.05em
  \raise.5ex\hbox{\the\scriptfont0 0}
  \kern-.15em/\kern-.15em
  \lower.25ex\hbox{\the\scriptfont0 00}}

\newcommand{\febs}{\textsc{Febs}}
\newcommand{\epos}{\textsc{Epos}}
\newcommand{\adesd}{\textsc{Adesd}}
\newcommand{\emote}{\textsc{EposMote}}
\newcommand{\adhop}{\textsc{ADHOP}}
\newcommand{\tinyos}{\textsc{TinyOS}}
\newcommand{\mantis}{\textsc{MantisOS}}
\newcommand{\contiki}{\textsc{Contiki}}
\newcommand{\glmsim}{\textsc{GloMoSim}}
\newcommand{\nsga}{\textsc{Nsga-II}}
\newcommand{\nsgaparam}{$\alpha=100, \mu=20, \lambda=20, g=50$}

\def\blfootnote{\xdef\@thefnmark{}\@footnotetext}

\begin{document}
% paper title
\title{On the Monitoring of System-Level Energy Consumption of Battery-Powered
Embedded Systems}

% author names and affiliations
% use a multiple column layout for up to three different
% affiliations
\author{\IEEEauthorblockN{Arliones Hoeller Jr}
\IEEEauthorblockA{Department of Automation and Systems Engineering\\
Federal University of Santa Catarina\\
Florian{\'o}polis, Brazil\\
arliones@das.ufsc.br}
\and
\IEEEauthorblockN{Ant{\^o}nio Augusto Fr{\"o}hlich}
\IEEEauthorblockA{Software/Hardware Integration Lab\\
Federal University of Santa Catarina\\
Florian{\'o}polis, Brazil\\
guto@lisha.ufsc.br}}

% avoiding spaces at the end of the author lines is not a problem with
% conference papers because we don't use \thanks or \IEEEmembership


% for over three affiliations, or if they all won't fit within the width
% of the page, use this alternative format:
%
%\author{\authorblockN{Michael Shell\authorrefmark{1},
%Homer Simpson\authorrefmark{2},
%James Kirk\authorrefmark{3},
%Montgomery Scott\authorrefmark{3} and
%Eldon Tyrell\authorrefmark{4}}
%\authorblockA{\authorrefmark{1}School of Electrical and Computer Engineering\\
%Georgia Institute of Technology,
%Atlanta, Georgia 30332--0250\\ Email: mshell@ece.gatech.edu}
%\authorblockA{\authorrefmark{2}Twentieth Century Fox, Springfield, USA\\
%Email: homer@thesimpsons.com}
%\authorblockA{\authorrefmark{3}Starfleet Academy, San Francisco, California 96678-2391\\
%Telephone: (800) 555--1212, Fax: (888) 555--1212}
%\authorblockA{\authorrefmark{4}Tyrell Inc., 123 Replicant Street, Los Angeles, California 90210--4321}}


% use only for invited papers
%\specialpapernotice{(Invited Paper)}

% make the title area
\maketitle
\thispagestyle{plain}


% insert page header and footer here for IEEE PDF Compliant
\fancypagestyle{plain}{
\fancyhf{}  % clear all header and footer fields
\fancyfoot[L]{}
\fancyfoot[C]{}
\fancyfoot[R]{}
\renewcommand{\headrulewidth}{0pt}
\renewcommand{\footrulewidth}{0pt}
}


\pagestyle{fancy}{
\fancyhf{}
\fancyfoot[R]{}}
\renewcommand{\headrulewidth}{0pt}
\renewcommand{\footrulewidth}{0pt}



\begin{abstract}
This paper addresses an approach for accurately measuring energy consumption on
battery-powered embedded systems which can be adequately tuned in order to
enhance a set of timing and energy consumption requirements for mission critical
systems. We introduce a software-based accounting scheme which is calibrated by
low-precision battery state-of-charge reads through a battery voltage model. We
then perform an offline multi-objective optimization procedure using \nsga~to
find good candidates to the period at which battery consumption information
should be updated. Such candidates might guarantee timing constraints (i.e., no
deadline misses), minimize residual energy after a pre-defined system lifetime,
and maximize system utilization during system lifetime. We considered a simple
scheduler which will reserve battery charge to run hard real-time tasks during a
pre-defined lifetime and will prevent best-effort tasks from running whenever
accounted battery state-of-charge is bellow the current reserve. We evaluated
our approach by performing a set of case-studies and show that there is no
direct relation between battery information update frequency and the composite
of objectives defined here.
\let\thefootnote\relax\footnotetext{This work have been partially funded by CNPq
grant 479009/2009-0. Arliones Hoeller Jr is also supported in part by CAPES
grant RH-TVD 006/2008.}
\end{abstract}

\begin{IEEEkeywords}
accounting, energy, embedded systems, real-time scheduling, adaptive task
periods
\end{IEEEkeywords}


% For peer review papers, you can put extra information on the cover
% page as needed:
% \begin{center} \bfseries EDICS Category: 3-BBND \end{center}
%
% for peerreview papers, inserts a page break and creates the second title.
% Will be ignored for other modes.
\IEEEpeerreviewmaketitle

\section{Introduction}
\label{sec:intro}

Low energy consumption is an important non-functional requirement for the design
of battery-powered embedded systems. Reliable information on the system energy
source is of paramount importance to design an energy-efficient system.
% Nowadays, with cheaper lithium batteries becoming available, such batteries are
% being increasingly used in embedded systems. Lithium batteries present different
% electrical characteristics when compared to ordinary batteries (alkaline),
% including enhanced capacity at smaller sizes and, most important, stabler
% voltage levels.
%Measuring current isn't feasible, so it is usual to keep track of battery voltage
In order to keep track of exact energy consumption at runtime, traditional
approaches rely on continuous measurements of the amount of current drained from
the battery. Besides the additional hardware required to perform this task,
%(a shunt resistor between system and battery connected to an ADC)
software support for sampling such circuitry may compromise system performance
due to the requirement of fine grained information needed to sample this
continuous signal.

\fig{sampled_real_discharge}{Sampled discharge plotted against real discharge
for a CR2 series Panasonic lithium
battery~\cite{Panasonic:Lithium:2006}.}{scale=.333}

To cope with this requirement mobile systems provide means to measure battery
voltage by which it is possible to infer battery charge through an approximate
discharge model for a given battery. This approach, however, brings limitations
to the task of estimating battery charge.
%which adds to the \emph{good} voltage stability of Lithium-based batteries.
Among the reported problems~\cite{Mundra:2008,Penella:2010} there is one of
special interest for the task of precisely estimating battery charge on
embedded systems: the low accuracy and long response time of voltage-based
battery state-of-charge models. This problem is related to the diminished
precision of such voltage measurements implied by low resolution
analog-to-digital converters and to the oscillation on the battery voltage-levels
due to load variations. It can be easily illustrated by
Fig.~\ref{fig:sampled_real_discharge}. The figure shows that during most of
system lifetime the response time in terms of battery voltage varies greatly.
For instance, if an energy-related decision lowers system utilization on day 50,
this decision may last until the next expected voltage drop, around day 100,
making an unique decision stay in effect for $14\%$ of system lifetime. We
can also see that the scheduler misses the opportunity of raising system utilization
because its monitor is not able to rapidly inform that the system is not using
the expected amount of energy. Instead of gradually raising system utilization
and keeping it higher during a longer period of time, it only acts by the end of
a given period, causing bursts in system utilization for short periods of time.

In this paper we propose a software-based energy consumption accounting scheme.
We rely on the fact that energy consumption is a cross-cutting concern,
orthogonal to all system components~\cite{Lohmann:2005}, to enable accurate
monitoring of energy consumption in a component-based operating system for
embedded systems. This scheme monitors the execution of operating system
functionalities. In order to make this approach feasible, we derived three
different profiles for energy accounting which can be applied to devices with
different operational behavior. We evaluate the effect of our approach on system
performance. This paper focus exclusively in the proposed model. An
implementation of the system was also performed for the
\epos~platform~\cite{Project:EPOS:2010} by extending \epos' power management
scheme~\cite{Froehlich:2011}, altought the implementation details where kept out
of this paper due to length limitations.

The remaining of this paper is organized as follows.
Section~\ref{sec:related} presents an overview of related work.
Section~\ref{sec:account} describes the software-based energy accounting
approach proposed in this paper, analyzing it through a didactic case-study.
% Section~\ref{sec:impl_epos} describes the implementation of this approach in the
% \epos~Project.
Section~\ref{sec:case} presents a case-study of the proposed approach on a
wireless sensor network system.
Section~\ref{sec:concl} summarizes this paper and gives some insights on future
work.

\section{Related work}
\label{sec:related}

% There are two sorts of third-party developments that are related to the present
% work: those related to software-based accounting of energy usage and those
% which address the deployment of multi-objective optimization methods on
% real-time systems for the control of energy consumption. This section gives an
% overview of studied related work.
Related work were divided into those related to software-based accounting of
energy usage and those which address the deployment of meta-heuristic and
multi-objective optimization methods on energy-aware real-time systems.

\subsection{Accounting of Battery Usage}

We investigated similar battery monitoring schemes in operating systems for
wireless sensor networks, including \tinyos~\cite{Polastre:2005},
\mantis~\cite{Bhatti:2005} and \contiki~\cite{Dunkels:2004}. All three systems
provide access to battery voltage measurements through an ADC interface.
% \tinyos~and \contiki, however, have other work either from the original authors
% or from third-parties that are related to the present work.
None of them provide complete voltage models by which it would be possible to
infer battery state-of-charge from voltage levels.
% Also, previous work have already discussed the problems related to the frequency
% in which operating mode migration happens~\cite{Hoeller:DIPES:2006}. These work
% already address questions concerning time overhead and additional energy
% consumption during these migrations~\cite{Seo:2011}.

Yang et al.~\cite{Yang:2007} built an extension to \tinyos~that enables
software-based accounting of energy consumption and battery lifetime estimation
for the \textsc{Mica2} sensor node. They monitor the time each hardware
component stays in an operating mode by intercepting mode changes and
accumulating drawn current during this period, decrementing it from the
initially informed battery charge. Their approach, however, requires
modifications on every monitored system component and may pose unnecessary
overheads in situations where devices change operating mode too often, as it
would be the case of a radio transceiver in a low-power listen mode, where the
transceiver is periodically switched on and off to check for incoming messages.
Weissel and Kellner~\cite{Weissel:2006} used an event-based energy accounting
mechanism to compute energy on the \textsc{BTnode} platform~\cite{Beutel:2004}
running \tinyos. They, however, didn't implement the presented concepts,
limiting their work to the analysis of the implementation possibilities.

Dunkels et al.~\cite{Dunkels:2007} implemented a time-based energy accounting
system for \contiki~running on \textsc{Tmote Sky}~\cite{Polastre:2005} platform.
Their model, as does the approaches used in \tinyos, demand for modification in
several different operating system modules (i.e., drivers), what may make the
system difficult to maintain. They also show that the lack of calibration with
real information in their system may be the cause of significant errors in
energy estimations.

\subsection{Optimization Methods in Energy-Aware Real-Time Systems}

Energy optimization for real-time systems has long been a subject of great
interest in the real-time community~\cite{Weng:2003}. A plurality of
works have been published that apply optimization techniques on real-time system
models to find good tradeoffs between energy consumption and operating
frequency/voltage of CPUs (DVS - Dynamic Voltage Scaling)~\cite{Chen:2007},
on/off status of peripheral devices (DPM - Dynamic Power Management), or
both~\cite{Jha:2001}. All these works, although important to the design of
energy-aware real-time systems, are outside the scope of this paper for they are
orthogonal to the work presented here.

Chantem et al.~\cite{Chantem:2009} proposed a generalized elastic scheduling
framework for real-time tasks based on Buttazzo's elastic
model~\cite{Buttazzo:1998} which may adapt task's elastic periods online based
on one specific (generic) performance metric. Optimal period adjustments are
then performed by a heuristic proposed by them. Although energy-related metrics
may be used as the optimization objective, authors didn't explore this.
Eker et al.~\cite{Eker:2000} and Cervin et al.~\cite{Cervin:2002} show the
application of optimization theory to solve the period selection problem at
runtime by performing adaptive adjustments of periods based on a control
performance metric.
Bini and Natale~\cite{Bini:2005} devised an optimal search algorithm to minimize
tasks' frequencies by performing incremental improvements on one specific
performance metric by using a branch and bound search over a predefined
feasibility region of the domain of task frequencies until the global optimum is
reached. The algorithm applies to fixed-priority scheduling schemes and may be
only applicable offline due to its high complexity.
% Multi-objective optimization have also been applied for solving multiprocessor
% scheduling problems such as the multiprocessor task
% assignment~\cite{Miryani:2009} and the task scheduling problem in heterogeneous
% systems~\cite{Chitra:2010}.

To the best of our knowledge, no work explored the effects of the period at
which battery state-of-charge information is made available for a real-time
energy-aware scheduler.

\section{Battery Level Monitoring by Event Accounting}
\label{sec:account}

%general introduction. pessimistic bias. show desired behavior.
Traditional voltage-based battery monitoring may lead to a pessimistic bias of
an energy-aware task scheduler. In order to enhance the precision of the battery
monitor, we propose a software-based scheme to account for energy consumption.
This scheme is based on the premise that energy is consumed by hardware, not
software, but it is the software that controls and monitors hardware activity.
Considering that the operating system layer abstracts hardware access to
application, it is straightforward to assume that the operating system is the
entity with most knowledge about hardware activity, thus being able to monitor
system-wide energy consumption.

\subsection{Energy consumption profiles}
\label{sec:profiles}

We analyzed usual hardware behavior and modeled three different profiles to
account for energy consumption: time-based measurement, event-based measurement,
and combined measurement.
%time-based
The time-based profile is used to account for energy consumption of devices
draining constant current over time when in a specific operating mode, as
(\ref{eq:en_dev_time}) shows.
%event-based
The event-based profile is used in devices for which operation can be mapped to
specific events (e.g., sensor sampling). As shown in (\ref{eq:en_dev_ev}),
events are accounted for and energy is updated periodically based on this
accounting.
%both
In some devices, however, both approaches may be used. For instance, a radio
that stays in a low-power listen mode has a base energy consumption (computed by
the time-based profile) and extra energy consumption when data actually arrives
(computed by the event-based profile). As shown in (\ref{eq:en_tm_ev}).

\begin{eqnarray}
E_{tm}(dev) = (t_{end} - t_{begin}) \times I_{dev,mode} \label{eq:en_dev_time}\\
E_{ev}(dev) = \sum_{event\_counters} E_i * counter \label{eq:en_dev_ev}\\
E_{tot}(dev) = E_{tm}(dev) + E_{ev}(dev) \label{eq:en_tm_ev}
\end{eqnarray}
, where $E_{tm}(dev)$ is the energy consumption in the time-based profile for a
specific device, $t_{end}$ and $t_{begin}$ denote timestamps, and $I$ is the
current of $dev$ at a given $mode$. $E_{ev}$ denotes the sum of energy
($E_i$) consumed by the observed events ($counters$). $E_{tot}$ is the energy
consumption in the combined profile.



\subsection{Battery state-of-charge monitoring}

% further describe system runtime operation.
% present battery monitoring algorithm - battery charge updated periodically by
% means of event-counters which are updated at run-time.
At runtime, battery charge is updated with the accounted data of each device.
These accounting information, however, need to be collected in order to update
battery charge. The frequency in which these data are collected directly affects
the accuracy of the proposed battery state-of-charge monitor. Recalling the
curves in Fig.~\ref{fig:sampled_real_discharge}, we may say that high update
frequencies would draw a curve close to the ``real charge'', while low update
frequencies would approximate the ``sensed charge'' curve. In this section we
describe how to achieve a satisfactory curve close to the ``real charge'' curve
of Fig.~\ref{fig:sampled_real_discharge} by adequately adjusting the battery
update frequency.

We start by analyzing the update frequency for the time-based profile. It is not
the intent of the present work to investigate issues related to the frequency of
operating mode migrations or their time and energy overheads, for such problems
have already been extensively addressed~\cite{Hoeller:DIPES:2006,Seo:2011}.
Thus, this profile can adhere to such migration models by the inclusion of an
extra routine like $energy\_migration\_update$ on Fig.~\ref{prg:batt_updates}
to compute elapsed time and consumed energy during these migrations as described
by (\ref{eq:en_dev_time}). The execution time for this routine, shown in
Tab.~\ref{tab:time_overhead} is constant and can be easily obtained and
integrated to any transition model, being either real-time or not. Additionally,
to prevent the system from loosing control of battery discharge when devices
stay in a certain operating mode for long periods, an active component
periodically collects energy consumption information from all devices and
updates battery charge.

\prg{pascal}{batt_updates}{Algorithms for energy accounting.}

\tab{time_overhead}{Processing overhead of the energy accounter on an
ARM7-TDMI processor.}

For the event-based profile, however, the battery charge update approach needs
to be different to avoid unnecessary processing overheads. For instance, suppose
that a hypothetic system monitors an event that is the reception of a byte from
a network interface. Network protocols will seldom use only one byte to perform
communications, thus, as can be seen in Tab.~\ref{tab:time_overhead}, system
performance may benefit from periodic updates of an accumulated counter. In
order to do that, an active object was modeled as an extra task on the system
which is responsible for collecting accounted information of the event-based
profile.

It is important to note that the accounting mechanism employed in this scheme,
although more accurate, is still pessimistic once it is based on the worst-case
energy consumption (WCEC) of events and components' operating modes. Thus, it is
expected that the accounted energy consumption reaches the value read from the
battery voltage model before it shows a drop in voltage. It is safe, however, to
assume that the information from the voltage model is a secure bound to battery
charge, although conservative. Then, we may correct the battery charge to the
maximum value between the accounted charge and the one estimated by the voltage
model (as shown in (\ref{eq:batt_update})).

\begin{eqnarray}
E_{batt} = max\left(E_{volt} , E_{batt} - \sum_{i = 0}^{\#devs} E_{tot}(i)\right)
\label{eq:batt_update}
\end{eqnarray}

Finally, the active component with the task of periodically updating the battery
charge is responsible for collecting accounted information from both event-based
and time-based profiles. The algorithm is the one at the procedure
$energy\_update\_total$ of Fig.~\ref{prg:batt_updates}. It is important to
note that timestamps and event counters are reset every time energy accounting
is updated (by $energy\_migration\_update$ and $energy\_event\_update$), thus
making sure that no energy consumed is accounted for twice. This algorithmic
approach assumes initialization of $Battery$ with the nominal capacity of the
battery in use.


\subsection{On the Freshness of Battery Information}
\label{sec:frequency}

To understand the accounter behavior further we performed an exploration of
system design space to be able to determine the frequency at which accounted
data should be gathered provided that the system has a pre-defined requirement
of operation lifetime. We used a multi-objective optimization method based on
the \textit{Non-dominated Sorting Genetic Algorithm II}~(\nsga)~\cite{Deb:2002}
which is able to find good solution candidates for the frequency of the
collector task, i.e., those closer to the Pareto front. The optimization is
executed with two objectives: to minimize residual energy and to maximize the
execution rate of best-effort tasks. We use a simple scheduling mechanism in
which hard real-time tasks execute regardless of system energy availability and
best-effort tasks run only when the system is still able to guarantee energy
availability for hard real-time tasks. Priorities in the scheduling queue are
assigned through a Rate Monotonic policy provided that all hard real-time tasks
have higher priorities than any best-effort task, regardless of their period. We
also assume that initial battery charge is enough to guarantee hard real-time
tasks' executions during the expected lifetime.

\tab{hyp-taskset}{Hypothetic application tasks'
parameters\protect\footnotemark[1].}

\footnotetext[1]{T: task; P: period in $ms$; WCET: worst-case execution time in
$ms$; WCEC: worst-case energy consumption in $\eta Ah$; 1-hour: energy consumption for
the targeted lifetime (1 hour) in $mAh$.}

\footnotetext[2]{This is a worst-case scenario as values of ``P'' and, as
consequence, ``1-hour'', are to be defined by the optimization.}



Now we analyze a hypothetic application. Tab.~\ref{tab:hyp-taskset} shows the
parameters for this application, comprised by two hard real-time tasks ($H_1$
and $H_2$), one best-effort task ($B_1$) and the energy accounter collector task
($H_C$), ordered according to their priority. We kept the collector task as a
hard real-time task for two reasons. First, it is the only way to guarantee that
the defined period for the collector task will be respected, once best-effort
tasks may be prevented from executing. Second, if eventually the system stops
the execution of best-effort tasks and the collector task is a best-effort task,
the battery information will no longer be updated, thus being this a dead-end
for the energy-aware scheduler.

In order to evaluate the system instances (individuals) generated during the
optimization process we integrated a real-time simulator to the optimizer. This
system simulates a rate monotonic queue with two levels of priorities, being the
first one the task's class and the second one the rate monotonic priority
itself. This made it possible the separation between hard real-time and
best-effort tasks. The simulator also monitors energy consumption of tasks and
controls battery discharge based on informed worst-case energy consumption
(WCEC) of tasks. In order to achieve a more realistic behavior we consider that
all tasks actually use their WCEC for 75\% of the jobs. The remaining 25\% of
the jobs have a random energy consumption uniformly distributed between 50\% and
100\% of the WCEC. This generates a slack on the energy budget that can be used
by the best-effort tasks, as would actually happen on real systems. Although
naive, this simple probabilistic assumption helps to understand and analyze the
approach. A more consistent approach will be considered later in the case study
of Section~\ref{sec:case}.

\fig{hyp-solutions}{All solutions for the hypothetical
application.}{width=.9\columnwidth}

\tab{hyp-solutions}{Solutions for the frequency of the collector task for the
hypothetical application.}

We ran \nsga~with a population size of 100 individuals, being 20 of them
selected as parents, generating 20 offsprings, repeating the process during 50
generations (iterations) (\nsgaparam).
%With these parameters the optimizer was able to find four good solutions for
%the collector task frequency, shown at Tab.~\ref{tab:hyp-solutions}.
The best solutions found by the optimizer, i.e., those at the Pareto front, are
shown in Tab.~\ref{tab:hyp-solutions}. By analyzing
Fig.~\ref{fig:hyp-solutions}, which shows all found solutions, it is possible to
observe here the wide spread of results obtained from the optimization process.
It is also important to note the non-linear behavior of the observed parameters
in relation to frequency, showing why the solution to this problem benefits from
the application of meta-heuristic methods like \nsga.

%\figthree
% \figtwo
% {hyp-bet_freq}{Execution rate of best-effort tasks.}
% {hyp-batt_freq}{Residual energy after projected lifetime.}
% %{hyp-lost_freq}{Total of lost hard deadlines.}
% {Optimization objectives plotted against variations on the frequency of the
% collector task for the hypothetical application.}

% \fig{hyp-bet_freq}{Results for the objective ``execution rate of best-effort
% tasks'' after optimization procedure for the hypothetical
% application.}{width=\columnwidth}
% 
% \fig{hyp-batt_freq}{Results for the objective ``residual energy'' after
% optimization procedure for the hypothetical application.}{width=\columnwidth}

\section{Implementation for the EPOS Project}
\label{sec:impl_epos}

The \epos~Project (Embedded Parallel Operating System) aims at automating the
development of embedded systems so that developers can concentrate on the
applications. \epos~relies on the Application-Driven Embedded System Design
(\adesd)~\cite{Frohlich:2001} method to guide the development of both software
and hardware components that can be automatically adapted to fulfill the
requirements of particular applications. \epos~features a set of tools to
support developers in selecting, configuring, and plugging components into its
application-specific framework. The combination of methodology, components,
frameworks, and tools enable the automatic generation of application-specific
embedded system instances~\cite{Project:EPOS:2010}.

\subsection{The EPOSMote}

Besides run time support system and tools, the \epos~Pro\-ject has driven the
development of hardware platforms, being the \emote~among them. \emote~is a
modular platform for wireless sensor network applications. It has three
different modules as shown in Fig.~\ref{fig:emote2-mc13224v-block_diagram}.
The \emph{Processing Module} incorporates the core processing and communication
components of the system. There are two different versions of this module: one
based on Atmel's ZigBit package and another, used in this work, based on
Freescales' System-on-Chip MC13224V. Both present RF transceivers compatible
with IEEE 802.15.4 standard. Power supply and I/O interfaces were factored out
on \emote's design to allow adaptation to specific applications. The power
interface has separated signals for the power source ($V_{cc}$, $V_{dd}$ and
$Gnd$) and an $I^2C$ interface for communication with the processing module. The
I/O interface make 32 pins available for custom designs including a bypass of
the power source, all $ADC$ channels, $SPI$, $UART$ and several $GPIO$ pins. The
\emote~Project developed a \emph{Start-Up} board to be connected to the I/O
interface featuring $USB$ converter, a thermistor, a 3-axis accelerometer, LEDs,
and buttons that was also used in this work.

\fig{emote2-mc13224v-block_diagram}{EPOSMote's block diagram.}{scale=.54}

To be able to account for energy consumption on the \emote~we first need to
analyze its power characteristics and build its energy model. These information
were collected from the devices' datasheets, when available, or measured using
an oscilloscope in current mode. Tab.~\ref{tab:emote-energy_currents} shows
values of current drains of system devices in different operating modes to be
used by time-based accounters, while Tab.~\ref{tab:emote-energy_consumptions}
show drained battery charge for monitored events to be used by event-based
accounters.

\tab{emote-energy_currents}{Current drawn by EPOSMote's components in different
operating modes for time-based accounting.}

\tab{emote-energy_consumptions}{Battery charge used by EPOSMote's components in
monitored events for event-based accounting.}


\subsection{The EPOS Power Manager}

Once power management is a non-functional property of computing
systems~\cite{Lohmann:2005} the \epos' power manager was modeled as an software
aspect~\cite{Mens:1997}, thus being its implementation orthogonal to the
implementation of other components in \epos' framework. In \epos, aspects are
implemented as constructs called \emph{Scenario
Adapters}~\cite{Frohlich:SCI:2000}, which relies on C++'s static metaprogramming
capability (templates) and doesn't imply in the use of extra tools such as
aspect weavers.

\wfig{pm-uml-classes}{UML class diagram of EPOS' power manager.}{scale=.99}

Fig.~\ref{fig:pm-uml-classes} shows a class diagram for the \epos' power
manager modeled as a scenario adapter.

The base \texttt{Power\_Manager} wraps the target class (to which the aspect may
be applied) by inheritance and function overriding (the \emph{Adapter Design
Pattern}). Additional methods may them be easily included, as is the case of the
\texttt{power} methods in the base \texttt{Power\_Manager} class.
\texttt{Power\_Manager} is, in turn, a facade (the \emph{Facade Design Pattern})
to other functionalities related to power management being implemented by other
components. For instance, \texttt{Power\_Manager\_Shared} and
\texttt{Power\_Manager\_Instances} are responsible for, respectively,
controlling of operating modes for shared components, and keeping of object
references for system-wide power management actions~\cite{Hoeller:DIPES:2006}.

In this work, \epos' power manager was extended to include the energy
consumption accounting functionality. This was done by aggregating the
\texttt{Power\_Manager\_Accounter} class which, in turn, implements the energy
consumption profiles described at Section~\ref{sec:profiles} with the algorithms
listed in Fig.~\ref{prg:batt_updates}. Accounting of events on the event-based
profile are performed through the \texttt{account(e:Event)} method, which may be
called in a \texttt{wrapped\_method()} if the event generated by such a method
is a monitored one. As concerning overhead issues, it is important to note that
the \texttt{account(e:Event)} method, which increments an event counter, is an
inline function, thus incurring in no overheads related to function calls at
runtime. Also, the branches that implement the facade at the \texttt{power(m:OP\_Mode)}
method of \texttt{Power\_Manager} use constant boolean values which, in \epos,
are defined at configuration time, before compilation. As such, these are
subject to compiler optimizations which are able to remove these branches in the
final system binary. Tab.~\ref{tab:pm-overhead} presents the impact of the
proposed accounter on \epos~in terms of code and data memory usage, showing that
the accounting mechanism aggregates 2,768 bytes of code (ROM) and 70 bytes of
data (RAM) over the original fully functional \texttt{Power\_Manager}.

\tab{pm-overhead}{Memory footprint overhead of the EPOS power manager.(?COMO
TIRA A HLINE DE BAIXO DO SETUP?)}


\section{Case Study}
\label{sec:case}

In this section we present the deployment of the approach described in this
paper in a mobility-enabled wireless sensor network running the Ant-based
Dynamic Hop Optimization Protocol (\adhop) over an IP network using IEEE
802.15.4. \adhop~is a self-configuring, reactive routing protocol inspired by
the HOPNET protocol for \emph{Mobile Ad Hoc Networks}~(MANETs) and designed with
the typical limitations of sensor nodes in mind, energy in
particular~\cite{Okazaki:2011}. \adhop's reactive component relies on an
\emph{Ant Colony Optimization} algorithm to discover and maintain routes. Ants
are sent out to track routes, leaving a trail of pheromone on their way back.
Routes with a higher pheromone deposit are preferred for data exchange.

\wtab{adzrp-taskset}{\adhop~case-study tasks' parameters.}

With the purpose of corroborating the approach presented in this paper we made a
few modifications to the \adhop~in order to make it energy-aware. In order to do
that, we separated \adhop~tasks between mandatory (hard real-time) tasks and
optional (best-effort) tasks. The main idea behind this setup was to homogenize
the battery discharge for every node in the network to enhance the lifetime of
the network as a whole. Considering the radio the most energy-hungry component
in a wireless sensing node, we toke the design decision of modeling the
\emph{ants} of \adhop~as best-effort tasks, as shown by the task set at
Table~\ref{tab:adzrp-taskset}. By doing this, the basic node functionality of
sensing a value (task $Sense$) and forwarding it through the radio to a
sink-node (task $Forward$) where modeled as hard real-time tasks, and the
functionality of forwarding other nodes' packets (and ants) when acting as a
``router'' was modeled as two best-effort tasks, one for monitoring the channel
for arriving messages ($Low Power Listen$), and another to effectively receive
the message and route it to another node ($Route$).

We set the lifetime objective for this system to 25 days. By analyzing the task
set it is possible to compute the total energy consumption of hard real-time
tasks for the desired lifetime to be of $601.88 mAh$, thus, the initial battery
charge for the system has to be greater than that. We analyzed this system
% in two situations: using a small Panasonic CR-2 $3V$ battery with a total
% capacity of $850 mAh$, and using a larger battery set comprised of two Panasonic
% AM-3PI $1.5V$ batteries, with a total capacity of $5,740 mAh$. In the first
% scenario, impact on network performance would be two significant, thus 
using a small Panasonic CR-2 $3V$ battery with a total capacity of $850 mAh$.

In order to enhance the significance of our results, we simulated larger
networks (with up to 200 nodes) using the \emph{Global Mobile Information System
Simulator} (\glmsim). In this setup, nodes were programmed to communicate
intensively and move randomly within a simulated grid of 700 x 400 meters for 25
days, thus stimulating both the routing protocol and the power management
mechanisms. \glmsim~was integrated to the same \nsga~optimizer described in
Section~\ref{sec:frequency}, and the optimization process ran with the same
parameters (\nsgaparam). The results of this optimization are shown in
Figure~\ref{fig:adzrp-solutions} (simulated solutions),
Table~\ref{tab:adzrp-solutions} (best solutions, i.e., those at the Pareto
front), and Figures~\ref{fig:adzrp-bet_freq} and~\ref{fig:adzrp-batt_freq}
(observed parameters plotted against the collector task's frequency).

\fig{adzrp-solutions}{All solutions for the \adhop~case-study.}{width=\columnwidth}

\tab{adzrp-solutions}{Solutions for the frequency of the collector task for
the \adhop~case-study.}

\figtwo{adzrp-bet_freq}{Execution rate of best-effort tasks.}
{adzrp-batt_freq}{Residual energy after projected lifetime.}
{Optimization objectives plotted against variations on the frequency of the
collector task for the \adhop~case-study.}

Additionally, we analyze the impact on the network performance by comparing the
obtained results with the data originally published by
Okazaki~\cite{Okazaki:2011}. Figure~\ref{fig:adzrp-avg_node_energy} shows a
reduction on the average energy consumed by each node on the network while
Figure~\ref{fig:adzrp-avg_node_lifetime} shows the expected enhancement on the
average battery lifetime of nodes. It is important to note that all nodes in
the network lived for, at least, 25 days as expected, being that the reason why
the average lifetime stayed around 30 days or above.

\figtwo{adzrp-avg_node_energy}{15 minutes energy
consumption.}{adzrp-avg_node_lifetime}{Battery lifetime.}{Average energy-related
parameters for the simulated \adhop~setup.}

Besides the good results from the energy consumption perspective, we observed an
important decrease on the overall network quality, as shown in
Figures~\ref{fig:adzrp-broken_routes} and~\ref{fig:adzrp-delivery_ratio} for,
respectively, the ``Broken routes'' and ``Delivery ratio'' parameters. These
contrast, however, with the obtained results on ``Link failures'' shown by
Figure~\ref{fig:adzrp-link_failures}, which shows that \adhop~deals well with
the broken routes, allowing undelivered packets to be re-routed and finally
delivered. Future work on energy-aware scheduling, which were not focus of the
present work, will rely on the presently proposed accounting mechanism to enable
fairer scheduling of such tasks. Initial studies have already began on flexible
schedulers such as Ramanathan's (m,k)-firm scheduler~\cite{Ramanathan:1997}
and Buttazzo's elastic model~\cite{Buttazzo:1998}.

\figthree{adzrp-broken_routes}{Broken routes.}{adzrp-delivery_ratio}{Delivery
ratio}{adzrp-link_failures}{Link failures}{Network quality impact for the \adhop
case-study.}


\section{Conclusion}
\label{sec:concl}

We presented a software implementation of an energy consumption accounter for
battery-operated embedded systems. We modeled the accounter and implemented it
in a simulation environment and in a real platform
(\epos~\cite{Project:EPOS:2010}), altought the implementation details in the
real platform have been kept out due to the page limit. In order to lower the
processing overhead imposed by our approach we extracted runtime parameters of
energy consumption and execution time of a given application and submitted it to
a meta-heuristic optimizer (\nsga). The objective is to look for good solutions
for the period at which battery-related information should be updated in order
to maximize system utilization while minimizing residual energy and guaranteeing
a pre-defined system lifetime (mission duration). A case study on an IP-based
network running over IEEE 802.15.4 sensing nodes showed promising results. It
showed that the approach was able to control energy consumption on the network
in a way that none of the nodes ran out of battery before the pre-defined
mission duration has elapsed.

On going studies are extending the work in this paper by deploying fairer
scheduling mechanisms to reduce the impact on system quality. This effort rely
on flexible task scheduling schemes
% , such as \textit{(m,k)-firm}~\cite{Ramanathan:1997} or the elastic
% model~\cite{Buttazzo:1998},
to be put in place of the current egoist approach of preventing tasks'
execution and using network-wide battery charge information provided by the
accounter described herein as parameter for \adhop's pheromone generation
function.


% Reminder: the "draftcls" or "draftclsnofoot", not "draft", class option
% should be used if it is desired that the figures are to be displayed while
% in draft mode.


% An example of a floating figure using the graphicx package.
% Note that \label must occur AFTER (or within) \caption.
% For figures, \caption should occur after the \includegraphics.
%
% \begin{figure}
% \centering
% \includegraphics[width=2.5in]{myfigure}
% where an .eps filename suffix will be assumed under latex,
% and a .pdf suffix will be assumed for pdflatex
% \caption{Simulation Results}
% \label{fig_sim}
% \end{figure}


% An example of a double column floating figure using two subfigures.
% (The subfigure.sty package must be loaded for this to work.)
% The subfigure \label commands are set within each subfigure command, the
% \label for the overall fgure must come after \caption.
% \hfil must be used as a separator to get equal spacing
%
% \begin{figure*}
% \centerline{\subfigure[Case I]{\includegraphics[width=2.5in]{subfigcase1}
% where an .eps filename suffix will be assumed under latex,
% and a .pdf suffix will be assumed for pdflatex
% \label{fig_first_case}}
% \hfil
% \subfigure[Case II]{\includegraphics[width=2.5in]{subfigcase2}
% where an .eps filename suffix will be assumed under latex,
% and a .pdf suffix will be assumed for pdflatex
% \label{fig_second_case}}}
% \caption{Simulation results}
% \label{fig_sim}
% \end{figure*}


% An example of a floating table. Note that, for IEEE style tables, the
% \caption command should come BEFORE the table. Table text will default to
% \footnotesize as IEEE normally uses this smaller font for tables.
% The \label must come after \caption as always.
%
% \begin{table}
% increase table row spacing, adjust to taste
% \renewcommand{\arraystretch}{1.3}
% \caption{An Example of a Table}
% \label{table_example}
% \begin{center}
% Some packages, such as MDW tools, offer better commands for making tables
% than the plain LaTeX2e tabular which is used here.
% \begin{tabular}{|c||c|}
% \hline
% One & Two\\
% \hline
% Three & Four\\
% \hline
% \end{tabular}
% \end{center}
% \end{table}


% use section* for acknowledgement
\section*{Acknowledgment}
% optional entry into table of contents (if used)
% \addcontentsline{toc}{section}{Acknowledgment}
Authors would like to thank the support given by Alexandre Massayuki Okazaki and
Rafael Luiz Cancian on the integration of the work herein with, respectively,
the \adhop~routing protocol and the \adesd's optimization tool.


% trigger a \newpage just before the given reference
% number - used to balance the columns on the last page
% adjust value as needed - may need to be readjusted if
% the document is modified later
% \IEEEtriggeratref{8}
% The "triggered" command can be changed if desired:
% \IEEEtriggercmd{\enlargethispage{-5in}}


% references section
% NOTE: BibTeX documentation can be easily obtained at:
% http://www.ctan.org/tex-archive/biblio/bibtex/contrib/doc/


% can use a bibliography generated by BibTeX as a .bbl file
% standard IEEE bibliography style from:
% http://www.ctan.org/tex-archive/macros/latex/contrib/supported/IEEEtran/bibtex
\bibliographystyle{IEEEtran.bst}
\bibliography{paper}

% argument is your BibTeX string definitions and bibliography database(s)
% \bibliography{IEEEabrv,../bib/paper}
%
% <OR> manually copy in the resultant .bbl file
% set second argument of \begin to the number of references
% (used to reserve space for the reference number labels box)
% \begin{thebibliography}{1}
% 
% \bibitem{IEEEhowto:kopka}
% H.~Kopka and P.~W. Daly, \emph{A Guide to {\LaTeX}}, 3rd~ed.\hskip 1em plus
%   0.5em minus 0.4em\relax Harlow, England: Addison-Wesley, 1999.
% \end{thebibliography}


% that's all folks
\end{document} 

