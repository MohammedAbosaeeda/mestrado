
%% bare_conf.tex
%% V1.3
%% 2007/01/11
%% by Michael Shell
%% See:
%% http://www.michaelshell.org/
%% for current contact information.
%%
%% This is a skeleton file demonstrating the use of IEEEtran.cls
%% (requires IEEEtran.cls version 1.7 or later) with an IEEE conference paper.
%%
%% Support sites:
%% http://www.michaelshell.org/tex/ieeetran/
%% http://www.ctan.org/tex-archive/macros/latex/contrib/IEEEtran/
%% and
%% http://www.ieee.org/

%%*************************************************************************
%% Legal Notice:
%% This code is offered as-is without any warranty either expressed or
%% implied; without even the implied warranty of MERCHANTABILITY or
%% FITNESS FOR A PARTICULAR PURPOSE!
%% User assumes all risk.
%% In no event shall IEEE or any contributor to this code be liable for
%% any damages or losses, including, but not limited to, incidental,
%% consequential, or any other damages, resulting from the use or misuse
%% of any information contained here.
%%
%% All comments are the opinions of their respective authors and are not
%% necessarily endorsed by the IEEE.
%%
%% This work is distributed under the LaTeX Project Public License (LPPL)
%% ( http://www.latex-project.org/ ) version 1.3, and may be freely used,
%% distributed and modified. A copy of the LPPL, version 1.3, is included
%% in the base LaTeX documentation of all distributions of LaTeX released
%% 2003/12/01 or later.
%% Retain all contribution notices and credits.
%% ** Modified files should be clearly indicated as such, including  **
%% ** renaming them and changing author support contact information. **
%%
%% File list of work: IEEEtran.cls, IEEEtran_HOWTO.pdf, bare_adv.tex,
%%                    bare_conf.tex, bare_jrnl.tex, bare_jrnl_compsoc.tex
%%*************************************************************************

% *** Authors should verify (and, if needed, correct) their LaTeX system  ***
% *** with the testflow diagnostic prior to trusting their LaTeX platform ***
% *** with production work. IEEE's font choices can trigger bugs that do  ***
% *** not appear when using other class files.                            ***
% The testflow support page is at:
% http://www.michaelshell.org/tex/testflow/



% Note that the a4paper option is mainly intended so that authors in
% countries using A4 can easily print to A4 and see how their papers will
% look in print - the typesetting of the document will not typically be
% affected with changes in paper size (but the bottom and side margins will).
% Use the testflow package mentioned above to verify correct handling of
% both paper sizes by the user's LaTeX system.
%
% Also note that the "draftcls" or "draftclsnofoot", not "draft", option
% should be used if it is desired that the figures are to be displayed in
% draft mode.
%
\documentclass[conference]{IEEEtran}
% Add the compsoc option for Computer Society conferences.
%
% If IEEEtran.cls has not been installed into the LaTeX system files,
% manually specify the path to it like:
% \documentclass[conference]{../sty/IEEEtran}


% Added to solve the error: "! Package babel Error: You haven't defined the language ENGLISH yet."
% Reference: http://www.michaelshell.org/tex/ieeetran/
\makeatletter
\def\markboth#1#2{\def\leftmark{\@IEEEcompsoconly{\sffamily}\MakeUppercase{\protect#1}}%
\def\rightmark{\@IEEEcompsoconly{\sffamily}\MakeUppercase{\protect#2}}}
\makeatother



% Some very useful LaTeX packages include:
% (uncomment the ones you want to load)

\usepackage[english]{babel} % for multilingual support
% \usepackage{babel} % for multilingual support

\usepackage[utf8]{inputenc} % for use utf8

% *** MISC UTILITY PACKAGES ***
%
%\usepackage{ifpdf}
% Heiko Oberdiek's ifpdf.sty is very useful if you need conditional
% compilation based on whether the output is pdf or dvi.
% usage:
% \ifpdf
%   % pdf code
% \else
%   % dvi code
% \fi
% The latest version of ifpdf.sty can be obtained from:
% http://www.ctan.org/tex-archive/macros/latex/contrib/oberdiek/
% Also, note that IEEEtran.cls V1.7 and later provides a builtin
% \ifCLASSINFOpdf conditional that works the same way.
% When switching from latex to pdflatex and vice-versa, the compiler may
% have to be run twice to clear warning/error messages.






% *** CITATION PACKAGES ***
%
%\usepackage{cite}
% cite.sty was written by Donald Arseneau
% V1.6 and later of IEEEtran pre-defines the format of the cite.sty package
% \cite{} output to follow that of IEEE. Loading the cite package will
% result in citation numbers being automatically sorted and properly
% "compressed/ranged". e.g., [1], [9], [2], [7], [5], [6] without using
% cite.sty will become [1], [2], [5]--[7], [9] using cite.sty. cite.sty's
% \cite will automatically add leading space, if needed. Use cite.sty's
% noadjust option (cite.sty V3.8 and later) if you want to turn this off.
% cite.sty is already installed on most LaTeX systems. Be sure and use
% version 4.0 (2003-05-27) and later if using hyperref.sty. cite.sty does
% not currently provide for hyperlinked citations.
% The latest version can be obtained at:
% http://www.ctan.org/tex-archive/macros/latex/contrib/cite/
% The documentation is contained in the cite.sty file itself.






% *** GRAPHICS RELATED PACKAGES ***
%
\ifCLASSINFOpdf
  \usepackage[pdftex]{graphicx}
  \usepackage[caption=false]{subfig} % Para usar duas ou mais figuras como uma só.
  % declare the path(s) where your graphic files are
  % \graphicspath{{../pdf/}{../jpeg/}}
  % and their extensions so you won't have to specify these with
  % every instance of \includegraphics
  % \DeclareGraphicsExtensions{.pdf,.jpeg,.png}
\else
  % or other class option (dvipsone, dvipdf, if not using dvips). graphicx
  % will default to the driver specified in the system graphics.cfg if no
  % driver is specified.
  % \usepackage[dvips]{graphicx}
  % declare the path(s) where your graphic files are
  % \graphicspath{{../eps/}}
  % and their extensions so you won't have to specify these with
  % every instance of \includegraphics
  % \DeclareGraphicsExtensions{.eps}
\fi
% graphicx was written by David Carlisle and Sebastian Rahtz. It is
% required if you want graphics, photos, etc. graphicx.sty is already
% installed on most LaTeX systems. The latest version and documentation can
% be obtained at:
% http://www.ctan.org/tex-archive/macros/latex/required/graphics/
% Another good source of documentation is "Using Imported Graphics in
% LaTeX2e" by Keith Reckdahl which can be found as epslatex.ps or
% epslatex.pdf at: http://www.ctan.org/tex-archive/info/
%
% latex, and pdflatex in dvi mode, support graphics in encapsulated
% postscript (.eps) format. pdflatex in pdf mode supports graphics
% in .pdf, .jpeg, .png and .mps (metapost) formats. Users should ensure
% that all non-photo figures use a vector format (.eps, .pdf, .mps) and
% not a bitmapped formats (.jpeg, .png). IEEE frowns on bitmapped formats
% which can result in "jaggedy"/blurry rendering of lines and letters as
% well as large increases in file sizes.
%
% You can find documentation about the pdfTeX application at:
% http://www.tug.org/applications/pdftex





% *** MATH PACKAGES ***
%
%\usepackage[cmex10]{amsmath}
% A popular package from the American Mathematical Society that provides
% many useful and powerful commands for dealing with mathematics. If using
% it, be sure to load this package with the cmex10 option to ensure that
% only type 1 fonts will utilized at all point sizes. Without this option,
% it is possible that some math symbols, particularly those within
% footnotes, will be rendered in bitmap form which will result in a
% document that can not be IEEE Xplore compliant!
%
% Also, note that the amsmath package sets \interdisplaylinepenalty to 10000
% thus preventing page breaks from occurring within multiline equations. Use:
%\interdisplaylinepenalty=2500
% after loading amsmath to restore such page breaks as IEEEtran.cls normally
% does. amsmath.sty is already installed on most LaTeX systems. The latest
% version and documentation can be obtained at:
% http://www.ctan.org/tex-archive/macros/latex/required/amslatex/math/





% *** SPECIALIZED LIST PACKAGES ***
%
%\usepackage{algorithmic}
% algorithmic.sty was written by Peter Williams and Rogerio Brito.
% This package provides an algorithmic environment fo describing algorithms.
% You can use the algorithmic environment in-text or within a figure
% environment to provide for a floating algorithm. Do NOT use the algorithm
% floating environment provided by algorithm.sty (by the same authors) or
% algorithm2e.sty (by Christophe Fiorio) as IEEE does not use dedicated
% algorithm float types and packages that provide these will not provide
% correct IEEE style captions. The latest version and documentation of
% algorithmic.sty can be obtained at:
% http://www.ctan.org/tex-archive/macros/latex/contrib/algorithms/
% There is also a support site at:
% http://algorithms.berlios.de/index.html
% Also of interest may be the (relatively newer and more customizable)
% algorithmicx.sty package by Szasz Janos:
% http://www.ctan.org/tex-archive/macros/latex/contrib/algorithmicx/




% *** ALIGNMENT PACKAGES ***
%
%\usepackage{array}
% Frank Mittelbach's and David Carlisle's array.sty patches and improves
% the standard LaTeX2e array and tabular environments to provide better
% appearance and additional user controls. As the default LaTeX2e table
% generation code is lacking to the point of almost being broken with
% respect to the quality of the end results, all users are strongly
% advised to use an enhanced (at the very least that provided by array.sty)
% set of table tools. array.sty is already installed on most systems. The
% latest version and documentation can be obtained at:
% http://www.ctan.org/tex-archive/macros/latex/required/tools/


%\usepackage{mdwmath}
%\usepackage{mdwtab}
% Also highly recommended is Mark Wooding's extremely powerful MDW tools,
% especially mdwmath.sty and mdwtab.sty which are used to format equations
% and tables, respectively. The MDWtools set is already installed on most
% LaTeX systems. The lastest version and documentation is available at:
% http://www.ctan.org/tex-archive/macros/latex/contrib/mdwtools/


% IEEEtran contains the IEEEeqnarray family of commands that can be used to
% generate multiline equations as well as matrices, tables, etc., of high
% quality.


%\usepackage{eqparbox}
% Also of notable interest is Scott Pakin's eqparbox package for creating
% (automatically sized) equal width boxes - aka "natural width parboxes".
% Available at:
% http://www.ctan.org/tex-archive/macros/latex/contrib/eqparbox/





% *** SUBFIGURE PACKAGES ***
%\usepackage[tight,footnotesize]{subfigure}
% subfigure.sty was written by Steven Douglas Cochran. This package makes it
% easy to put subfigures in your figures. e.g., "Figure 1a and 1b". For IEEE
% work, it is a good idea to load it with the tight package option to reduce
% the amount of white space around the subfigures. subfigure.sty is already
% installed on most LaTeX systems. The latest version and documentation can
% be obtained at:
% http://www.ctan.org/tex-archive/obsolete/macros/latex/contrib/subfigure/
% subfigure.sty has been superceeded by subfig.sty.



%\usepackage[caption=false]{caption}
%\usepackage[font=footnotesize]{subfig}
% subfig.sty, also written by Steven Douglas Cochran, is the modern
% replacement for subfigure.sty. However, subfig.sty requires and
% automatically loads Axel Sommerfeldt's caption.sty which will override
% IEEEtran.cls handling of captions and this will result in nonIEEE style
% figure/table captions. To prevent this problem, be sure and preload
% caption.sty with its "caption=false" package option. This is will preserve
% IEEEtran.cls handing of captions. Version 1.3 (2005/06/28) and later
% (recommended due to many improvements over 1.2) of subfig.sty supports
% the caption=false option directly:
%\usepackage[caption=false,font=footnotesize]{subfig}
%
% The latest version and documentation can be obtained at:
% http://www.ctan.org/tex-archive/macros/latex/contrib/subfig/
% The latest version and documentation of caption.sty can be obtained at:
% http://www.ctan.org/tex-archive/macros/latex/contrib/caption/




% *** FLOAT PACKAGES ***
%
%\usepackage{fixltx2e}
% fixltx2e, the successor to the earlier fix2col.sty, was written by
% Frank Mittelbach and David Carlisle. This package corrects a few problems
% in the LaTeX2e kernel, the most notable of which is that in current
% LaTeX2e releases, the ordering of single and double column floats is not
% guaranteed to be preserved. Thus, an unpatched LaTeX2e can allow a
% single column figure to be placed prior to an earlier double column
% figure. The latest version and documentation can be found at:
% http://www.ctan.org/tex-archive/macros/latex/base/



%\usepackage{stfloats}
% stfloats.sty was written by Sigitas Tolusis. This package gives LaTeX2e
% the ability to do double column floats at the bottom of the page as well
% as the top. (e.g., "\begin{figure*}[!b]" is not normally possible in
% LaTeX2e). It also provides a command:
%\fnbelowfloat
% to enable the placement of footnotes below bottom floats (the standard
% LaTeX2e kernel puts them above bottom floats). This is an invasive package
% which rewrites many portions of the LaTeX2e float routines. It may not work
% with other packages that modify the LaTeX2e float routines. The latest
% version and documentation can be obtained at:
% http://www.ctan.org/tex-archive/macros/latex/contrib/sttools/
% Documentation is contained in the stfloats.sty comments as well as in the
% presfull.pdf file. Do not use the stfloats baselinefloat ability as IEEE
% does not allow \baselineskip to stretch. Authors submitting work to the
% IEEE should note that IEEE rarely uses double column equations and
% that authors should try to avoid such use. Do not be tempted to use the
% cuted.sty or midfloat.sty packages (also by Sigitas Tolusis) as IEEE does
% not format its papers in such ways.





% *** PDF, URL AND HYPERLINK PACKAGES ***
%
%\usepackage{url}
% url.sty was written by Donald Arseneau. It provides better support for
% handling and breaking URLs. url.sty is already installed on most LaTeX
% systems. The latest version can be obtained at:
% http://www.ctan.org/tex-archive/macros/latex/contrib/misc/
% Read the url.sty source comments for usage information. Basically,
% \url{my_url_here}.





% *** Do not adjust lengths that control margins, column widths, etc. ***
% *** Do not use packages that alter fonts (such as pslatex).         ***
% There should be no need to do such things with IEEEtran.cls V1.6 and later.
% (Unless specifically asked to do so by the journal or conference you plan
% to submit to, of course. )

% Commands to insert figures ---------------------------------------------------
\newcommand{\fig}[4][ht]{
  \begin{figure}[#1] {\centering\scalebox{#2}{\includegraphics{fig/#3}}\par}
    \caption{#4\label{fig:#3}}
  \end{figure}
}
% fig usage:
% \fig{<scale>}{<file>}{<caption>}
% e.g.: \fig{.4}{uml/uml_comportamental_dia}{Diagramas comportamentais da UML}
% The figure label will be "fig:" plus <file>.
% The figure file must lie in the "fig" directory.

\newcommand{\figtwocolumn}[4][ht]{
  \begin{figure*}[#1] {\centering\scalebox{#2}{\includegraphics{fig/#3}}\par}
    \caption{#4\label{fig:#3}}
  \end{figure*}
}

% Para colocar 2 figuras como uma só - dispostas horizontalmente
\newcommand{\multfigtwoh}[6][htbp]{
\begin{figure*}[#1]
  \centering
  \subfloat[]{\label{fig:#3}\scalebox{#2}{\includegraphics{fig/#3}}}
  \subfloat[]{\label{fig:#4}\scalebox{#2}{\includegraphics{fig/#4}}}
  \caption{#6}
  \label{fig:#5}
\end{figure*}
}
% e.g.
%\multfigtwoh{.65}{fig_plot_time_orig}{fig_plot_time_mod}
%{fig_plot_time_all}
%{Original (a) and modified (b) benchmarks execution time comparison.}

% Para colocar 2 figuras como uma só - dispostas verticalmente
\newcommand{\multfigtwov}[6][htbp]{
\begin{figure}[#1]
  \centering
  \subfloat[]{\label{fig:#3}\scalebox{#2}{\includegraphics{fig/#3}}}\\
  \subfloat[]{\label{fig:#4}\scalebox{#2}{\includegraphics{fig/#4}}}
  \caption{#6}
  \label{fig:#5}
\end{figure}
}
% e.g.
%\multfigtwov{.35}{fig_epos_mem_framework}{fig_epos_mem_framework_spm}
%{fig_epos_mem_framework_all}
%{EPOS memory mapping before (a) and after (b) using the new framework}

% Commands to used code as figure ----------------------------------------------
\usepackage{listings}
\lstset{keywordstyle=\bfseries, flexiblecolumns=true}
\lstloadlanguages{[ANSI]C++,HTML}
\lstdefinestyle{prg} {basicstyle=\small\sffamily, lineskip=-0.2ex, showspaces=false}

% C++
\newcommand{\progcpp}[3][tbp]{
 \begin{figure}[#1]
     \lstinputlisting[language=C++,style=prg]{fig/#2.cc}
   \caption{#3\label{progcpp:#2}}
 \end{figure}
}

% ------------------------------------------------------------------------------


% correct bad hyphenation here
\hyphenation{op-tical net-works semi-conduc-tor}


\begin{document}
%
% paper title
% can use linebreaks \\ within to get better formatting as desired
\title{System-Level Verification of Embedded Operating Systems Components}



% author names and affiliations
% use a multiple column layout for up to three different
% affiliations
\author{\IEEEauthorblockN{Mateus Krepsky Ludwich and Antônio Augusto Fröhlich}
\IEEEauthorblockA{Laboratory for Software and Hardware Integration (LISHA)\\
Federal University of Santa Catarina (UFSC)\\
P.O.Box 476, 880400900 - Florianópolis - SC - Brasil\\
Email: \{mateus,guto\}@lisha.ufsc.br
}
}

% conference papers do not typically use \thanks and this command
% is locked out in conference mode. If really needed, such as for
% the acknowledgment of grants, issue a \IEEEoverridecommandlockouts
% after \documentclass

% for over three affiliations, or if they all won't fit within the width
% of the page, use this alternative format:
% 
%\author{\IEEEauthorblockN{Michael Shell\IEEEauthorrefmark{1},
%Homer Simpson\IEEEauthorrefmark{2},
%James Kirk\IEEEauthorrefmark{3}, 
%Montgomery Scott\IEEEauthorrefmark{3} and
%Eldon Tyrell\IEEEauthorrefmark{4}}
%\IEEEauthorblockA{\IEEEauthorrefmark{1}School of Electrical and Computer Engineering\\
%Georgia Institute of Technology,
%Atlanta, Georgia 30332--0250\\ Email: see http://www.michaelshell.org/contact.html}
%\IEEEauthorblockA{\IEEEauthorrefmark{2}Twentieth Century Fox, Springfield, USA\\
%Email: homer@thesimpsons.com}
%\IEEEauthorblockA{\IEEEauthorrefmark{3}Starfleet Academy, San Francisco, California 96678-2391\\
%Telephone: (800) 555--1212, Fax: (888) 555--1212}
%\IEEEauthorblockA{\IEEEauthorrefmark{4}Tyrell Inc., 123 Replicant Street, Los Angeles, California 90210--4321}}




% use for special paper notices
%\IEEEspecialpapernotice{(Invited Paper)}




% make the title area
\maketitle


\begin{abstract}
%\boldmath
The increasing complexity of embedded operating systems is pushing their design
to System-Level, leading to the convergence between software and hardware.
In such scenario, it is highly desirable to verify system properties formally,
regardless of whether their components are going to be implemented
in software or hardware.
In this paper, we introduce an approach to verify
functional correctness and safety properties of
embedded operating system components formally and at System-Level.
In order to demonstrate our approach, we present a scheduler of an
embedded operating system showing that such scheduler follows its specification
regardless of the domain it is instantiated.
The verified code was subsequently compiled using GCC yielding a software
instance and using CatapultC yielding a hardware instance of the scheduler.
\end{abstract}

% IEEEtran.cls defaults to using nonbold math in the Abstract.
% This preserves the distinction between vectors and scalars. However,
% if the conference you are submitting to favors bold math in the abstract,
% then you can use LaTeX's standard command \boldmath at the very start
% of the abstract to achieve this. Many IEEE journals/conferences frown on
% math in the abstract anyway.

% no keywords




% For peer review papers, you can put extra information on the cover
% page as needed:
% \ifCLASSOPTIONpeerreview
% \begin{center} \bfseries EDICS Category: 3-BBND \end{center}
% \fi
%
% For peerreview papers, this IEEEtran command inserts a page break and
% creates the second title. It will be ignored for other modes.
\IEEEpeerreviewmaketitle

% ------------------------------------------------------------------------------
% ------------------------------------------------------------------------------
\section{Introduction} \label{intro}
% + Introduction
% 
% The very first letter is a 2 line initial drop letter followed
% by the rest of the first word in caps.
%
% form to use if the first word consists of a single letter:
% \IEEEPARstart{A}{demo} file is ....
%
% form to use if you need the single drop letter followed by
% normal text (unknown if ever used by IEEE):
% \IEEEPARstart{A}{}demo file is ....
%
% Some journals put the first two words in caps:
% \IEEEPARstart{T}{his demo} file is ....
%
% Here we have the typical use of a "T" for an initial drop letter
% and "HIS" in caps to complete the first word.
% \IEEEPARstart{T}{his} demo file is intended.

\IEEEPARstart{E}{nergy} consumption is a determining factor when designing wireless sensor networks.
As a consequence, battery lifetime is a limitation on the development of such systems.
Therefore, the idea of extracting energy from the environment has become attractive.
Looking to the energy consumption problem, the intelligent usage of the stored energy contributes to extend the sensor nodes' longevity.
Consequently, energy schedulers have been developed in order to adequately assess the energy consumption and adapt the system accordingly to the available amount of energy.
The purpose of this work is to adapt a solar energy harvesting circuit to supply energy to low power wireless platforms, i.e., those that operate under $50~mW$.
Simultaneously, we aim at improving the performance of the energy-aware task scheduler in wireless sensor network systems by providing fine-grained battery and environmental monitoring.

Among a number of energy sources that have been studied so far, solar has proved to be one of the most effective~\cite{Roundy:2003}.
The solar energy conversion through photovoltaic (PV) cells is better performed at an optimum operating voltage.
Operating a solar panel on this voltage results in transferring to the system the maximum amount of power available.
In this context, \emph{maximum power point tracker circuits} (MPPT) have been proposed.
The drawback is that MPPT circuitry may introduce losses to a solar harvesting system.
Concerning low-power applications, it may be more energy efficient to have a good matching between the solar panel and the energy storage unit~\cite{Raghunathan:2005}.
This well matched system is than able to work close to the maximum power point with less power loss.

In this work, an evaluation of the proposed harvesting circuit is performed in order to show improvements on an energy-aware task scheduler~\cite{Hoeller:SMC:2011}.
It is shown that the combination of the proposed circuit with the cited scheduler not only extended the longevity of the wireless sensor network, but also improved system quality.

The paper is organized as follows:
Section~\ref{fund} presents the fundamentals of solar energy harvesting and energy-aware task scheduler.
Section~\ref{design} discusses the design of the harvesting circuit under the perspective of low power wireless platforms.
Section~\ref{case} presents the evaluation of the harvesting circuit and a case study showing the improvements on system quality.
Finally, section~\ref{concl} closes the paper.

% ------------------------------------------------------------------------------


% ------------------------------------------------------------------------------
% ------------------------------------------------------------------------------
\section{Strategies for ME Optimization} \label{sota}
% TODO: Go back to here to customize this section.
% Currently it is to centered on ME optimization. 
% It does not mention interfaces, etc.

% + Related Work
% This section make an overview of the strategies to optimize the time performance of ME:
% algorithmic optimizations (fast-search algorithms, macroblock subsampling,
% sample truncation, multi-resolution ME, subsampled motion-field estimation),
%  parallelization of algorithms (our case), and
% hardware implementations of algorithms.
%
% Focus on works related to ME parallelization / distribution.
% Maybe compare to the work of Ronaldo's student: Ricardo Kintschner also about
% ME optimization on Cell (Webmedia2011).
% Look for other works directly comparable.
%
% Also: ?[and reviews techniques about designing for interfaces]?
% 

% TODO Maybe move this to SOTA
% The \emph{Block-Matching Algorithm} (BMA), which searches for similar blocks and
% generates the motion vectors, is mainly responsible for ME being so time
% consuming.
% Therefore one strategy for optimizing BMA is the \emph{fast-search}, which looks
% only in specific points of the search window, while a similar block is being
% searched.
% Another strategy is to perform ME hierarchically, computing motion vectors for a
% specific frame region, and refining them in each level, which is known as
% \emph{multi-resolution} motion estimation.
% Other strategies look into finding parallelism in BMAs, in order to run ME
% stages simultaneously.
% For all strategies there are also hardware implementations, based on optimized
% functional units (such as vector operations) or based on replication of
% functional units, to explore parallelism.
% Block-Matching Algorithms using fast-search improve time performance of ME, but
% they can find suboptimal motion vectors because they do not search in all
% positions of the search window. Multi-resolution ME works with different
% resolutions of one frame, successively refining the found motion vectors.
% This increases the ME time if the search is performed sequentially as in
% \cite{ChiaChunLin:FastAlgPlusArch:2006} or demands for replicated hardware
% functional units, as in \cite{ChiaChunLin:PMRME:2007}.
% Similarly, parallel and hardware implementations come at the cost of replicated
% or dedicated functional units.

% There are two major goals in motion estimation optimization: to improve the 
% compression rate and to reduce the total encoding time. 
% Improving the compression rate is achieved by finding the best possible motion
% vectors, which means motion vectors that will generate the smallest residual
% difference during the motion compensation (MC). Reducing the total encoding
% time is achieved by finding the motion vectors in the smallest possible period 
% of time. 
% Several tools in H.264 are used to find the best possible motion vectors; 
% besides looking in all positions of a search window (i.e. full-search), it is 
% possible to search in several reference frames (backwards or forwards), and it 
% is possible to perform block-matching using sub-pel precision 
% (half and quarter of a pel) \cite{citeulike:1269699}. 
% Finding the best motion vectors, very often, goes against finding the motion 
% vectors more quickly. 
% In this work we focus on motion estimation optimizations which aim to reduce the
% total encoding time, therefore we are not going into the details of techniques 
% for finding the best motion vectors possible, but they can be found in 
% \cite{YuWenHuang:Complex:2006}, \cite{YepingSu:MF:2006}, \cite{Ma:MF:2009}, 
% and \cite{XiangLi:MF:2004}. 
% It is important to notice that all techniques for ME optimization must take into
% consideration keeping the video quality of the generated bitstream.

% P2
% Como otimizar a ME - estratégias.
There are several strategies to optimize the execution time of ME:
fast-search algorithms, macroblock subsampling, sample truncation, 
multi-resolution ME, subsampled motion-field estimation, 
and parallel and hardware implementations of algorithms.

% P3 ... Pn-1
% Um paragrafo para cada estratégia dizendo:
% + how these solutions match/ contribute to Goals { g1, g2, .., gn} and Features {f1 , f2 , ..., fn}
% + And what set of features F is still missing? (GAP)
%
% P3 - fast-search
\emph{Fast-search algorithms} are block-matching algorithms that look only in 
specific positions of the search window \cite{SunNingning:TSS:2009, LaiManPo:4SS:1996,
ShipingZhu:DS:2009, Tourapis:PMVFAST:2001, HoiMingWong:EPMVFAST:2005, LiangGeeChen:TDL:1991}.
The search window defines the region of the reference frame that is scanned for a macroblock
partition similar to the current one. 
Only the motion vectors that correspond to the match with the lowest \emph{motion cost} 
are chosen. 
The main drawback of this approach is that, since some positions of the search 
window are discarded, it is possible to find suboptimal motion vectors.

% P4 macroblock subsampling and sample truncation
Two other strategies to optimize ME during block-matching are macroblock 
subsampling and sample truncation. 
Macroblock subsampling takes into consideration only a macroblock partition 
(i.e. some samples of a macroblock) while the matching for a 
position of the search window is being performed. 
Sample truncation is performed by ignoring the least significant bits of a 
sample. 
These strategies have been used separately in \cite{liu:sub:1993} (subsampling)
and in \cite{DBLP:journals/tcsv/HeTCL00}, and \cite{ChiaChunLin:PMRME:2007} 
(truncation). 

% P5 multi-resolution ME
\emph{Multi-resolution motion estimation} is the strategy in which the motion 
vectors are computed for distinct resolutions of the same frame. 
Motion vectors computed in a more coarse level can be successively refined until
the finest level (higher resolution). 
If the search is performed sequentially as in 
\cite{ChiaChunLin:FastAlgPlusArch:2006}, the time of ME can be increased due to
the dependencies between distinct levels. 
On the other hand, if the search is executed in parallel for each resolution 
level, as in \cite{ChiaChunLin:PMRME:2007}, hardware functional units need to be
replicated. 
A similar technique is \emph{subsampled motion-field estimation} 
\cite{liu:sub:1993}, which is based on the assumption that motion vectors of neighboring 
blocks tends to be similar. Thus, for each block, only a set of motion 
vectors (a motion-field) is computed, while the others are interpolated.

% P7 | P6 parallel and hardware implementations of algorithms
Other strategies for optimizing motion estimation are based on finding 
parallelism in ME stages, especially in the block-matching algorithms, in order
to execute them simultaneously. 
These parallel strategies commonly have been the base for dedicated hardware implementations. 
The \emph{Sum of Absolute Differences} (SAD) is a metric of error used in 
block-matching algorithms. This technique is frequently parallelized using functional 
units in hardware \cite{ChiaChunLin:PMRME:2007}, 
and \cite{HoyoungChang:HW:2009}. 
Hardware implementations of shared buffers for frame data are also
common \cite{HoyoungChang:HW:2009}, \cite{HaibingYin:HW:2010}.

% ------------------------------------------------------------------------------


% ------------------------------------------------------------------------------
\section{Building a Trustful Infrastructure for Future Internet}
\label{sec:solution}
The Internet architecture demonstrate inefficiency and problems in several and large areas, such as mobility, real-time applications,
failures (e.g. equipment, software bugs, and configuration mistakes), and especially in pervasive security problems \cite{Rexford:2010}.
Moreover, the Internet lacks effective solutions in terms of scalability and sustainability, 
consuming much more energy and hindering the management of countless sensor devices that are so important for several applications in the Future Internet.
Hence, we propose the use of a stack of communication protocols (UDP@NDN@C-MAC), in the scope of the EPOSMote project,
designed specifically to guarantee a trustful communication
%Our solution also includes EPOSMote II, an embedded platform. Thus, 
while still compromised with the low utilization of resources (processing, memory, power and communication bandwidth).
%and the use of EPOSMote II which is an embedded platform and represents a typical Future Internet device.

\subsection{EPOSMote}
The EPOSMote is an open hardware project~\cite{eposmote}. Initially it aimed at 
the development of a wireless sensor network module, and focused on environment 
monitoring. Its first version, the EPOSMote I, features an 8-bit AVR microcontroller, 
IEEE 802.15.4 communication capability and a small set of sensors.

As the project evolved a second version arose, with the objective of delivering a 
hardware platform to allow research on energy harvesting, biointegration, and 
MEMS-based sensors. The EPOSMote II focus on modularization, and thus is composed 
by interchangeable modules for each function.

Figure \ref{emote2-block_diagram} shows an overview of the EPOSMote II architecture.
Its hardware is designed as a layer architecture composed by a main module,
a sensoring module, and a power module. The main module is responsible for processing
and communication. It is based on the Freescale MC13224V microcontroller~\cite{mc13224v}, which possess 
a 32-bit ARM7 core, an IEEE 802.15.4-compliant transceiver, 128kB of flash memory, 80kB of ROM memory
and 96kB of RAM memory. We have developed a startup sensoring module, which contains some sensors  
(temperature and accelerometer), leds, switches, and a micro USB (that can also be used as power supply). 
Figure \ref{emote2-mc13224v-pictures-real_white_background} shows the development kit which is slightly 
larger than a R\$1 coin, on the left the sensoring module, and on the right the main module.

\fig{.45}{emote2-block_diagram}{Architectural overview of EPOSMote II.}

\fig{.07}{emote2-mc13224v-pictures-real_white_background}{EPOSMote II SDK side-by-side with a R\$1 coin.}

\subsection{C-MAC}
C-MAC is a highly configurable MAC protocol for WSNs realized as a framework of
medium access control strategies that can be combined to produce
application-specific protocols~\cite{steiner:2010}. It enables application
programmers to configure several communication parameters (e.g.  synchronization,
contention, error detection, acknowledgment, packing, etc) to adjust the protocol
to the specific needs of their applications. The framework was implemented in C++ 
using static metaprogramming techniques (e.g. templates, inline functions, and 
inline assembly), thus ensuring that configurability does not come at expense of 
performance or code size. The main C-MAC configuration points include:

\textbf{Physical layer configuration:} These are the configuration points defined
by the underlying transceiver (e.g. frequency, transmit power, date rate).

\textbf{Synchronization and organization:} Provides mechanisms to send or receive
synchronization data to organize the network and synchronize the nodes duty
cycle.

\textbf{Collision-avoidance mechanism:} Defines the contention mechanisms used to
avoid collisions. May be comprised of a carrier sense algorithm (e.g. CSMA-CA),
the exchange of contention packets (\emph{Request to Send} and \emph{Clear to
Send}), or a combination of both.

\textbf{Acknowledgment mechanism:} The exchange of \emph{ack} packets to
determine if the transmission was successful, including preamble acknowledgements.

\textbf{Error handling and security:} Determine which mechanisms will be used to
ensure the consistency of data (e.g. CRC check) and the data security.

The Future Internet will be composed by a wide range of both applications and devices, 
each with its own requirements and available resources. Through C-MAC configurability we
can provide the most adequate MAC functionalities for each case, instead of providing a 
general non-optimal solution for all of them.

\subsection{NDN}
Communication in NDN is impelled by the data consumers.
Nodes that are interested in a content transmit \emph{Interest} packets, which contains the name of the requested data. %selector, nonce
Every node that receives the \emph{Interest} and have the requested data can respond with a \emph{Data} packet that follows back the path from which the \emph{Interest} came. %content name, signature, signed info, data
It is important to notice that one \emph{Data} satisfies one \emph{Interest}, thus ensuring flow balance in the network.
Since the content being exchanged is identified by its name, all nodes interested in the same content can share transmissions (considering a broadcast medium, which is the case for most Future Internet devices).

NDN packet forwarding engine has three main data structures: the FIB (Forwarding Information Base), which is used to forward \emph{Interest} packets to potential sources; 
the ContentStore, which is a buffer memory used to maximize the sharing of packets; 
and the PIT (Pending Interest Table), which is used to keep track of \emph{Interest} packets so that \emph{Data} packets can be sent to its requester(s).

When a node receives an \emph{Interest} packet it searches for its content name, looking for a match primarily at the ContentStore, then the PIT, and ultimately at the FIB.
If there is a match at the ContentStore, it is sent and the \emph{Interest} discarded.
Otherwise, if there is a match at the PIT, the set of requesting interfaces for that data is updated, and the \emph{Interest} discarded (at this point an \emph{Interest} in this data has already been sent).
Otherwise, if there is a match at the FIB, the \emph{Interest} is sent towards the data, and a new PIT entry is created. 
In case there is no match for the \emph{Interest} then it is discarded.

As for the \emph{Data} packet they simply follow the chain of PIT entries back to the original requester(s).
When a node receives a \emph{Data} packet it searches for its content name. 
If there is a ContentStore match, then the \emph{Data} is a duplicate and is discarded.
%A FIB match means there are no matching PIT entries, so the \emph{Data} is unsolicited and it is discarded.
In case of a PIT match, the data is validated, added to the ContentStore, and sent to the set of requesting interfaces from the corresponding PIT entry.

In NDN the name in every packet is bound to its content with a signature.
This enables data integrity and provenance, allowing consumers to trust the data they receive regardless of how the data came to them.
To provide content protection and access control NDN uses encryption.
The encryption of content or names is transparent to the network, since to NDN it is all just named binary data.
%The signature algorithm used may be selected by the content publisher, 
%and chosen to meet performance requirements such as latency or computational cost of signature generation or verification.
Nevertheless, NDN does not mandate any particular key distribution scheme, signature, or encryption algorithm.

\subsection{UDP}
The User Datagram Protocol has been chosen for its simplicity. Its simple transmission 
model avoids unnecessary overhead, since it does not handle reliability, ordering, 
and data integrity, leaving these characteristics to be treated in other layers if necessary, which is a 
perfect blend with the rest of our protocol stack.


% ------------------------------------------------------------------------------
% ------------------------------------------------------------------------------
\section{Practical Experiments} \label{eval}
% + Practical Experiments
% First shows speedup and quality results for DMEC
% using 1 (without partitioning) to 6 worker threads.
% Show that speedup is high and quality is kept acceptable.
% 
% Then, show speedup and quality results for DMEC integrated to JM and compares
%  to the original JM (and, if possible, to other works).
% 

% + JM
% + Dizer como realizamos os experimentos. E/ou quais as variáves observadas:
%   Evaluate the component in isolation DMEC to show its speedup.
%    And how it scales.
%   Evaluate the component in JM to show sppeedup and PSNR.
We have evaluated DMEC in two stages.
First, in order to verify how DMEC's performance scales from 
one to six \emph{Workers} instances, we have evaluated all DMEC implementations 
in a test case.
The test case application mimics the behavior of an H.264 encoder: it provides 
DMEC with pictures, obtain the ME results 
(motion vectors and motion cost), and checks if the results are correct.
Secondly, in order to assess DMEC influence on the final video quality, we have
evaluated all DMEC implementations in the
JM H.264 Reference Encoder~\cite{site:jm}.
The PSNR degradation is computed as the absolute PNSR difference between the
original encoder and the optimized ones.

% P: Dizer pq focamos em luma
For inter macroblock modes in H.264 (i.e. modes related to the ME),
the motion cost for chrominance components derives from the motion cost for 
luminance components~\cite{1101854}. 
Consequently the PSNR for chrominance components derives from the PSNR for 
luminance components. 
For this reason, in this paper we focus on the PSNR variation of the luminance 
component.

Figure \ref{fig:dmec-speedup_workers} show the speedup of DMEC in there
test case application with a different number of \emph{Workers} instances.
For such test, we have used an arbitrary set of pictures with a resolution of
1080p (Full-HD).
The speedup is normalized to one \emph{Worker} instance (speedup of 1X).

% TODO
% \textit{Comments about results in: Cell BE, Muticore IA32, and HW.}
It is worth to mention that for each number of \emph{Worker} instances
a different partition mode was used, according to
Figure \ref{fig:picture_partition}.
For one \emph{Worker} instance we have used the ``Single Partition'',
for two \emph{Worker} instances we have used the ``2x1'' partition and so on,
up to the ``2x3'' partition mode (used for six \emph{Worker} instances).

Besides the additional performance obtained by using a higher number of \emph{Worker} instances,
the partition mode also has influence on the speedup.
The reason of such influence is that, during the partitioning process, the dimensions of the
search window shrinks, thus reducing the area of the picture searched for
similarities.

\fig{.45}{dmec-speedup_workers}{Time performance scalability of DMEC}

% Sobre RD curves
% Figures XXX show the speedup of DMEC while tested already integrated to JM.
% The obtained values are compared to the ones obtained while using the original
% JM, without DMEC.
In order to evaluate in details the behavior of DMEC for distinct values 
of encoding bit-rates, we have used the BD-PSNR (Bjøntegaard Delta PSNR) metric
using the following values of QP (Quantization Parameter): 16,20,24,28; as 
described in \cite{gisle_bjntegaard_calculation_2001}.
It is important to evaluate quality (PSNR) for distinct bit-rates to test 
whether the approach can be used in distinct scenarios of application.
Figure \ref{fig:crowd-bitrate_psnr} shows the rate-distortions (RD) curves using the
original JM encoder and the optimized encoder using DMEC.
The video sequence used for this curves was \texttt{Crowd Run}, a 1080p sequence with  a
high ammount of motion.
Lower values of bit-rate are obtained for higher values of QP since by using
higher values for QP more data is discarded, thus increasing the
compression ratio. 
The two curves very near from each other indicates that the DMEC
presents a good rate-distortion performance for all the evaluated bit-rates.

\fig{.45}{crowd-bitrate_psnr}{RD curve of a 1080p video sequence}

We have evaluated also the speedup obtained in the
% encoding time
ME run time
while using
DMEC for the same QP values we used for BD-PSNR.
Figure \ref{fig:crowd-bitrate_speedup} shows the obtained values while using
6 \emph{Worker} instances.
For Muticore IA32, a speedup of around 9 times is obtained for all
bit-rate values.
For Cell BE this value is about 2 times.
A small speedup for the Cell BE, while compared to Multicore IA32 and the dedicated hardware, is due
to the memory transferences (picture samples and ME results) which is performed using
the DMA requisitions of Cell BE.


\fig{.45}{crowd-bitrate_speedup}{Speedup vs bit-rate of a 1080p sequence}
%
% \multfigtwov{.65}{bd_psnr}{bd_speedup} {bd} {RD curve (a) and speedup vs bit-rate (b) of 1080p sequence}

% Discussion


% Falar do paralelismo / particionamento de dados
% Qualidade ficou boa mesmo particionando e desempenho aumentou: speedup ~ 70%
%The strategy of ME distribution based on picture partitioning has been shown 
%effective.
% We have obtained a speedup higher than XXX\% without loosing quality.
%Data partitioning is effective because the visual interdependence between
%partitions is not significant to influence on the encoding quality, and allows
%for a speedup because it enables the simultaneous processing of each picture
%partition.

% - Falar da comunicação
% - Necessidade via espaço de endereçamentos diferentes
% A arquitetura Cell Broad Band demonstrou-se uma arquitetura interessante para o
% processamento paralelo de vídeo, pois possui unidades funcionais dedicadas 
% (i.e. SPEs) para processamento de dados. 
% A principal dificuldade encontrada em trabalhar-se com o Cell foi a capacidade 
% limitada da memória local das SPEs.
% Outra dificuldade foi lidar com as transferências de memória entre SPE e PPE. 
% Isto em fato foi superado pelas estratégias que desenvolvemos de baferização e 
% também com a utilização do Element Interconect Bus (EIB) do Cell que realiza 
% DMAs com altas taxas de transferências.
% 
% - Solução 1: Buffer de preditores, contribuiu bastante
% A estratégia de armazenamento de preditores nas SPEs foi significativa no 
% aumento do desempenho, pois vetores de movimentos necessários para o cálculo da
% ME não precisam ser consultados na memória principal. É coerente a decisão de 
% manter uma cópia local destes vetores, pois todos os vetores que a ME irá 
% precisar foram calculados pela partição em questão e por nenhuma outra.


% ------------------------------------------------------------------------------


% ------------------------------------------------------------------------------
% ------------------------------------------------------------------------------
\section{Conclusão} \label{concl}

Após realizarmos os testes concluímos a viabilidade da implementação do protocolo PTP para realizar a sincronização dos tempos de relógio em um sistema operacional embarcado, pois conseguimos manter o {\it offset} próximo a 0 segundo. Isso nos fornece uma base para trabalharmos a implementação com o intuito de obter um {\it offset} na faixa de sub-milissegundos. Obtendo essa precisão conseguiríamos garantir a aplicação, por exemplo, da implementação para sistemas de sensoriamento oceânico, como é abordado em \cite{DelRio2012}. Onde é feita uma abordagem sobre a implementação do protocolo PTP para distribuir os tempos de relógio em uma rede Ethernet de Sensoriamento Marinho(MSN). O fato da necessidade de um protocolo deste tipo para tal escopo se dá pelo fato de sinais GPS não estarem disponíveis devido à atenuação da água no fundo do mar e à requisitos de sincronização de instrumentos marítimos, tais como sismógrafos.




% ------------------------------------------------------------------------------

% if have a single appendix:
%\appendix[Proof of the Zonklar Equations]
% or
%\appendix  % for no appendix heading
% do not use \section anymore after \appendix, only \section*
% is possibly needed

% use appendices with more than one appendix
% then use \section to start each appendix
% you must declare a \section before using any
% \subsection or using \label (\appendices by itself
% starts a section numbered zero.)
%


% \appendices
% \section{Proof of the First Zonklar Equation}
% Appendix one text goes here.
% 
% % you can choose not to have a title for an appendix
% % if you want by leaving the argument blank
% \section{}
% Appendix two text goes here.


% use section* for acknowledgement
% \section*{Acknowledgment}
% 
% 
% The authors would like to thank...





% trigger a \newpage just before the given reference
% number - used to balance the columns on the last page
% adjust value as needed - may need to be readjusted if
% the document is modified later
%\IEEEtriggeratref{8}
% The "triggered" command can be changed if desired:
%\IEEEtriggercmd{\enlargethispage{-5in}}

% references section

% can use a bibliography generated by BibTeX as a .bbl file
% BibTeX documentation can be easily obtained at:
% http://www.ctan.org/tex-archive/biblio/bibtex/contrib/doc/
% The IEEEtran BibTeX style support page is at:
% http://www.michaelshell.org/tex/ieeetran/bibtex/
%\bibliographystyle{IEEEtran}
% argument is your BibTeX string definitions and bibliography database(s)
%\bibliography{IEEEabrv,../bib/paper}
%
% <OR> manually copy in the resultant .bbl file
% set second argument of \begin to the number of references
% (used to reserve space for the reference number labels box)
% References
\bibliographystyle{IEEEtran}
\bibliography{hw,fm,os}


\end{document}
