\documentclass[12pt]{article}

\usepackage[brazil]{babel}
\usepackage{sbc-template}
\usepackage{graphicx,url}
\usepackage[utf8]{inputenc}  
\usepackage{listings}
\usepackage{subfigure}
\lstset{keywordstyle=\bfseries, flexiblecolumns=true}
\lstloadlanguages{[ANSI]C,[ANSI]C++,[plain]TeX}
\lstdefinestyle{prg}{
  basicstyle=\small\sffamily,
%  lineskip=-0.2ex,
  showspaces=false
}

\sloppy

% \title{Um estudo de caso de análise da estimativa de tempo de execução em pior
% caso de algoritmos utilizando a ferramenta Bound-T}
\title{Estimativa de tempo de execução no pior caso por análise estática de
código utilizando o \textsc{Bound-T}}

% \author{Arliones Hoeller Jr.\inst{1}, Emílio Wuerges\inst{2}}
% \author{Arliones Hoeller Jr.}
\author{Arliones Hoeller Jr.}%\inst{1} e Antônio Augusto Fröhlich\inst{2}}

\address{Departamento de Automação e Sistemas (DAS)\\
Universidade Federal de Santa Catarina (UFSC)\\
Florianópolis -- SC -- Brazil\\
arliones@das.ufsc.br}%\\
% \nextinstitute
% Laboratório de Integração Software/Hardware (LISHA)\\
% Universidade Federal de Santa Catarina (UFSC)\\
% Florianópolis -- SC -- Brazil\\
% guto@lisha.ufsc.br}
% \nextinstitute
% System Design Automation Lab (LAPS)\\
% Federal University of Santa Catarina (UFSC)\\
% PO Box 476 -- 88.040-900 -- Florianópolis -- SC -- Brazil\\
% arliones@lisha.ufsc.br}

\newcommand{\fig}[4][ht!]{
  \begin{figure}[#1]
    {\centering{\includegraphics[#4]{fig/#2}}\par}
    \caption{#3}
    \label{fig:#2}
  \end{figure}
}

\newcommand{\prg}[4][ht!]{
  \begin{figure}
    \begin{center}
      \makebox[\width]
    {\centering\lstinputlisting[language=#2,style=prg]{fig/#3.prg}\par}
      \caption{#4}\label{prg:#3}
    \end{center}
  \end{figure}
}

\usepackage{multirow}
%\setlength{\tabcolsep}{1mm}
\newcommand{\tab}[4][ht!]{
  \begin{table}
    {\centering\footnotesize\textsf{\input{fig/#2.tab}}\par}
    \caption{#3}\label{tab:#2}
  \end{table}
}

\newcommand{\boundt}{\textsc{Bound-T}}

\begin{document}

\maketitle

\begin{abstract}
This paper presents a study of the \boundt~tool. \boundt~is a tool used to
perform static analysis of code for determining worst-case execution time (WCET)
of programs. This study was performed in the context of a course on real-time
systems within the Automation and Systems Engineering Graduate Program at the
Federal University of Santa Catarina, where students proposed a set of programs
to be analyzed by several WCET tools. The paper briefly describes the
\boundt~tool, evaluates its performance when analyzing a benchmark for real-time
applications, and presents the results obtained when analyzing the proposed
programs. We present the structure of annotations needed to make most of these
analysis possible and identify a few limitations on the tool.
\end{abstract}

\begin{resumo}
Este artigo apresenta um estudo da ferramenta \boundt. O \boundt~é uma
ferramenta de análise estática de código para determinar o tempo de execução no
pior-caso (WCET) de programas. O estudo está contextualizado numa disciplina de
sistemas de tempo-real do Programa de Pós-Graduação em Engenharia de Automação e
Sistemas da Universidade Federal de Santa Catarina onde os alunos propuseram um
conjunto de programas-desafio a serem analisados por diferentes ferramentas de
análise de WCET. Este trabalho descreve a ferramenta \boundt, avalia seu
desempenho frente a um benchmark para aplicações de tempo-real, e apresenta os
resultados obtidos por esta ferramenta ao analisar os desafios propostos. São
apresentadas estruturas de anotações realizadas para viabilizar a análise dos
programas, bem como limitações encontradas neste processo.
\let\thefootnote\relax\footnotetext{Este trabalho foi parcialmente financiado
pela CAPES (Arliones Hoeller Jr., projeto RH-TVD 006/2008).}
\end{resumo}

\section{Introduction}
\label{sec:intro}

Low energy consumption is an important non-functional requirement for the design
of battery-powered embedded systems. Reliable information on the system energy
source is of paramount importance to design an energy-efficient system.
% Nowadays, with cheaper lithium batteries becoming available, such batteries are
% being increasingly used in embedded systems. Lithium batteries present different
% electrical characteristics when compared to ordinary batteries (alkaline),
% including enhanced capacity at smaller sizes and, most important, stabler
% voltage levels.
%Measuring current isn't feasible, so it is usual to keep track of battery voltage
In order to keep track of exact energy consumption at runtime, traditional
approaches rely on continuous measurements of the amount of current drained from
the battery. Besides the additional hardware required to perform this task,
%(a shunt resistor between system and battery connected to an ADC)
software support for sampling such circuitry may compromise system performance
due to the requirement of fine grained information needed to sample this
continuous signal.

\fig{sampled_real_discharge}{Sampled discharge plotted against real discharge
for a CR2 series Panasonic lithium
battery~\cite{Panasonic:Lithium:2006}.}{scale=.333}

To cope with this requirement mobile systems provide means to measure battery
voltage by which it is possible to infer battery charge through an approximate
discharge model for a given battery. This approach, however, brings limitations
to the task of estimating battery charge.
%which adds to the \emph{good} voltage stability of Lithium-based batteries.
Among the reported problems~\cite{Mundra:2008,Penella:2010} there is one of
special interest for the task of precisely estimating battery charge on
embedded systems: the low accuracy and long response time of voltage-based
battery state-of-charge models. This problem is related to the diminished
precision of such voltage measurements implied by low resolution
analog-to-digital converters and to the oscillation on the battery voltage-levels
due to load variations. It can be easily illustrated by
Fig.~\ref{fig:sampled_real_discharge}. The figure shows that during most of
system lifetime the response time in terms of battery voltage varies greatly.
For instance, if an energy-related decision lowers system utilization on day 50,
this decision may last until the next expected voltage drop, around day 100,
making an unique decision stay in effect for $14\%$ of system lifetime. We
can also see that the scheduler misses the opportunity of raising system utilization
because its monitor is not able to rapidly inform that the system is not using
the expected amount of energy. Instead of gradually raising system utilization
and keeping it higher during a longer period of time, it only acts by the end of
a given period, causing bursts in system utilization for short periods of time.

In this paper we propose a software-based energy consumption accounting scheme.
We rely on the fact that energy consumption is a cross-cutting concern,
orthogonal to all system components~\cite{Lohmann:2005}, to enable accurate
monitoring of energy consumption in a component-based operating system for
embedded systems. This scheme monitors the execution of operating system
functionalities. In order to make this approach feasible, we derived three
different profiles for energy accounting which can be applied to devices with
different operational behavior. We evaluate the effect of our approach on system
performance. This paper focus exclusively in the proposed model. An
implementation of the system was also performed for the
\epos~platform~\cite{Project:EPOS:2010} by extending \epos' power management
scheme~\cite{Froehlich:2011}, altought the implementation details where kept out
of this paper due to length limitations.

The remaining of this paper is organized as follows.
Section~\ref{sec:related} presents an overview of related work.
Section~\ref{sec:account} describes the software-based energy accounting
approach proposed in this paper, analyzing it through a didactic case-study.
% Section~\ref{sec:impl_epos} describes the implementation of this approach in the
% \epos~Project.
Section~\ref{sec:case} presents a case-study of the proposed approach on a
wireless sensor network system.
Section~\ref{sec:concl} summarizes this paper and gives some insights on future
work.

% \section{Related work}
\label{sec:related}

% There are two sorts of third-party developments that are related to the present
% work: those related to software-based accounting of energy usage and those
% which address the deployment of multi-objective optimization methods on
% real-time systems for the control of energy consumption. This section gives an
% overview of studied related work.
Related work were divided into those related to software-based accounting of
energy usage and those which address the deployment of meta-heuristic and
multi-objective optimization methods on energy-aware real-time systems.

\subsection{Accounting of Battery Usage}

We investigated similar battery monitoring schemes in operating systems for
wireless sensor networks, including \tinyos~\cite{Polastre:2005},
\mantis~\cite{Bhatti:2005} and \contiki~\cite{Dunkels:2004}. All three systems
provide access to battery voltage measurements through an ADC interface.
% \tinyos~and \contiki, however, have other work either from the original authors
% or from third-parties that are related to the present work.
None of them provide complete voltage models by which it would be possible to
infer battery state-of-charge from voltage levels.
% Also, previous work have already discussed the problems related to the frequency
% in which operating mode migration happens~\cite{Hoeller:DIPES:2006}. These work
% already address questions concerning time overhead and additional energy
% consumption during these migrations~\cite{Seo:2011}.

Yang et al.~\cite{Yang:2007} built an extension to \tinyos~that enables
software-based accounting of energy consumption and battery lifetime estimation
for the \textsc{Mica2} sensor node. They monitor the time each hardware
component stays in an operating mode by intercepting mode changes and
accumulating drawn current during this period, decrementing it from the
initially informed battery charge. Their approach, however, requires
modifications on every monitored system component and may pose unnecessary
overheads in situations where devices change operating mode too often, as it
would be the case of a radio transceiver in a low-power listen mode, where the
transceiver is periodically switched on and off to check for incoming messages.
Weissel and Kellner~\cite{Weissel:2006} used an event-based energy accounting
mechanism to compute energy on the \textsc{BTnode} platform~\cite{Beutel:2004}
running \tinyos. They, however, didn't implement the presented concepts,
limiting their work to the analysis of the implementation possibilities.

Dunkels et al.~\cite{Dunkels:2007} implemented a time-based energy accounting
system for \contiki~running on \textsc{Tmote Sky}~\cite{Polastre:2005} platform.
Their model, as does the approaches used in \tinyos, demand for modification in
several different operating system modules (i.e., drivers), what may make the
system difficult to maintain. They also show that the lack of calibration with
real information in their system may be the cause of significant errors in
energy estimations.

\subsection{Optimization Methods in Energy-Aware Real-Time Systems}

Energy optimization for real-time systems has long been a subject of great
interest in the real-time community~\cite{Weng:2003}. A plurality of
works have been published that apply optimization techniques on real-time system
models to find good tradeoffs between energy consumption and operating
frequency/voltage of CPUs (DVS - Dynamic Voltage Scaling)~\cite{Chen:2007},
on/off status of peripheral devices (DPM - Dynamic Power Management), or
both~\cite{Jha:2001}. All these works, although important to the design of
energy-aware real-time systems, are outside the scope of this paper for they are
orthogonal to the work presented here.

Chantem et al.~\cite{Chantem:2009} proposed a generalized elastic scheduling
framework for real-time tasks based on Buttazzo's elastic
model~\cite{Buttazzo:1998} which may adapt task's elastic periods online based
on one specific (generic) performance metric. Optimal period adjustments are
then performed by a heuristic proposed by them. Although energy-related metrics
may be used as the optimization objective, authors didn't explore this.
Eker et al.~\cite{Eker:2000} and Cervin et al.~\cite{Cervin:2002} show the
application of optimization theory to solve the period selection problem at
runtime by performing adaptive adjustments of periods based on a control
performance metric.
Bini and Natale~\cite{Bini:2005} devised an optimal search algorithm to minimize
tasks' frequencies by performing incremental improvements on one specific
performance metric by using a branch and bound search over a predefined
feasibility region of the domain of task frequencies until the global optimum is
reached. The algorithm applies to fixed-priority scheduling schemes and may be
only applicable offline due to its high complexity.
% Multi-objective optimization have also been applied for solving multiprocessor
% scheduling problems such as the multiprocessor task
% assignment~\cite{Miryani:2009} and the task scheduling problem in heterogeneous
% systems~\cite{Chitra:2010}.

To the best of our knowledge, no work explored the effects of the period at
which battery state-of-charge information is made available for a real-time
energy-aware scheduler.

\section{Análise do programa exemplo}
\label{sec:exemplo}

Inicialmente, analisamos em detalhes um programa disponibilizado como exemplo na
documentação do \boundt~pelo qual é possível observar características e
limitações da ferramenta para o ARM7. O exemplo consiste de dois arquivos
\texttt{C}, \texttt{main.c} e \texttt{routines.c}. Há também três
\textit{headers} (*.h) e um arquivo em linguagem de montagem do ARM7,
\texttt{Startup.s}. O código destes arquivos estão disponíveis no site da
Tidorum~\cite{Tidorum:BoundT_ARM7}~\footnote{Por limitação no número de páginas,
listagens de código-fonte foram removidos deste documento, porém estão
disponíveis através da Internet.}. O programa analisado é composto por diversas
chamadas de função, sendo o grafo destas chamadas o apresentado na
Figura~\ref{fig:cg_main}.

\fig{cg_main}{Grafo de chamadas do programa exemplo.}{width=\columnwidth}

Alguns pontos neste exemplo devem ser destacados por serem importantes no
processo de análise estática do programa:
\begin{itemize}
  \item Na função \texttt{Count} há um laço cujo número de iterações depende do
  valor de um dos parâmetros da função (\texttt{u});
  \item Na função \texttt{Ones} há um laço cujo número de iterações depende da
  posição do bit não nulo mais significante dentro de um dos seus parâmetros
  (\texttt{x}). A função conta o número de bits não nulos em \texttt{x};
  \item Na função \texttt{Solve} há um laço cujo número de iterações é limitado
  por uma constante (\texttt{8}). Contudo, o laço pode ser terminado mais cedo
  por um \texttt{break}.
\end{itemize}

Para compilar e ligar o programa, a Tidorum utilizou o Keil/ARM RealView (armcc)
em ambiente Windows. Como nós utilizamos o Gnu GCC, no Linux, foi necessário
remover o arquivo \texttt{Startup.s}, que estava fora do padrão do nosso
compilador/montador. Para substituí-lo, foi utilizado código de inicialização
disponibilizado pelo próprio GCC.

O \textsc{Bound-T} não conseguiu determinar um limite de tempo para o programa
automaticamente, precisando para isso de anotações sobre o comportamento de
alguns pontos do programa. Estas anotações estão na Figura~\ref{prg:assertions},
e são comentadas abaixo:
\begin{itemize}
  \item Na função \texttt{Ones}, uma anotação é necessária para determinar o
  número de repetições do laço \texttt{while(x)} que repete uma vez para cada
  bit não nulo de \texttt{x}. Como \texttt{x} é um valor de 32-bits, o laço
  repete, no máximo, 32 vezes;
  \item Na função \texttt{Solve}, uma anotação é necessária para determinar o
  número de repetições do laço \texttt{for}. Como \texttt{k} é o número de bits
  '1' em \texttt{*x}, que é um valor de 32 bits, \texttt{k} é no máximo 32. Este
  valor limita o parâmetro \texttt{u} da chamada de \texttt{Count}, e define um
  limite para o laço em \texttt{Count};
  \item Na função \texttt{main} há um laço eterno para bloquear o programa ao
  final. Uma anotação é necessária para evitar que o \textsc{Bound-T} pare a
  análise do programa. Como neste ponto toda a computação desejada já deve ter
  sido feita, uma anotação é incluída para que a análise considere apenas uma
  iteração para este laço.
\end{itemize}

\prg{TeX}{assertions}{Anotações do \boundt~para o programa exemplo.}

O resultado da análise do WCET deste programa é de 12.680 ciclos de máquina.
Também foi possível obter a profundidade máxima da pilha do programa, que foi de
44 bytes. A saída da ferramenta para esta análise é a que foi apresentada na
introdução.


\section{Análise do \textit{benchmark} WCET do MRTC}
\label{sec:benchmark}

O grupo MRTC (Mälardalen Real-Time research Centre), da Universidade de
Mälardalen, Suécia, desenvolveu um benchmark para avaliação de ferramentas de
análise de WCET no contexto do projeto SWEET~\cite{Gustafsson:2010}. Muitos dos
programas deste benchmark foram utilizados no \textit{WCET Challenge 2006}. Os
programas estão disponíveis publicamente~\cite{MRTC:Benchmark}.

Os programas foram compilados utilizando o Gnu GCC conforme a linha abaixo, ou
seja, para o processador ARM7-TDMI, sem otimizações (-O0) e com informação de
depuração (-g):
\begin{verbatim}
   arm-gcc -mcpu=arm7tdmi -O0 -g -o <saída>.elf <entrada>.c
\end{verbatim}

Em seguida foram submetidos à análise pelo \boundt~conforme a linha abaixo, onde
``main'' é a função cujo WCET deseja-se estimar:
\begin{verbatim}
   boundt_arm7 -arm7 <entrada>.elf main
\end{verbatim}

A Tabela~\ref{tab:benchmark} mostra os resultados da análise de WCET para alguns
programas do benchmark. Não foram feitas tentativas de adicionar anotações para
resolver os problemas encontrados nestes benchmarks. O objetivo era avaliar até
onde o \boundt~seria capaz de ir automaticamente. Basicamente, exceto para o
programa ``crc'', em que a ferramenta deixou de funcionar durante a análise, os
problemas encontrados foram a impossibilidade de determinar limites de valores
para variáveis controladoras de laços (\textit{Unbounded}), presença de grafos
de fluxo de execução irredutíveis (\textit{Irreductible Flow Graph}) e a
presença de recursão no código (\textit{Recursion}). Para os programas que o
\boundt~conseguiu realizar a análise é apresentado o valor do WCET em ciclos de
máquina.

\tab{benchmark}{Benchmark WCET da MRTC da Universidade de Mälardalen.}

Ainda nesta tabela, é apresentado um índice de complexidade do programa em
questão. Tal índice foi montado utilizando uma soma ponderada de fatores
considerados complicantes do processo de análise estática de código-fonte de
modo que, quanto maior a complexidade atribuída ao programa, mais difícil é o
processo de análise estática do mesmo. Como exemplo de fatores complicantes de
análise estática pode-se citar a presença de recursão, ponteiros para funções ou
relação com bibliotecas externas ao programa (e.g., \texttt{libc}). Os fatores
empregados são os mesmos apresentados pelo MRTC na descrição dos programas
propostos~\cite{MRTC:Benchmark}.

%\section{Programa proposto para o desafio de WCET}
No contexto em que estes testes foram realizados, a disciplina de Sistemas de
Tempo-Real do Programa de Pós-Graduação em Engenharia de Automação e Sistemas da
Universidade Federal de Santa Catarina, diversos grupos de alunos propuseram
desafios para serem analisados por diversas ferramentas de análise de WCET.
Sendo assim, o problema apresentado como desafio para disciplina e analisado com
sucesso pelo \boundt~é uma execução sequencial dos programas considerados mais
complexos pela análise apresentada na Tabela~\ref{tab:benchmark}, ou seja,
\texttt{qsort-exam}, \texttt{select} e \texttt{fac} do benchmark apresentado
acima.

Para resolver o problema proposto no \boundt~foi necessário adicionar as
anotações apresentadas na Figura~\ref{prg:anotacoes_desafio}. O subprograma
``sort'' apresenta uma série de laços alinhados, que podem ser facilmente
expressos pela linguagem de anotações do \boundt~conforme apresentado na figura.
No caso do ``sort'', anotações foram necessárias para limitar o laço principal
(\texttt{loop that contains (loop that contains loop)}) e os cinco laços
aninhados (\texttt{all 5 loops in (\ldots)}). Anotações similares foram
realizadas para viabilizar a análise do subprograma ``select''.

\prg{TeX}{anotacoes_desafio}{Anotações do \boundt~para o programa desafio
proposto.}

Para possibilitar a análise do subprograma ``fac", contudo, outro tipo de
anotação se fez necessária. O \boundt~não resolve recursões sozinho. Para
resolver as chamadas recursivas à função que calcula o fatorial (\texttt{fac}),
foi necessário quebrar as chamadas em 2, conforme apresentado na
Figura~\ref{prg:fac}, e anotar que estas chamadas não repetem nunca. Assim, o
WCET pode ser calculado manualmente ao final. Os resultados obtidos foram
5.496.330 ciclos para o subprograma \texttt{qsort}, 1.107.269 ciclos para o
subprograma \texttt{select} e 30 ciclos para o subprograma \texttt{fac},
totalizando 6.603.648 ciclos para todo o desafio. O WCET real, contudo, precisa
levar em conta o número de repetições da função recursiva. Para obtê-lo, rodamos
o analisador apenas sobre a função da recursão, considerando o tempo de
\texttt{fac} como zero:

\begin{verbatim}
$  boundt_arm7 -arm7  -assert annotations.txt  desafio facA
Wcet:desafio:desafio.c:facA:220-225:24
\end{verbatim}

Como os subprograma \texttt{fac} calcula 6 valores de fatorial (fatoriais de 0 a
5), o tempo final é dado por $WCET = (6 \times 24) + 6.603.648 = 6.606.528$
ciclos.

\prg{C++}{fac}{Código modificado para permitir a análise de recursão.}


\section{Análise dos desafios propostos}

Esta seção apresenta os resultados obtidos na análise pelo \boundt~dos desafios
propostos pelas outras equipes da disciplina de tempo-real. Cada subseção
seguinte trata de um dos desafios, com uma rápida descrição de cada um, o
arquivo de anotações necessário para a análise e o resultado. A
Tabela~\ref{tab:resultados} sumariza os resultados gerados pelo \boundt~para o
WCET dos desafios propostos. Como pode ser observado, a ferramenta não foi capaz
de avaliar os programas em dois dos desafios\footnote{Devido às limações em
número de páginas, o código-fonte dos desafios aqui apresentados foram
suprimidos e disponibilizados online em:
\texttt{http://www.lisha.ufsc.br/$\sim$arliones/WCET\_Fontes.zip}}.

\tab{resultados}{Resultados gerados pelo \boundt~para os desafios.}


\subsection{Tabela Hash}

Este desafio apresenta algoritmos para a manipulação de uma tabela hash. Uma
anotação foi necessária para amarrar o laço na linha 90 do programa. Como já
constava no comentário existente no código, o limite de iterações do laço está
associado ao tamanho máximo de uma string, que está definido no arquivo
\texttt{main.h} como 512 bytes, logo, o laço repete 512 vezes. 

Outro problema encontrado que necessitou de anotações foi a existência de
chamadas dinâmicas a funções, ou seja, uso de ponteiros para funções. O
\boundt~permite resolver este problema através de uma anotação específica em que
se passam os valores que estes ponteiros podem assumir, ou seja, as funções que
podem ser atribuídas a estes ponteiros. O arquivo de anotações gerado está na
Figura~\ref{prg:renan_assert}.
% A saída gerada pelo \boundt~para esta análise está na
% Figura~\ref{prg:renan_output}. Como pode ser visto, mesmo com anotações para
% ``driblar'' as chamadas dinâmicas, o Bound-T continua emitindo avisos de que
% estas chamadas estão ocorrendo.
O WCET para a função main do desafio ``Tabela Hash'' foi de 54.272 ciclos.

\prg{TeX}{renan_assert}{Anotações para análise de WCET do desafio Tabela Hash.}

%\prg{TeX}{renan_output}{Saída da análise de WCET do desafio Tabela Hash.}

\subsection{Float bit a bit}

Este desafio implementa algumas operações binárias sobre uma variável de ponto
flutuante (\emph{float}) implementadas através de uma função que é invocada
através de um ponteiro. Também são realizadas cópias de memória utilizando a
função \texttt{memcpy} da libc. Uma anotação foi necessária para resolver a
chamada dinâmica a função. Esta anotação mapeia o ponteiro \texttt{pf} a seu
único valor possível: \texttt{fe}. As outras anotações foram necessárias para
resolver os limites dos laços de cópia de memória dentro da função
\texttt{memcpy} da \texttt{libc}. Analisando o código fonte da aplicação, o pior
caso de chamada ao \texttt{memcpy} realiza a cópia de 23 bytes. As anotações da
Figura~\ref{prg:delcino_assert} levam isto em consideração. O resultado da
análise foi um WCET de 2.870 ciclos.

\prg{TeX}{delcino_assert}{Anotações para análise de WCET do desafio float bit a
bit.}

%\prg{TeX}{delcino_output}{Saída da análise de WCET do desafio float bit a
%bit.}


\subsection{Transformada Discreta de Fourier}

A Transformada de Fourrier é largamente utilizada em diversos cenários,
geralmente, para decompor sinais em suas componentes de frequência e amplitude.
A implementação apresentada neste desafio é uma versão da Transformada Discreta
de Fourrier operando sobre variáveis inteiras (ponto fixo).

O desafio proposto faz uso intensivo da biblioteca matemática \texttt{libm} que,
por sua vez, faz uso de uma série de estruturas e funções internas do GCC e da
\texttt{libc}. A análise deste código se mostrou um desafio praticamente
impossível. Vale destacar que, nestes casos, abordages que utilizam técnicas de
medição ao invés de análise estática de código são mais indicadas. Nestes casos
estas dependências pouco influenciam nos resultados de
análise~\cite{Wilhelm:2008}.

Acreditamos que, devido à complexidade dos grafos de fluxo e chamadas de função
gerados para esta aplicação pela interação com a \texttt{libm} o \boundt~não foi
capaz de gerar uma análise confiável. O principal problema encontrado dentro do
código das bibliotecas foi a existência de uma série de chamadas dinâmicas a
funções. Outro erro que nos levou a acreditar que o \boundt~estava operando de
forma inconsistente foi ele ter alegado a existência de uma recursão na função
\texttt{fft} do desafio. Analisando o código, contudo, não fomos capazes de
encontrar esta recursão.
% Os erros apresentados pelo \boundt~na análise desta aplicação estão na
% Figura~\ref{prg:fred_output}.

% \prg{TeX}{fred_output}{Saída da análise de WCET do desafio transformada de
% Fourier discreta.}

\subsection{Codec ADPCM}

Este desafio implementa algoritmos de codificação e decodificação do formato
ADPCM (Adaptive Differential Pulse-Code Modulation). Aparentemente, um bug no
\boundt, a análise da função main deste programa falhou. Descobrimos que, se
incluirmos limites por anotações às funções \texttt{\_\_aeabi\_ui2f} e
\texttt{\_\_aeabi\_i2f}, o bug não ocorria. Logo, analisamos os WCET destas
funções em separado, obtendo 42 ciclos como WCET da \texttt{\_\_aeabi\_ui2f} e
40 ciclos como WCET da \texttt{\_\_aeabi\_i2f}, podendo assim incluir estes WCET
como anotações (Figura~\ref{prg:juliano_assert}) e proceder a análise e
resolução dos demais laços cujos limites não puderam ser definidos.

A partir daí, vários laços cujos limites não puderam ser definidos surgiram para
as funções matemáticas \texttt{\_\_divsf3} e \texttt{\_\_mulsf3}. A
implementação destas funções, internas ao compilador GCC para ARM, estão em
linguagem de montagem no arquivo \texttt{ieee754-sf.S}, cuja análise se mostrou
uma tarefa bastante complexa. Sendo assim, visando completar a análise da
aplicação, consideramos tempo de execução zero para estas funções. Para obter
resultados mais precisos, contudo, uma análise mais detalhada destas funções em
linguagem de montagem é necessária. Como consequência, a análise de tamanho de
pilha da ferramenta deixa de funcionar adequadamente.

Os laços nas funções \texttt{factorial}, \texttt{power} e \texttt{cosine}
tiveram seus limites máximos (de pior caso) definidos baseados no valor do
iterador \texttt{i} utilizado na função \texttt{cosine}. Haviam também chamadas
dinâmicas a funções através de ponteiros para funções. Contudo, estes ponteiros
são inicializados no início da função \texttt{main}, logo, têm um único
comportamento, facilmente definido por anotações. As anotações utilizadas estão
na Figura~\ref{prg:juliano_assert}.

% \prg{TeX}{juliano_output2}{Saída da análise de WCET das funções matemáticas do
% desafio Codec ADPCM.}

Mesmo acreditando que todas as anotações estejam corretas, não foi possível
estimar o WCET desta aplicação porque a ferramenta \boundt~não finalizou a
análise. Todas as demais análises terminaram em menos de 5 segundos de execução.
A análise desta aplicação rodou por, aproximadamente, 1 hora até consumir toda a
memória disponível na máquina (4 GB). Provavelmente isto deve estar relacionado
a um bug da ferramenta.

% \prg{TeX}{juliano_output1}{Saída com erros da análise de WCET do desafio Codec
% ADPCM.}

\prg{TeX}{juliano_assert}{Anotações para análise de WCET do desafio Codec
ADPCM.}

% \prg{TeX}{juliano_output3}{Saída da análise de WCET do desafio Codec ADPCM
% com anotações.}

% 
% \subsection{Programa cnt.c do Benchmark do MRTC}
% 
% Este programa popula e soma uma matriz 10x10 de números inteiros. Novamente,
% alguns laços cujos limites não puderam ser definidos surgiram para funções
% matemáticas, sendo desta vez para a função \texttt{\_\_aeabi\_ddiv}. A
% implementação desta função também está no arquivo \texttt{ieee754-sf.S}, cuja
% análise em linguagem de montagem se mostrou uma tarefa bastante complexa.
% Sendo assim, visando completar a análise da aplicação, consideramos tempo de
% execução zero para esta função (Figura~\ref{prg:anton_assert}). Assim como na
% aplicação anterior, estas anotações tiveram como consequência o
% comprometimento da análise de pilha. O WCET deste desafio foi de 12.881
% ciclos.
% 
% \prg{TeX}{anton_assert}{Anotações para análise de WCET do desafio cnt.c.}
% 
% %\prg{TeX}{anton_output}{Saída com erros da análise de WCET do desafio cnt.c.}

%\input{files/4_epos.tex}
\section{Conclusão}

Este trabalho apresentou as principais características e funcionalidades da
ferramenta de análise de tempo de execução de pior caso (WCET) \boundt, da
Tidorum Ltd. Uma análise geral da ferramenta foi realizada seguida da
apresentação de um exemplo disponível junto à documentação da ferramenta.

No contexto da disciplina DAS9007 - Sistemas de Tempo Real do Programa de
Pós-Graduação em Engenharia de Automação e Sistemas (PGEAS) da UFSC, um desafio
foi proposto em que alunos, divididos em grupos, deveriam estudar ferramentas de
estimativa de WCET e propôr aplicações-desafio aos demais grupos. Cada grupo,
então, deveria analisar as aplicações propostas com as ferramentas que
estudaram.

Neste trabalho, foi posposta uma aplicação baseada em alguns dos programas
utilizados pelo benchmark de WCET do MRTC, da Universidade de Mälardalen,
Suécia. A análise dos demais desafios propostos também foi realizada utilizando
o Bound-T, o que revelou uma limitação da ferramenta no que diz respeito a
integração com código mais complexo presente em bibliotecas do sistema (e.g.,
$libm$ e $libc$). A ferramenta também apresentou alguns bugs. Alguns destes bugs
foram, de algum modo, contornados. Outros problemas, contudo, inviabilizaram o
uso da ferramenta para a análise de dois dos desafios propostos (Transformada
Discreta de Fourrier e Codec ADPCM).

Como resultado do estudo, percebemos que as ferramentas disponíveis atualmente,
de modo geral, não apresentam um nível de maturidade no qual sua aplicação a
qualquer programa seja possível. Mesmo nas ferramentas comerciais mais
utilizadas (categoria na qual o \boundt~se insere), uma série de anotações
precisam ser feitas para viabilizar a análise. Embora muitas destas limitações
existam devido a questões teóricas relacionadas a problemas não computáveis
(e.g., \textit{The Halting Problem}), muitas limitações devem-se a bugs ou
incompletudes. Em alguns casos, chegar a limites razoáveis para estas anotações
pode ser uma tarefa bastante difícil, o que leva, frequentemente, ao uso de
margens seguras, fazendo da estimativa de WCET um valor muito além do valor
real, tornando sistemas de tempo-real superdimensionados.

Como trabalho futuro, pretende-se utilizar o \boundt~para permitir a análise de
WCET de aplicações utilizando o sistema operacional
\textsc{Epos}~\cite{Marcondes:WSO:2009}. Análises iniciais já apresentaram
diversos desafios nesta tarefa, como na interação entre o sistema operacional e
periféricos e na implementação de funções clássicas de sistemas operacionais
como sincronizadores com espera ociosa e troca de contexto. Espera-se, através
deste estudo, viabilizar uma estrutura de anotações que torne viável a análise
do WCET do código do sistema operacional \textsc{Epos}, viabilizando assim a
análise de suas aplicações.

%
\newpage
\appendix
\section{main.c}
\begin{verbatim}
#include "types.h"
#include "routines.h"

int main (void)
{
   uint x = 1;

   Count25 (&x);
   /* This can be bounded automatically. */

   Foo7 (&x);
   /* This can be bounded automatically because Foo7 calls */
   /* Count with a static value for Count.u.               */

   Foo (6, &x);
   /* This can be bounded automatically, because Foo.num = 6,  */
   /* which makes Count.u = 9, which bounds the loop in Count. */

   Solve (&x);
   /* The loop in Solve can be bounded automatically, but not */
   /* the loop in Count when called from Solve.               */

   while (1) x++;
   /* This is a simple eternal loop. Bound-T will report it */
   /* and include one execution in the WCET.                */
}
\end{verbatim}

\newpage
\section{routines.c}
\begin{verbatim}
#include "types.h"
#include "routines.h"

void Count25 (uint *x)
{
   count_t u = 25;

   for (; u > 0; u -= 2)
   {
      *x = *x + u;
   }
}

void Count (count_t u, uint *x)
{
   for (; u > 0; u -= 2)
   {
      *x = *x + u;
   }
}

void Foo (count_t num, uint *x)
{
   Count (num + 3, x);
   /* The loop in Count depends on Count.u = num + 3. */
}

void Foo7 (uint *x)
{
   *x = *x + 10;

   Count (7, x);
   /* The loop in Count depends on Count.u = 7. */

   *x = *x - 8;
}

void Solve (uint *x)
{
   count_t i;
   volatile count_t k;

   for (i = 0; i < 8; i++)
   /* The bounds on this loop are static. */
   {
      k = Ones (*x);
      /* k is now the number of '1' bits in *x.          */
      /* This would be quite hard to analyse statically. */

      if (k == 0) break;
      /* This can make the for-loop stop before its full    */
      /* number of iterations. Bound-T uses the full number */
      /* for the Worst Case Time.                           */

      Count (k, x);
      /* The loop in Count depends on Count.u = k, which is  */
      /* hard to analyse statically. An assertion is needed. */
   }
}

count_t Ones (uint x)
{
   count_t v = 0;

   while (x)
   /* This is not a 'counter' loop, so Bound-T cannot find */
   /* its bounds automatically. An assertion is needed.    */
   {
      if (x & 1) v ++;
      x >>= 1;
   }

   return v;
}
\end{verbatim}

\newpage
\section{desafio.c}
\label{anx:desafio}
\begin{verbatim}
#define SWAP(a,b) temp=(a);(a)=(b);(b)=temp;
#define M 7
#define NSTACK 50

void
sort(unsigned long n);

float           arr_sort[20] = {
    5, 4, 10.3, 1.1, 5.7, 100, 231, 111, 49.5, 99,
    10, 150, 222.22, 101, 77, 44, 35, 20.54, 99.99, 88.88
};

int             istack[100];

void
sort(unsigned long n)
{
    unsigned long   i, ir = n, j, k, l = 1;
    int             jstack = 0;
    int             flag;
    float           a, temp;

    flag = 0;
    while (1) {
        if (ir - l < M) {
            for (j = l + 1; j <= ir; j++) {
                a = arr_sort[j];
                for (i = j - 1; i >= l; i--) {
                    if (arr_sort[i] <= a)
                        break;
                    arr_sort[i + 1] = arr_sort[i];
                }
                arr_sort[i + 1] = a;
            }
            if (jstack == 0)
                break;
            ir = istack[jstack--];
            l = istack[jstack--];
        } else {
            k = (l + ir) >> 1;
            SWAP(arr_sort[k], arr_sort[l + 1])
                if (arr_sort[l] > arr_sort[ir]) {
                SWAP(arr_sort[l], arr_sort[ir])
            }
            if (arr_sort[l + 1] > arr_sort[ir]) {
                SWAP(arr_sort[l + 1], arr_sort[ir])
            }
            if (arr_sort[l] > arr_sort[l + 1]) {
                SWAP(arr_sort[l], arr_sort[l + 1])
            }
            i = l + 1;
            j = ir;
            a = arr_sort[l + 1];
            for (;;) {
                i++;
                while (arr_sort[i] < a)
                    i++;
                j--;
                while (arr_sort[j] > a)
                    j--;
                if (j < i)
                    break;
                SWAP(arr_sort[i], arr_sort[j]);
            }
            arr_sort[l + 1] = arr_sort[j];
            arr_sort[j] = a;
            jstack += 2;

            if (ir - i + 1 >= j - l) {
                istack[jstack] = ir;
                istack[jstack - 1] = i;
                ir = j - 1;
            } else {
                istack[jstack] = j - 1;
                istack[jstack - 1] = l;
                l = i;
            }
        }
    }
}

int
qsort_exam_main(void)
{
    sort(19);
    return 0;
}

float           select(unsigned long k, unsigned long n);

#define SWAP(a,b) temp=(a);(a)=(b);(b)=temp;

float           arr[20] = {
    5, 4, 10.3, 1.1, 5.7, 100, 231, 111, 49.5, 99,
    10, 150, 222.22, 101, 77, 44, 35, 20.54, 99.99, 888.88
};


float
select(unsigned long k, unsigned long n)
{
    unsigned long   i, ir, j, l, mid;
    float           a, temp;
    int             flag, flag2;

    l = 1;
    ir = n;
    flag = flag2 = 0;
    while (!flag) {
        if (ir <= l + 1) {
            if (ir == l + 1)
                if (arr[ir] < arr[l]) {
                    SWAP(arr[l], arr[ir])
                }
            flag = 1;
        } else if (!flag) {
            mid = (l + ir) >> 1;
            SWAP(arr[mid], arr[l + 1])
                if (arr[l + 1] > arr[ir]) {
                SWAP(arr[l + 1], arr[ir])
            }
            if (arr[l] > arr[ir]) {
                SWAP(arr[l], arr[ir])
            }
            if (arr[l + 1] > arr[l]) {
                SWAP(arr[l + 1], arr[l])
            }
            i = l + 1;
            j = ir;
            a = arr[l];
            while (!flag2) {
                i++;
                while (arr[i] < a)
                    i++;
                j--;
                while (arr[j] > a)
                    j--;
                if (j < i)
                    flag2 = 1;
                if (!flag2)
                    SWAP(arr[i], arr[j]);

            }
            arr[l] = arr[j];
            arr[j] = a;
            if (j >= k)
                ir = j - 1;
            if (j <= k)
                l = i;
        }
    }
    return arr[k];
}

int
select_main(void)
{
    select(10, 20);

    return 0;
}

int             fac(int n);

int
facA(int n)
{
        if (n == 0)
                return 1;
        else
                return (n * fac(n - 1));
}

int
fac(int n)
{
        return facA(n);
}


int
fac_main(void)
{
        int             i;
        int             s = 0;

        for (i = 0; i <= 5; i++)
                s += facA(i);

        return (s);
}

int main() {
  qsort_exam_main();
  select_main();
  fac_main();
}
\end{verbatim}



\section*{Agradecimentos}
Os autores agradecem aos professores da disciplina de sistemas de tempo-real do
Programa de Pós-Graduação em Engenharia de Automação e Sistemas da Universidade
Federal de Santa Catarina, Rômulo Silva de Oliveira e Carlos Barros Montez, pela
orientação e valiosas discussões em sala de aula. Também agradecem ao colega
Emílio Wuerges e aos demais alunos que cursaram a disciplina em 2010 pelo apoio
na realização do trabalho e pelos programas-desafio utilizados nos experimentos.
Agradecemos também à Tidorum Ltd. pela licença acadêmica do \boundt.

\bibliographystyle{sbc}
\bibliography{paper}

\end{document}
