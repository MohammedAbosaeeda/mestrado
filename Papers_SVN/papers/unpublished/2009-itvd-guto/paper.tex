\begin{filecontents*}{example.eps}
%!PS-Adobe-3.0 EPSF-3.0
%%BoundingBox: 19 19 221 221
%%CreationDate: Mon Sep 29 1997
%%Creator: programmed by hand (JK)
%%EndComments
gsave
newpath
  20 20 moveto
  20 220 lineto
  220 220 lineto
  220 20 lineto
closepath
2 setlinewidth
gsave
  .4 setgray fill
grestore
stroke
grestore
\end{filecontents*}

\documentclass{svjour3}
%\documentclass[twocolumn]{article}
\usepackage[latin1]{inputenc}
\usepackage{graphicx}

\newcommand{\fig}[4][ht]{
  \begin{figure}[#1] {\centering\scalebox{#2}{\includegraphics{fig/#3}}\par}
    \caption{#4\label{fig:#3}}
  \end{figure}
}

%\pagestyle{empty}

\begin{document}

\title{Interactive Digital Television for Developing Countries: a
  software/hardware perspective}

\author{Ant�nio Augusto Fr�hlich \and Valter Roesler}

\institute{Ant�nio Augusto Fr�hlich \at
  Federal University of Santa Catarina\\
  Software/Hardware Integration Lab\\
  88040-900 Florian�polis - SC - Brazil\\
  E-mail: \texttt{guto@lisha.ufsc.br}\\
  \texttt{http://www.lisha.ufsc.br/$\sim$guto}
  \and
  Valter Roesler \at
  Federal University of Rio Grande do Sul\\
  Informatics Institute\\
  91501-970 Porto Alegre - RS - Brazil\\
  E-mail: \texttt{roesler@inf.ufrgs.br}\\
  \texttt{http://www.inf.ufrgs.br/$\sim$roesler}
}

\date{Received: date / Accepted: date}

\maketitle

\begin{abstract}

  \emph{Interactive Digital Television} is mostly a matter of
  entertainment for developed countries, but bears great potential to
  narrow the digital divide in developing countries. The television's
  return channel might become the family's gate to the Information
  Society. For this to come true, however, digital receivers and
  return channel technologies must reach cost levels that enable mass
  penetration. In this essay, we discuss design decisions around the
  receiver that can drastically impact the success of interactive
  digital television in the developing world.

  \keywords{Interactive Digital Television \and Multimedia \and Middleware \and
    Set-top box}

\end{abstract}


\section{Introduction}

% the title
When an author assumes the risk of entitling a paper with provocative
terms, he has no better way to begin that paper than by explaining the
provocation. After all, most readers might think that there is no
significant difference between interactive digital television in
developing and in developed countries---specially as regards the
software/hardware design of associated gadgets. Furthermore, even if
there were significant differences, it is a fact the most of the
previous attempts to promote digital-inclusion by means of extremely
low-cost gadgets ended overcame by mass production of ordinary devices.
Nevertheless, before going further with this reasoning, it is perhaps
important to clearly define the context in which the term ``interactive
digital television'' will be used in this paper.

% iDTV
Getting involved in more than one controversy in a single paper would
not be wise, so no attempt to define ``interactive digital television''
will be made here. However, two facts about the technology surrounding
iDTV seems to be evident:

\begin{itemize}
\item Interactive digital television, despite being digital and
  interactive, remains \emph{television}. Users are now very well
  familiarized with their TV sets and any technological enhancement must
  preserve that relationship. Rebooting a TV set while watching a
  program, for instance, is an unthinkable situation.
  
\item It will take some more time until television experts devise the
  possibilities embodied in the new technology and we get a broader
  notion of how ``interaction'' will take place. In the near horizon,
  interactive digital television seems to point out to a merger of
  standard television plus that what we now call the \emph{Web}.
\end{itemize}

Indeed, there is no need to go deeper into the discussion about what
interactive digital television is, or will be, in order to make the
point about the very distinctive cases for developed and developing
countries, if we accept the idea that, for now, interactive digital
television is being built on standard television content plus services
derived from those currently available on the \emph{Web}:

% iDTV in developed countries
In \textbf{developed} countries, the microcomputer/Internet revolution
took place mostly in the 90s. Nowadays, much of the population of those
countries has plain access to the \emph{Web} at home, workplace and
public places (e.g.  schools, libraries, etc). Because it was born
earlier, the \emph{Information Society} in developed countries evolved
relatively independent from the digital television scenario. Indeed,
even personal communication and Internet access developed as two
distinct industries that only now are showing signals of true
convergence. In this scenario of independent media (i.e. TV) and
communication (i.e. \emph{Web}) services, developed countries underwent
the digital television race looking for high added-value services, such
as high definition video and audio, intelligent video recorders, and a
large set of portal-like applications. This makes sense, since this
industry targets customers who have already acquired, at least, a TV
set, a microcomputer and a broadband access to the
Internet~\cite{Duncan:2006}.  Aggregating value to the
television is a logical way to attain mass marketing conditions.

% iDTV in developing coutries - motivation
On the other hand, the reality of \textbf{developing} countries is
quite different: television is mostly ubiquitous, but the population
of these countries still have limited access to the
Internet~\cite{Crampton:2006,Monte:2006}. Indeed, thinking in terms of
iDTV, the \emph{return channel} is by itself a big challenge for
countries with deficient telephony infrastructure. In this regard,
investments are being made in power line, satellite, and cell
communication, which might provide answers to localized questions.

% % a bit of reallity
% The problem in relation to the return channel is that it is a service normally paid, reducing the quantity of people who can afford it. However, the number of people in developing countries who have pre-paid cell phones is increasing rapidly. One way to reach most of the population would be to sell "pre-paid interactivity TV cards", which would give rights to \emph{n} interactivities through TV, without catching the televiewer in a monthly bill.

% According to United
% Nations\footnote{http://esa.un.org/un-energy/pdf/UN-ENRG_paper.pdf 
%   2005.}, about 1.6 billion people around the world don't have even
% access to electrical energy. In Brazil the numbers are more comfortable,
% but there is about 10 to
% 15% of the population in this
  % situation\footnote{http://www.cgu.unicamp.br/energia2020/papers/paper_Poppe.pdf.},
  % totalizing 4 to 5 million homes or 20 million
  % people. Controversially, the cellular coverage is about 90% of the
  % brazilian area\footnote{www.teleco.com.br.}, leading to a conclusion
  % that, even without electrical energy, some people can talk via cell
  % phone. That happens, for example, in regions like Manaus, in Amazon,
  % where people have cell reachability, but don't have electrical
  % energy. Paradoxally, they must run a bicycle or a crank to have
  % electrical energy to watch TV, but the return channel is granted by
  % the cell infrastructure.

Nonetheless, supposing that a return channel infrastructure is on the
way for developing countries, iDTV becomes the most concrete
digital-inclusion mechanism available at short-term. A low-cost,
web-enabled set-top box could be the pivot of a plan to catapult
developing countries directly into the digital society of today,
bypassing the microcomputer+modem lap. Such a gadget, besides acting as
the traditional TV gateway, could be the gateway to e-government,
e-learning, e-health, and other e-services to come.

% some examples for social inclusion
% It is important to understand that interactive digital TV is a valuable
% tool for social inclusion, and some of the benefits are:

% \begin{itemize}
% \item Receiving of the retirement comprovant every month.

% \item Make an appointment to medical assistance.

% \item Use of text editor and spreadsheet, for example, plugging a printer in the set-top box and allowing the televiewer to use tools \emph{office like}, aiming social inclusion through digital inclusion.
% \end{itemize}

% iDTV in developing coutries - limitations
Of course, for these vague ideas to become a concrete plan, many other
variables must be considered besides the return channel. For instance,
there are concerns that the low-resolution TV sets now in use will not
be able to display the \emph{Web} as we know today\footnote{Standard TV
  sets support resolutions of about 500 pixels, while the \emph{Web} is
  now at roughly double that resolution.}.  The most critical variable,
however, seems to be bound to the cost of the set-top boxes that will
have to be installed in millions of houses.  And this is exactly the
point in which iDTV strategies for developed and developing countries
differ the most: while developed countries are envisioning complex (and
expensive) services to be the promoters of the new media, developing
countries must seek for solutions that are extremely affordable.

% no heterodox solutions
Based on previous experiments, it is also reasonable to suppose that any
heterodox proposal, like closed standards, incompatible content formats,
specific marketing conditions, will not help in building an universal
information society---at most they will create digital islands. In this
context, this paper brings a reasoning about two technical approaches
that could fulfil the requirements of iDTV for both developed and
developing countries.

The reminder of this paper presents a reasoning about an API for iDTV,
discusses the currently available technologies and finally presents the
authors perspectives about the discussed matters.


\section{An API for iDTV}

As stated before, interactive digital television is yet in its infancy
and a comprehensive characterization of its applications will not be
available soon. By the time iDTV applications become commonplace, much
of the hardware and software limitations of today will probably be
overcame, so they should not be allowed to dictate any major guideline
of application development. The significant points in delineating an
Application Program Interface~(API) for iDTV are what we now know about
such applications and also the concern not to prevent foreseeable
innovations. Some of the facts about iDTV applications we can list today
are:

\begin{itemize}
\item They will be sent to set-top boxes along with TV programs,
  multiplexed in video and audio streams, but somehow confined as not to
  allow applications to impact the stability of ordinary television
  (watching will still be the priority for most users);
\item They will be executed in set-top boxes of different architectures
  and functionalities;
\item Applications, at least initially, will have the traditional remote
  control as the main human interaction interface. Otherwise, typical
  devices of ordinary PCs, like mouse and keyboard, will be used;
\item Some applications will require data to be sent back through a
  return channel in a way very similar to the \emph{Web} applications of
  today;
\item Some will explore concepts such as Personal Video Recorders~(PVR)
  and will require large and fast storage;
\item A few will challenge the establishment, possibly revolutionising
  the way we watch TV today.
\end{itemize}

Although limited, this small set of remarks about applications gives
rise to some important issues for the design of a tentative API for
iDTV.  Extensibility, for instance, comes forth as one of the main
requisites, so that the operating system (or middleware) implementing
the API will be able to evolve by improving its services and also by
adding new services. The interface of such a \emph{run-time support
  system} for interactive digital television could be seen as a kind of
\emph{Interactive Digital Television Language}~(iDTV-L), and thus a
first consideration about its design could be whether that language
should be a \emph{imperative} language or a \emph{declarative}
language\footnote{It is not a goal of this paper to approach the rich
  taxonomic discussion about formal languages. The term
  \emph{declarative language} is being used here to designate
  \emph{markup} languages such as \texttt{HTML} e \texttt{XML}, while
  \emph{imperative languages} designate programming languages like C++
  and \textsc{Java}.}.

Having our iDTV-L to be a \textbf{imperative language}, such as those we
currently use to develop software, would allow iDTV application
developers to ``program'' the set-top box in order to execute tasks it
was not initially devised for. In this way, application developers could
simply ignore the high-level services provided by the API and implement
their own services, without any aid from the original set-top box
manufacturer.  Not even an operating system upgrade would be necessary.
Nonetheless, a imperative language of this nature would require a
complex interpreter/compiler to be part of the native set of services
provided by the API. This might be a major disadvantage as it probably
prevents the implementation of set-top boxes based on the premises of
embedded systems, guiding them towards the realm of ordinary personal
computers. The impact in cost such a design decision would have would
probably hinder the use of the API in developing countries.

Alternatively, our iDTV-L could be a \textbf{declarative language} that
would allow us to specify ``what'' is to be done by the STB without
being concerned about ``how'' it will take place.  The big advantage of
a language of this nature comes from high-level elements such as:
\texttt{TV Program} (to be handled by the receiving-decoding-rendering
subsystem); \texttt{URL} (to be handled by a web-browser);
\texttt{Program Tag} (to be handled by a PVR for recording); etc. A
linguistic set of elements of this kind would probably allow for a
slimmer API implementation, since each OS could implement it considering
the particularities of the set-top box architecture.  Families of
set-top boxes could implement specific subsets of the API, ranging from
simple DTV receivers to full-fledged iDTV devices. However, any eventual
modification or inclusion of services would require a system upgrade and
would have to be carried out in agreement with manufacturers.

In order to differentiate these two alternatives, perhaps it is
necessary to go one step deeper in our decision tree. On the
\textbf{imperative language} branch of the tree, we will soon find a
decision of whether the language should be compiled to each STB
architecture or whether it should be interpreted, thus bringing about a
virtual machine. Translating our iDTV-L to \textbf{real machine} code
would show a series of advantages, specially as regards resource
utilization, and consequently would foster the design of receivers in
the realm of embedded systems. Nonetheless, this decision would also
imply in TV programs being recompiled for each STB architecture.
Considering the diversity of architectures commonly deployed in embedded
systems, this branch is likely to be ruled-out. On the other hand,
translating iDTV-L to \textbf{virtual machine} code would eliminate this
problem, but not without bringing about additional costs. A virtual
machine demands bringing a good fraction of the compilation system to
run-time.

Thinking about a imperative language for a virtual machine without
remembering \textsc{Java} is virtually impossible nowadays. But could
\textsc{Java} itself be our iDTV-L? Could it be the API on which the
future generation of TV programs will be developed? The fact is that,
like \textsc{Java} or not, few people would give us credit if we started
a discussion about a ``new'' imperative language at the level of
elementary computations for a virtual machine. On the other hand, a
higher-level imperative language would suffer from the same deficiencies
of the declarative language approach: modifications or additions would
require coordinated upgrades.  Therefore, the decision tree about the
nature of iDTV-L we have been building meets a crucial choice: the
imperative language \textsc{Java} or a novel declarative language.

% The advantage of using Java over other imperative languages is because
% Java is a type safe language [hoskin:java], where it is not possible
% to freely cast integers to be pointers or vice-versa, confining the
% application to its own address space. Another Java robustness is that
% there's no "free" operation to release memory, eliminating some
% programming problems encountered in other languages. Java uses a
% "garbage collector" to free memory.

Apropos, \textsc{Java} is already deeply entangled in the development of
some current DTV commercial initiatives, such as MHP~\cite{MHP} and
DASE~\cite{DASE}. Indeed, this might be the appropriate choice for
developed countries, because, while accepting the higher costs
associated to the run-time support system of \textsc{Java}, they also
bring about all the advantages of adopting a widespread imperative
language, including easy access to skilled human resources. Adopting
\textsc{Java} as it is today for the iDTV-L, however, could put the
costs of iDTV at prohibitive levels for developing countries. As an
example, consider a video-on-demand~(VoD) application written in
\textsc{Java} and executed in a standard PC: besides the application
itself, we will find a huge operating system kernel, a virtual machine
(JVM), a set of run-time support libraries (\emph{classpath}) and a
media framework (e.g. JMF).  The overhead resulting from this
architecture is so high that only high-end PCs are able to
satisfactorily execute the application.  Non-functional requirements,
such as response time and quality of service can hardly be met in such a
scenario.

Going back to the context of declarative languages, the same VoD
application could be easily described by a \emph{complex element} that
would include information about servers, content and protocols. The
description would subsequently be parsed by a sort of server that would
invoke native OS services accordingly. This approach would certainly
incur in far less overhead than the previous \textsc{Java} stack and
perhaps could be implemented as a low-cost embedded system. It is
important to remember, however, that this approach would constrain iDTV
programs to that what is made available through the API. Any extension
would imply in upgrading the OS on the STB, thus binding content
developers with equipment manufacturers in a delicate matter.

A declarative language could handle this limitation by offering lower
level elements, such as \texttt{TCP Connection}, \texttt{File},
\texttt{Graphic Window}, and others. This would considerably increase
the complexity of the language interpretation engine, but this, when
compared to \textsc{Java}, would probably not be enough to exclude the
alternative\footnote{In a \textsc{Java} run-time environment, the
  interpreter (i.e.  bytecode parser) is not one of the most significant
  causes of overhead~\cite{Rayside:1999}.}. More important, perhaps, is
the impact such an approach would have on security. A low-level API
would give access to basic OS services, including storage management and
communication via the return-channel. If not addressed consistently,
this could open the way to a whole new era of cybernetic crimes,
bringing the actual chaos of personal computing into the realm of
television: viruses, interruptions, lost of stored programs, privacy
corruption, etc. One could argue that \textsc{Java}, for being a
low-level imperative language (if compared to our high-level declarative
language, not to C or assembly) is exposed to the same problems. This
might be true in theory, but in practice, \textsc{Java} security issues
have been consequently explored by Web application developers in the
last decade, so that it would not be fair to suppose that both
strategies would be exposed to the same level of security flaws.

\fig{.8}{tree}{Decision tree for an hypothetical iDTV API.}

At this point, the decision tree about an API for iDTV has only two
strong branches: \textsc{Java} and a novel declarative language with
high-level elements. \textsc{Java} is a well-know imperative language,
what makes additional explanations unnecessary, but what about this
hypothetical declarative language? What would it look like? Well,
considering the Web of today, we could think about this language as a
collection of XML DTDs and Schemas. An iDTV program would thus be
described by an element whose attributes would include, for instance,
available audio and subtitle languages. The OS would parse that tag and
react to it invoking services to notify the user about the multilingual
features of the program. Other tags could be multiplexed along with the
program's audio and video streams, for instance, to indicate that there
is a chat room for the program, to supply program-related URLs, to setup
a poll, etc.  Whenever such a high-level tag is parsed by the OS, one or
more native services are invoked.

In this scenario, HTML would certainly be one of the constituent DTDs of
our iDTV-L. Furthermore, interaction events, like a mouse click, could
be handled by script languages embedded into the iDTV-L program in a way
similar to \textsc{Java-Script} embedded in HTML. On the broadcaster
side, the analogy would take PHP instead. It is important to notice that
this level of Web accessibility is now available in a variety of complex
embedded systems, such as cell phones and PDAs. A consistent
re-engineering of those systems is likely to succeed in bridging the gap
to low-cost iDTV terminals.


\section{Currently Existing Approaches}

The first approaches to realize an DTV API resulted in proprietary
middlewares such as \textsc{OpenTV}, \textsc{NDS}, \textsc{Canal}+,
\textsc{PowerTV} and \textsc{MicrosoftTV} that shaped a vertical market,
in which the televiewer depended on the same company to provide the
service and the set-top box. Recently, open standards started to attract
the attention of the industry and are now reshaping the market towards a
more flexible scenery. This evolution is represented in the time-line in
figure~\ref{fig:openstd}.

\fig{.5}{openstd}{Evolution of DTV middleware standards.}

Since the publication of MHEG-5 standard in 1997, many other open
standards followed, like MHEG-6 and Digital Audio and Video
Council~(DAVIC) in 1998 and JavaTV in 1999. These last two plus Havi
formed the basis of DVB Multimedia Home Platform~(MHP)~\cite{MHP}
released in 2000.  After that, OpenCable Application Platform~(OCAP) was
developed based on MHP and, in 2001, all these standards started to
converge toward Globally Executable MHP~(GEM)~\cite{GEM}, which was also
standardized as J.200 by ITU-T in 2003~\cite{ITU-T:J.202}.

The first open standards, like for instance MHEG-5, followed the
declarative approach. Today, however, iDTV middlewares tend to combine
both approaches, including support for declarative and imperative
subsets.  HMP 1.1, for instance, supports HTML in its Internet Access
profile, which is known as DVB-HTML.

Another middleware that supports both approaches is the brazilian
Ginga~\cite{GINGA}, which is divided in \textsc{Ginga-J} (imperative
approach), based on \textsc{Java}, and \textsc{Ginga-NCL} (declarative
approach), based on the Nested Context Language~(NCL)~\cite{NCL}.

The official Chinese middleware for digital TV is designed around five
layers~\cite{Guang:2004}, including a system layer with three blocks:
API packages (creates an abstraction layer to the applications),
\textsc{Java-VM} (run imperative applications) and web engine (run
declarative applications).

Pablo Cesar proposes an extension to the GEM standard in order to
support a declarative approach, namely XHTML, XForms, CSS, Synchronized
Multimedia Integration Language~(SMIL) and 3D graphics, thus improving
the support for applications~\cite{Cesar:2005}.

In all this initiatives, the declarative portion of the API deliveries
ready-to-use services that can be efficiently implemented on the native
architecture of the set-top box; while the imperative portion of the API
maintains a flexibility compromise with application developers.
Nonetheless, even this kind of hybrid middleware, as far as the authors
are concerned, are reported to have been implemented only in high-end
gadgets with PC-class processors (i.e. 32-bit with MMU) and large
storage capabilities. 


\section{Final Considerations}

Interactive Digital Television seems to be following quite different
tracks in developed and in developing countries: while it is more of a
technological enhancement of ordinary TV for the first category---where
most citizens already have Internet access, the second sees it as a
concrete possibility to bring millions of digitally excluded citizens
into the information society. Nonetheless, it is known from past
experiences that isolated developing countries initiatives are unlikely
to succeed, specially considering the obvious pressure from the global
content industry.  From the technological point of view, the solution
would be an iDTV that has all the features demanded by consumers from
developed countries while being affordable to consumers in developing
ones.

In this paper, we presented a reasoning about the implementation of
Interactive Digital Television in developing countries focused on the
API that must be implemented in every set-top box in order to support
iDTV programs.  This reasoning guided us towards two major alternatives:
adopting \textsc{Java} as a substratum for higher-level iDTV frameworks,
or developing a new declarative language based on XML. The \textsc{Java}
approach has its main advantage on the flexibility to develop new iDTV
programs independently from the original API implementation, after all
\textsc{Java} is an object-oriented language and higher-level frameworks
can always be bypassed.  The main disadvantage is the complexity of the
associated run-time environment, which implies in higher-cost set-top
boxes. From the XML approach, the main advantage comes from the
possibility of a leaner implementation of the iDTV API, what could
enable the production of relatively low-cost terminals. The main
disadvantage, however, is that such implementation would have to be done
almost from scratch and would require explicit upgrades in order to
support additional features.

The hybrid approach of combining declarative and imperative languages to
build the iDTV API, which seems to be the direction for current open
standards, is a recognition of the gap between flexibility and set-top
box cost. However, initiatives to develop more efficient run-time
support systems for \textsc{Java}, which could make it more comfortably
fit in the embedded systems scenery are under way withing several
research groups.  For instance, the Kilo Virtual Machine (KVM), which is
the reference implementation of J2ME by Sun Microsystems, requires a
32-bit processor and about 1/2 Mbyte of RAM to run an ordinary
application, what is far too much for the desired scenario, but is
already a fantastic advance if compared to the counterpart PC with
\textsc{Windows}/\textsc{Linux} and J2RE\cite{Rayside:1999}. These
approaches might give us the answer to many problems in the path of iDTV
to digital inclusion in the developing world.

\bibliographystyle{spbasic}
\bibliography{guto,os,tvd}


\end{document}


% Nonetheless, having both imperative and declarative approaches in the
% middleware raises some performance issues for the set-top box. The
% middleware implementation must be reliable and efficient, while at the
% same time it must handle buggy and malicious applications properly~\cite
% {morris:itvstd}. One issue is the mapping of applications to Virtual
% Machines (VM). The data carousel is constantly throwing applications on
% the set-top boxes to be executed. If all applications execute on the
% same VM, the middleware developer would have some difficulties to assure
% they are truly independent from one another. On the other side, if each
% application runs on its own VM, the middleware developer must coordinate
% the communication and behavior of them.

% Another technical difficulty is to choose the correct number of threads
% for each application [morris:itvstd]. Only one thread could be too slow,
% and the question is to find a good number without affect the performance
% or the rest of the system.

% There are many questions to middleware developers, like if the VM must
% support just-in-time compiling, which can increase the performance, but,
% on the other side, it will slow down the application loading and demands
% more memory, which could be used to caching objects from the broadcast
% file system in order to speed up loading times.

% DVB-MHP. Digital V�deo Broadcasting Multimedia Home Plataform. 2003- 
% 2006. Site oficial da plataforma MHP. Dispon�vel em <http://www.mhp.org>. 
% Acesso em: 15 de junho de 2006. 

% MHP
% ETSI ~ European Telecommunications Standards Institute. Digital V�deo 
% Broadcasting (DVB). Multimedia Home Platform (MHP) Specification 1.1.1. 
% v.1.1.2. Rev.1. Fran�a: ETSI, 2003.

% ETSI ~ European Telecommunications Standards Institute. Digital V�deo 
% Broadcasting (DVB). Multimedia Home Platform (MHP) Specification 1.1.2. 
% Draft. Fran�a: ETSI, 2005. 

% GEM
% ETSI ~ European Telelecommunications Standards Institute. Digital Video 
% Broadcasting (DVB). Globaly Executable MHP version 1.0.2. Fran�a: ETSI, 
% 2005. 

% DASE
% ATSC: DTV APPLICATION SOFTWARE ENVIRONMENT LEVEL 1 (DASE-1) - PART 1: INTRODUCTION, ARCHITECTURE, AND COMMON FACILITIES. Doc. A/100-1. March, 2003.
