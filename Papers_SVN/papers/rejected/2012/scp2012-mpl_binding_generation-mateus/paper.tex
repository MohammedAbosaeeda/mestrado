\documentclass[final,1p,times]{elsarticle}
% \documentclass[final,3p,times,twocolumn]{elsarticle}

\usepackage[english]{babel}	% for multilingual support

\usepackage[utf8]{inputenc} % for use utf8
\usepackage{graphicx}

\usepackage[caption=false]{subfig} % Para usar duas ou mais figuras como uma só.

\usepackage{algorithmic}
\usepackage{algorithm}

\usepackage{verbatim} % Para usar comentários em bloco.


% ------------------------------------------------------------------------------

% Command to use code as figure ------------------------------------------------
\usepackage{listings}
\lstset{keywordstyle=\bfseries, flexiblecolumns=true}
\lstloadlanguages{[ANSI]C++,HTML}
\lstdefinestyle{prg} {basicstyle=\small\sffamily, lineskip=-0.2ex, showspaces=false}

% header C
\newcommand{\headerc}[3][tbp]{
 \begin{figure}[#1]
     \lstinputlisting[language=C,style=prg]{fig/#2.h}
   \caption{#3\label{headerc:#2}}
 \end{figure}
}

% C
\newcommand{\progc}[3][tbp]{
 \begin{figure}[#1]
     \lstinputlisting[language=C,style=prg]{fig/#2.c}
   \caption{#3\label{progc:#2}}
 \end{figure}
}

% header C++
\newcommand{\headercpp}[3][tbp]{
 \begin{figure}[#1]
     \lstinputlisting[language=C++,style=prg]{fig/#2.hh}
   \caption{#3\label{headercpp:#2}}
 \end{figure}
}


% C++
\newcommand{\progcpp}[3][tbp]{
 \begin{figure}[#1]
     \lstinputlisting[language=C++,style=prg]{fig/#2.cc}
   \caption{#3\label{progcpp:#2}}
 \end{figure}
}

% Java
\newcommand{\progjava}[3][tbp]{
 \begin{figure}[#1]
     \lstinputlisting[language=Java,style=prg]{fig/#2.java}
   \caption{#3\label{progjava:#2}}
 \end{figure}
}


% Para colocar 2 programas como um só - dispostos horizontalmente
\newcommand{\multprogjavatwoh}[5][htbp]{
\begin{figure*}[#1]
  \centering

\subfloat[]{\label{fig:#2}\lstinputlisting[language=Java,style=prg
]{fig/#2.java}}

\subfloat[]{\label{fig:#3}\lstinputlisting[language=Pascal,
style=prg]{fig/#3.lua}}
  \caption{#5}
  \label{fig:#4}
\end{figure*}
}
% e.g.
%\multfigtwoh{fig_plot_time_orig}{fig_plot_time_mod}
%{fig_plot_time_all}
%{Original (a) and modified (b) benchmarks execution time comparison.}


% Lua
\newcommand{\proglua}[3][tbp]{
 \begin{figure}[#1]
     \lstinputlisting[language=Pascal,style=prg]{fig/#2.lua}
   \caption{#3\label{proglua:#2}}
 \end{figure}
}

% KCL
\newcommand{\progkcl}[3][tbp]{
 \begin{figure}[#1]
     \lstinputlisting[language=C++,style=prg]{fig/#2.kcl}
   \caption{#3\label{progkcl:#2}}
 \end{figure}
}

% OCL
\newcommand{\progocl}[3][tbp]{
 \begin{figure}[#1]
     \lstinputlisting[language=C++,style=prg]{fig/#2.ocl}
   \caption{#3\label{progocl:#2}}
 \end{figure}
}

% OIL
\newcommand{\progoil}[3][tbp]{
 \begin{figure}[#1]
     \lstinputlisting[language=C++,style=prg]{fig/#2.oil}
   \caption{#3\label{progoil:#2}}
 \end{figure}
}

% ------------------------------------------------------------------------------

% Commands to insert figures ---------------------------------------------------
\newcommand{\figu}[4][ht]{
  \begin{figure}[#1] {\centering\scalebox{#2}{\includegraphics{fig/#3}}\par}
    \caption{#4\label{fig:#3}}
  \end{figure}
}

\newcommand{\fig}[4][ht]{
  \begin{figure}[#1] {\centering\scalebox{#2}{\includegraphics{fig/#3}}\par}
    \caption{#4\label{fig:#3}}
  \end{figure}
}
% fig usage:
% \fig{<scale>}{<file>}{<caption>}
% e.g.: \fig{.4}{uml/uml_comportamental_dia}{Diagramas comportamentais da UML}
% The figure label will be "fig:" plus <file>.
% The figure file must lie in the "fig" directory.

\newcommand{\figtwocolumn}[4][ht]{
  \begin{figure*}[#1] {\centering\scalebox{#2}{\includegraphics{fig/#3}}\par}
    \caption{#4\label{fig:#3}}
  \end{figure*}
}

\newcommand{\figb}[4][hb]{
  \begin{figure}[#1] {\centering\scalebox{#2}{\includegraphics{fig/#3}}\par}
    \caption{#4\label{fig:#3}}
  \end{figure}
}

% Para colocar 2 figuras como uma só - dispostas horizontalmente
\newcommand{\multfigtwoh}[6][htbp]{
\begin{figure*}[#1]
  \centering
  \subfloat[]{\label{fig:#3}\scalebox{#2}{\includegraphics{fig/#3}}}
  \subfloat[]{\label{fig:#4}\scalebox{#2}{\includegraphics{fig/#4}}}
  \caption{#6}
  \label{fig:#5}
\end{figure*}
}
% e.g.
%\multfigtwoh{.65}{fig_plot_time_orig}{fig_plot_time_mod}
%{fig_plot_time_all}
%{Original (a) and modified (b) benchmarks execution time comparison.}

% Para colocar 2 figuras como uma só - dispostas verticalmente
\newcommand{\multfigtwov}[6][htbp]{
\begin{figure}[#1]
  \centering
  \subfloat[]{\label{fig:#3}\scalebox{#2}{\includegraphics{fig/#3}}}\\
  \subfloat[]{\label{fig:#4}\scalebox{#2}{\includegraphics{fig/#4}}}
  \caption{#6}
  \label{fig:#5}
\end{figure}
}
% e.g.
%\multfigtwov{.35}{fig_epos_mem_framework}{fig_epos_mem_framework_spm}
%{fig_epos_mem_framework_all}
%{EPOS memory mapping before (a) and after (b) using the new framework}


% Alguns outros comandos:
\newcommand{\lua}{\textsc{Lua}}
\newcommand{\java}{\textsc{Java}}
\newcommand{\python}{\textsc{Python}}
\newcommand{\ruby}{\textsc{Ruby}}


% ------------------------------------------------------------------------------
\journal{Science of computer programming}

\begin{document}

\begin{frontmatter}

% Títulos já usados em papers publicados sobre o mesmo tema
% + Método para Abstração de Componentes de Hardware para Sistemas Embarcados (2012:Dissertação)
% + Abstracting hardware devices to embedded Java applications (2011:IADIS)
% + Abstraindo dispositivos de hardware para aplicações Java embarcadas (2011:SBESC)
% título em inglês para o IEEE: Interfacing Hardware Devices to Embedded Java
% + Interfacing Operating Systems Components with Embedded Java Applications (2009:I2TS)

% \title{MPL binding generation as an AOP problem}
% \title{Efficient generation of wrappers for hardware devices in the embedded
% system scenario}
% \title{MPL Binding Generation for Embedded Systems as Aspect Weaving}
% \title{Managed Programming Languages Binding Generation for Embedded Systems Components}
\title{Binding Embedded System Components in Managed Programming Languages}

\author{Mateus Krepsky Ludwich and Antônio Augusto Fröhlich}

\address{Laboratory for Software and Hardware Integration (LISHA)\\
Federal University of Santa Catarina (UFSC)\\
Florianópolis, Brazil\\
\{mateus,guto\}@lisha.ufsc.br
}


\begin{abstract}
\documentclass[11pt]{article} 
%\documentclass[10pt,twocolumn]{article} 
\usepackage{latex8}
\usepackage{times}
\usepackage{epsf}

\newcommand{\tild}{\raisebox{-.75ex}{\~{}}}
\newcommand{\rb}[1]{\raisebox{1.5ex}[-1.5ex]{#1}}
\newcommand{\puteps}[3]{
	\begin{figure}[htb]
	\begin{center}
	\leavevmode
	\epsfbox{#1}
	\end{center}
	\vskip -1.5em
	\caption{\label{#2}#3}
	\end{figure}
} 
\newcommand{\putfig}[5]{
	\begin{figure}[htb]
	\begin{center}
	\leavevmode
	\epsfxsize=#2
	\epsfysize=#3
	\epsfbox{#1}
	\end{center}
	\vskip -1.5em
	\caption{\label{#4}#5}
	\end{figure}
} 

\begin{document}
\pagestyle{empty}
\setlength{\parskip}{1.5ex}
\setlength{\parindent}{3em}

\title{\Large\bf Tailor-made Operating Systems for\\Embedded Parallel Applications}
%\author{\begin{tabular}[t]{c@{\extracolsep{8em}}c}
%Ant\^onio Augusto Fr\"ohlich\thanks{Work partially supported by Federal University of Santa Catarina and by Funda\c{c}\~ao Coordena\c{c}\~ao de Aperfeicoamento de Pessoal de N\'ivel Superior.} & Wolfgang Schr\"oder-Preikschat\\
%\\
%GMD FIRST & University of Magdeburg\\
%Rudower Chaussee 5 & Universit\"atsplatz 2\\
%D-12489 Berlin, Germany & D-39106 Magdeburg, Germany\\
%{\tt guto@first.gmd.de} & {\tt wosch@cs.uni-magdeburg.de}
%\end{tabular}}

\author{\begin{tabular}[t]{c@{\extracolsep{8em}}c}
Ant\^onio Augusto Fr\"ohlich & Wolfgang Schr\"oder-Preikschat\\
\\
GMD FIRST & University of Magdeburg\\
Rudower Chaussee 5 & Universit\"atsplatz 2\\
D-12489 Berlin, Germany & D-39106 Magdeburg, Germany\\
{\tt guto@first.gmd.de} & {\tt wosch@cs.uni-magdeburg.de}
\end{tabular}}

\date{}
\maketitle

\thispagestyle{empty}

\section*{\center Extended Abstract}

The boom of embedded systems in the recent years projects a near future replete of complex embedded applications, for instance navigation, computer vision and particularly automotive systems. Many of these applications demand performance levels that can only be achieved by parallelization  and therefore new operating systems and tools are to be conceived.

	Our experiences developing run time support systems for ordinary, i.e., non-embedded, parallel applications \cite{Preikschat:94, Froehlich:96b} convinced us that adjectives such as "generic", "global" and "all purpose" do not fit together with "high performance", whereas different parallel applications have quite different requirements regarding operating systems. Even apparently flexible designs like micro-kernel based operating systems may imply in waste of resources that, otherwise, could be used by applications. The reality for embedded parallel applications can not be other and thus we should give each application its own operating system.

	The promotion of configurability has been properly addressed by the PURE operating system \cite{Schoen:98}. PURE is designed as a collection of configurable system objects that can be regarded as building-blocks to be put to together according to applications demands. In PURE, an application can get the operating system that exactly fulfills its requirements. Although doing much for performance, reusability and maintainability, strategies like this are usually not enough to support application programmers, since, as the development of these systems advance, the number of available building-blocks grows quickly to reach hundreds or thousands, and, in such a situation, selecting and configuring the proper building-blocks becomes a nightmare. This gap between what the operating system offers and what the applications need can compromise the success of such strategies and are to be overcame.

	This paper proposes an innovative strategy to deal with this gap, both, by reducing the building-block collection  complexity and by automating the selection and configuration process. In this proposal, the application can be conceived in a more abstract and flexible way and then be submitted to a tool that will generated a tailored operating system to support its embedded parallel execution.


\subsection*{Adapting System Objects}

	As a tailorable operating system, PURE is designed to yield a large number of building-blocks, which in turn are to be put together to compose the tailor-made operating system. In order to achieve high performance, these building-blocks must be fine tuned to each of the execution scenarios aimed, therefore, a reasonable set of building-blocks will comprise a large number of elements. However, when we take a deeper look inside these building-blocks, we realize that those designed to present the same functionality in different scenarios are indeed very similar. We also realize that the differences often follow a pattern along members conceived to support the same scenario.

	In this way, we propose the system objects to be implemented as independent from the execution scenario as possible. They would then be put together with the aid of some sort of "glue" specific to each scenario. We named these "glue" scenario adapters, since they will adapt an existing system object to a certain scenario. These adapters are designed not to insert overhead on the path from applications to system objects.

	The approach of writing pieces of software that are independent from certain aspects and later adapting them to a give scenario has been referred to as Aspect Oriented Programming \cite{Kiczales:97}. However, there are significant differences in our proposal to make system objects adaptable. We are not proposing a language to describe aspects neither tools to automatically combine  aspects and aspect-independent components.  The problem we want to tackle is the complexity of the system objects repository, which will be reduced by isolating scenario dependencies in a separate software structure, the scenario adapters.


\subsection*{Automatically Generating  a Tailored Operating System}

	Supposing we have a reasonable set of operating system building-blocks, the generation of the "best possible" operating system for a given parallel application that will execute in a given scenario could be carried out like following:

\begin{enumerate}
\setlength{\parskip}{0cm}
\item Let the application programmer select the system objects (he believes) he needs;
\item Let the application programmer describe the execution scenario for the application;
\item Resolve dependencies among the selected system objects and among them and the described scenario;
\item Compile (link) the just configured operating system.
\end{enumerate}
\setlength{\parskip}{1.5ex}

	These steps, however, will not generate the "best possible" operating system because of one of the following: (a) the supposition that we have a reasonable set of system objects to select for is false and the application programmer will not find the system objects he is looking for; (b) the set of system objects is so complex that the application programmer gets confused and fail on selecting the proper objects or on describing the execution scenario. This frustrating situation becomes evident when the programmer realizes he has to modify the application to adapt it to the so called "tailored operating system" or when he runs the application and cannot see any benefit over using a standard operating system. This second remark is only not wors because, as far as we now, there are no standard operating systems, like Unix or Windows NT, to support embedded parallel applications.

	Our proposal to generate a tailored operating system starts top-down at the application. We believe the application programmer should be allowed to express his expectations regarding the operating system simply by writing down well-known system object invocations (system calls in non object oriented systems) while coding the application. By well-know system object invocations we mean that the operating system services will be made available to applications via objects commonly accepted by the parallel system community, such as threads, tasks, address spaces, channels, ports, etc.

	That is, during application design and implementation, the programmer will rely on a set of well-known system objects inflated interfaces. These interfaces are defined according to the fundamental law of object orientation that says \cite{booch:94}: look at the real world while looking for objects. The question of where in the real world can one find a thread can be easily answered when we realize that threads are human conventions, just like numbers are, and that a couple of classical computer science books should comprise most widely accepted conventions.  Nevertheless, our users, i.e., embedded parallel application programmers, are welcome to suggest modification or extensions for these interfaces.

	With these interfaces in hand, it should no longer be a problem for a skilled parallel application programmer to select how he wants to express process communication, thread synchronization or any other operating system service. An application conceived in this fashion can now be submitted to a tool that will extract a blueprint for the operating system to be generated. Actually, this tool (figure \ref{interfaces}) has the solely task of identifying the system objects referred by the application and the methods invoked on them. The inflated interfaces described above are only used until this point. Each of the implicitly selected system objects can now be assembled using only the building-blocks that are really necessary to grant the functionality required by the invoked methods.

% size = {484 x 309}
% scale = 1/5.15
\putfig{interfaces.eps}{94mm}{60mm}{interfaces}{Extracting an operating system blueprint from the application.}

	Our primary operating system blueprint is, unfortunately, not complete. We got good hints about how the ideal operating system for a given application should look like, but there are aspects that can not be deduced by analyzing the application's source code. As an example, we could consider the decision of generating an operating system that supports multiple processes in protected address spaces based on a micro-kernel or an operating system for a single, possibly multi-threaded, process to be embedded on the application. The fact that the application's source does not show any syntactic evidence that multiple processes may need to run concurrently in a single processor, does not necessarily mean that this situation will not happen. The multi-task support may be required because the application may need to span more processes than the available number of processors, and this will not be perceivable until the user tell us something about the intended execution scenario. Other factors such as target architecture, number of processors available, network architecture and topology are fundamental to tailor a good operating system, but are usually not expressed inside the application. Therefore, we still need user intervention to describe the application's execution scenario, however, the description of the available resources will be due to the operating system developers and the interaction with the user will be done via visual tools. 

	Refining the blueprint obtained when analyzing the application's code with the context information acquired via this visual tool will render a precise description of how the ideal operating system for a given application should look like. This refined blueprint is still comprised of system objects interfaces to be satisfied, but those inflated interfaces we started with are now reduced to small, scenario specific ones. For example, the inflated thread interface from the first step may have included remote invocation and migration, but reached the final step as a simple single-task, priority scheduled thread for a certain microcontroller. This refined operating system blueprint can now be used to select the respective building-blocks from the repository. A representation of an application tailored opperating system generated according to this model is depictet in the the figure \ref{global_view}.

% size={366 x 276}
% scale=1/6,9
\putfig{global_view.eps}{53mm}{40mm}{global_view}{An operating system tailored to an application.}

	 It is important to understand that at the early stages of the operating system development, very often a required building-block will not yet be available. Even then, the proposed strategy is of great value, since the operating system developers got a precise description for the missing building-blocks. In many cases, a missing building-block will be quickly (automatically) adapted from another scenario using the scenario adapters described earlier.

	Only if the operating system developers are not able to deliver the requested building-blocks in a time considered acceptable by the user, either because a building-block with that functionality have not yet been implemented for any scenario or because the requested scenario is radically different from the currently supported scenarios, we will shock the user asking him to select the best option from the available set of system objects (scenario adapters) and to adapt his program. In this way, our strategy ends where most tailorable operating systems start. Moreover, after some development effort, the combination of scenario adapters and system objects shall satisfy the big majority of parallel embedded applications.

\bibliographystyle{latex8}
\bibliography{operating_systems,software_engineering,guto}
 
\end{document}


\end{abstract}

\begin{keyword}
Binding Generation, Aspect-Oriented Programming, Embedded Systems, Foreign Function Interface

\end{keyword}

\end{frontmatter}


%-------------------------------------------------------------------------------
% Section <Introduction>
% ------------------------------------------------------------------------------
\section{Introduction} \label{intro}
% + Introduction
% 
% The very first letter is a 2 line initial drop letter followed
% by the rest of the first word in caps.
%
% form to use if the first word consists of a single letter:
% \IEEEPARstart{A}{demo} file is ....
%
% form to use if you need the single drop letter followed by
% normal text (unknown if ever used by IEEE):
% \IEEEPARstart{A}{}demo file is ....
%
% Some journals put the first two words in caps:
% \IEEEPARstart{T}{his demo} file is ....
%
% Here we have the typical use of a "T" for an initial drop letter
% and "HIS" in caps to complete the first word.
% \IEEEPARstart{T}{his} demo file is intended.

\IEEEPARstart{E}{nergy} consumption is a determining factor when designing wireless sensor networks.
As a consequence, battery lifetime is a limitation on the development of such systems.
Therefore, the idea of extracting energy from the environment has become attractive.
Looking to the energy consumption problem, the intelligent usage of the stored energy contributes to extend the sensor nodes' longevity.
Consequently, energy schedulers have been developed in order to adequately assess the energy consumption and adapt the system accordingly to the available amount of energy.
The purpose of this work is to adapt a solar energy harvesting circuit to supply energy to low power wireless platforms, i.e., those that operate under $50~mW$.
Simultaneously, we aim at improving the performance of the energy-aware task scheduler in wireless sensor network systems by providing fine-grained battery and environmental monitoring.

Among a number of energy sources that have been studied so far, solar has proved to be one of the most effective~\cite{Roundy:2003}.
The solar energy conversion through photovoltaic (PV) cells is better performed at an optimum operating voltage.
Operating a solar panel on this voltage results in transferring to the system the maximum amount of power available.
In this context, \emph{maximum power point tracker circuits} (MPPT) have been proposed.
The drawback is that MPPT circuitry may introduce losses to a solar harvesting system.
Concerning low-power applications, it may be more energy efficient to have a good matching between the solar panel and the energy storage unit~\cite{Raghunathan:2005}.
This well matched system is than able to work close to the maximum power point with less power loss.

In this work, an evaluation of the proposed harvesting circuit is performed in order to show improvements on an energy-aware task scheduler~\cite{Hoeller:SMC:2011}.
It is shown that the combination of the proposed circuit with the cited scheduler not only extended the longevity of the wireless sensor network, but also improved system quality.

The paper is organized as follows:
Section~\ref{fund} presents the fundamentals of solar energy harvesting and energy-aware task scheduler.
Section~\ref{design} discusses the design of the harvesting circuit under the perspective of low power wireless platforms.
Section~\ref{case} presents the evaluation of the harvesting circuit and a case study showing the improvements on system quality.
Finally, section~\ref{concl} closes the paper.

% ------------------------------------------------------------------------------


% ------------------------------------------------------------------------------
% Section <SOTA>
% -----------------------------------------------------------------------------
\section{Trabalhos relacionados}
\label{sec:related_work}
% Nosso trabalho aborda a questão de como prover acesso a dispositivos de hardware
% ao Java e como prover este acesso de uma forma bem estruturada levando em 
% consideração todos os requisitos do cenário de sistemas embarcados.
%
A linguagem de programação Java é desprovida do conceito de \emph{ponteiro}, 
presente em linguagens como C e C++. 
O endereço das \emph{variáveis de referência}, utilizadas para acessar objetos Java,
é conhecido apenas pela JVM, a qual trata de todos os acessos à
memória. Como a maioria dos dispositivos de hardware são mapeados em endereços de
memória, acessá-los diretamente é um problema para a linguagem Java. 
FFI é a abordagem
utilizada por Java para superar esta limitação uma vez que ela permite ao Java
utilizar construções, como ponteiros C/C++, para acessar diretamente dispositivos
de hardware.
FFIs também tem sido utilizadas por plataformas Java na reutilização de código
escrito em outras linguagens de programação como C e C++ e para embarcar JVMs em 
aplicações nativas permitindo as mesmas acessar funcionalidades 
Java \cite{Liang:1999}.
%(\cite{Liang:1999}, \cite{1288968}).

\emph{Java Native Interface} (JNI) é a principal FFI Java, a qual é utilizada na
plataforma \emph{Java Standard Edition} \cite{Liang:1999}. 
Na JNI, a interface entre código nativo e Java é realizada durante o tempo de 
execução do programa. Isto significa que, durante a execução de um programa, 
a JVM procura e carrega a implementação dos métodos marcados como nativos 
(métodos que possuem a palavra reservada \emph{native} em suas assinaturas).
Usualmente a implementação dos métodos nativos é armazenada em uma biblioteca 
ligada dinamicamente.
Este mecanismo de busca e carga de métodos aumenta a necessidade de memória em
tempo de execução e o tamanho da JVM. Por esta razão eles são evitados em 
sistemas embarcados.

% NOTA: Não estou certo destas limitações da KNI, apesar de estarem na 
% especificação da mesma.
A plataforma \emph{Java Micro Edition} (JME) utiliza uma FFI ``leve'',
chamada de \emph{K Native Interface} (KNI) \cite{_k_2002}. 
A KNI não carrega métodos nativos dinamicamente na JVM, evitando o sobrecusto 
de memória da JNI. 
Na KNI a interface entre Java e código nativo é realizada estaticamente, 
durante o tempo de compilação. 
Entretanto decisões de projeto da KNI impõem algumas limitações. 
A KNI proíbe a criação de objetos Java (exceto de strings) a partir do código 
nativo. 
% É proibida também a chamada de métodos Java, a partir do código nativo. 
Além disto, na KNI os únicos métodos nativos que podem ser invocados são aqueles 
pré-compilados na JVM. 
Não há uma Interface de Programação de Aplicações 
(API - Application Programming Interface) em nível Java para invocar outros 
métodos nativos. % cuidado com esta afirmação. Frase original: There is no Java-level API to invoke others native methods.
Como consequência, é difícil de criar novos controladores de dispositivos de 
hardware utilizando-se a KNI.

% NOTA: Os último argumento, de chamar Java a partir do C... é fraco
A FFI da KESO, utilizada neste trabalho, foca em sistemas embarcados. 
Assim como a KNI, a FFI da KESO não realiza carga dinâmica de métodos nativos.
Entretanto, diferentemente da KNI, a FFI da KESO provê aos programadores uma 
API em nível Java para criação de novas interfaces com código nativo. 
Também não exite problema do código nativo chamar código Java, uma vez que 
KESO e a FFI da KESO geram código C.

O tarefa de escrita de adaptadores para código nativo pode ser facilitada de
duas maneiras, por APIs de alto nível e por ferramentas geradoras. As APIs de
alto nível fornecem métodos específicos para auxiliar na criação desses adaptadores,
enquanto as ferramentas geradoras podem gerar parte de adaptadores ou adaptadores
completos a partir de análise de código nativo ou a partir de uma especificação
em mais alto nível.
SWIG e a biblioteca de função estrangeira de Python \emph{ctypeslib} são exemplos
de ferramentas que geram adaptadores a partir de arquivos \emph{headers} C/C++
como entrada. O primeiro suporta diversas linguagens como saída, como por exemplo
Python, D e Java. O segundo foca em programas Python 
\cite{swig-site},\cite{ctypeslib-site}.
Ravit et al. apresenta uma ferramenta que tem como objetivo prover funcionalidades
da linguagem de mais alto nível (tuplas, por exemplo) para serem utilizadas no 
código dos adaptadores. A ferramenta proposta por Ravit et al. gera adaptadores
Python a partir de código escrito em C e descrições de interface, as quais 
contêm, dentre outras, informações sobre funções e seus respectivos parâmetros
\cite{Ravitch:2009:AGL:1542476.1542516}.
Outras soluções, como a linguagem \emph{Jeannie}, misturam código C e Java em 
um único programa a partir do qual geram adaptadores JNI 
automaticamente \cite{1297030}.
A FFI da KESO, utilizada neste trabalho, provê uma API baseada em aspectos para
ajudar na criação de adaptadores. É possível especificar quais pontos do programa
Java serão afetados pela criação dos adaptadores, assim como qual código deve
ser gerado para cada ponto do programa Java a ser afetado.

% The binding code can be checked for correctness and bug detection. Tools such 
% as J-BEAM and Ilea perform bug checking based on the source code, using static
% analysis techniques\cite{jbeam:2008}, \cite{Ilea:2007}.
% Lee et al. deal with bug detection dynamically, when the FFI code is been 
% used \cite{Lee:2010:JSD:1809028.1806601}.
% Their tool Jinn synthesizes dynamic bug detectors for FFIs from Finite State 
% Machines whose encode FFI constrains that should be tested. The FFI targeted 
% for the Java language is the JNI which contains hundreds os API calls. 
% Although not aborted by this paper, similar bug detectors can be adapted to 
% check bindings generated using the KESO FFI.


% -----------------------------------------------------------------------------



% ------------------------------------------------------------------------------
% Section <DERCS>
% ------------------------------------------------------------------------------
\section{DERCS} \label{dercs}
% + Intro
Our method uses an intermediate representation to describe hardware mediators and
FFI aspects.
This section describes the meta-model we have used and
Section \ref{sec:proposal} describes the method we have proposed.

% NOTE: Maybe move
% + Why we have used DERCS
% and
% + How we have used DERCS
% To here.

% + DERCS def
Model-Driven Engineering (MDE) proposes the development of complete
computational systems from high level specifications which are transformed,
in one of more phases, in order to generate the final system. % cite MDE
In such context, the Distributed Embedded Real-Time Compact Specification
(DERCS) is a specification/meta-model used to represent platform independent
models.
Such models, according to the
Aspect-oriented Model-Driven Engineering for Real-Time systems (AMoDE-RT)
methodology, are used together with the description of the target platform and
mapping rules to generate the final system, which encompass software and hardware
elements \cite{Wehrmeister:2009}.

% + DERCS and AMoDE-RT
According to the AMoDE-RT methodology,
a DERCS model is generated from UML's class and sequence diagrams, and
from diagrams that specify non-functional elements as aspects.
Therefore, the DERCS meta-model defines
structural elements, behavioral elements, and aspect-oriented elements,
encompassing distinct visions into a single model.

% + DERCS's Structural and Behavioral Elements
% Figures \ref{fig:structural_elements} and \ref{fig:behavioral_elements}
% show, respectively, the structural and behavioral elements defined by DERCS.
Among the structural elements of DERCS are classes, attributes,
methods, and parameters.
The behavioral elements determine the behavior of a method, detailing which
messages such method can send to objects, which are its
local variables, actions, etc.
The structural and behavioral elements of DERCS have semantics similar to the
semantics of object-oriented programming languages.
Details concerning structural and behavioral elements of DERCS can be found at
\cite{Wehrmeister:2009}.

% \figtwocolumn{1}{structural_elements}{DERCS structural elements. Adapted from \cite{Wehrmeister:2009}.}

% \figtwocolumn{1}{behavioral_elements}{DERCS behavioral elements. Adapted from \cite{Wehrmeister:2009}.}

% + Aspect-Oriented Elements
Figure \ref{fig:ao_elements} shows the aspect-oriented elements
defined by DERCS.
Among the aspect-oriented elements are aspects, pointcuts,
structural adaptations, and behavioral adaptations.
Those elements correspond to the concepts of Aspect-Oriented Programming (AOP),
as defined by \cite{Kiczales:1997} whereas
aspect adaptations (\emph{AspectAdaptation})
represent the concept of \emph{advices}.
An aspect adaptation can be structural (\emph{StructuralAdaptation}) or
behavioral (\emph{BehavioralAdaptation}).
Structural adaptations modify the structure of the elements of the model, for
example, adding methods to a class or parameters to a method.
On the other hand, behavioral adaptations modify the behavior of the elements of
the model, for example, executing tasks before or after the execution of a
method behavior, or completely changing a method's behavior.

\figtwocolumn{1}{ao_elements}{DERCS aspect-oriented elements. Adapted from \cite{Wehrmeister:2009}.}

% + For what DERCS was used
The aspect-oriented elements of DERCS meta-model can be used to modify the
interface of a component in order to adapt such component according to what is
required by its client.
As next section shows, we have explored this feature of DERCS in order to
adapt a hardware mediator interface for what is expected by a specify FFI.

% \clearpage

% ------------------------------------------------------------------------------


% ------------------------------------------------------------------------------
% Section <proposal>
\section{Building a Trustful Infrastructure for Future Internet}
\label{sec:solution}
The Internet architecture demonstrate inefficiency and problems in several and large areas, such as mobility, real-time applications,
failures (e.g. equipment, software bugs, and configuration mistakes), and especially in pervasive security problems \cite{Rexford:2010}.
Moreover, the Internet lacks effective solutions in terms of scalability and sustainability, 
consuming much more energy and hindering the management of countless sensor devices that are so important for several applications in the Future Internet.
Hence, we propose the use of a stack of communication protocols (UDP@NDN@C-MAC), in the scope of the EPOSMote project,
designed specifically to guarantee a trustful communication
%Our solution also includes EPOSMote II, an embedded platform. Thus, 
while still compromised with the low utilization of resources (processing, memory, power and communication bandwidth).
%and the use of EPOSMote II which is an embedded platform and represents a typical Future Internet device.

\subsection{EPOSMote}
The EPOSMote is an open hardware project~\cite{eposmote}. Initially it aimed at 
the development of a wireless sensor network module, and focused on environment 
monitoring. Its first version, the EPOSMote I, features an 8-bit AVR microcontroller, 
IEEE 802.15.4 communication capability and a small set of sensors.

As the project evolved a second version arose, with the objective of delivering a 
hardware platform to allow research on energy harvesting, biointegration, and 
MEMS-based sensors. The EPOSMote II focus on modularization, and thus is composed 
by interchangeable modules for each function.

Figure \ref{emote2-block_diagram} shows an overview of the EPOSMote II architecture.
Its hardware is designed as a layer architecture composed by a main module,
a sensoring module, and a power module. The main module is responsible for processing
and communication. It is based on the Freescale MC13224V microcontroller~\cite{mc13224v}, which possess 
a 32-bit ARM7 core, an IEEE 802.15.4-compliant transceiver, 128kB of flash memory, 80kB of ROM memory
and 96kB of RAM memory. We have developed a startup sensoring module, which contains some sensors  
(temperature and accelerometer), leds, switches, and a micro USB (that can also be used as power supply). 
Figure \ref{emote2-mc13224v-pictures-real_white_background} shows the development kit which is slightly 
larger than a R\$1 coin, on the left the sensoring module, and on the right the main module.

\fig{.45}{emote2-block_diagram}{Architectural overview of EPOSMote II.}

\fig{.07}{emote2-mc13224v-pictures-real_white_background}{EPOSMote II SDK side-by-side with a R\$1 coin.}

\subsection{C-MAC}
C-MAC is a highly configurable MAC protocol for WSNs realized as a framework of
medium access control strategies that can be combined to produce
application-specific protocols~\cite{steiner:2010}. It enables application
programmers to configure several communication parameters (e.g.  synchronization,
contention, error detection, acknowledgment, packing, etc) to adjust the protocol
to the specific needs of their applications. The framework was implemented in C++ 
using static metaprogramming techniques (e.g. templates, inline functions, and 
inline assembly), thus ensuring that configurability does not come at expense of 
performance or code size. The main C-MAC configuration points include:

\textbf{Physical layer configuration:} These are the configuration points defined
by the underlying transceiver (e.g. frequency, transmit power, date rate).

\textbf{Synchronization and organization:} Provides mechanisms to send or receive
synchronization data to organize the network and synchronize the nodes duty
cycle.

\textbf{Collision-avoidance mechanism:} Defines the contention mechanisms used to
avoid collisions. May be comprised of a carrier sense algorithm (e.g. CSMA-CA),
the exchange of contention packets (\emph{Request to Send} and \emph{Clear to
Send}), or a combination of both.

\textbf{Acknowledgment mechanism:} The exchange of \emph{ack} packets to
determine if the transmission was successful, including preamble acknowledgements.

\textbf{Error handling and security:} Determine which mechanisms will be used to
ensure the consistency of data (e.g. CRC check) and the data security.

The Future Internet will be composed by a wide range of both applications and devices, 
each with its own requirements and available resources. Through C-MAC configurability we
can provide the most adequate MAC functionalities for each case, instead of providing a 
general non-optimal solution for all of them.

\subsection{NDN}
Communication in NDN is impelled by the data consumers.
Nodes that are interested in a content transmit \emph{Interest} packets, which contains the name of the requested data. %selector, nonce
Every node that receives the \emph{Interest} and have the requested data can respond with a \emph{Data} packet that follows back the path from which the \emph{Interest} came. %content name, signature, signed info, data
It is important to notice that one \emph{Data} satisfies one \emph{Interest}, thus ensuring flow balance in the network.
Since the content being exchanged is identified by its name, all nodes interested in the same content can share transmissions (considering a broadcast medium, which is the case for most Future Internet devices).

NDN packet forwarding engine has three main data structures: the FIB (Forwarding Information Base), which is used to forward \emph{Interest} packets to potential sources; 
the ContentStore, which is a buffer memory used to maximize the sharing of packets; 
and the PIT (Pending Interest Table), which is used to keep track of \emph{Interest} packets so that \emph{Data} packets can be sent to its requester(s).

When a node receives an \emph{Interest} packet it searches for its content name, looking for a match primarily at the ContentStore, then the PIT, and ultimately at the FIB.
If there is a match at the ContentStore, it is sent and the \emph{Interest} discarded.
Otherwise, if there is a match at the PIT, the set of requesting interfaces for that data is updated, and the \emph{Interest} discarded (at this point an \emph{Interest} in this data has already been sent).
Otherwise, if there is a match at the FIB, the \emph{Interest} is sent towards the data, and a new PIT entry is created. 
In case there is no match for the \emph{Interest} then it is discarded.

As for the \emph{Data} packet they simply follow the chain of PIT entries back to the original requester(s).
When a node receives a \emph{Data} packet it searches for its content name. 
If there is a ContentStore match, then the \emph{Data} is a duplicate and is discarded.
%A FIB match means there are no matching PIT entries, so the \emph{Data} is unsolicited and it is discarded.
In case of a PIT match, the data is validated, added to the ContentStore, and sent to the set of requesting interfaces from the corresponding PIT entry.

In NDN the name in every packet is bound to its content with a signature.
This enables data integrity and provenance, allowing consumers to trust the data they receive regardless of how the data came to them.
To provide content protection and access control NDN uses encryption.
The encryption of content or names is transparent to the network, since to NDN it is all just named binary data.
%The signature algorithm used may be selected by the content publisher, 
%and chosen to meet performance requirements such as latency or computational cost of signature generation or verification.
Nevertheless, NDN does not mandate any particular key distribution scheme, signature, or encryption algorithm.

\subsection{UDP}
The User Datagram Protocol has been chosen for its simplicity. Its simple transmission 
model avoids unnecessary overhead, since it does not handle reliability, ordering, 
and data integrity, leaving these characteristics to be treated in other layers if necessary, which is a 
perfect blend with the rest of our protocol stack.


% ------------------------------------------------------------------------------
% Section <Evaluation>
% ------------------------------------------------------------------------------
\section{Practical Experiments} \label{eval}
% + Practical Experiments
% First shows speedup and quality results for DMEC
% using 1 (without partitioning) to 6 worker threads.
% Show that speedup is high and quality is kept acceptable.
% 
% Then, show speedup and quality results for DMEC integrated to JM and compares
%  to the original JM (and, if possible, to other works).
% 

% + JM
% + Dizer como realizamos os experimentos. E/ou quais as variáves observadas:
%   Evaluate the component in isolation DMEC to show its speedup.
%    And how it scales.
%   Evaluate the component in JM to show sppeedup and PSNR.
We have evaluated DMEC in two stages.
First, in order to verify how DMEC's performance scales from 
one to six \emph{Workers} instances, we have evaluated all DMEC implementations 
in a test case.
The test case application mimics the behavior of an H.264 encoder: it provides 
DMEC with pictures, obtain the ME results 
(motion vectors and motion cost), and checks if the results are correct.
Secondly, in order to assess DMEC influence on the final video quality, we have
evaluated all DMEC implementations in the
JM H.264 Reference Encoder~\cite{site:jm}.
The PSNR degradation is computed as the absolute PNSR difference between the
original encoder and the optimized ones.

% P: Dizer pq focamos em luma
For inter macroblock modes in H.264 (i.e. modes related to the ME),
the motion cost for chrominance components derives from the motion cost for 
luminance components~\cite{1101854}. 
Consequently the PSNR for chrominance components derives from the PSNR for 
luminance components. 
For this reason, in this paper we focus on the PSNR variation of the luminance 
component.

Figure \ref{fig:dmec-speedup_workers} show the speedup of DMEC in there
test case application with a different number of \emph{Workers} instances.
For such test, we have used an arbitrary set of pictures with a resolution of
1080p (Full-HD).
The speedup is normalized to one \emph{Worker} instance (speedup of 1X).

% TODO
% \textit{Comments about results in: Cell BE, Muticore IA32, and HW.}
It is worth to mention that for each number of \emph{Worker} instances
a different partition mode was used, according to
Figure \ref{fig:picture_partition}.
For one \emph{Worker} instance we have used the ``Single Partition'',
for two \emph{Worker} instances we have used the ``2x1'' partition and so on,
up to the ``2x3'' partition mode (used for six \emph{Worker} instances).

Besides the additional performance obtained by using a higher number of \emph{Worker} instances,
the partition mode also has influence on the speedup.
The reason of such influence is that, during the partitioning process, the dimensions of the
search window shrinks, thus reducing the area of the picture searched for
similarities.

\fig{.45}{dmec-speedup_workers}{Time performance scalability of DMEC}

% Sobre RD curves
% Figures XXX show the speedup of DMEC while tested already integrated to JM.
% The obtained values are compared to the ones obtained while using the original
% JM, without DMEC.
In order to evaluate in details the behavior of DMEC for distinct values 
of encoding bit-rates, we have used the BD-PSNR (Bjøntegaard Delta PSNR) metric
using the following values of QP (Quantization Parameter): 16,20,24,28; as 
described in \cite{gisle_bjntegaard_calculation_2001}.
It is important to evaluate quality (PSNR) for distinct bit-rates to test 
whether the approach can be used in distinct scenarios of application.
Figure \ref{fig:crowd-bitrate_psnr} shows the rate-distortions (RD) curves using the
original JM encoder and the optimized encoder using DMEC.
The video sequence used for this curves was \texttt{Crowd Run}, a 1080p sequence with  a
high ammount of motion.
Lower values of bit-rate are obtained for higher values of QP since by using
higher values for QP more data is discarded, thus increasing the
compression ratio. 
The two curves very near from each other indicates that the DMEC
presents a good rate-distortion performance for all the evaluated bit-rates.

\fig{.45}{crowd-bitrate_psnr}{RD curve of a 1080p video sequence}

We have evaluated also the speedup obtained in the
% encoding time
ME run time
while using
DMEC for the same QP values we used for BD-PSNR.
Figure \ref{fig:crowd-bitrate_speedup} shows the obtained values while using
6 \emph{Worker} instances.
For Muticore IA32, a speedup of around 9 times is obtained for all
bit-rate values.
For Cell BE this value is about 2 times.
A small speedup for the Cell BE, while compared to Multicore IA32 and the dedicated hardware, is due
to the memory transferences (picture samples and ME results) which is performed using
the DMA requisitions of Cell BE.


\fig{.45}{crowd-bitrate_speedup}{Speedup vs bit-rate of a 1080p sequence}
%
% \multfigtwov{.65}{bd_psnr}{bd_speedup} {bd} {RD curve (a) and speedup vs bit-rate (b) of 1080p sequence}

% Discussion


% Falar do paralelismo / particionamento de dados
% Qualidade ficou boa mesmo particionando e desempenho aumentou: speedup ~ 70%
%The strategy of ME distribution based on picture partitioning has been shown 
%effective.
% We have obtained a speedup higher than XXX\% without loosing quality.
%Data partitioning is effective because the visual interdependence between
%partitions is not significant to influence on the encoding quality, and allows
%for a speedup because it enables the simultaneous processing of each picture
%partition.

% - Falar da comunicação
% - Necessidade via espaço de endereçamentos diferentes
% A arquitetura Cell Broad Band demonstrou-se uma arquitetura interessante para o
% processamento paralelo de vídeo, pois possui unidades funcionais dedicadas 
% (i.e. SPEs) para processamento de dados. 
% A principal dificuldade encontrada em trabalhar-se com o Cell foi a capacidade 
% limitada da memória local das SPEs.
% Outra dificuldade foi lidar com as transferências de memória entre SPE e PPE. 
% Isto em fato foi superado pelas estratégias que desenvolvemos de baferização e 
% também com a utilização do Element Interconect Bus (EIB) do Cell que realiza 
% DMAs com altas taxas de transferências.
% 
% - Solução 1: Buffer de preditores, contribuiu bastante
% A estratégia de armazenamento de preditores nas SPEs foi significativa no 
% aumento do desempenho, pois vetores de movimentos necessários para o cálculo da
% ME não precisam ser consultados na memória principal. É coerente a decisão de 
% manter uma cópia local destes vetores, pois todos os vetores que a ME irá 
% precisar foram calculados pela partição em questão e por nenhuma outra.


% ------------------------------------------------------------------------------

% ------------------------------------------------------------------------------
\subsection{Distributed Motion Estimation}
ME is a technique employed to explore the similarity between 
neighboring pictures in a video sequence.
Figure \ref{fig:motion_estimation} illustrates the ME process for the
neighboring pictures \emph{A} and \emph{B}.
By searching for similarities between these two pictures, it is possible
to determine which blocks from picture \emph{A} are also found in
picture \emph{B}.
Such displacement of picture blocks is encoded by \emph{motion vectors}
(represented by the small arrows in the bottom side of
Figure \ref{fig:motion_estimation}).
Exploring the similarity between neighboring pictures allows for
difference-based encoding, thus increasing the compression rate of the generated
bitstream \cite{citeulike:1269699}.
As ME is a significant stage for H.264 encoding, the component
we have adapted to work with embedded \java~and~\lua~uses an optimized version
of ME.
In order to improve the performance of ME, the component uses a data
partitioning strategy where the motion estimation for each partition of the
picture is performed in parallel in a specific functional unit, such as a core
of a multicore processor.
However, such complexity is hidden from the point of view of the user of the
component (e.g. H.264 encoder), which only sees a component to perform ME
by using the \emph{match} method.

\figtwocolumn{.4}{motion_estimation}{Motion Estimation.}


% (4) About the DMEC App: Java and Lua, mimics the H.264 encoder
In order to use the ME component from \java~and \lua, first we wrote an
application that mimics the behavior of an H.264 encoder then, we have generated
binding code classes for \java~and \lua.

Figure \ref{progjava:dmec_java_app} shows the Java version of the
encoder-like application.
Figure \ref{proglua:dmec_lua_app} shows the same application written in \lua.
The application provides the component with pictures,
get from the component the ME results (motion vectors and motion cost), and
checks if the results are correct.

\progjava{dmec_java_app}{Motion Estimation Java application.}

\proglua{dmec_lua_app}{Motion Estimation Lua application.}

% \multprogjavatwoh{dmec_java_app}{dmec_lua_app}{ME Java and Lua applications.}


% (5) KESO wrapper and a Lua wrapper for DMEC (text in msc thesis: p6)
% IMPORTANT TO SAY TODO: This shows the reuse between distinct FFIs
Using \emph{EBG}, we have generated binding code classes for the ME component
in order to integrate it with \java~and \lua.
Figure \ref{progc:dmec_wrappers4lua} shows the binding code wrapping the
\emph{match} method to provide it for \lua.
Figure \ref{progc:dmec_wrappers4nano_vm} shows the equivalent code for NanoVM,
and Figure \ref{progjava:dmec_wrappers4java} shows it for KESO (both \java).
It is not our goal to describe these figures in detail.
Instead, such figures are an illustration that the same functional component
of the binding can be \emph{reused} for distinct FFIs.
% Na real o reuso é meio que a nível de DERCS..., mas tudo bem.
% Aqui mostra depois do weaving e depois da geração de código.
In all figures it is shown that the ME
parameters: the \emph{current} and the \emph{reference} picture
(which correspond, respectively, to picture \emph{A} and \emph{B} from
Figure \ref{fig:motion_estimation}) are selected and then used on the invocation
of the \emph{match} method.
Then, the \emph{match} method returns the motion vectors and
the motion cost (grouped together by the class \emph{PictureMotionCounterpart}).
% The output generated by EBG for KESO FFI,
% shown at Figure \ref{progjava:dmec_wrappers4java},
% is different from the output for \lua~and NanoVM FFI due to the
% generational approach of KESO.
% Instead of generating the binding code, EBG generates a \emph{weavelet} class
% (following the KESO FFI) which is then used by the KESO compiler to generate
% the final binding code.
% 
% It recovers the motion estimator object from the \lua's stack, as well the ME
% parameters: the \emph{current} and the \emph{reference} picture
% (which correspond, respectively, to picture \emph{A} and \emph{B} from
% Figure \ref{fig:motion_estimation}).
% Then, it invokes the \emph{match} method which returns the motion vectors and
% the motion cost (grouped together by the class \emph{PictureMotionCounterpart}).
% Finally, it puts back in the \lua's stack the ME's result and
% returns the control to the \lua~runtime.
% Figure \ref{progc:dmec_wrappers4nano_vm} shows the same binding code for the
% NanoVM FFI.
% Its structure is very similar to the \lua~binding code: it invokes the motion
% estimator and it uses the NanoVM stack to get the parameters and publish the
% ME return.
% The output generated by EBG for KESO FFI,
% shown at Figure \ref{progjava:dmec_wrappers4java},
% is different from the output for \lua~and NanoVM FFI due to the
% generational approach of KESO.
% Instead of generating the binding code, EBG generates a \emph{weavelet} class
% (following the KESO FFI) which is then used by the KESO compiler to generate
% the final binding code.


\progc{dmec_wrappers4lua}{Binding code for \emph{match} method (Lua FFI)}

\progc{dmec_wrappers4nano_vm}{Binding code for \emph{match} method (NanoVM FFI)}

\progjava{dmec_wrappers4java}{Binding code for \emph{match} method (KESO FFI)}


% (6) Performance evaluation: time overhead for match method
In order to evaluate the feasibility of the generated binding code classes, we
have evaluated them according to performance and memory consumption.
For performance evaluation, first we have measured the time overhead caused by
the binding code of the \emph{match} method.
%according to the equation \ref{eq:time_overhead}.
Then, we have calculated how such overhead affects the ME throughput in order to
see the impact in ME overall performance.
All time measurements were performed using the IA32 \emph{Time-Stamp Clock} (TSC),
which is a 64-bit register that counts the number if cycles since reset.
For memory usage evaluation, first we have measured how many bytes the
binding code for the \emph{match} method has.
Then, we have calculated how much of the total system size is caused by such
overhead.
All memory measurements were performed using the utility \emph{size} of GNU for
the IA32.

% (7) Performance evaluation: throughput for C++, for KESO, and for Lua.
% IMPORTANT TO SAY TODO: This evaluation show that our wrappers does not impact
% on the overall performance of the application.
% The time requirements for ME processing are still fulfilled
Table \ref{tab:dmec_time_overhead} shows the time overhead of the \emph{match}
method for each FFI used.
The column \emph{Device} contains the device time, which is the time of the
\emph{match} method while accessed directly through a C++ program, without the
use of any binding.
Such time is independent of the FFI used.
The column \emph{Binding} shows the overhead just for the binding code
of the match method, column \emph{VM} shows the runtime environment support
(e.g. virtual machine) overhead for decoding an instruction of native method
invoking, and column \emph{Total} show the overall overhead (binding plus VM).
% Comparate KESO, NanoVM and Lua.
% some partial conclusion
% As one can see, for all the FFIs the time overhead was less than X\%.


%Time overhead			
%FFI	Device (µs)	Binding Overhead (µs)	VM Overhead (µs) Total Overhead (µs)
%Lua	9.68362E+005	2.73753E-003	6.29973E+001         6.30000E+001
%NanoVM	9.68362E+005	3.97423E-003	1.96960E+001         1.97000E+001
%KESO	9.68362E+005	8.26280E-003	1.18113E+003         1.18113E+003


\begin{table*}[t]
\begin{center}
\caption{Time overhead}
\begin{tabular}{|c|c|c|c|c|c|}
\hline
\textbf{FFI} & \textbf{Device ($\mu s$)} & \textbf{Binding ($\mu s$)} & \textbf{VM ($\mu s$)} & \textbf{Total ($\mu s$)}\\
\hline
Lua    & 9.68362E+005 & 2.73753E-003 & 6.29973E+001 & 6.30000E+001 \\
\hline
NanoVM & 9.68362E+005 & 3.97423E-003 & 1.96960E+001 & 1.97000E+001 \\
\hline
KESO   & 9.68362E+005 & 8.26280E-003 & 1.18113E+003 & 1.18113E+003 \\
\hline
\end{tabular}
\label{tab:dmec_time_overhead}
\end{center}
\end{table*}

% TODO: Checkar se vou usar estes valores de 30 fps mesmo.
% Até se vou usar fps
% Se usar, é necessário definir o que é fps: frames per second.
% Essa argumentação é interessante. Mas para isso o DMEC teria de entregar
% realmente a 33 fps (ou perto disso).
% Se nao for o DMEC outras opção seria alguma applicação que use a UART
% (e.g. TSG / uart do modem)
% Ou ainda o NIC::send do radio do EPOS mote II para uma aplicação de
% transmissão de audio (ver trabalho do Rufino).
In order to evaluate the time overhead impact on the overall ME performance, we
have calculated how such overhead affects the ME throughput.
Table \ref{tab:me_throughput} shows the results.
The column \emph{Original Throughput} shows the original ME throughput
(without using any binding code) while
the column \emph{New Throughput} shows the ME throughput while using each
FFI/MPL evaluated.
Considering that the required throughput for the ME is one Frame-per-Second
(FPS), the new throughput is acceptable for all FFIs considered.
The overhead caused by the binding code and the VM is independent of the
\emph{Device} time.
It only depends on the number and the type of methods arguments, and method
return plus the VM decoding time of a native method invoke instruction.
Theoretically, considering a very fast ME 
(e.g. with a device time in the order of dozens of $\mu s$),
the total overhead (also around dozens of $\mu s$) would be prohibitive 
because it would be in the same order of the device time (representing an overhead around 100\%).
However, this is not the case in the practice since an real-time ME is on the
order of 30 FPS which represents a device time of 33 ms that is still one order
of magnitude bigger than the obtained overhead.
Thus, in practice all FFIs used will continue to meet the time requirements.

%ME throughput		
%FFI	Original throughput (FPS)	New throughput (FPS)
%Lua	1.032672	1.032604
%NanoVM	1.032672	1.032651
%KESO	1.032672	1.031416

\begin{table*}[t]
\begin{center}
\caption{ME throughput}
\begin{tabular}{|c|c|c|}
\hline
% \textbf{FFI} & \textbf{Original Throughput ($fps$)} & \textbf{New Throughput ($fps$)}\\
\textbf{FFI} & \textbf{Original Throughput (FPS)} & \textbf{New Throughput (FPS)}\\
\hline
Lua    & 1.032672 & 1.032604 \\
\hline
NanoVM & 1.032672 & 1.032651 \\
\hline
KESO   & 1.032672 & 1.031416 \\
\hline
\end{tabular}
\label{tab:me_throughput}
\end{center}
\end{table*}

% (8) Memory evaluation: memory overhead for KESO and for Lua.
In order to evaluate the memory overhead caused by the generated binding code
classes, we have measured
how many bytes the binding code for the \emph{match} method has.
Table \ref{tab:dmec_memory_overhead} shows the obtained values
(column \emph{Overhead})
for each FFI used.
The same table shows the total system footprint, including application, and the
MPL runtime (composed by the VM and EPOS), and how much of this size is caused
by such the binding code classes.
The memory available for the platform where DMEC runs
%(256MB)
% (4GB)
is far away bigger
than the new footprint obtained while using the binding code classes.
Thus, for all FFIs, the generated binding code classes fulfill the memory
usage requirements.

%Memory overhead and its impact on the total footprint			
%FFI	Overhead (byte)	Footprint (byte)	Impact (%)
%Lua	800	1576512	0.05
%NanoVM	1453	1459501	0.1
%KESO	201	1463201	0.01

\begin{table*}[t]
\begin{center}
\caption{Memory overhead and its impact on the total footprint}
\begin{tabular}{|c|c|c|c|}
\hline
\textbf{FFI} & \textbf{Overhead (byte)} & \textbf{Footprint (byte)} & \textbf{Impact (\%)}\\
\hline
Lua    & 800  & 1576512 & 0.05 \\
\hline
NanoVM & 1453 & 1459501 & 0.1 \\
\hline
KESO   & 201  & 1463201 & 0.01 \\
\hline
\end{tabular}
\label{tab:dmec_memory_overhead}
\end{center}
\end{table*}



% (9) Talks about Hardware / Software platform. DMEC@HW - HLS.
% Não me parece muito relavante falar sobre isso se o foco do paper é reuso de
% binding code entre FFIs distintas.

\clearpage

% ------------------------------------------------------------------------------

% % ------------------------------------------------------------------------------
% NOTA TODO: Revisar / Reescrever esta subseção: Serial Communication
\subsection{Serial Communication}
% NOTA:  Acho que é interessante citar os meus papers anteriores IADIS AC 2011.
% Algo como, resultados iniciais mostrados para KESO em \cite...
The first case study evaluated as a synthetic application which uses the
\emph{Universal Asynchronous Receiver Transmitter} (UART) hardware device for
serial communication.
Such application, was evaluated in \java, using the KESO FFI, for the
architectures IA32, AVR8, and PPC32.
The Figure \ref{progjava:uart_app} shows the \java~source of the application,
which uses the UART hardware mediator in order to write characters on a serial
device.

\progjava{uart_app}{UART application.}

The \java~ class \emph{UART}, generated as one of the outputs of EBG, is the
\java~counterpart of the UART hardware mediator and has only native methods
without implementation.
The other output of EBG is the native code adapter already tailored to be
integrated with the KESO JVM.
This integration process is show in details in our previous work
\cite{Ludwich:IADIS_AC:2011}.
At such work the UART application using our approach is compared to the a
equivalent application written using the \emph{Java Standard Edition} platform,
which uses the \emph{Java Native Interface} (JNI) as FFI, running on Linux.
Our approach is around 38X faster than using JNI and Linux.

% ------------------------------------------------------------------------------
% % ------------------------------------------------------------------------------
% NOTA TODO: Revisar / Reescrever esta subseção: Temperature Sensing
\subsection{Temperature Sensing}
The third case study is an application for temperature sensing.
This is a distributed application, composed by a \emph{Sensor} node which
measures the temperature and sends the obtained measures to a \emph{Sink} node
which receive the temperature values and process them.
The communication between the nodes is performed by radio on the context of a
\emph{Wireless Sensor Network} (WSN).

The Figure \ref{progjava:sensor_app} shows the application executed by the
Sensor node, and the Figure \ref{progjava:sink_app} shows the application
executed by the Sink node.
Both applications are written in \java.
We have generated native code adapters for the temperature sensor
mediator (\emph{Temperature\_Sensor}), and for the network interface card
(\emph{NIC}) mediator which abstracts the radio used for communicating between
the nodes.

\progjava{sensor_app}{Sensor application.}
\progjava{sink_app}{Sink application.}

The target FFIs were KESO FFI and NanoVM FFI.
The application remains the same for both virtual machines.
The application which runs on KESO JVM was deployed in the AVR8 (8 bit)
architecture, and the application which runs on NanoVM was deployed in the ARM7
architecture (32 bit).
The platform used was the \emph{EPOS Mote} (version AVR8 and ARM7).
\emph{EPOS Mote} is an open source and open hardware mote for WSN which runs
the EPOS system \cite{EposMoteSite}.

In order to evaluate the performance of the generated native code adapters,
we have measured the time overhead for accessing the hardware mediators used by
the temperature sensing application.
Table \ref{tab:tempsensing_app_time_overhead_keso_epos_avr8} shows the
results for the AVR8 architecture using the KESO FFI.
Table \ref{tab:tempsensing_app_time_overhead_nanovm_epos_arm7} shows the
results for the ARM7 architecture using the NanoVM FFI.
For the temperature sensor device was measured the time of the \emph{sample}
method which samples the temperature from the environment.
For the NIC has measured the time for send an arbitrary message.
%
\begin{table*}[t]
\begin{center}
\begin{tabular}{|c|c|c|c|}
\hline
\textbf{Method} & \textbf{Total ($\mu s$)} &
\textbf{Device ($\mu s$)} &
\textbf{Overhead (\%)}\\
\hline
\emph{Temp::sample} & 334.22 & 330 & 1.26 \\
\hline
\emph{NIC::send} & 8586.22 & 8580 & 0.072 \\
\hline
\end{tabular}
\caption{Time overhead generated by the native code adapters.
Architecture AVR8, KESO FFI.}
\label{tab:tempsensing_app_time_overhead_keso_epos_avr8}
\end{center}
\end{table*}
%
\begin{table*}[t]
\begin{center}
\begin{tabular}{|c|c|c|c|}
\hline
\textbf{Method} & \textbf{Total ($\mu s$)} &
\textbf{Device ($\mu s$)} &
\textbf{Overhead (\%)}\\
\hline
\emph{Temp::sample} & 946.04 & 942 & 0.43 \\
\hline
\emph{NIC::send} & 958.39 & 950 & 0.87 \\
\hline
\end{tabular}
\caption{Time overhead generated by the native code adapters.
Architecture ARM7, NanoVM FFI.}
\label{tab:tempsensing_app_time_overhead_nanovm_epos_arm7}
\end{center}
\end{table*}
%
%
The ``Total'' time and the ``Device'' time from the two tables are not
comparable, since they are obtained from distinct platforms.
However, the overhead value is, and it is less than 1.3\% for all devices.
The time overhead generated by NanoVM should be bigger than in KESO JVM because
NanoVM has an overhead for interpreting the \java~bytecode.
KESO JVM does not have this kind of overhead since all bytecode is translated
to C and then to native code.
However, this is not seen comparing the tables because the bytecode
interpretation overhead was not measured for NanoVM.
The measure overhead from NanoVM comes specifically from the native method,
which uses ``push'' and ``pop'' operations to interact with NanoVM stack for
obtaining the method's arguments and for returning the method's results.

% ------------------------------------------------------------------------------


% ------------------------------------------------------------------------------
% Section <Discussion>
% ------------------------------------------------------------------------------
\section{Conclusões}
\label{sec:discussion}
% Aplicações para sistemas embarcados usualmente necessitam interagir com
% diversos tipos de dispositivos de hardware como sensores, atuadores, 
% transmissores, receptores e \emph{timers}.
%
% Interface de função estrangeira é o mecanismo adotado por Java para superar as
% limitações da linguagem e permitir acesso direto a memória e a dispositivos de
% hardware. 
% Entretanto, como mostrado na seção \ref{sec:related_work}, as principais FFIs 
% Java não são eficientes em termos de consumo de recursos, ou possuem limitações
% de projeto que dificultam o desenvolvimento de novas interfaces entre Java e os
% dispositivos de hardware.
%
Neste artigo, apresentou-se um meio de realizar a interface entre componentes 
de hardware e aplicações Java para sistemas embarcados. 
Isto foi obtido utilizando-se a interface de função estrangeira da JVM KESO e o
EPOS.

% O EPOS permite o desenvolvimento de aplicações portáveis, independentes de
% especificidades de máquina. 
% Isto é conseguido utilizando-se o conceito de mediadores de hardware, os quais
% sustentam um contrato de interface entre abstrações de sistemas e a máquina.

% O JVM KESO compila o bytecode de uma aplicação Java em código C e gera as partes
% da JVM necessárias pela aplicação. 
% A FFI da KESO também utiliza esta abordagem estática, gerando o código C 
% especificado nas classes \emph{Weavelet}. 
% Então o código C gerado pelo compilador KESO e pela FFI da KESO são compilados
% em conjunto em código nativo, utilizando-se um compilador C padrão.

Nós avaliamos nossa abordagem em termos de desempenho, consumo de memória e 
portabilidade.
Para a aplicação utilizando o mediador de hardware da UART o sobrecusto 
de tempo obtido foi menos de 0.04 \% do tempo total de execução da aplicação e
nossa solução é 38 vezes mais rápido do que a JNI da Sun.
O consumo de memória para tal aplicação foi de 33KB, incluindo todo o suporte
de ambiente de execução, o qual é adequado para diversos sistemas embarcados.
Utilizando o EPOS nós obtivemos portabilidade para várias plataformas e 
utilizando o conceito de componentes híbridos podemos utilizar os mesmos 
adaptadores de código nativo tanto para componentes implementados em 
hardware como implementados em software.

Visando avaliar nossa abordagem em uma aplicação real, nós escrevemos
adaptadores de código nativo para um componente o qual realizada estimativa de
movimento para codificação de vídeo H.264.

% ------------------------------------------------------------------------------



% ------------------------------------------------------------------------------
% References
\bibliographystyle{elsarticle-num}
\bibliography{hw,os,pl,mm,se}


\end{document}

%------------------------------------------------------------------------------

