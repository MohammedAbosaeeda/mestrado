\section{Case Study}
\label{sec:case}

In this section we present the deployment of the approach described in this
paper in a mobility-enabled wireless sensor network running the Ant-based
Dynamic Hop Optimization Protocol (\adhop) over an IP network using IEEE
802.15.4. \adhop~is a self-configuring, reactive routing protocol inspired by
the HOPNET protocol for \emph{Mobile Ad Hoc Networks}~(MANETs) and designed with
the typical limitations of sensor nodes in mind, energy in
particular~\cite{Okazaki:2011}. \adhop's reactive component relies on an
\emph{Ant Colony Optimization} algorithm to discover and maintain routes. Ants
are sent out to track routes, leaving a trail of pheromone on their way back.
Routes with a higher pheromone deposit are preferred for data exchange.

\wtab{adzrp-taskset}{\adhop~case-study tasks' parameters.}

With the purpose of corroborating the approach presented in this paper we made a
few modifications to the \adhop~in order to make it energy-aware. In order to do
that, we separated \adhop~tasks between mandatory (hard real-time) tasks and
optional (best-effort) tasks. The main idea behind this setup was to homogenize
the battery discharge for every node in the network to enhance the lifetime of
the network as a whole. Considering the radio the most energy-hungry component
in a wireless sensing node, we toke the design decision of modeling the
\emph{ants} of \adhop~as best-effort tasks, as shown by the task set at
Table~\ref{tab:adzrp-taskset}. By doing this, the basic node functionality of
sensing a value (task $Sense$) and forwarding it through the radio to a
sink-node (task $Forward$) where modeled as hard real-time tasks, and the
functionality of forwarding other nodes' packets (and ants) when acting as a
``router'' was modeled as two best-effort tasks, one for monitoring the channel
for arriving messages ($Low Power Listen$), and another to effectively receive
the message and route it to another node ($Route$).

We set the lifetime objective for this system to 25 days. By analyzing the task
set it is possible to compute the total energy consumption of hard real-time
tasks for the desired lifetime to be of $601.88 mAh$, thus, the initial battery
charge for the system has to be greater than that. We analyzed this system
% in two situations: using a small Panasonic CR-2 $3V$ battery with a total
% capacity of $850 mAh$, and using a larger battery set comprised of two Panasonic
% AM-3PI $1.5V$ batteries, with a total capacity of $5,740 mAh$. In the first
% scenario, impact on network performance would be two significant, thus 
using a small Panasonic CR-2 $3V$ battery with a total capacity of $850 mAh$.

In order to enhance the significance of our results, we simulated larger
networks (with up to 200 nodes) using the \emph{Global Mobile Information System
Simulator} (\glmsim). In this setup, nodes were programmed to communicate
intensively and move randomly within a simulated grid of 700 x 400 meters for 25
days, thus stimulating both the routing protocol and the power management
mechanisms. \glmsim~was integrated to the same \nsga~optimizer described in
Section~\ref{sec:frequency}, and the optimization process ran with the same
parameters (\nsgaparam). The results of this optimization are shown in
Figure~\ref{fig:adzrp-solutions} (simulated solutions),
Table~\ref{tab:adzrp-solutions} (best solutions, i.e., those at the Pareto
front), and Figures~\ref{fig:adzrp-bet_freq} and~\ref{fig:adzrp-batt_freq}
(observed parameters plotted against the collector task's frequency).

\fig{adzrp-solutions}{All solutions for the \adhop~case-study.}{width=\columnwidth}

\tab{adzrp-solutions}{Solutions for the frequency of the collector task for
the \adhop~case-study.}

\figtwo{adzrp-bet_freq}{Execution rate of best-effort tasks.}
{adzrp-batt_freq}{Residual energy after projected lifetime.}
{Optimization objectives plotted against variations on the frequency of the
collector task for the \adhop~case-study.}

Additionally, we analyze the impact on the network performance by comparing the
obtained results with the data originally published by
Okazaki~\cite{Okazaki:2011}. Figure~\ref{fig:adzrp-avg_node_energy} shows a
reduction on the average energy consumed by each node on the network while
Figure~\ref{fig:adzrp-avg_node_lifetime} shows the expected enhancement on the
average battery lifetime of nodes. It is important to note that all nodes in
the network lived for, at least, 25 days as expected, being that the reason why
the average lifetime stayed around 30 days or above.

\figtwo{adzrp-avg_node_energy}{15 minutes energy
consumption.}{adzrp-avg_node_lifetime}{Battery lifetime.}{Average energy-related
parameters for the simulated \adhop~setup.}

Besides the good results from the energy consumption perspective, we observed an
important decrease on the overall network quality, as shown in
Figures~\ref{fig:adzrp-broken_routes} and~\ref{fig:adzrp-delivery_ratio} for,
respectively, the ``Broken routes'' and ``Delivery ratio'' parameters. These
contrast, however, with the obtained results on ``Link failures'' shown by
Figure~\ref{fig:adzrp-link_failures}, which shows that \adhop~deals well with
the broken routes, allowing undelivered packets to be re-routed and finally
delivered. Future work on energy-aware scheduling, which were not focus of the
present work, will rely on the presently proposed accounting mechanism to enable
fairer scheduling of such tasks. Initial studies have already began on flexible
schedulers such as Ramanathan's (m,k)-firm scheduler~\cite{Ramanathan:1997}
and Buttazzo's elastic model~\cite{Buttazzo:1998}.

\figthree{adzrp-broken_routes}{Broken routes.}{adzrp-delivery_ratio}{Delivery
ratio}{adzrp-link_failures}{Link failures}{Network quality impact for the \adhop
case-study.}

