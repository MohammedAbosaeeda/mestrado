\documentclass[12pt]{article} % SBESC 2011
\usepackage{sbc-template} % SBESC 2011

% \usepackage[latin1]{inputenc}	% for Latin languages
% \usepackage[T1]{fontenc}	% for ISO and UTF characters
\usepackage[brazil]{babel}	% for multilingual support

\usepackage[utf8]{inputenc} % for use utf8
% \usepackage{graphicx}
\usepackage{graphicx,url} % SBESC 2011

% -----------------------------------------------------------------------------

% Command to use code as figure -----------------------------------------------
\usepackage{listings}
\lstset{keywordstyle=\bfseries, flexiblecolumns=true}
\lstloadlanguages{[ANSI]C++,HTML}
\lstdefinestyle{prg} {basicstyle=\small\sffamily, lineskip=-0.2ex, showspaces=false}

\newcommand{\prgcpp}[3][tbp]{
 \begin{figure}[#1]
     \lstinputlisting[language=C++,style=prg]{fig/#2.cc}
   \caption{#3\label{prg:#2}}
 \end{figure}
}

\newcommand{\prgjava}[3][tbp]{
 \begin{figure}[#1]
     \lstinputlisting[language=Java,style=prg]{fig/#2.java}
   \caption{#3\label{prgjava:#2}}
 \end{figure}
}

\newcommand{\prgkcl}[3][tbp]{
 \begin{figure}[#1]
     \lstinputlisting[language=C++,style=prg]{fig/#2.kcl}
   \caption{#3\label{prgkcl:#2}}
 \end{figure}
}

\newcommand{\oclspec}[3][tbp]{
 \begin{figure}[#1]
     \lstinputlisting[language=C++,style=prg]{fig/#2.ocl}
   \caption{#3\label{oclspec:#2}}
 \end{figure}
}


%------------------------------------------------------------------------------

% Commands to insert figures --------------------------------------------------
\newcommand{\figu}[4][ht]{
  \begin{figure}[#1] {\centering\scalebox{#2}{\includegraphics{fig/#3}}\par}
    \caption{#4\label{fig:#3}}
  \end{figure}
}

\newcommand{\fig}[4][ht]{
  \begin{figure}[#1] {\centering\scalebox{#2}{\includegraphics{fig/#3}}\par}
    \caption{#4\label{fig:#3}}
  \end{figure}
}
% fig usage:
% \fig{<scale>}{<file>}{<caption>}
% e.g.: \fig{.4}{uml/uml_comportamental_dia}{Diagramas comportamentais da UML}
% The figure label will be "fig:" plus <file>.
% The figure file must lie in the "fig" directory.

\newcommand{\figtwocolumn}[4][ht]{
  \begin{figure*}[#1] {\centering\scalebox{#2}{\includegraphics{fig/#3}}\par}
    \caption{#4\label{fig:#3}}
  \end{figure*}
}

\newcommand{\figb}[4][hb]{
  \begin{figure}[#1] {\centering\scalebox{#2}{\includegraphics{fig/#3}}\par}
    \caption{#4\label{fig:#3}}
  \end{figure}
}

%------------------------------------------------------------------------------
\sloppy

% \title{Abstracting Hardware Devices to Embedded Java Applications}
% \title{Um método para abstração de dispositivos de hardware para aplicações Java embarcadas}
\title{Abstraindo dispositivos de hardware para aplicações Java embarcadas}
%\title{Abstraindo dispositivos de hardware para Java embarcado}


\author{Mateus Krepsky Ludwich\inst{1}, Antônio Augusto Fröhlich\inst{1}}

%\address{Laboratory for Software and Hardware Integration -- LISHA\\ 
%Federal University of Santa Catarina -- UFSC\\ 
%P.O.Box 476, 880400900 - Florianópolis - SC - Brazil\\
%\email{\{mateus,guto\}@lisha.ufsc.br}
%}

\address{Laboratório de Integração de Software e Hardware (LISHA)\\ 
Universidade Federal de Santa Catarina (UFSC)\\ 
Caixa postal 476, 880400900 - Florianópolis - SC - Brasil\\
\email{\{mateus,guto\}@lisha.ufsc.br}
}


\begin{document}

\maketitle

% NOTA geral: ler o paper todo depois e ver se a palavra adaptador é uma 
% tradução adequada para binding/wrapper de interface nativa.
% Na dissertação de mestrado estou usando o termo interface e código de interface.
%
% NOTA "nova": decidi usar o tempo adaptador de código nativo, para designar
% o código de binding.

\begin{abstract}
Access to hardware devices is an important requirement to be fulfilled by Java
implementations targeting embedded systems because the interaction between the
embedded system and the environment where it is inserted on is performed by 
these devices.
In this paper we introduce a method for abstracting hardware devices to 
embedded Java applications. 
We have evaluated our method in terms of performance,
memory footprint, and portability. 
The applicability of our method was tested for abstracting simple hardware
devices for serial communication and for abstracting more complex components
such as a motion estimator for H.264 video coding. 
\end{abstract}

\begin{resumo} % Reduzir para 10 linhas.
O acesso a dispositivos de hardware é
um importante requisito a ser atendido por implementações Java para
sistemas embarcados pois, a interação entre o sistema embarcado e o ambiente 
no qual ele esta inserido é realizada por meio destes dispositivos.
Neste artigo nós apresentamos um método de como abstrair dispositivos de 
hardware para aplicações Java embarcadas. 
Nós avaliamos nosso método em termos de desempenho, consumo de memória e 
portabilidade. 
A aplicabilidade do nosso método foi testada na 
abstração de dispositivos de hardware simples para comunicação
serial e em componentes mais complexos como um estimador de movimento  
para codificação de vídeo H.264.

% NOTA: texto alternativo
% Interface de função estrangeira é o mecanismo utilizado por linguagens de 
% altíssimo nível, como Java, na abstração de dispositivos de hardware para serem
% utilizados pelo desenvolvedor do software embarcado. Entretanto o mecanismo de
% interface por si só não especifica como esta abstração deve ser feita.
% Neste artigo nós apresentamos um método de como abstrair
% dispositivos de hardware para aplicações Java embarcadas. Nós avaliamos nosso 
% método em termos de desempenho, consumo de memória e portabilidade. 
% A aplicabilidade do nosso método foi testada na 
% abstração de dispositivos de hardware simples para comunicação
% serial e em componentes mais complexos como um estimador de movimento  
% em codificação de vídeo H.264.

% NOTA: se não houvesse limite de 10 linhas eu iria usar este resumo aqui:
% Acesso a dispositivos de hardware é um importante requisito para sistemas 
% embarcados uma vez que tais dispositivos são utilizados para realizar a 
% interação entre o sistema embarcado e o ambiente no qual ele esta inserido.
% Paralelamente a isto, o uso de linguagens de altíssimo nível, como Java, 
% facilita o desenvolvimento de sistemas embarcados porque elas proveem 
% funcionalidades como orientação a objetos, gerenciamento automático de memória e
% proteção de memória.
% Neste artigo nós apresentamos um método de como abstrair
% dispositivos de hardware para aplicações Java embarcadas. Nós avaliamos nosso 
% método em termos de desempenho, consumo de memória e portabilidade. Além disso, 
% nós desenvolvemos adaptadores Java para um componente que realiza estimativa de 
% movimento em codificação de vídeo H.264, demonstrando a aplicabilidade da nossa
% abordagem em um cenário real.

\end{resumo}

% \category{D.3.3}{Programming Languages}{Language Constructs and Features}[Classes and objects]
% \category{D.3.4}{Programming Languages}{Processors}[Run-time environments]
% \category{D.4.4}{Operating Systems}{Communications Management}[Input/output]
% \category{D.4.7}{Operating Systems}{Organization and Design}[Real-time systems and embedded systems]
%
% \terms
% Design, Languages, Reliability
% 
% \keywords
% Java, Embedded Systems, Foreign Function Interface

%------------------------------------------------------------------------------
% Section <Introduction>
% ------------------------------------------------------------------------------
\section{Introduction} \label{intro}
% + Introduction
% 
% The very first letter is a 2 line initial drop letter followed
% by the rest of the first word in caps.
%
% form to use if the first word consists of a single letter:
% \IEEEPARstart{A}{demo} file is ....
%
% form to use if you need the single drop letter followed by
% normal text (unknown if ever used by IEEE):
% \IEEEPARstart{A}{}demo file is ....
%
% Some journals put the first two words in caps:
% \IEEEPARstart{T}{his demo} file is ....
%
% Here we have the typical use of a "T" for an initial drop letter
% and "HIS" in caps to complete the first word.
% \IEEEPARstart{T}{his} demo file is intended.

\IEEEPARstart{E}{nergy} consumption is a determining factor when designing wireless sensor networks.
As a consequence, battery lifetime is a limitation on the development of such systems.
Therefore, the idea of extracting energy from the environment has become attractive.
Looking to the energy consumption problem, the intelligent usage of the stored energy contributes to extend the sensor nodes' longevity.
Consequently, energy schedulers have been developed in order to adequately assess the energy consumption and adapt the system accordingly to the available amount of energy.
The purpose of this work is to adapt a solar energy harvesting circuit to supply energy to low power wireless platforms, i.e., those that operate under $50~mW$.
Simultaneously, we aim at improving the performance of the energy-aware task scheduler in wireless sensor network systems by providing fine-grained battery and environmental monitoring.

Among a number of energy sources that have been studied so far, solar has proved to be one of the most effective~\cite{Roundy:2003}.
The solar energy conversion through photovoltaic (PV) cells is better performed at an optimum operating voltage.
Operating a solar panel on this voltage results in transferring to the system the maximum amount of power available.
In this context, \emph{maximum power point tracker circuits} (MPPT) have been proposed.
The drawback is that MPPT circuitry may introduce losses to a solar harvesting system.
Concerning low-power applications, it may be more energy efficient to have a good matching between the solar panel and the energy storage unit~\cite{Raghunathan:2005}.
This well matched system is than able to work close to the maximum power point with less power loss.

In this work, an evaluation of the proposed harvesting circuit is performed in order to show improvements on an energy-aware task scheduler~\cite{Hoeller:SMC:2011}.
It is shown that the combination of the proposed circuit with the cited scheduler not only extended the longevity of the wireless sensor network, but also improved system quality.

The paper is organized as follows:
Section~\ref{fund} presents the fundamentals of solar energy harvesting and energy-aware task scheduler.
Section~\ref{design} discusses the design of the harvesting circuit under the perspective of low power wireless platforms.
Section~\ref{case} presents the evaluation of the harvesting circuit and a case study showing the improvements on system quality.
Finally, section~\ref{concl} closes the paper.

% ------------------------------------------------------------------------------


% -----------------------------------------------------------------------------
% Section <Java requirements for embedded systems>
% \label{sec:related_work}
% -----------------------------------------------------------------------------
\section{Trabalhos relacionados}
\label{sec:related_work}
% Nosso trabalho aborda a questão de como prover acesso a dispositivos de hardware
% ao Java e como prover este acesso de uma forma bem estruturada levando em 
% consideração todos os requisitos do cenário de sistemas embarcados.
%
A linguagem de programação Java é desprovida do conceito de \emph{ponteiro}, 
presente em linguagens como C e C++. 
O endereço das \emph{variáveis de referência}, utilizadas para acessar objetos Java,
é conhecido apenas pela JVM, a qual trata de todos os acessos à
memória. Como a maioria dos dispositivos de hardware são mapeados em endereços de
memória, acessá-los diretamente é um problema para a linguagem Java. 
FFI é a abordagem
utilizada por Java para superar esta limitação uma vez que ela permite ao Java
utilizar construções, como ponteiros C/C++, para acessar diretamente dispositivos
de hardware.
FFIs também tem sido utilizadas por plataformas Java na reutilização de código
escrito em outras linguagens de programação como C e C++ e para embarcar JVMs em 
aplicações nativas permitindo as mesmas acessar funcionalidades 
Java \cite{Liang:1999}.
%(\cite{Liang:1999}, \cite{1288968}).

\emph{Java Native Interface} (JNI) é a principal FFI Java, a qual é utilizada na
plataforma \emph{Java Standard Edition} \cite{Liang:1999}. 
Na JNI, a interface entre código nativo e Java é realizada durante o tempo de 
execução do programa. Isto significa que, durante a execução de um programa, 
a JVM procura e carrega a implementação dos métodos marcados como nativos 
(métodos que possuem a palavra reservada \emph{native} em suas assinaturas).
Usualmente a implementação dos métodos nativos é armazenada em uma biblioteca 
ligada dinamicamente.
Este mecanismo de busca e carga de métodos aumenta a necessidade de memória em
tempo de execução e o tamanho da JVM. Por esta razão eles são evitados em 
sistemas embarcados.

% NOTA: Não estou certo destas limitações da KNI, apesar de estarem na 
% especificação da mesma.
A plataforma \emph{Java Micro Edition} (JME) utiliza uma FFI ``leve'',
chamada de \emph{K Native Interface} (KNI) \cite{_k_2002}. 
A KNI não carrega métodos nativos dinamicamente na JVM, evitando o sobrecusto 
de memória da JNI. 
Na KNI a interface entre Java e código nativo é realizada estaticamente, 
durante o tempo de compilação. 
Entretanto decisões de projeto da KNI impõem algumas limitações. 
A KNI proíbe a criação de objetos Java (exceto de strings) a partir do código 
nativo. 
% É proibida também a chamada de métodos Java, a partir do código nativo. 
Além disto, na KNI os únicos métodos nativos que podem ser invocados são aqueles 
pré-compilados na JVM. 
Não há uma Interface de Programação de Aplicações 
(API - Application Programming Interface) em nível Java para invocar outros 
métodos nativos. % cuidado com esta afirmação. Frase original: There is no Java-level API to invoke others native methods.
Como consequência, é difícil de criar novos controladores de dispositivos de 
hardware utilizando-se a KNI.

% NOTA: Os último argumento, de chamar Java a partir do C... é fraco
A FFI da KESO, utilizada neste trabalho, foca em sistemas embarcados. 
Assim como a KNI, a FFI da KESO não realiza carga dinâmica de métodos nativos.
Entretanto, diferentemente da KNI, a FFI da KESO provê aos programadores uma 
API em nível Java para criação de novas interfaces com código nativo. 
Também não exite problema do código nativo chamar código Java, uma vez que 
KESO e a FFI da KESO geram código C.

O tarefa de escrita de adaptadores para código nativo pode ser facilitada de
duas maneiras, por APIs de alto nível e por ferramentas geradoras. As APIs de
alto nível fornecem métodos específicos para auxiliar na criação desses adaptadores,
enquanto as ferramentas geradoras podem gerar parte de adaptadores ou adaptadores
completos a partir de análise de código nativo ou a partir de uma especificação
em mais alto nível.
SWIG e a biblioteca de função estrangeira de Python \emph{ctypeslib} são exemplos
de ferramentas que geram adaptadores a partir de arquivos \emph{headers} C/C++
como entrada. O primeiro suporta diversas linguagens como saída, como por exemplo
Python, D e Java. O segundo foca em programas Python 
\cite{swig-site},\cite{ctypeslib-site}.
Ravit et al. apresenta uma ferramenta que tem como objetivo prover funcionalidades
da linguagem de mais alto nível (tuplas, por exemplo) para serem utilizadas no 
código dos adaptadores. A ferramenta proposta por Ravit et al. gera adaptadores
Python a partir de código escrito em C e descrições de interface, as quais 
contêm, dentre outras, informações sobre funções e seus respectivos parâmetros
\cite{Ravitch:2009:AGL:1542476.1542516}.
Outras soluções, como a linguagem \emph{Jeannie}, misturam código C e Java em 
um único programa a partir do qual geram adaptadores JNI 
automaticamente \cite{1297030}.
A FFI da KESO, utilizada neste trabalho, provê uma API baseada em aspectos para
ajudar na criação de adaptadores. É possível especificar quais pontos do programa
Java serão afetados pela criação dos adaptadores, assim como qual código deve
ser gerado para cada ponto do programa Java a ser afetado.

% The binding code can be checked for correctness and bug detection. Tools such 
% as J-BEAM and Ilea perform bug checking based on the source code, using static
% analysis techniques\cite{jbeam:2008}, \cite{Ilea:2007}.
% Lee et al. deal with bug detection dynamically, when the FFI code is been 
% used \cite{Lee:2010:JSD:1809028.1806601}.
% Their tool Jinn synthesizes dynamic bug detectors for FFIs from Finite State 
% Machines whose encode FFI constrains that should be tested. The FFI targeted 
% for the Java language is the JNI which contains hundreds os API calls. 
% Although not aborted by this paper, similar bug detectors can be adapted to 
% check bindings generated using the KESO FFI.


% -----------------------------------------------------------------------------



% -----------------------------------------------------------------------------
% Section <KESO>
%\input{tex/keso}

% -----------------------------------------------------------------------------
% Section <proposal>
%\section{Building a Trustful Infrastructure for Future Internet}
\label{sec:solution}
The Internet architecture demonstrate inefficiency and problems in several and large areas, such as mobility, real-time applications,
failures (e.g. equipment, software bugs, and configuration mistakes), and especially in pervasive security problems \cite{Rexford:2010}.
Moreover, the Internet lacks effective solutions in terms of scalability and sustainability, 
consuming much more energy and hindering the management of countless sensor devices that are so important for several applications in the Future Internet.
Hence, we propose the use of a stack of communication protocols (UDP@NDN@C-MAC), in the scope of the EPOSMote project,
designed specifically to guarantee a trustful communication
%Our solution also includes EPOSMote II, an embedded platform. Thus, 
while still compromised with the low utilization of resources (processing, memory, power and communication bandwidth).
%and the use of EPOSMote II which is an embedded platform and represents a typical Future Internet device.

\subsection{EPOSMote}
The EPOSMote is an open hardware project~\cite{eposmote}. Initially it aimed at 
the development of a wireless sensor network module, and focused on environment 
monitoring. Its first version, the EPOSMote I, features an 8-bit AVR microcontroller, 
IEEE 802.15.4 communication capability and a small set of sensors.

As the project evolved a second version arose, with the objective of delivering a 
hardware platform to allow research on energy harvesting, biointegration, and 
MEMS-based sensors. The EPOSMote II focus on modularization, and thus is composed 
by interchangeable modules for each function.

Figure \ref{emote2-block_diagram} shows an overview of the EPOSMote II architecture.
Its hardware is designed as a layer architecture composed by a main module,
a sensoring module, and a power module. The main module is responsible for processing
and communication. It is based on the Freescale MC13224V microcontroller~\cite{mc13224v}, which possess 
a 32-bit ARM7 core, an IEEE 802.15.4-compliant transceiver, 128kB of flash memory, 80kB of ROM memory
and 96kB of RAM memory. We have developed a startup sensoring module, which contains some sensors  
(temperature and accelerometer), leds, switches, and a micro USB (that can also be used as power supply). 
Figure \ref{emote2-mc13224v-pictures-real_white_background} shows the development kit which is slightly 
larger than a R\$1 coin, on the left the sensoring module, and on the right the main module.

\fig{.45}{emote2-block_diagram}{Architectural overview of EPOSMote II.}

\fig{.07}{emote2-mc13224v-pictures-real_white_background}{EPOSMote II SDK side-by-side with a R\$1 coin.}

\subsection{C-MAC}
C-MAC is a highly configurable MAC protocol for WSNs realized as a framework of
medium access control strategies that can be combined to produce
application-specific protocols~\cite{steiner:2010}. It enables application
programmers to configure several communication parameters (e.g.  synchronization,
contention, error detection, acknowledgment, packing, etc) to adjust the protocol
to the specific needs of their applications. The framework was implemented in C++ 
using static metaprogramming techniques (e.g. templates, inline functions, and 
inline assembly), thus ensuring that configurability does not come at expense of 
performance or code size. The main C-MAC configuration points include:

\textbf{Physical layer configuration:} These are the configuration points defined
by the underlying transceiver (e.g. frequency, transmit power, date rate).

\textbf{Synchronization and organization:} Provides mechanisms to send or receive
synchronization data to organize the network and synchronize the nodes duty
cycle.

\textbf{Collision-avoidance mechanism:} Defines the contention mechanisms used to
avoid collisions. May be comprised of a carrier sense algorithm (e.g. CSMA-CA),
the exchange of contention packets (\emph{Request to Send} and \emph{Clear to
Send}), or a combination of both.

\textbf{Acknowledgment mechanism:} The exchange of \emph{ack} packets to
determine if the transmission was successful, including preamble acknowledgements.

\textbf{Error handling and security:} Determine which mechanisms will be used to
ensure the consistency of data (e.g. CRC check) and the data security.

The Future Internet will be composed by a wide range of both applications and devices, 
each with its own requirements and available resources. Through C-MAC configurability we
can provide the most adequate MAC functionalities for each case, instead of providing a 
general non-optimal solution for all of them.

\subsection{NDN}
Communication in NDN is impelled by the data consumers.
Nodes that are interested in a content transmit \emph{Interest} packets, which contains the name of the requested data. %selector, nonce
Every node that receives the \emph{Interest} and have the requested data can respond with a \emph{Data} packet that follows back the path from which the \emph{Interest} came. %content name, signature, signed info, data
It is important to notice that one \emph{Data} satisfies one \emph{Interest}, thus ensuring flow balance in the network.
Since the content being exchanged is identified by its name, all nodes interested in the same content can share transmissions (considering a broadcast medium, which is the case for most Future Internet devices).

NDN packet forwarding engine has three main data structures: the FIB (Forwarding Information Base), which is used to forward \emph{Interest} packets to potential sources; 
the ContentStore, which is a buffer memory used to maximize the sharing of packets; 
and the PIT (Pending Interest Table), which is used to keep track of \emph{Interest} packets so that \emph{Data} packets can be sent to its requester(s).

When a node receives an \emph{Interest} packet it searches for its content name, looking for a match primarily at the ContentStore, then the PIT, and ultimately at the FIB.
If there is a match at the ContentStore, it is sent and the \emph{Interest} discarded.
Otherwise, if there is a match at the PIT, the set of requesting interfaces for that data is updated, and the \emph{Interest} discarded (at this point an \emph{Interest} in this data has already been sent).
Otherwise, if there is a match at the FIB, the \emph{Interest} is sent towards the data, and a new PIT entry is created. 
In case there is no match for the \emph{Interest} then it is discarded.

As for the \emph{Data} packet they simply follow the chain of PIT entries back to the original requester(s).
When a node receives a \emph{Data} packet it searches for its content name. 
If there is a ContentStore match, then the \emph{Data} is a duplicate and is discarded.
%A FIB match means there are no matching PIT entries, so the \emph{Data} is unsolicited and it is discarded.
In case of a PIT match, the data is validated, added to the ContentStore, and sent to the set of requesting interfaces from the corresponding PIT entry.

In NDN the name in every packet is bound to its content with a signature.
This enables data integrity and provenance, allowing consumers to trust the data they receive regardless of how the data came to them.
To provide content protection and access control NDN uses encryption.
The encryption of content or names is transparent to the network, since to NDN it is all just named binary data.
%The signature algorithm used may be selected by the content publisher, 
%and chosen to meet performance requirements such as latency or computational cost of signature generation or verification.
Nevertheless, NDN does not mandate any particular key distribution scheme, signature, or encryption algorithm.

\subsection{UDP}
The User Datagram Protocol has been chosen for its simplicity. Its simple transmission 
model avoids unnecessary overhead, since it does not handle reliability, ordering, 
and data integrity, leaving these characteristics to be treated in other layers if necessary, which is a 
perfect blend with the rest of our protocol stack.

% -----------------------------------------------------------------------------
\section{Interface entre dispositivos de hardware e linguagens de altíssimo nível}
\label{sec:proposal}
% <N> + Fala do problema de acessar HW. Falar que usamos mediadores HW EPOS e FFI KESO.
% <N> Falar também que UART é usada como exemplo.
%
% + Qual é a proposta?
Nossa proposta para realizar a interface entre linguagens de programação de
altíssimo nível e dispositivos de hardware é baseada em exportar estes 
dispositivos para a API da linguagem. 
Os dispositivos de hardware a serem exportados possuem uma interface bem 
definida, utilizando o conceito de mediadores de hardware da ADESD. 
Nós utilizamos a interface de função estrangeira da JVM KESO de forma a exportar
mediadores EPOS para Java. 
Esta seção explica como nós abstraímos componentes de hardware para serem 
utilizados por aplicações Java embarcadas. A utilização da abordagem é 
exemplificada abstraindo-se para Java um dispositivo de hardware 
transmissor-receptor assíncrono universal 
(UART - \emph{Universal Asynchronous Receiver Transmitter}).

%\subsection{Hardware devices access}
% + Conceito de HW mediators do EPOS 
O EPOS utiliza o conceito de mediadores de hardware para abstrair 
especificidades de dispositivos de hardware distintos. 
Os mediadores de hardware sustentam um contrato de interface entre as abstrações
de sistema (e.g. \emph{threads}) e o hardware, permitindo à estas abstrações 
independência de plataforma \cite{Polpeta:2004}.
A geração da implementação do mediador para uma plataforma específica é executada 
em tempo de compilação. 
Utilizando-se técnicas de meta programação e \emph{inlining} de funções é 
possível dissolver os mediadores entre as abstrações que os utilizam, o que 
evita sobrecusto de tempo no uso de mediadores.

% KESO FFI
A KESO provê uma interface de função estrangeira para realizar a interface entre
código Java e código C ou C++. 
A FFI da KESO utiliza uma abordagem estática assim como a KNI da Sun, realizando
a interface entre Java e código nativo em tempo de compilação. 
A interface de função estrangeira da KESO é chamada de 
\emph{KESO Native Interface (KNI)}, mas neste artigo ela é referida como 
FFI da KESO para evitar confusões com a KNI da Sun.

% AOP concepts
O projeto da FFI da KESO adota alguns conceitos de Programação Orientada a 
Aspectos (AOP - Aspect-Oriented Programming). 
%\cite{Kiczales97aspect-orientedprogramming}. 
Utilizando a FFI da KESO é possível ``escrever'' pontos de corte 
(\emph{point cuts}) especificando os pontos de junção (\emph{join points}) de um
programa Java (como por exemplo, métodos e classes Java) que irão ser afetados 
pelos conselhos (\emph{advices}) fornecidos. 
Um \emph{advice}, neste caso, é o código que representa a implementação do um 
método nativo. 
A figura \ref{fig:weavelet_extension_generic} ilustra a utilização da FFI da
KESO.
Os aspectos (\emph{aspects}), os quais agrupam os \emph{point cuts} e os 
\emph{advices} são representados na API da FFI da KESO pela classe 
abstrata \emph{Weavelet}. 
Estendendo a classe \emph{Weavelet} e implementando alguns de seus métodos, é
possível especificar quais métodos e classes Java serão afetados e qual código 
nativo deve ser gerado. 

\fig{.3}{weavelet_extension_generic}{Utilizando a FFI KESO}

% + É o parágrafo principal de todo o paper.
% Diz que usamos KESO FFI para criar um binding para cada mediador EPOS.
% Diz que escrevemos Weavelets para especificar o código gerado. E mais ou menos
%  como eles operam: pointcuts, advices, wove...
% Mostra a figura: acessando HW devices
Nós utilizamos a FFI do KESO para criar um adaptador para cada mediador EPOS que
deve ser acessado pelo Java, provendo Java com componentes de hardware. 
A abordagem utilizada é mostrada na figura \ref{fig:accessing_hw}. 
A classe Java, a qual representa a contraparte Java do mediador de hardware que 
está sendo abstraído, especifica assinaturas de métodos mas não os implementa, 
uma vez que tais métodos representam métodos nativos.
Então, uma classe \emph{weavelet} da FFI do KESO é utilizada para especificar a
implementação de cada método nativo. 
Mais especificamente, a classe \emph{weavelet} especifica quais métodos da 
classe Java deseja-se interceptar 
(o equivalente aos \emph{pointcuts} de uma linguagem de AOP) e quais os 
respectivos códigos que devem ser gerados 
(o equivalente aos \emph{advices} em AOP). 
Durante a tradução de bytecode Java em C, o compilador da KESO entrelaça (\emph{wove})
as assinaturas dos métodos nativos da classe Java com os \emph{advices} 
especificados pelo \emph{weavelet}, gerando o código dos adaptadores que realizam
a interface entre a classe Java e o mediador de hardware EPOS.

\fig{.2}{accessing_hw}{Interface Hardware/Java} 

% + KESO FFI API/Framework
A implementação de cada método nativo, especificada no \emph{weavelet}, é 
basicamente uma chamada para cada método do mediador EPOS que está sendo 
exportado ao Java.
Na implementação do método \emph{affectMethod} é possível especificar um padrão
o qual representa um elemento do programa Java
(uma assinatura de método, por exemplo), assim como qual código deverá ser gerado 
quando o compilador KESO reconhece tal padrão.
O trecho de código da figura \ref{prgjava:weav_uart_put} mostra a especificação de um
adaptador de código nativo a ser gerado para um método virtual chamado 
de \emph{m1}. 
Este método possui um único parâmetro do tipo carácter e não possui retorno
(tipo \emph{void}).
O \emph{eposHWMediator} é um campo da classe adaptadora o qual aponta para o
objeto mediador de hardware do EPOS. É possível adicionar campos nas classes
adaptadoras implementando-se o método \emph{addFields} da classe abstrata 
\emph{Weavelet}.

\prgjava{weav_uart_put}{Especificando os adaptadores de código nativo}

% + Fala que objetos EPOS podem ser destruídos pelo GC do KESO, usando 
% finalizers
Quando cria-se um objeto Java (chamando-se \emph{new}) que representa uma 
abstração de hardware, o objeto adaptador de código nativo, pode alocar objetos 
EPOS como por exemplo o \emph{eposHWMediator} do programa da figura 
\ref{prgjava:weav_uart_put}.
O objeto Java será desalocado automaticamente pelo coletor de lixo da KESO JVM
e o objeto EPOS, segundo nossa abordagem, é desalocado chamando-se o método 
\emph{finalizer} da classe Java. Como o coletor de lixo da KESO JVM utiliza
algoritmos determinísticos é garantido que os \emph{finalizers} de Java são 
sempre chamados.

% +Comentar mais vantagens do uso do KESO - otimizações 
% Nota: retirar este parágrafo para caber em 12 páginas para o sbesc 2011?
A FFI da KESO é integrada com o compilador KESO então, durante a compilação de
bytecode Java em C, instâncias de classes \emph{weavelet} são criadas e 
utilizadas na geração de código nativo. Apesar de que o código especificado por 
uma \emph{weavelet} não é objeto das análises estáticas executadas pelo 
compilador KESO, a FFI da KESO ainda apresenta algumas vantagens interessantes,
as quais nos motivaram a utilizá-la. 
Por exemplo, se o compilador KESO identifica que o código da aplicação não 
utiliza algum método nativo, ele não gera o código nativo para aquele método, 
reduzindo o consumo de memória, o que é altamente desejável em um cenário de
sistemas embarcados.

%\subsection{UART example}
% + Escrevemos uma aplicação de UART. 
% * Figura código Java da aplicação
Nós escrevemos uma pequena aplicação utilizando o mediador de hardware UART para
ilustrar nossa proposta de abstração de dispositivos de hardware para Java
embarcado.
A aplicação, mostrada pelo programa da figura \ref{prgjava:application}, utiliza
a UART para escrever caracteres em um dispositivo serial.

\prgjava{application}{Exemplo UART}

% + O que é Java, o que é C/C++
% * Figura aplicação UART (parte Java e parte C/C++ área cinza escuro e cinza claro)
A classe Java \emph{UART} é a contraparte Java da classe UART do mediador de 
hardware EPOS e possui apenas métodos nativos sem quaisquer implementação.
Então, o \emph{UART\_Weavelet} é utilizado para especificar a implementação de
cada método da \emph{UART}. A abordagem é a mesma apresentada pelo programa da
figura \ref{prgjava:weav_uart_put}.
A figura \ref{fig:overall_uart_application}, por sua vez, mostra a arquitetura
geral do programa exemplo UART, mostrando as principais classes envolvidas e
distinguindo a parte de código Java (abaixo da área cinza-escuro) da parte de
código C/C++ (abaixo da área cinza-claro). Pode-se observar que as classes em
Java correspondem à quase todo código escrito, incluindo a classe da aplicação
\emph{UART\_Test}, a classe \emph{UART} do lado Java e o \emph{UART\_Weavelet}.
% Our application 
% class \emph{UART\_Test} extends the KESO class \emph{Task}. This means that an
% instance of our class will be implemented by an OSEK task created by KESO, 
% during the system generation. The \emph{launch} method is the task's entry point.

\figtwocolumn{.3}{overall_uart_application}{Arquitetura do programa UART}


% -----------------------------------------------------------------------------



% -----------------------------------------------------------------------------
% Section <Evaluation>
% ------------------------------------------------------------------------------
\section{Practical Experiments} \label{eval}
% + Practical Experiments
% First shows speedup and quality results for DMEC
% using 1 (without partitioning) to 6 worker threads.
% Show that speedup is high and quality is kept acceptable.
% 
% Then, show speedup and quality results for DMEC integrated to JM and compares
%  to the original JM (and, if possible, to other works).
% 

% + JM
% + Dizer como realizamos os experimentos. E/ou quais as variáves observadas:
%   Evaluate the component in isolation DMEC to show its speedup.
%    And how it scales.
%   Evaluate the component in JM to show sppeedup and PSNR.
We have evaluated DMEC in two stages.
First, in order to verify how DMEC's performance scales from 
one to six \emph{Workers} instances, we have evaluated all DMEC implementations 
in a test case.
The test case application mimics the behavior of an H.264 encoder: it provides 
DMEC with pictures, obtain the ME results 
(motion vectors and motion cost), and checks if the results are correct.
Secondly, in order to assess DMEC influence on the final video quality, we have
evaluated all DMEC implementations in the
JM H.264 Reference Encoder~\cite{site:jm}.
The PSNR degradation is computed as the absolute PNSR difference between the
original encoder and the optimized ones.

% P: Dizer pq focamos em luma
For inter macroblock modes in H.264 (i.e. modes related to the ME),
the motion cost for chrominance components derives from the motion cost for 
luminance components~\cite{1101854}. 
Consequently the PSNR for chrominance components derives from the PSNR for 
luminance components. 
For this reason, in this paper we focus on the PSNR variation of the luminance 
component.

Figure \ref{fig:dmec-speedup_workers} show the speedup of DMEC in there
test case application with a different number of \emph{Workers} instances.
For such test, we have used an arbitrary set of pictures with a resolution of
1080p (Full-HD).
The speedup is normalized to one \emph{Worker} instance (speedup of 1X).

% TODO
% \textit{Comments about results in: Cell BE, Muticore IA32, and HW.}
It is worth to mention that for each number of \emph{Worker} instances
a different partition mode was used, according to
Figure \ref{fig:picture_partition}.
For one \emph{Worker} instance we have used the ``Single Partition'',
for two \emph{Worker} instances we have used the ``2x1'' partition and so on,
up to the ``2x3'' partition mode (used for six \emph{Worker} instances).

Besides the additional performance obtained by using a higher number of \emph{Worker} instances,
the partition mode also has influence on the speedup.
The reason of such influence is that, during the partitioning process, the dimensions of the
search window shrinks, thus reducing the area of the picture searched for
similarities.

\fig{.45}{dmec-speedup_workers}{Time performance scalability of DMEC}

% Sobre RD curves
% Figures XXX show the speedup of DMEC while tested already integrated to JM.
% The obtained values are compared to the ones obtained while using the original
% JM, without DMEC.
In order to evaluate in details the behavior of DMEC for distinct values 
of encoding bit-rates, we have used the BD-PSNR (Bjøntegaard Delta PSNR) metric
using the following values of QP (Quantization Parameter): 16,20,24,28; as 
described in \cite{gisle_bjntegaard_calculation_2001}.
It is important to evaluate quality (PSNR) for distinct bit-rates to test 
whether the approach can be used in distinct scenarios of application.
Figure \ref{fig:crowd-bitrate_psnr} shows the rate-distortions (RD) curves using the
original JM encoder and the optimized encoder using DMEC.
The video sequence used for this curves was \texttt{Crowd Run}, a 1080p sequence with  a
high ammount of motion.
Lower values of bit-rate are obtained for higher values of QP since by using
higher values for QP more data is discarded, thus increasing the
compression ratio. 
The two curves very near from each other indicates that the DMEC
presents a good rate-distortion performance for all the evaluated bit-rates.

\fig{.45}{crowd-bitrate_psnr}{RD curve of a 1080p video sequence}

We have evaluated also the speedup obtained in the
% encoding time
ME run time
while using
DMEC for the same QP values we used for BD-PSNR.
Figure \ref{fig:crowd-bitrate_speedup} shows the obtained values while using
6 \emph{Worker} instances.
For Muticore IA32, a speedup of around 9 times is obtained for all
bit-rate values.
For Cell BE this value is about 2 times.
A small speedup for the Cell BE, while compared to Multicore IA32 and the dedicated hardware, is due
to the memory transferences (picture samples and ME results) which is performed using
the DMA requisitions of Cell BE.


\fig{.45}{crowd-bitrate_speedup}{Speedup vs bit-rate of a 1080p sequence}
%
% \multfigtwov{.65}{bd_psnr}{bd_speedup} {bd} {RD curve (a) and speedup vs bit-rate (b) of 1080p sequence}

% Discussion


% Falar do paralelismo / particionamento de dados
% Qualidade ficou boa mesmo particionando e desempenho aumentou: speedup ~ 70%
%The strategy of ME distribution based on picture partitioning has been shown 
%effective.
% We have obtained a speedup higher than XXX\% without loosing quality.
%Data partitioning is effective because the visual interdependence between
%partitions is not significant to influence on the encoding quality, and allows
%for a speedup because it enables the simultaneous processing of each picture
%partition.

% - Falar da comunicação
% - Necessidade via espaço de endereçamentos diferentes
% A arquitetura Cell Broad Band demonstrou-se uma arquitetura interessante para o
% processamento paralelo de vídeo, pois possui unidades funcionais dedicadas 
% (i.e. SPEs) para processamento de dados. 
% A principal dificuldade encontrada em trabalhar-se com o Cell foi a capacidade 
% limitada da memória local das SPEs.
% Outra dificuldade foi lidar com as transferências de memória entre SPE e PPE. 
% Isto em fato foi superado pelas estratégias que desenvolvemos de baferização e 
% também com a utilização do Element Interconect Bus (EIB) do Cell que realiza 
% DMAs com altas taxas de transferências.
% 
% - Solução 1: Buffer de preditores, contribuiu bastante
% A estratégia de armazenamento de preditores nas SPEs foi significativa no 
% aumento do desempenho, pois vetores de movimentos necessários para o cálculo da
% ME não precisam ser consultados na memória principal. É coerente a decisão de 
% manter uma cópia local destes vetores, pois todos os vetores que a ME irá 
% precisar foram calculados pela partição em questão e por nenhuma outra.


% ------------------------------------------------------------------------------


% -----------------------------------------------------------------------------
% Section <Real-word application>
% ------------------------------------------------------------------------------
\section{Aplicação real}
\label{sec:case_study}
% Distributed Motion Estimation for H.264 encoding.
% 
% - Tells about the role of ME in H.264 
% -  Shows our model of distributed ME component 
% Wrapper do Coordenador
% Wrapper do Worker
% Wrapper do Cronometro
% //- Used mediators/components: DMA mediator (used to exchange data)
% //Barrier mediator/abstraction (used for synchronization between Coordinator and Workers threads)
% 
% // + Evaluation
% 
% Avaliações específicas do encoder ME na seção de estudo de caso mesmo.
% - More general: How many frames per second our component process? 
% (comparation with JM encoder)
% 
% - More specific: Time overhead of wrappers 
% (comparation with JNI, and others)
%
Como forma de avaliar nossa proposta em uma aplicação real, nós desenvolvemos
um componente Java o qual computa estimativa de movimento 
(ME - \emph{Motion Estimation}) para codificação de vídeo H.264.
ME é utilizada para explorar a similaridade entre imagens
vizinhas em uma sequencia de vídeo, permitindo que elas sejam codificadas 
diferencialmente, aumentando a taxa de compressão do 
\emph{bitstream} gerado. % \cite{citeulike:1269699}.
ME é um estágio significante da codificação H.264, pois ele consome 
aproximadamente 90\% do tempo total do processo de 
codificação \cite{XiangLi:2004}.

Visando aumentar o desempenho da ME, nosso componente utiliza uma estratégia de
particionamento de dados, aonde a estimativa de movimento de cada partição da
imagem é realizada em paralelo pelo módulo \emph{Worker}, o qual executa em
uma unidade funcional específica, como por exemplo um núcleo de um processador
multinúcleo.
Existe também o módulo \emph{Coordinator}, o qual é responsável por definir a
partição de imagem para cada instância do módulo \emph{Worker} e por prover a
eles as imagens a serem processadas.
O \emph{Coordinator} é também responsável por agrupar os resultados gerados pelos
módulos \emph{Worker} (i.e. custos de movimento e vetores de movimento) e por
entregar estes resultados de volta para o codificador. 
A figura \ref{fig:threads} ilustra a interação entre os módulos 
\emph{Coordinator} e \emph{Worker}.

\figtwocolumn{.25}{threads}{Interação entre \emph{Coordinator} e \emph{Workers}}

Nosso componente é chamado de \emph{Distributed Motion Estimation Component} - 
DMEC (Componente de Estimativa de Movimento Distribuída), uma vez que o 
cálculo da ME é realizado de forma paralela pelos módulos \emph{Worker}.
Entretanto, esta complexidade é ocultada da aplicação Java 
(e.g. codificador H.264), o qual enxerga apenas um componente para calculo de ME,
realizada pelo método \emph{match}.
O DMEC é implementado como um componente em C++ e é exportado para Java utilizando
a estratégia apresentada na seção \ref{sec:proposal}, aonde é desenvolvido um 
adaptador de código nativo para cada objeto sendo abstraído.

Atualmente o DMEC é implementado por componentes de software, aonde os módulos
\emph{Coordinator} e \emph{Workers} são threads executando em núcleos distintos
de um processador multinúcleo.
Apesar disso o DMEC pode ser implementado por componentes de hardware 
preservando as mesmas interfaces disponíveis na versão em software.
Isso pode ser obtido por meio o uso do conceito de \emph{componentes híbridos}
do EPOS. 
%\cite{Marcondes:IESS:2009}.
Neste caso, nossos adaptadores de código nativo também permanecem os mesmos.
Em um cenário de implementação em hardware \emph{Coordinator} e \emph{Worker}
são IPs (\emph{Intelectual Properties}) de um multiprocessador em chip
(MPSoC - \emph{Multiprocessor System-on-Chip}) e a comunicação entre eles é realizada
por um sistema de interconexão em chip, como por exemplo os descritos por 
\cite{Javaid:2010:OSL:1878961.1878978} e \cite{Popovici:2009:FAC:1509633.1509681}.

% + Dizer que foi feita uma aplicação em Java que realiza um unit test do 
% DMEC, simulando um encoder H.264
Nós escrevemos uma aplicação 100\% Java para utilizar nosso componente DMEC.
Do ponto de vista da ME, a aplicação desenvolvida atua como um codificador
H.264: ela provê ao DMEC imagens para serem processadas e utiliza os resultados
entregues pelo componente verificando se eles estão corretos.
O programa da figura \ref{prgjava:app} mostra o principal método da aplicação.
O DMEC é utilizado como um objeto Java usual.
% NOTA: dizer mais coisa aqui, falar do Wrapper, fazer propaganda.

\prgjava{app}{Aplicação Java DMEC}

% Dizer os resultados
% // + Evaluation
% 
% Avaliações específicas do encoder ME na seção de estudo de caso mesmo.
% - More general: How many frames per second our component process? 
% (comparation with JM encoder)
% 
% - More specific: Time overhead of wrappers 
% (comparation with JNI, and others) pode ficar na secão anterior, não?
%
% TODO: Dizer os resultados. 
% \begin{itemize}
% \item Tempo de processamento dos frames. (Maybe comparation with JM encoder)
% \item Time overhead of wrappers (comparation with JNI, and others)
% \item Application size and footprint
% \end{itemize}

% ------------------------------------------------------------------------------



% -----------------------------------------------------------------------------
% Section <Discussion>
% ------------------------------------------------------------------------------
\section{Conclusões}
\label{sec:discussion}
% Aplicações para sistemas embarcados usualmente necessitam interagir com
% diversos tipos de dispositivos de hardware como sensores, atuadores, 
% transmissores, receptores e \emph{timers}.
%
% Interface de função estrangeira é o mecanismo adotado por Java para superar as
% limitações da linguagem e permitir acesso direto a memória e a dispositivos de
% hardware. 
% Entretanto, como mostrado na seção \ref{sec:related_work}, as principais FFIs 
% Java não são eficientes em termos de consumo de recursos, ou possuem limitações
% de projeto que dificultam o desenvolvimento de novas interfaces entre Java e os
% dispositivos de hardware.
%
Neste artigo, apresentou-se um meio de realizar a interface entre componentes 
de hardware e aplicações Java para sistemas embarcados. 
Isto foi obtido utilizando-se a interface de função estrangeira da JVM KESO e o
EPOS.

% O EPOS permite o desenvolvimento de aplicações portáveis, independentes de
% especificidades de máquina. 
% Isto é conseguido utilizando-se o conceito de mediadores de hardware, os quais
% sustentam um contrato de interface entre abstrações de sistemas e a máquina.

% O JVM KESO compila o bytecode de uma aplicação Java em código C e gera as partes
% da JVM necessárias pela aplicação. 
% A FFI da KESO também utiliza esta abordagem estática, gerando o código C 
% especificado nas classes \emph{Weavelet}. 
% Então o código C gerado pelo compilador KESO e pela FFI da KESO são compilados
% em conjunto em código nativo, utilizando-se um compilador C padrão.

Nós avaliamos nossa abordagem em termos de desempenho, consumo de memória e 
portabilidade.
Para a aplicação utilizando o mediador de hardware da UART o sobrecusto 
de tempo obtido foi menos de 0.04 \% do tempo total de execução da aplicação e
nossa solução é 38 vezes mais rápido do que a JNI da Sun.
O consumo de memória para tal aplicação foi de 33KB, incluindo todo o suporte
de ambiente de execução, o qual é adequado para diversos sistemas embarcados.
Utilizando o EPOS nós obtivemos portabilidade para várias plataformas e 
utilizando o conceito de componentes híbridos podemos utilizar os mesmos 
adaptadores de código nativo tanto para componentes implementados em 
hardware como implementados em software.

Visando avaliar nossa abordagem em uma aplicação real, nós escrevemos
adaptadores de código nativo para um componente o qual realizada estimativa de
movimento para codificação de vídeo H.264.

% ------------------------------------------------------------------------------



% -----------------------------------------------------------------------------
% References
\bibliographystyle{sbc}
\bibliography{hw,os,pl-pt-BR,mm}


\end{document}

%------------------------------------------------------------------------------



