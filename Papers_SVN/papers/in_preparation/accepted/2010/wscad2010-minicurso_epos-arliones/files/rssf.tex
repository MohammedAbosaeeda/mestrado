\section{\Rssf}
\label{sec:rssf}

%introdução
Avanços recentes nos projetos de dispositivos eletrônicos e a miniaturização
levou ao surgimento de um novo conjunto de aplicações para computadores na forma
de microsensores sem-fio de baixa potência. Estes microsensores são equipados
com dispositivos de sensoriamento analógicos ou digitais (e.g., temperatura,
campo magnético, som), um processador digital, um transceptor de comunicação
sem-fio (e.g., rádio de baixa potência, infra-vermelho) e um módulo de
alimentação (e.g., bateria, célula foto-sensível). Cada sensor, individualmente,
é capaz de obter uma visão local de seu ambiente e de coordenar e se comunicar
com outros sensores para criar uma visão global do objeto de estudo alvo da
aplicação.

\fig{wsn}{\Rssf.}{width=\columnwidth}

A ideia de uma rede auto-gerenciada composta por dispositivos autônomos que
coletam e enviam dados através de um enlace sem-fio traz à tona uma série de
novos desafios ao projeto das plataformas (hardware). Para que não sejam
intrusivos e operem autonomamente por longos períodos de tempo, os nodos
sensores precisam ser pequenos e consumir pouca energia. Além disso, nodos
sensores precisam ser modulares e permitir uso de diferentes tipos de
dispositivos sensores para permitir que uma única plataforma possa ser empregada
nos diversos tipos de aplicações, podendo assim serem adaptados de acordo com as
especificidades de cada aplicação. De modo similar, o hardware de comunicação
deve permitir ampla configuração do canal de dados, possibilitando que
diferentes aplicações se beneficiem de diferentes esquemas de modulação ou de
controle de acesso ao meio (MAC). Com o aumento da complexidade das tecnologias
de \rssf a necessidade de software para suporte à execução composto por sistemas
operacionais e componentes abstratos de alto nível (e.g., middleware) se torna
essencial.

Neste contexto, desde o ano de 2003 o \lisha (Laboratório de Integração
Software/Hardware) vem trabalhando com \rssf. Trabalhos desenvolvidos no \lisha
no escopo do Projeto EPOS~\cite{Project:Emote:2010} desenvolveram suporte de
sistema operacional para a abstração dos mecanismos de:
\begin{itemize}
    \item aquisição de dados~\cite{Wanner:ETFA:2006};
    \item comunicação~\cite{Wanner:IESS:2007};
    \item gerência do consumo de energia~\cite{Hoeller:WSO:2006};
    \item outros serviços comuns de sistema operacional como alocação de
    memória e escalonamento de tarefas~\cite{Marcondes:ETFA:2006};     
\end{itemize}

Mais recentemente, o \lisha passou a dedicar esforços no desenvolvimento de
plataformas próprias de \rssf, o que culminou com o projeto dos módulos de
sensoriamento \emote I, baseado em uma arquitetura \textsc{Avr} de 8 bits, em
2009 e \emote II, baseado em uma arquitetura \textsc{Arm7} de 32 bits, em 2010.

Este curso prevê uma revisão dos aspectos das tecnologias de \rssf, buscando
descrever as tecnologias e apresentar como estas foram tratadas tanto na
implementação do sistema operacional \epos quanto no projeto dos módulos de
sensoriamento \emote I e II. Para isso, esta primeira seção apresenta uma
contextualização da tecnologia seguido de exemplos de aplicações das \rssf. A
seção~\ref{sec:modulos} apresenta requisitos comumente procurados em módulos de
\rssf e apresenta alternativas de módulos disponíveis. A seção~\ref{sec:sensores}
apresenta tipos de sensores, métodos de interfaceamento e técnicas de aquisição
de dados. A seção~\ref{sec:MAC} revisa técnicas de controle de acesso ao meio,
apresenta topologias e descreve alguns dos MACs mais utilizados em \rssf. A
seção~\ref{sec:so} apresenta características de sistemas operacionais para
\rssf. A seção~\ref{sec:exercicios} lista alguns exercícios para utilização do
\epos na plataforma \emote.

%aplicações
\subsection{As aplicações de \rssf}

Novas tecnologias, sistemas ou plataformas surgem, quase que exclusivamente, por
um único motivo: atender a uma determinada demanda, ou seja, uma aplicação! No
caso das \rssf não poderia ser diferente. A criação das \rssf foi motivada por
aplicações militares como vigilância de campos de batalha e são hoje utilizados
em diversas áreas de aplicações civis e industriais, incluindo monitoramento e
controle de processos industriais, supervisão de maquinário, monitoramento de
ambientes ou habitats, automação doméstica, entre outras.

Esta variada gama de aplicações implica na existência de grande variação no
conjunto de requisitos que uma \rssf precisa atender, incluindo requisitos
possivelmente conflitantes, ou seja, características das quais um determinado
tipo de aplicação se beneficia, eventualmente, pode tornar o uso da tecnologia
proibitiva a outra aplicação. Para dar conta desta variabilidade, \rssf precisam
ser amplamente configuráveis. E, para ser configurável, um sistema de \rssf
precisa levar em consideração as características específicas dos diferentes
tipos de aplicações existentes.

\fig{mottola_taxonomy}{Uma taxonomia para aplicações de
\rssf~\cite{Mottola:2010}}{width=\columnwidth}

\tab{mottola_classification}{Exemplos de aplicações classificadas segundo a
taxonomia proposta por Mottola~\cite{Mottola:2010}.}

Motolla e Picco~\cite{Mottola:2010} publicaram um excelente estudo em que,
dentre outras coisas, classificaram diferentes aplicações de \rssf com o
objetivo de destacar as diferenças entre estas aplicações. A
Figura~\ref{fig:mottola_taxonomy} apresenta uma taxonomia pela qual as
características das aplicações podem ser classificadas. Neste trabalho, os
autores ainda analisam uma série de aplicações publicadas em canais científicos,
classificando-as segundo a taxonomia proposta. Alguns exemplos destas
classificações estão na Tabela~\ref{tab:mottola_classification}. Cada uma das
características da taxonomia por eles proposta podem ser assim interpretadas:
\begin{itemize}
    \item \textit{Objetivo:} dependendo do objetivo das redes, isto é, apenas
    monitorar um objeto de estudo ou, além de monitorar, também atuar sobre este
    objeto, a topologia empregada pode sofrer modificações, especialmente devido
    à inclusão, neste último tipo de objetivo, de módulos atuadores capazes de
    interagir com o objeto de estudo.
    \item \textit{Padrão de interação:} é, normalmente, dependente do
    \textit{objetivo}. ``Muitos-para-um'', mais comumente empregado, é utilizado
    quando dados de diversos nodos são coletados por um nodo central.
    Comunicação ``um-para-muitos'' é normalmente utilizada para envio de
    comandos de configuração nas redes, e ``muitos-para-muitos'' é mais comum em
    situações onde há múltiplos concentradores de informação, o que geralmente
    ocorre na presença de atuadores.
    \item \textit{Mobilidade:} em configurações ``estáticas'' nenhum elemento da
    rede se move após a implantação. Em configurações com ``nodos móveis''
    alguns elementos das redes são móveis, como, por exemplo, em aplicações de
    monitoramento de habitats em que alguns nodos estão implantados em animais.
    Já em configurações com ``concentradores móveis'', normalmente, são
    indiferentes quanto à mobilidade dos demais nodos já que, neste caso, a
    diferença básica reside no fato de que a coleta de dados é realizada de modo
    oportunístico quando os concentradores se aproximam dos demais nodos.
    \item \textit{Espaço:} diz respeito à semântica dos dados recolhidos.
    Configurações ``globais'' são aquelas em que dados de sensores individuais
    não são úteis, sendo úteis apenas as informações extraídas da análise das
    medições realizadas por todos os sensores em uma rede. Já configurações
    ``regionais'' são aquelas o objeto de interesse está localizado em uma
    região limitada, sendo, neste caso, leituras individuais relevantes.
    \item \textit{Tempo:} do ponto de vista temporal a operação dos nodos em uma
    \rssf pode ser classificada em ``periódica'', quando nodos realizam
    leituras de seus sensores e enviam os dados para processamento na rede
    periodicamente, ou ``orientada a eventos'', quando nodos sensores permanecem
    em modo quiescente observando os dados lidos por seus sensores e enviam
    dados na rede apenas quando um determinado evento é detectado (e.g., o
    valor lido em um determinado sensor ultrapassa um limite pré-definido).
\end{itemize}
