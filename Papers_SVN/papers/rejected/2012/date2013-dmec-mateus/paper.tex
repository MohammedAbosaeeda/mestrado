\documentclass[10pt,conference]{IEEEtran}

\usepackage[english]{babel} % for multilingual support
% \usepackage{babel} % for multilingual support

\usepackage[utf8]{inputenc} % for use utf8

% *** GRAPHICS RELATED PACKAGES ***
%
\ifCLASSINFOpdf
  \usepackage[pdftex]{graphicx}
  \usepackage[caption=false]{subfig} % Para usar duas ou mais figuras como uma só.
  % declare the path(s) where your graphic files are
  % \graphicspath{{../pdf/}{../jpeg/}}
  % and their extensions so you won't have to specify these with
  % every instance of \includegraphics
  % \DeclareGraphicsExtensions{.pdf,.jpeg,.png}
\else
  % or other class option (dvipsone, dvipdf, if not using dvips). graphicx
  % will default to the driver specified in the system graphics.cfg if no
  % driver is specified.
  % \usepackage[dvips]{graphicx}
  % declare the path(s) where your graphic files are
  % \graphicspath{{../eps/}}
  % and their extensions so you won't have to specify these with
  % every instance of \includegraphics
  % \DeclareGraphicsExtensions{.eps}
\fi

% Commands to insert figures ---------------------------------------------------
\newcommand{\fig}[4][ht]{
  \begin{figure}[#1] {\centering\scalebox{#2}{\includegraphics{fig/#3}}\par}
    \caption{#4\label{fig:#3}}
  \end{figure}
}
% fig usage:
% \fig{<scale>}{<file>}{<caption>}
% e.g.: \fig{.4}{uml/uml_comportamental_dia}{Diagramas comportamentais da UML}
% The figure label will be "fig:" plus <file>.
% The figure file must lie in the "fig" directory.

\newcommand{\figtwocolumn}[4][ht]{
  \begin{figure*}[#1] {\centering\scalebox{#2}{\includegraphics{fig/#3}}\par}
    \caption{#4\label{fig:#3}}
  \end{figure*}
}

% Para colocar 2 figuras como uma só - dispostas horizontalmente
\newcommand{\multfigtwoh}[6][htbp]{
\begin{figure*}[#1]
  \centering
  \subfloat[]{\label{fig:#3}\scalebox{#2}{\includegraphics{fig/#3}}}
  \subfloat[]{\label{fig:#4}\scalebox{#2}{\includegraphics{fig/#4}}}
  \caption{#6}
  \label{fig:#5}
\end{figure*}
}
% e.g.
%\multfigtwoh{.65}{fig_plot_time_orig}{fig_plot_time_mod}
%{fig_plot_time_all}
%{Original (a) and modified (b) benchmarks execution time comparison.}

% Para colocar 2 figuras como uma só - dispostas verticalmente
\newcommand{\multfigtwov}[6][htbp]{
\begin{figure}[#1]
  \centering
  \subfloat[]{\label{fig:#3}\scalebox{#2}{\includegraphics{fig/#3}}}\\
  \subfloat[]{\label{fig:#4}\scalebox{#2}{\includegraphics{fig/#4}}}
  \caption{#6}
  \label{fig:#5}
\end{figure}
}
% e.g.
%\multfigtwov{.35}{fig_epos_mem_framework}{fig_epos_mem_framework_spm}
%{fig_epos_mem_framework_all}
%{EPOS memory mapping before (a) and after (b) using the new framework}


%Reduce the paper size
\setlength{\intextsep}{4pt}
\setlength{\floatsep}{4pt}
\setlength{\dblfloatsep}{4pt}
\setlength{\abovecaptionskip}{1pt}
\setlength{\belowcaptionskip}{1pt}
%\setlength{\intextsep}{-1ex} % remove extra space above and below in-line float
%Options    
%\floatsep - Space between floats. \dblfloatsep for 2 column format
%\intextsep - Space above and below in-line text floats
%\abovecaptionskip - Space above float caption
%\belowcaptionskip - Space below float caption


\begin{document}
%
% paper title
% can use linebreaks \\ within to get better formatting as desired
\title{On the Engineering of a Component \\for Distributed Motion Estimation}
%
%
% author names and IEEE memberships
% note positions of commas and nonbreaking spaces ( ~ ) LaTeX will not break
% a structure at a ~ so this keeps an author's name from being broken across
% two lines.
% use \thanks{} to gain access to the first footnote area
% a separate \thanks must be used for each paragraph as LaTeX2e's \thanks
% was not built to handle multiple paragraphs
%

\author{Authors omitted for blind review}
%\author{Mateus~Krepsky~Ludwich,
        %Alexandre~Massayuki~Okazaki,
        %Tiago~Rogério~Mück,
        %and~Antônio~Augusto~Fröhlich,~\IEEEmembership{Federal University of Santa Catarina}% <-this % stops a space
%}

% note the % following the last \IEEEmembership and also \thanks -
% these prevent an unwanted space from occurring between the last author name
% and the end of the author line. i.e., if you had this:
%
% \author{....lastname \thanks{...} \thanks{...} }
%                     ^------------^------------^----Do not want these spaces!
%
% a space would be appended to the last name and could cause every name on that
% line to be shifted left slightly. This is one of those "LaTeX things". For
% instance, "\textbf{A} \textbf{B}" will typeset as "A B" not "AB". To get
% "AB" then you have to do: "\textbf{A}\textbf{B}"
% \thanks is no different in this regard, so shield the last } of each \thanks
% that ends a line with a % and do not let a space in before the next \thanks.
% Spaces after \IEEEmembership other than the last one are OK (and needed) as
% you are supposed to have spaces between the names. For what it is worth,
% this is a minor point as most people would not even notice if the said evil
% space somehow managed to creep in.


% NOTE TODO re-enable latter
% The paper headers
% \markboth{Journal of \LaTeX\ Class Files,~Vol.~6, No.~1, January~2007}%
% {Shell \MakeLowercase{\textit{et al.}}: Bare Demo of IEEEtran.cls for Journals}
% The only time the second header will appear is for the odd numbered pages
% after the title page when using the twoside option.
%
% *** Note that you probably will NOT want to include the author's ***
% *** name in the headers of peer review papers.                   ***
% You can use \ifCLASSOPTIONpeerreview for conditional compilation here if
% you desire.




% If you want to put a publisher's ID mark on the page you can do it like
% this:
%\IEEEpubid{0000--0000/00\$00.00~\copyright~2007 IEEE}
% Remember, if you use this you must call \IEEEpubidadjcol in the second
% column for its text to clear the IEEEpubid mark.



% use for special paper notices
%\IEEEspecialpapernotice{(Invited Paper)}




% make the title area
\maketitle


\begin{abstract}
%\boldmath
The increasing availability of computational resources is enabling the
construction of optimized encoders for H.264 and other video
standards.
In such scenario, keeping the same interface between the encoding stages 
is highly desirable since it allows for the use of new optimized versions of
algorithms while keeping intact the overall design of the encoder.
In this paper we present the design and implementation of a component for
\textit{motion estimation} which keeps the same interface for all of its
three implementations: Cell BE, Multicore IA32, and dedicated hardware.
Motion estimation is an important stage for optimization since it consumes around
90\% of the total encoding time of raw video into H.264.
The proposed component is evaluated according to encoding time and PSNR, showing
that it is possible to develop an optimized component while keeping a single
interface among distinct implementations.
\end{abstract}

% IEEEtran.cls defaults to using nonbold math in the Abstract.
% This preserves the distinction between vectors and scalars. However,
% if the journal you are submitting to favors bold math in the abstract,
% then you can use LaTeX's standard command \boldmath at the very start
% of the abstract to achieve this. Many IEEE journals frown on math
% in the abstract anyway.

% Note that keywords are not normally used for peerreview papers.
\begin{IEEEkeywords}\\
I.4 [Image Processing and Computer Vision]: Compression (Coding) \\
I.3 [Computer Graphics]: Parallel processing \\
D.2 [Software Engineering]: Modules and Interfaces
\end{IEEEkeywords}

% = Categories and Subject Descriptors
% I.4 [Image Processing and Computer Vision]: Compression (Coding)
% I.3 [Computer Graphics] Parallel processing
% D.2 [Software Engineering]: Modules and Interfaces
% 
% = General Terms Design
% Algorithms
% Performance
% 
% = Keywords / Index Terms
% Motion Estimation, Components, Designing for interface, Video encoding, H.264




% For peer review papers, you can put extra information on the cover
% page as needed:
% \ifCLASSOPTIONpeerreview
% \begin{center} \bfseries EDICS Category: 3-BBND \end{center}
% \fi
%
% For peerreview papers, this IEEEtran command inserts a page break and
% creates the second title. It will be ignored for other modes.
\IEEEpeerreviewmaketitle

% ------------------------------------------------------------------------------
% ------------------------------------------------------------------------------
\section{Introduction} \label{intro}
% + Introduction
% 
% The very first letter is a 2 line initial drop letter followed
% by the rest of the first word in caps.
%
% form to use if the first word consists of a single letter:
% \IEEEPARstart{A}{demo} file is ....
%
% form to use if you need the single drop letter followed by
% normal text (unknown if ever used by IEEE):
% \IEEEPARstart{A}{}demo file is ....
%
% Some journals put the first two words in caps:
% \IEEEPARstart{T}{his demo} file is ....
%
% Here we have the typical use of a "T" for an initial drop letter
% and "HIS" in caps to complete the first word.
% \IEEEPARstart{T}{his} demo file is intended.

\IEEEPARstart{E}{nergy} consumption is a determining factor when designing wireless sensor networks.
As a consequence, battery lifetime is a limitation on the development of such systems.
Therefore, the idea of extracting energy from the environment has become attractive.
Looking to the energy consumption problem, the intelligent usage of the stored energy contributes to extend the sensor nodes' longevity.
Consequently, energy schedulers have been developed in order to adequately assess the energy consumption and adapt the system accordingly to the available amount of energy.
The purpose of this work is to adapt a solar energy harvesting circuit to supply energy to low power wireless platforms, i.e., those that operate under $50~mW$.
Simultaneously, we aim at improving the performance of the energy-aware task scheduler in wireless sensor network systems by providing fine-grained battery and environmental monitoring.

Among a number of energy sources that have been studied so far, solar has proved to be one of the most effective~\cite{Roundy:2003}.
The solar energy conversion through photovoltaic (PV) cells is better performed at an optimum operating voltage.
Operating a solar panel on this voltage results in transferring to the system the maximum amount of power available.
In this context, \emph{maximum power point tracker circuits} (MPPT) have been proposed.
The drawback is that MPPT circuitry may introduce losses to a solar harvesting system.
Concerning low-power applications, it may be more energy efficient to have a good matching between the solar panel and the energy storage unit~\cite{Raghunathan:2005}.
This well matched system is than able to work close to the maximum power point with less power loss.

In this work, an evaluation of the proposed harvesting circuit is performed in order to show improvements on an energy-aware task scheduler~\cite{Hoeller:SMC:2011}.
It is shown that the combination of the proposed circuit with the cited scheduler not only extended the longevity of the wireless sensor network, but also improved system quality.

The paper is organized as follows:
Section~\ref{fund} presents the fundamentals of solar energy harvesting and energy-aware task scheduler.
Section~\ref{design} discusses the design of the harvesting circuit under the perspective of low power wireless platforms.
Section~\ref{case} presents the evaluation of the harvesting circuit and a case study showing the improvements on system quality.
Finally, section~\ref{concl} closes the paper.

% ------------------------------------------------------------------------------


% ------------------------------------------------------------------------------
% ------------------------------------------------------------------------------
\section{Strategies for ME Optimization} \label{sota}
% TODO: Go back to here to customize this section.
% Currently it is to centered on ME optimization. 
% It does not mention interfaces, etc.

% + Related Work
% This section make an overview of the strategies to optimize the time performance of ME:
% algorithmic optimizations (fast-search algorithms, macroblock subsampling,
% sample truncation, multi-resolution ME, subsampled motion-field estimation),
%  parallelization of algorithms (our case), and
% hardware implementations of algorithms.
%
% Focus on works related to ME parallelization / distribution.
% Maybe compare to the work of Ronaldo's student: Ricardo Kintschner also about
% ME optimization on Cell (Webmedia2011).
% Look for other works directly comparable.
%
% Also: ?[and reviews techniques about designing for interfaces]?
% 

% TODO Maybe move this to SOTA
% The \emph{Block-Matching Algorithm} (BMA), which searches for similar blocks and
% generates the motion vectors, is mainly responsible for ME being so time
% consuming.
% Therefore one strategy for optimizing BMA is the \emph{fast-search}, which looks
% only in specific points of the search window, while a similar block is being
% searched.
% Another strategy is to perform ME hierarchically, computing motion vectors for a
% specific frame region, and refining them in each level, which is known as
% \emph{multi-resolution} motion estimation.
% Other strategies look into finding parallelism in BMAs, in order to run ME
% stages simultaneously.
% For all strategies there are also hardware implementations, based on optimized
% functional units (such as vector operations) or based on replication of
% functional units, to explore parallelism.
% Block-Matching Algorithms using fast-search improve time performance of ME, but
% they can find suboptimal motion vectors because they do not search in all
% positions of the search window. Multi-resolution ME works with different
% resolutions of one frame, successively refining the found motion vectors.
% This increases the ME time if the search is performed sequentially as in
% \cite{ChiaChunLin:FastAlgPlusArch:2006} or demands for replicated hardware
% functional units, as in \cite{ChiaChunLin:PMRME:2007}.
% Similarly, parallel and hardware implementations come at the cost of replicated
% or dedicated functional units.

% There are two major goals in motion estimation optimization: to improve the 
% compression rate and to reduce the total encoding time. 
% Improving the compression rate is achieved by finding the best possible motion
% vectors, which means motion vectors that will generate the smallest residual
% difference during the motion compensation (MC). Reducing the total encoding
% time is achieved by finding the motion vectors in the smallest possible period 
% of time. 
% Several tools in H.264 are used to find the best possible motion vectors; 
% besides looking in all positions of a search window (i.e. full-search), it is 
% possible to search in several reference frames (backwards or forwards), and it 
% is possible to perform block-matching using sub-pel precision 
% (half and quarter of a pel) \cite{citeulike:1269699}. 
% Finding the best motion vectors, very often, goes against finding the motion 
% vectors more quickly. 
% In this work we focus on motion estimation optimizations which aim to reduce the
% total encoding time, therefore we are not going into the details of techniques 
% for finding the best motion vectors possible, but they can be found in 
% \cite{YuWenHuang:Complex:2006}, \cite{YepingSu:MF:2006}, \cite{Ma:MF:2009}, 
% and \cite{XiangLi:MF:2004}. 
% It is important to notice that all techniques for ME optimization must take into
% consideration keeping the video quality of the generated bitstream.

% P2
% Como otimizar a ME - estratégias.
There are several strategies to optimize the execution time of ME:
fast-search algorithms, macroblock subsampling, sample truncation, 
multi-resolution ME, subsampled motion-field estimation, 
and parallel and hardware implementations of algorithms.

% P3 ... Pn-1
% Um paragrafo para cada estratégia dizendo:
% + how these solutions match/ contribute to Goals { g1, g2, .., gn} and Features {f1 , f2 , ..., fn}
% + And what set of features F is still missing? (GAP)
%
% P3 - fast-search
\emph{Fast-search algorithms} are block-matching algorithms that look only in 
specific positions of the search window \cite{SunNingning:TSS:2009, LaiManPo:4SS:1996,
ShipingZhu:DS:2009, Tourapis:PMVFAST:2001, HoiMingWong:EPMVFAST:2005, LiangGeeChen:TDL:1991}.
The search window defines the region of the reference frame that is scanned for a macroblock
partition similar to the current one. 
Only the motion vectors that correspond to the match with the lowest \emph{motion cost} 
are chosen. 
The main drawback of this approach is that, since some positions of the search 
window are discarded, it is possible to find suboptimal motion vectors.

% P4 macroblock subsampling and sample truncation
Two other strategies to optimize ME during block-matching are macroblock 
subsampling and sample truncation. 
Macroblock subsampling takes into consideration only a macroblock partition 
(i.e. some samples of a macroblock) while the matching for a 
position of the search window is being performed. 
Sample truncation is performed by ignoring the least significant bits of a 
sample. 
These strategies have been used separately in \cite{liu:sub:1993} (subsampling)
and in \cite{DBLP:journals/tcsv/HeTCL00}, and \cite{ChiaChunLin:PMRME:2007} 
(truncation). 

% P5 multi-resolution ME
\emph{Multi-resolution motion estimation} is the strategy in which the motion 
vectors are computed for distinct resolutions of the same frame. 
Motion vectors computed in a more coarse level can be successively refined until
the finest level (higher resolution). 
If the search is performed sequentially as in 
\cite{ChiaChunLin:FastAlgPlusArch:2006}, the time of ME can be increased due to
the dependencies between distinct levels. 
On the other hand, if the search is executed in parallel for each resolution 
level, as in \cite{ChiaChunLin:PMRME:2007}, hardware functional units need to be
replicated. 
A similar technique is \emph{subsampled motion-field estimation} 
\cite{liu:sub:1993}, which is based on the assumption that motion vectors of neighboring 
blocks tends to be similar. Thus, for each block, only a set of motion 
vectors (a motion-field) is computed, while the others are interpolated.

% P7 | P6 parallel and hardware implementations of algorithms
Other strategies for optimizing motion estimation are based on finding 
parallelism in ME stages, especially in the block-matching algorithms, in order
to execute them simultaneously. 
These parallel strategies commonly have been the base for dedicated hardware implementations. 
The \emph{Sum of Absolute Differences} (SAD) is a metric of error used in 
block-matching algorithms. This technique is frequently parallelized using functional 
units in hardware \cite{ChiaChunLin:PMRME:2007}, 
and \cite{HoyoungChang:HW:2009}. 
Hardware implementations of shared buffers for frame data are also
common \cite{HoyoungChang:HW:2009}, \cite{HaibingYin:HW:2010}.

% ------------------------------------------------------------------------------


% ------------------------------------------------------------------------------
% ------------------------------------------------------------------------------
\section{DMEC} \label{dmec}
% + DMEC
% Describes the Distributed Motion Estimation Component (DMEC).
% Describes its parallelization Strategy (picture partition) as well the
% Coordinator-Worker model


% 1º parte independente de distribuído
% * Diagrama de classes: Interface PictureMotionEstimator                         (fig 3.1 relat tec): pme.dia
% Dizer que DMEC foi projetado seguindo os conceitos de ME.
% Engenharia de domínio da codificação de vídeo.
The Distributed Motion Estimation Component (DMEC) was developed following a
domain engineering process.
Before we explain how we have designed DMEC it is worth reviewing some
ME concepts from the domain of video encoding.
ME is a technique employed to explore the
similarity between neighboring pictures in a video sequence.
Figure \ref{fig:motion_estimation} illustrates the ME process for the neighboring
pictures \emph{A} and \emph{B}.
By searching for similarities between these two pictures, it is possible to determine
which blocks from picture \emph{A} are also found in picture \emph{B}.
Such displacement of picture blocks is encoded by \emph{motion vectors} 
(represented by the small arrows in the bottom side of
Figure \ref{fig:motion_estimation}).
Exploring the similarity between neighboring pictures allows for difference-based
encoding, thus increasing the compression rate of the generated 
bitstream \cite{citeulike:1269699}.

\fig{.4}{motion_estimation}{Motion Estimation}

Taking into consideration the concepts encompassed by the ME technique, we have
designed DMEC.
The central element of DMEC is described by the interface 
\emph{PictureMotionEstimator} shown by Figure \ref{fig:pme}.
Such interface describes entities that are responsible to perform ME of a whole
picture (or a partition of a picture).
An object of the type \emph{PictureMotionEstimator} knows all block modes of
the video standard in use (e.g. H.264), and works according to a specific
error metric, such as the Sum of Absolute Differences (SAD).
Such error metric is used to determine the similarity between neighboring
pictures.
The ME itself is computed by the \emph{match} method that takes the
current and the reference pictures
(respectively pictures \emph{A} and \emph{B} from Figure \ref{fig:motion_estimation})
and returns their ``counterpart'' which is composed by the motion vectors
and the motion cost.



% * Diagrama de classes: Retorno do match                                         (fig 3.2 relat tec): pmc.dia
% Dizer que retorna "tudo" é um componente bem auto contido.
DMEC was designed to be self contained, thus, the \emph{match} method computes ME
for all picture's macroblocks and for all block modes without any dependency
from another encoder element.
Therefore, the return of the method \emph{match}, which is conceptually shown
by Figure \ref{fig:pmc} is a multidimensional vector containing all motion
vectors and costs for each macroblock and block mode used in the ME.

\subsection{Parallelization Strategy}
% // DMEC: Parallelization Strategy //
% 2º parte distribuída
% * Figura particionamentos: todos os 6.                                          (TODO): picture_partition.XXX
% com ou sem o foreman. Pode ser sem, apenas retângulos.
% 
% * Figura threads/"atores" coordinator e workers                                 (from fig.git): dmec_thread_model-en.dia
% 
% * Diagrama de classes: módulos coordinator e workers                            (fig 3.4 relat tec): cnw.dia
% Motrar que vários algoritmos podem ser utilizados.


The parallelization strategy employed by DMEC is based on data partitioning,
in which the unit to be partitioned is the picture.
Figure \ref{fig:picture_partition} shows all picture partition modes available.
% It is worthy to mention that as all picture partitions dimensions must be 
% multiple to the macroblock dimension (16x16 pixels), it can occur having 
% partitions with distinct dimensions.
All picture partitions dimensions must be multiple to the macroblock dimension 
(16x16 pixels) to avoid having a macroblock broken between to neighboring
partitions.
If desired, it is possible to include new partition modes by specifying how the
partition should be performed.

% Fig showing all partitions types.
\fig{.2}{picture_partition}{Supported picture partitions modes}
% SINGLE_PARTITION = 1, 
% |0|
% 
% ONExTWO_PARTITION = 2,     
% |0|
% |1|
% 
% THREExONE_PARTITION = 3,
% |0|1|2|
% 
% TWOxTWO_PARTITION = 4,
% |0|1|
% |2|3|
% 
% THREE_TWOxTWO_PARTITION = 5,
% |0|1|2|
% |3 | 4|
% 
% THREExTWO_PARTITION = 6
% |0|1|2|
% |3|4|5|
% 

\fig{.4}{workers_and_bma}{Worker and Block Matching Algorithm}

In order to improve the performance of ME, each picture partition is then 
assigned to a \emph{Worker} module which executes in a specific functional unit,
such as a core of a multicore processor or a dedicated hardware element.
There is also a \emph{Coordinator} module, responsible to define the 
picture partition for each \emph{Worker} and to provide them with 
pictures to be processed.
The \emph{Coordinator} is also responsible to gather results generated by
\emph{Workers} (motion cost and motion vectors) and to deliver these results
back to the encoder. 
Figure \ref{fig:dmec_thread_model-en} illustrates the interaction
between the \emph{Coordinator} and \emph{Worker} modules.

\figtwocolumn{.4}{pme}{\emph{PictureMotionEstimator} interface}

% + Algoritmo
Each \emph{Worker} module computes ME by using a Block Matching Algorithm (BMA),
which itself also implements the \emph{PictureMotionEstimator} interface.
Thus, as shown by Figure \ref{fig:workers_and_bma}, all BMAs follow a single interface
therefore, it is possible to replace a BMA for another, according to the encoder
needs.

\subsection{Data Transference and Synchronization}
% // DMEC: Data Transference and Synchronization //
% * Diagrama de classes: TransferenceManager e SynchronizationManager             (TODO): dmec_trans_and_synch.dia
% Talvez seja um diagrama de sequencia até.
% 
% Mostrando como a transferencia de dados e sincronização são feitas.
% 
% É, diagrama de seq é bom.
% 
% Coordinator prepara pictures.
% Usa SynchronizationManager para Avisar que elas estão prontas (posso chamar de Barrier)
% Workers usam Transference Manager para obter as Pictures.
% Calculam ME e usar Transference Manager para divulgar os resultados
% Avisam Coordinator usando SynchronizationManager que o trabalho está pronto.
% 
% É meio um repeteco da figura thread model, mas é interessante para mostrar as
% interfaces de TransferenceManager e SynchronizationManager que serão implementadas
% pelo CELL, PC e Hardware...
The memory model used to exchange data between \emph{Coordinator} and
\emph{Worker} modules is a kind of shared memory, although, in practice, such
memory can be implemented as a memory block in a single address space 
(as is the case for our implementation for the IA32 architecture and
dedicated hardware), or
can be constituted by multiple memory blocks, each one using its own address
space
(as is the case for our implementation for the Cell BE architecture).

Figure \ref{fig:dmec_trans_and_synch} details the interaction between 
\emph{Coordinator} and \emph{Worker} modules.
At the beginning of the ME operation DMEC assumes that all pictures are in the
main memory.
Using the \emph{TransferenceManager} interface, \emph{Worker} modules can
obtain the samples of their picture partitions.
In the case of a memory block in a single address space, the operations of
\emph{TransferenceManager} are just pointer manipulations.
On the other hand, while using memory blocks in distinct address spaces, the
operations of \emph{TransferenceManager} are mapped to real 
memory transferences, such as Direct Memory Access (DMA) requisitions.
Also, by using the \emph{TransferenceManager}, \emph{Worker} modules 
can publish the computed motion vectors and motion costs.
% 
\emph{Worker} modules should wait for a signal from the \emph{Coordinator}
module indicating that there are pictures to the processed.
Similarly, the \emph{Coordinator} module should wait for the ME results 
generated by the \emph{Worker} modules.
Such synchronization operations are implemented by the 
\emph{SynchronizationManager} interface, which specifies barrier-like
mechanisms, which can be implemented using operating system operations or using
directly dedicated hardware elements.
In the case of a sequential operation with a single partition containing the
whole picture, the operations of \emph{SynchronizationManager} are canceled. 


% ------------------------------------------------------------------------------
% 
% Some notes
% DMEC Interfaces:
% 
% \emph{PictureMotionEstimator}
% 
% \emph{TransferenceManager}
% 
% \emph{SynchronizationManager}
% 
% 
% \emph{Coordinator} implements \emph{PictureMotionEstimator}
% 
% \emph{Worker} kind of implements \emph{PictureMotionEstimator}.
% The interface is other but it also performs match.
% The \emph{Worker} interface has a entry point ``run'', a match method,
% and some methods to gather the pictures and put the pictures back.
% 
% A \emph{Worker} uses the interfaces
% \emph{TransferenceManager} and \emph{SynchronizationManager}


% ------------------------------------------------------------------------------
% ------------------------------------------------------------------------------
\section{Implementation} \label{impl}
% + Implementation
% Describes our implementation for Linux@Cell, Linux@IA32, and Hardware (obtained using HLS).
% Emphasizes our communication (data transference and synchronization)
% interfaces are the same for both architectures, they only differs at implementation.

% ------------------------------------------------------------------------------


% ------------------------------------------------------------------------------
\subsection{Cell BE}
% ++ Cell BE
% Describes how data transference is implemented: DMA between PPE and SPEs.
% Describes how synchronization is implemented: Barriers like mechanisms using
% Cell MailBox

% = What are our goals in evaluating this hardware implementation.
The Cell BE is a processor architecture developed by IBM,
Sony, and Toshiba, targeting applications with high thread-level parallelism \cite{Gschwind:2006}.
Our goal on implementing DMEC at Cell BE is evaluating ME in a high performance
and distributed architecture.

% = Target architecture
Cell BE is an example of a
``Multi-computer-on-a-Chip'' architecture because it is composed by distinct
processor units, each one having its own memory address space.
The Cell BE architecture incorporates in the same chip nine independent cores:
a Power Processor Element (PPE) and eight
Synergistic Processor Elements (SPEs).
% 
The PPE is the main processing unit, responsible for managing all chip resources.
The SPEs are dedicated processing units supporting vectorized
floating point code execution.
Each SPE has its own local memory, called Local Storage (LC).
Hence, all communication between cores is performed explicitly though the
Element Interconnect BUS (EIB), a high performance bus.

% \fig{.4}{cell}{Cell BE architecture}

% = Memory model / Data transference
As Cell BE uses memory blocks in distinct address spaces,
we have mapped the operations of \emph{TransferenceManager} to
DMA requests using the EIB.
Using DMA requests, each \emph{Worker} module of DMEC obtain the samples
it needs to perform the ME and deliver back the computed ME results 
to the \emph{Coordinator} module.
% 
The local memory of SPEs is limited to 256KB, including data and code.
This imposed several limitations. For instance, using a partition mode with
six partitions of 640x544 each at a 1080p resolution requires 680KB for data storage.
Thus, only the samples that are currently been used
by ME computation (or samples that are going to be used in a near future) are
kept inside the SPEs local memory.
The samples transferences are hidden from the ME algorithm by the implementation of
the \emph{Picture} interface. The \emph{Picture}, in this case, has a local buffer
used as a cache and obtains sample as needed through the \emph{TransferenceManager}.
% Avaliamos mecanismos de \emph{double buffer} afim de esconder a latência das 
% transferências de memória via DMA entre a memória principal e a memória local 
% das SPEs.

Similarly, the implementation of \emph{PictureMotionCounterpart} for Cell BE
also uses DMA requests through \emph{TransferenceManager}, in order to
deliver the ME results to the main memory.
% TODO maybe move this to DMEC Section.
During the ME computation, previously calculated motion vectors can be used to
improve the ME for the current blocks.
Such motion vectors are called \emph{motion predictors}.
The implementation of \emph{PictureMotionCounterpart} for Cell BE contains an
internal buffer to keep motion predictors inside the local memory of
the \emph{Worker} module, avoiding unnecessary DMA transferences between
the \emph{Worker} and the \emph{Coordinator} modules.

% = Synchronization
The \emph{SynchronizationManager} interface, which specifies barrier-like
mechanisms, is implemented in Cell BE using the \emph{MailBox}, a
hardware resource present on each SPE.
By using specific MailBox messages, it is possible to specify when the
\emph{Worker} has completed the ME computation as well when there are new 
pictures to be processed.

% ------------------------------------------------------------------------------


% ------------------------------------------------------------------------------
\subsection{Multicore IA32}
% ++ Multicore IA32
% Describes how data transference is implemented: direct access though pointers.
% Describes how synchronization is implemented: Barriers like mechanisms using
% pthreads / EPOS semaphores

% = What are our goals in evaluating this hardware implementation.
Our goal on implementing DMEC on a Multicore Intel IA32 
architecture is evaluating ME for a high performance and Symmetric Multiprocessor (SPM).
% 
% = Target architecture
For our experiments we have used an Intel Core2 Quad, with four symmetric
cores.
% chroma.lisha.ufsc.br
% TODO add mais info aqui.
% \cite{IA32, ou Intel Core, 

% = Memory model / Data transference
% Each core on an Intel Core microarchitecture
At the main memory level,
all cores on a Multicore IA32 processor 
share the same memory address space.
Thus, for the Multicore IA32 architecture implementation,
the operations of the \emph{TransferenceManager} are mapped to direct memory access
through pointers.
The implementation of \emph{Picture} interface, in this case,
is straightforward. It contains the whole picture partition that is going to
be used by the \emph{Worker} module.

Similarly, the implementation of \emph{PictureMotionCounterpart} for
Multicore IA32  also uses the operations of \emph{TransferenceManager}.
As there is no practical memory limitation the implementation of 
\emph{PictureMotionCounterpart} for Multicore IA32, it contains all the
motion predictors necessary by the \emph{Worker} modules.

% = Synchronization
The \emph{SynchronizationManager} interface, which specifies barrier-like
mechanisms, is implemented using \emph{pthreads} semaphores of the
POSIX thread library.

% ------------------------------------------------------------------------------


% ------------------------------------------------------------------------------
\subsection{Dedicated hardware}
% ++ Dedicated hardware
% Describes how data transference is implemented: buses, etc.
% Describes how synchronization is implemented: Barriers like mechanisms
% implemented in hardware
In the last few years, advances in \textit{electronic design automation}~(EDA) tools are allowing
hardware synthesis from high-level behavioral models. This process is known as \textit{high-level
synthesis}~(HLS) and allows designers to describe hardware components using languages like C++, and
higher-level techniques, such as \textit{Object-Oriented Programming}~(OOP). In this work we
leverage on HLS technology to obtain a dedicated hardware implementation of DMEC. The main goal of
providing hardware DMEC is to show that the same interface defined for the previous
implementations can also be used in this case. Additionally, this section highlights the reusability
and flexibility of our implementation. Through a simple code refactoring process, we have converted
our multicore implementation to a synthesizable behavioral hardware model.

Figure \ref{fig:dmec_hw_hls} gives an overview of the HLS process and the refactored code. The HLS
tool we have used (Calypto's CatapultC~\cite{Calypto:Catapult}) does not work with the concept of
tasks and threads, but it extracts parallelism by exploiting the parallelization of \emph{loops}. To
cope with such approach, we have serialized the execution of the tasks performed by the
\emph{Coordinator} and the \emph{Workers}. The \emph{SynchronizationManager} now defines internal
buffers for the pictures and the resulting motion estimation vectors, coordinating the transfer
of data between these elements and the main memory. The ME algorithm is implemented in a loop in
which each iteration performs the task of one \emph{Worker}. By using \emph{synthesis
directives} we can guide the HLS process in order to obtain parallelism. Figure
\ref{fig:dmec_hw_hls}
shows that the directive \emph{UNROLL} is set for the \emph{workers\_loop}, indicating that this
loop should be fully unrolled. In the HLS context, fully unrolling a loop means that all
iterations are performed in parallel. There are other forms of loop parallelization. For instance, 
loops can be pipelined in order to reduce hardware area while keeping the same throughput. To match
the multicore behavior in which all workers execute in parallel, we have chosen to fully unroll
only the top-lovel loop. Additional directives were set to pipeline inner parts of the ME algorithm
whenever possible. 

\fig{0.6}{dmec_hw_hls}{High-level synthesis process for the generation of a dedicated hardware
implementation}

The output of the HLS process are \emph{register transfer level}~(RTL) descriptions in
VHDL or Verilog that can be used as input for most RTL
synthesis tools. CatapultC also generates cycle- and bit-accurate SystemC models
for design verification and evaluation. Figure \ref{fig:dmec_hw_eval} shows a block diagram of the
final RTL implementation generated by CatapultC and its integration with the evaluation
infrastructure.

\fig{0.6}{dmec_hw_eval}{HW DMEC evaluation infrastructure}

The final component uses a simple handshaking mechanism for input/output data. This interface is
adapted for the AMBA AXI4 bus, a widely used standard for on-chip interconnect of functional
blocks. Our performance evaluation is based on simulation of the SystemC model generated by
CatapultC. A model of the AXI4 bus model is provided for interfacing the cycle-accurate hardware
simulation with the remaining infrastructure.

In the software side, an implementation of the \emph{PictureMotionEstimator} interface serializes
the pictures and copies the data to the hardware component internal buffers using a memory mapped
interface. Once the ME is complete, the motion vectors are read and used to build the
\emph{PictureMotionCounterPart} object which is returned to the test application.


% ------------------------------------------------------------------------------


% ------------------------------------------------------------------------------
% ------------------------------------------------------------------------------
\section{Practical Experiments} \label{eval}
% + Practical Experiments
% First shows speedup and quality results for DMEC
% using 1 (without partitioning) to 6 worker threads.
% Show that speedup is high and quality is kept acceptable.
% 
% Then, show speedup and quality results for DMEC integrated to JM and compares
%  to the original JM (and, if possible, to other works).
% 

% + JM
% + Dizer como realizamos os experimentos. E/ou quais as variáves observadas:
%   Evaluate the component in isolation DMEC to show its speedup.
%    And how it scales.
%   Evaluate the component in JM to show sppeedup and PSNR.
We have evaluated DMEC in two stages.
First, in order to verify how DMEC's performance scales from 
one to six \emph{Workers} instances, we have evaluated all DMEC implementations 
in a test case.
The test case application mimics the behavior of an H.264 encoder: it provides 
DMEC with pictures, obtain the ME results 
(motion vectors and motion cost), and checks if the results are correct.
Secondly, in order to assess DMEC influence on the final video quality, we have
evaluated all DMEC implementations in the
JM H.264 Reference Encoder~\cite{site:jm}.
The PSNR degradation is computed as the absolute PNSR difference between the
original encoder and the optimized ones.

% P: Dizer pq focamos em luma
For inter macroblock modes in H.264 (i.e. modes related to the ME),
the motion cost for chrominance components derives from the motion cost for 
luminance components~\cite{1101854}. 
Consequently the PSNR for chrominance components derives from the PSNR for 
luminance components. 
For this reason, in this paper we focus on the PSNR variation of the luminance 
component.

Figure \ref{fig:dmec-speedup_workers} show the speedup of DMEC in there
test case application with a different number of \emph{Workers} instances.
For such test, we have used an arbitrary set of pictures with a resolution of
1080p (Full-HD).
The speedup is normalized to one \emph{Worker} instance (speedup of 1X).

% TODO
% \textit{Comments about results in: Cell BE, Muticore IA32, and HW.}
It is worth to mention that for each number of \emph{Worker} instances
a different partition mode was used, according to
Figure \ref{fig:picture_partition}.
For one \emph{Worker} instance we have used the ``Single Partition'',
for two \emph{Worker} instances we have used the ``2x1'' partition and so on,
up to the ``2x3'' partition mode (used for six \emph{Worker} instances).

Besides the additional performance obtained by using a higher number of \emph{Worker} instances,
the partition mode also has influence on the speedup.
The reason of such influence is that, during the partitioning process, the dimensions of the
search window shrinks, thus reducing the area of the picture searched for
similarities.

\fig{.45}{dmec-speedup_workers}{Time performance scalability of DMEC}

% Sobre RD curves
% Figures XXX show the speedup of DMEC while tested already integrated to JM.
% The obtained values are compared to the ones obtained while using the original
% JM, without DMEC.
In order to evaluate in details the behavior of DMEC for distinct values 
of encoding bit-rates, we have used the BD-PSNR (Bjøntegaard Delta PSNR) metric
using the following values of QP (Quantization Parameter): 16,20,24,28; as 
described in \cite{gisle_bjntegaard_calculation_2001}.
It is important to evaluate quality (PSNR) for distinct bit-rates to test 
whether the approach can be used in distinct scenarios of application.
Figure \ref{fig:crowd-bitrate_psnr} shows the rate-distortions (RD) curves using the
original JM encoder and the optimized encoder using DMEC.
The video sequence used for this curves was \texttt{Crowd Run}, a 1080p sequence with  a
high ammount of motion.
Lower values of bit-rate are obtained for higher values of QP since by using
higher values for QP more data is discarded, thus increasing the
compression ratio. 
The two curves very near from each other indicates that the DMEC
presents a good rate-distortion performance for all the evaluated bit-rates.

\fig{.45}{crowd-bitrate_psnr}{RD curve of a 1080p video sequence}

We have evaluated also the speedup obtained in the
% encoding time
ME run time
while using
DMEC for the same QP values we used for BD-PSNR.
Figure \ref{fig:crowd-bitrate_speedup} shows the obtained values while using
6 \emph{Worker} instances.
For Muticore IA32, a speedup of around 9 times is obtained for all
bit-rate values.
For Cell BE this value is about 2 times.
A small speedup for the Cell BE, while compared to Multicore IA32 and the dedicated hardware, is due
to the memory transferences (picture samples and ME results) which is performed using
the DMA requisitions of Cell BE.


\fig{.45}{crowd-bitrate_speedup}{Speedup vs bit-rate of a 1080p sequence}
%
% \multfigtwov{.65}{bd_psnr}{bd_speedup} {bd} {RD curve (a) and speedup vs bit-rate (b) of 1080p sequence}

% Discussion


% Falar do paralelismo / particionamento de dados
% Qualidade ficou boa mesmo particionando e desempenho aumentou: speedup ~ 70%
%The strategy of ME distribution based on picture partitioning has been shown 
%effective.
% We have obtained a speedup higher than XXX\% without loosing quality.
%Data partitioning is effective because the visual interdependence between
%partitions is not significant to influence on the encoding quality, and allows
%for a speedup because it enables the simultaneous processing of each picture
%partition.

% - Falar da comunicação
% - Necessidade via espaço de endereçamentos diferentes
% A arquitetura Cell Broad Band demonstrou-se uma arquitetura interessante para o
% processamento paralelo de vídeo, pois possui unidades funcionais dedicadas 
% (i.e. SPEs) para processamento de dados. 
% A principal dificuldade encontrada em trabalhar-se com o Cell foi a capacidade 
% limitada da memória local das SPEs.
% Outra dificuldade foi lidar com as transferências de memória entre SPE e PPE. 
% Isto em fato foi superado pelas estratégias que desenvolvemos de baferização e 
% também com a utilização do Element Interconect Bus (EIB) do Cell que realiza 
% DMAs com altas taxas de transferências.
% 
% - Solução 1: Buffer de preditores, contribuiu bastante
% A estratégia de armazenamento de preditores nas SPEs foi significativa no 
% aumento do desempenho, pois vetores de movimentos necessários para o cálculo da
% ME não precisam ser consultados na memória principal. É coerente a decisão de 
% manter uma cópia local destes vetores, pois todos os vetores que a ME irá 
% precisar foram calculados pela partição em questão e por nenhuma outra.


% ------------------------------------------------------------------------------


% ------------------------------------------------------------------------------
% ------------------------------------------------------------------------------
\section{Conclusão} \label{concl}

Após realizarmos os testes concluímos a viabilidade da implementação do protocolo PTP para realizar a sincronização dos tempos de relógio em um sistema operacional embarcado, pois conseguimos manter o {\it offset} próximo a 0 segundo. Isso nos fornece uma base para trabalharmos a implementação com o intuito de obter um {\it offset} na faixa de sub-milissegundos. Obtendo essa precisão conseguiríamos garantir a aplicação, por exemplo, da implementação para sistemas de sensoriamento oceânico, como é abordado em \cite{DelRio2012}. Onde é feita uma abordagem sobre a implementação do protocolo PTP para distribuir os tempos de relógio em uma rede Ethernet de Sensoriamento Marinho(MSN). O fato da necessidade de um protocolo deste tipo para tal escopo se dá pelo fato de sinais GPS não estarem disponíveis devido à atenuação da água no fundo do mar e à requisitos de sincronização de instrumentos marítimos, tais como sismógrafos.




% ------------------------------------------------------------------------------

% if have a single appendix:
%\appendix[Proof of the Zonklar Equations]
% or
%\appendix  % for no appendix heading
% do not use \section anymore after \appendix, only \section*
% is possibly needed

% use appendices with more than one appendix
% then use \section to start each appendix
% you must declare a \section before using any
% \subsection or using \label (\appendices by itself
% starts a section numbered zero.)
%


% \appendices
% \section{Proof of the First Zonklar Equation}
% Appendix one text goes here.
% 
% % you can choose not to have a title for an appendix
% % if you want by leaving the argument blank
% \section{}
% Appendix two text goes here.


% use section* for acknowledgement
%\section*{Acknowledgment}
%We would like to thank Ronaldo Husemann and Prof. Valter Roesler for the
%prolific discussions about motion estimation optimization.
%This work was partially funded by FINEP grant INF/FINEP 01.08.0287.00-REDEH264,
%project Rede H.264 SBTVD.
% TODO colocar número do projeto.
% PROJETO: INF/FINEP 01.08.0287.00-REDEH264
% NÚMERO:
% 06341 - X



% Can use something like this to put references on a page
% by themselves when using endfloat and the captionsoff option.
\ifCLASSOPTIONcaptionsoff
  \newpage
\fi



% trigger a \newpage just before the given reference
% number - used to balance the columns on the last page
% adjust value as needed - may need to be readjusted if
% the document is modified later
%\IEEEtriggeratref{8}
% The "triggered" command can be changed if desired:
%\IEEEtriggercmd{\enlargethispage{-5in}}

% references section

% can use a bibliography generated by BibTeX as a .bbl file
% BibTeX documentation can be easily obtained at:
% http://www.ctan.org/tex-archive/biblio/bibtex/contrib/doc/
% The IEEEtran BibTeX style support page is at:
% http://www.michaelshell.org/tex/ieeetran/bibtex/
%\bibliographystyle{IEEEtran}
% argument is your BibTeX string definitions and bibliography database(s)
%\bibliography{IEEEabrv,../bib/paper}
%
% <OR> manually copy in the resultant .bbl file
% set second argument of \begin to the number of references
% (used to reserve space for the reference number labels box)
% References
\bibliographystyle{IEEEtran}
\bibliography{hw,mm}


\end{document}
