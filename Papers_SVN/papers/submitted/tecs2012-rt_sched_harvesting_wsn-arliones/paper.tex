\documentclass[prodmode,acmtecs]{acmsmall}

% Package to generate and customize Algorithm as per ACM style
\usepackage[ruled]{algorithm2e}
\renewcommand{\algorithmcfname}{ALGORITHM}
\SetAlFnt{\small}
\SetAlCapFnt{\small}
\SetAlCapNameFnt{\small}
\SetAlCapHSkip{0pt}
\IncMargin{-\parindent}

% Metadata Information
\acmVolume{12}
\acmNumber{4}
\acmArticle{1}
\acmYear{2013}
\acmMonth{9}



\usepackage{multirow}
\usepackage{graphicx}
\usepackage{caption}
\usepackage{subcaption}
% \usepackage{subfigure}


% % Arliones' defined commands

\newcommand{\fig}[4][tb]{
  \begin{figure}[#1]
    {\centering{\includegraphics[#4]{fig/#2}}\par}
    \caption{#3}
    \label{fig:#2}
  \end{figure}
}

\newcommand{\wfig}[4][tb]{
  \begin{figure*}[#1]
    {\centering{\includegraphics[#4]{fig/#2}}\par}
    \caption{#3}
    \label{fig:#2}
  \end{figure*}
}

\newcommand{\figtwo}[8]{
\begin{figure}[tb]
	\centering
	\begin{subfigure}[b]{#8}
		\centering
		\includegraphics[#3]{fig/#1}
		\caption{#2}
	\end{subfigure}%
	\begin{subfigure}[b]{#8}
		\centering
		\includegraphics[#6]{fig/#4}
		\caption{#5}
	\end{subfigure}
	
    \caption{#7\label{figtwo:#1#4}}
\end{figure}
}

\newcommand{\mytabwnote}[4][h]{
  \begin{table}%
  \tbl{#3\label{tab:#2}}{%
    \input{fig/#2.tab}
  }
  \input{fig/#2.tabnote}%
  \end{table}
}

\newcommand{\mytab}[4][h]{
  \begin{table}%
  \tbl{#3\label{tab:#2}}{%
    \input{fig/#2.tab}
  }
  \end{table}
}

\newcommand{\mywtab}[4][h]{
  \begin{table*}%
  \tbl{#3\label{tab:#2}}{%
    \input{fig/#2.tab}
  }
  \end{table*}
}

\newcommand{\epos}{\textsc{Epos}}
\newcommand{\adesd}{\textsc{Adesd}}
\newcommand{\emote}{\textsc{EposMote}}
\newcommand{\adhop}{\textsc{ADHOP}}
\newcommand{\bgr}{\emph{BET Global Rate}}
\newcommand{\bwr}{\emph{BET Window Rate}}
\newcommand{\bwar}{\emph{BET Window Average Rate}}
\newcommand{\bwdr}{\emph{BET Window Derivative Rate}}

% Document starts
\begin{document}

% Page heads
\markboth{Arliones Hoeller Jr. and Ant{\^o}nio Augusto Fr{\"o}hlich}{Scheduling of energy usage for real-time WSN systems with energy-harvesting capability}

% Title portion
\title{Scheduling of energy usage for real-time wireless sensor network systems with the energy-harvesting capability}
% \title{RT scheduling in energy-harvesting WSN}
\author{ARLIONES HOELLER JR.
\affil{Federal University of Santa Catarina}
ANT{\^O}NIO AUGUSTO FR{\"O}HLICH
\affil{Federal University of Santa Catarina}}


\begin{abstract}
Wireless sensor network (WSN) systems present serious limitations in terms of energy availability and processing power.
To cope with the energy availability issue, WSN nodes equip energy harvesters (e.g. photovoltaic modules) and energy buffers (e.g. rechargeable batteries).
This approach, besides increasing energy availability, also brings uncertainty to the system when the efficiency of the energy harvester depends on variables that are difficult to predict (e.g. solar irradiation, temperature).
This paper presents a set of heuristics to schedule energy utilization of critical applications running on wireless sensor networks.
Experiments consider the impact of the scheduling heuristics on timing correctness and network quality.
We propose and evaluate four energy scheduling heuristics based on the recent history of harvested energy.
Results of simulations based on a real system implementation show the performance of the heuristics.% and compared to related work.
% One of the proposed heuristics is shown to perform better that others under large oscillations on energy availability.
%
% % 100-words abstract
% WSN systems present limitations in terms of energy availability.
% To cope with this limitation, WSN nodes equip energy harvesters (e.g. photovoltaic modules) and energy buffers (e.g. batteries).
% Although increasing the amount of energy available, the harvesters introduce uncertainty to the system because their efficiency depends on variables that are difficult to predict (e.g. solar irradiation, temperature).
% This paper evaluates the use of energy predictors when scheduling real-time tasks on WSNs.
% The predictors are heuristics based on recent history of harvested energy.
% Simulations based on a real system implementation consider the impact of the approach on real-time requirements and network quality.
\end{abstract}

\category{D.4.7}{Organization and Design}{Real-time systems and embedded systems}

\terms{Design, Experimentation, Performance}

\keywords{wireless sensor networks, energy harvesting, real-time scheduling}

\acmformat{Hoeller Jr., A., Fr{\"o}hlich, A.A. 2013. Scheduling of energy usage for real-time wireless sensor network systems with energy-harvesting capability.}

\begin{bottomstuff}
This work is supported in part by Brazilian RNP/CTIC effort towards the \textit{Internet of Things}.
A. Hoeller Jr. is also supported in part by CAPES, under grant RH-TVD 006/2008.

Author's addresses:
Arliones Hoeller Jr., Department of Automation and Systems Engineering, Federal University of Santa Catarina;
Ant{\^o}nio Augusto Fr{\"o}hlich, Department of Computer Science, Federal University of Santa Catarina.
\end{bottomstuff}

\maketitle

% ------------------------------------------------------------------------------
\section{Introduction} \label{intro}
% + Introduction
% 
% The very first letter is a 2 line initial drop letter followed
% by the rest of the first word in caps.
%
% form to use if the first word consists of a single letter:
% \IEEEPARstart{A}{demo} file is ....
%
% form to use if you need the single drop letter followed by
% normal text (unknown if ever used by IEEE):
% \IEEEPARstart{A}{}demo file is ....
%
% Some journals put the first two words in caps:
% \IEEEPARstart{T}{his demo} file is ....
%
% Here we have the typical use of a "T" for an initial drop letter
% and "HIS" in caps to complete the first word.
% \IEEEPARstart{T}{his} demo file is intended.

\IEEEPARstart{E}{nergy} consumption is a determining factor when designing wireless sensor networks.
As a consequence, battery lifetime is a limitation on the development of such systems.
Therefore, the idea of extracting energy from the environment has become attractive.
Looking to the energy consumption problem, the intelligent usage of the stored energy contributes to extend the sensor nodes' longevity.
Consequently, energy schedulers have been developed in order to adequately assess the energy consumption and adapt the system accordingly to the available amount of energy.
The purpose of this work is to adapt a solar energy harvesting circuit to supply energy to low power wireless platforms, i.e., those that operate under $50~mW$.
Simultaneously, we aim at improving the performance of the energy-aware task scheduler in wireless sensor network systems by providing fine-grained battery and environmental monitoring.

Among a number of energy sources that have been studied so far, solar has proved to be one of the most effective~\cite{Roundy:2003}.
The solar energy conversion through photovoltaic (PV) cells is better performed at an optimum operating voltage.
Operating a solar panel on this voltage results in transferring to the system the maximum amount of power available.
In this context, \emph{maximum power point tracker circuits} (MPPT) have been proposed.
The drawback is that MPPT circuitry may introduce losses to a solar harvesting system.
Concerning low-power applications, it may be more energy efficient to have a good matching between the solar panel and the energy storage unit~\cite{Raghunathan:2005}.
This well matched system is than able to work close to the maximum power point with less power loss.

In this work, an evaluation of the proposed harvesting circuit is performed in order to show improvements on an energy-aware task scheduler~\cite{Hoeller:SMC:2011}.
It is shown that the combination of the proposed circuit with the cited scheduler not only extended the longevity of the wireless sensor network, but also improved system quality.

The paper is organized as follows:
Section~\ref{fund} presents the fundamentals of solar energy harvesting and energy-aware task scheduler.
Section~\ref{design} discusses the design of the harvesting circuit under the perspective of low power wireless platforms.
Section~\ref{case} presents the evaluation of the harvesting circuit and a case study showing the improvements on system quality.
Finally, section~\ref{concl} closes the paper.

% ------------------------------------------------------------------------------

\section{System Models}
\label{sec:sys_models}

% Section intro
The definition of a system model allows a better understanding of the problem in hand.
This section presents two models: the \emph{Task Model} and the \emph{Energy Model}.
The first models the system quality and timing correctness issues.
The second, models aspects of energy consumption, production, and storage.

\subsection{Task Model}
\label{sec:app_model}

% Real-Time - Hard real-time - Flexible real-time (best-effort) - this approach (hard + BE)
Real-time scheduling is a common solution to ensure correctness and quality of critical applications.
In a hard real-time model, all instances of system tasks (jobs) must complete in time.
Flexible real-time models may also take into account a set of optional or adaptable tasks.
These flexible models may also be called best-effort models.
In this scenario, the real-time scheduling policy ensures correctness, while system quality levels may be expressed as a function of the execution rate or adaptation levels of a set of flexible tasks.
This work employs a hybrid model based on both hard real-time and best-effort tasks.

% Hard real-time at design phase
As a matter of fact, critical real-time systems are almost always designed considering energy sources that are consistent with system demands.
Also, the system designer must make energy saving decisions, such as voltage scaling and device hibernation, at design-time.
Thus, the designer can take such decisions into consideration while defining the energy budget necessary to sustain the system.
Complex, battery-operated, real-time embedded systems, such as satellites, autonomous vehicles, and even sensor networks, comprise a set of tasks that include both, critical and noncritical tasks.
A power manager for one such embedded system must follow the design-time decisions for critical parts while trying to optimize energy consumption by noncritical parts.

% Detail our scheduling approach
For the envisioned scenario -- i.e.  battery-operated, real-time, embedded systems harvesting energy from the environment -- the energy budget would be defined at design-time based on critical tasks, while noncritical tasks would be executed on a best-effort policy, considering not only the availability of time, but also of energy.
Along with the assumption that an autonomous power manager cannot interfere with the execution of hard real-time tasks, the separation of critical and noncritical tasks at design-time leads to the following scheduling strategy:
\begin{itemize}
\item The system handles hard real-time tasks as mandatory tasks, executed independently of the energy available at the moment.
These tasks are scheduled according to traditional algorithms such as Earliest Deadline First~(EDF) and Rate Monotonic~(RM)~\cite{Liu:1973}, either in their original form or extended to support DVS~\cite{Pillai:2001}.
\item Best-effort tasks, periodic or not, have lower priorities than the hard real-time ones and therefore only execute if no hard real-time tasks are ready to run.
Furthermore, the decision to dispatch a best-effort task must also take into consideration whether the remaining energy will be enough to schedule all activations of hard-real time tasks expected to occur until the system reaches the targeted lifetime.
\item During system execution, a speculative power manager is active identifying idle periods of components.
Whenever a best-effort job does not execute due to energy limitations, the number of idle periods of system components raises, thus promoting further energy savings.
\end{itemize}

% Point to epos' power manager for energy-monitoring and devices' power management
This scheduling strategy has only small implications in terms of process management at the operating system level;
however, in order to be implemented it requires a comprehensive power management infrastructure, like the one previously presented by the authors~\cite{Frohlich:IJDSN:2011,Hoeller:SMC:2011}.
In particular, the system needs battery monitoring services to support the scheduling decisions around best-effort tasks and component dependency maps to avoid power management decisions that could affect the execution of hard real-time tasks.


\subsection{Energy Model}
\label{sec:en_model}

% Must consider energy consumption, production, and storage.
In this work, three facets of energy must be taken into account: consumption, production, and storage.

% Energy accounting method (equations and update frequency)
Equations~\ref{eq:en_dev_time} through~\ref{eq:energy}, introduced in a previous work~\cite{Hoeller:SMC:2011}, define the employed energy consumption model.
Depending on the behavior of a given component, or the information available about its energy consumption, the designer may choose to monitor energy consumption based either on the time the device spends in a particular operating mode or on the events generated by the device.
$E_{tm}^i(d,\phi)$ defines the energy consumed by a single device ($d$) over time as a function of current drain ($I$) and time ($t$) spent in an operating mode ($m$) with a specific configuration ($\phi$).
$E_{ev}^i(d,\phi)$ is the sum of the energy consumed by events that are relevant in terms of energy consumption.
During execution, the system accounts for these events ($\chi$).
Each event has a known worst-case energy consumption ($E_e(\phi)$) that is subject to a specific configuration ($\phi$).
In the case of a device consuming energy in both ways, both energy consumption profiles can be applied. 
Finally, $E_{tot}^i(\phi)$ is the sum of the estimated energy consumption of all system devices ($\Delta$), during the $ith$ iteration.
\begin{eqnarray}
\label{eq:en_dev_time} E_{tm}^i(d,\phi) = (t_{end} - t_{begin}) \times I(d,m,\phi)\\
\label{eq:en_dev_ev} E_{ev}^i(d,\phi) = \sum_{e=0}^{|\chi|} (E_e(\phi) \times \chi_e)\\
\label{eq:energy} E_{tot}^i(\phi) = \sum_{d=0}^{|\Delta|}{(E_{tm}^i(d,\phi) + E_{ev}^i(d,\phi))}
\end{eqnarray}

$E_{tm}^i$ is updated either on every operating mode change or periodically, when $E_{ev}^i$ and $E_{tot}^i$ are also updated.
The period of the iterations ($i$) vary from application to application and has already been subject of previous studies~\cite{Hoeller:SMC:2011}.

% Accouter modification to include recharge
Energy production depends on the energy harvesting technology, and the used model should be detailed for each application.
Regardless of the harvesting technology, however, the amount of energy available to the system in a moment ($E_{batt}^i$) can be estimated as shown in Equation~\ref{eq:batt_charge}.
Given a previously known battery charge ($E_{batt}^{(i-1)}$), current battery charge comes from the subtraction of the amount of energy consumed in a period ($E_{tot}^i(\phi)$), and the sum of the amount of energy coming in from the harvesting system ($R_i$).
As the amount of stored energy cannot be infinite, the computed battery charge has to be limited by the battery capacity ($E_{batt}^{max}$).
\begin{equation}
\label{eq:batt_charge} E_{batt}^i = \min \left ( E_{batt}^{max} , E_{batt}^{(i-1)} - E_{tot}^{i}(\phi) + R_i \right )
\end{equation}

% Exemplify through photovoltaic data
For example, energy production through photovoltaic modules, besides being subject to characteristics of the mechanisms themselves, is also dependent on solar irradiance and temperature.
In solar systems, irradiance is the parameter that primarily influences energy production.
Figure~\ref{figtwo:plot_1dayplot_1year} shows that irradiance varies considerably during daytime (top), as does irradiation\footnote{Solar irradiation is the cumulative energy brought from solar irradiance over a period of time, i.e. $E_I = \int_{t_0}^{t_1} I dt$.} during the year (bottom).
% In contrast with the wind, 
It is possible to observe, however, that an important repetition pattern exist for sunlight irradiance: it is not present during nighttime and tends increase during the morning, until it reaches a peak around noon, to start to decrease again.
In order to schedule a system in an adequate way, an energy-aware scheduler should be aware of such variations.
By doing so, the system might store energy during the day to use at night, being, thus, able to reduce oscillations on system quality.

% \wfig{wind-epagri}{Ten minutes samples of average wind velocity and direction from August 20th to August 22nd, 2012, in Florian{\'op}olis, Brazil.\\TODO: english version of figure}{width=2\columnwidth}
\figtwo
{plot_1day}{Solar irradiance of January 1st, 1998.}{width=\columnwidth}
{plot_1year}{1998 monthly irradiation.}{width=\columnwidth}
{Sample solar irradiance and irradiation at Florian{\'o}polis, Brazil ($27^o$ latitude).}{0.5\textwidth}


% Energy buffer size
The size of the energy storage ($E_{batt}^{max}$ in Equation~\ref{eq:batt_charge}) is another key parameter in the present scenario once the system may not store all the excess energy due to the lack of capacity.
There are other parameters affecting the performance of the energy storage in either rechargeable batteries or supercapacitors, such as capacity degradation, aging, and operating temperature.
These issues, although relevant, are by now outside the scope of this work.

\section{System Design and Implementation}
\label{sec:design}

The \epos~Project (Embedded Parallel Operating System) aims at automating the development of embedded systems so that developers can concentrate on the applications~\cite{Project:EPOS:2012}.
\epos~relies on the Application-Driven Embedded System Design (\adesd) method~\cite{Frohlich:2001} to guide the development of both software and hardware components that are adaptable.
The adaptation of components takes place during design and aims at fulfilling the requirements of applications.
\epos~has a set of tools to support developers in selecting, configuring, and plugging components into an application-specific framework~\cite{Schulter:JOT:2007}.
The combination of methodology, components, frameworks, and tools enable the automatic generation of application-specific embedded system instances.

Besides the run time support system and tools, the \epos~Project has driven the development of hardware platforms, being the \emote~among them.
\emote~is a modular platform for wireless sensor network applications.
The implementation of the approach described in this paper used the \emote~platform.
The implementation also modified \epos's scheduler and power manager.
The following sections describe the implementation.

\subsection{\emote}
\label{sec:emote}

The \emote~is a modular platform for building wireless sensor network applications.
Figure~\ref{fig:emote2_epos-block_diagram} shows the block diagram with the three modules of the platform.
The \texttt{Processing Module} incorporates the core processing and communication components of the system.
There are two different versions of this module: one based on Atmel ZigBit System-in-a-Package (SiP) and another, used in this work, based on Freescale MC13224V System-on-a-Chip (SoC).
Both present RF transceivers compatible with IEEE 802.15.4 standard~\cite{IEEE802154:2006} and an integrated processor, which is an 8-bit AVR for the Atmel SiP and a 32-bit ARM7 for the Freescale SoC.

The application designer can adapt the hardware to the needs of each application through the power supply and IO interfaces factored out on \emote~design.
The power interface features separate signals for the power source (i.e. $V_{cc}$, $V_{dd}$ and $Gnd$) and an $I^2C$ interface for communicating with the processing module.
The IO interface has 34 pins available for custom designs, including a bypass of the power source, all $ADC$ channels, $SPI$, $UART$ and several $GPIO$ pins.
The \emote~Project developed a \emph{Start-Up} board to be connected to the IO interface that features a $USB$ converter, a thermistor, a 3-axis accelerometer, LEDs, and push buttons.
This work also used the \emph{Start-up} board.  

\wfig{emote2_epos-block_diagram}{Block diagram of the \emote~platform.}{width=.6\columnwidth}

To account for energy consumption of \emote~we first need to produce its power characterization and map it to the energy model described in Section~\ref{sec:en_model}.
This information comes either from the components datasheets, when available, or from measurements in a real system.
Table~\ref{tab:emote-energy_currents} shows values of current drains of system devices in different operating modes to be used by time-based accounters.
Table~\ref{tab:emote-energy_consumptions} shows how much energy each monitored event consumes.
The event accounting system uses this information to estimate the amount of consumed energy.

\mytab{emote-energy_currents}{Current drain of CPU and Radio components of the \emote.}

\mytab{emote-energy_consumptions}{Energy consumption of monitored events of the \emote.}

\subsection{\epos~Power Manager}
\label{sec:epospm}

Once energy is a non-functional property of computing systems~\cite{Lohmann:2005}, the design of \epos~power manager uses aspect-oriented programming~\cite{Mens:1997} to implement its functionalities in a way that is orthogonal to other components of the operating system.
\epos~implements aspects as constructs called \emph{Scenario Adapters}~\cite{Frohlich:SCI:2000} that rely on the static metaprogramming capability of C++ (templates).% and does not require the use of extra tools such as aspect weavers.

\wfig{uml-class-pm}{UML class diagram of \epos~power manager.}{width=\columnwidth}

Figure~\ref{fig:uml-class-pm} shows a class diagram of the \epos~power manager designed as a scenario adapter.
The base class \texttt{Power\_Manager} wraps the target class, i.e. the class that the aspect modifies.
The wrapping happens through inheritance and function overriding (from the \emph{Adapter Design Pattern}~\cite{GangOfFour:1994}).
Additional methods may be easily included, as is the case of the \texttt{power} methods in of the base \texttt{Power\_Manager} class.
\texttt{Power\_Manager} is, in turn, a facade (from the \emph{Facade Design Pattern}~\cite{GangOfFour:1994}) to other power management functionalities implemented by other components.
For instance, \texttt{Power\_Manager\_Shared} and \texttt{Power\_Manager\_Instances} are responsible for, respectively, controlling of operating modes for shared components, and keeping of object references for system-wide power management actions~\cite{Hoeller:DIPES:2006}.

An extension to \epos~power manager, the \texttt{Power\_Manager\_Accounter}, implements the energy model described in Section~\ref{sec:en_model}.
This extension enables the energy consumption accounting functionality in \epos.
The \texttt{account(e:Event)} method accounts the events using the event-based profile (Equation~\ref{eq:en_dev_ev}), which may be called in a wrapped method if the event generated by such a method is a monitored one.
As concerning overhead issues, it is important to note that \texttt{account(e:Event)}, which increments an event counter, is an inline function, thus incurring in no overheads due to function calls at runtime.
Also, the branches that implement the facade at \texttt{power(m:OP\_Mode)} of \texttt{Power\_Manager} use constant boolean values which \epos~defines at configuration time, before system generation.
As such, these are subject to compiler optimizations that remove the branches in the final binaries of the system.
Table~\ref{tab:pm-overhead} presents the impact of the proposed accounter in terms of code size and data memory usage.
As can be seen, the accounting mechanism aggregates 2,768 bytes of code (ROM) and 70 bytes of data (RAM) to the original, fully functional \texttt{Power\_Manager}.

\mytab{pm-overhead}{Memory footprint of \epos~power manager.}

\subsection{\epos~Real-Time Scheduling}
\label{sec:epossched}

Figure~\ref{fig:uml-class-ea_sched} shows the three main components forming the real-time scheduling support on \epos: \texttt{Thread}, \texttt{Criterion}, and \texttt{Scheduler}.
The \texttt{Thread} class represents an aperiodic task and defines its execution flow, with its own context and stack.
This class implements traditional thread functionalities, such as suspend, resume, sleep, and wake up operations.
The \texttt{Periodic\_Thread}\footnote{A periodic thread in \epos~is conceptually equivalent to a real-time periodic task.} provides support for periodic tasks by extending the \texttt{Thread} class and aggregating mechanisms related to the re-execution of the periodic task.
The \texttt{wait\_next} method performs a \texttt{p} operation on a semaphore, forcing the thread to sleep until it reaches the next activation instant.
Each periodic thread aggregates an \texttt{Alarm} object that is responsible for performing a \texttt{v} operation that, periodically, releases and wakes up the thread.

\wfig{uml-class-ea_sched}{UML class diagram of the scheduling structure of \epos.}{width=\columnwidth}

The \texttt{Scheduler} class and \texttt{Criterion} subclasses define the structure that realizes task scheduling.
Usually, object-oriented OS scheduler implementations use a hierarchy of specialized classes of an abstract scheduler class.
In this case, subclasses specialize the abstract class to provide different scheduling policies~\cite{Marcondes:EPS:2009}.
\epos~reduces the complexity of maintaining such hierarchy and promotes code reuse by detaching the scheduling policy (here represented by the \texttt{Criterion} subclasses) from its mechanism (e.g., data structure implementations as lists and heaps).
The data structure in the scheduler class uses the defined scheduling criterion to order the tasks accordingly.
During compilation, the \texttt{Trait}\footnote{A trait class is a template class that associates information of a component at compile time.} class of \texttt{Thread} defines the scheduling criterion.
For example, \texttt{typedef Scheduling\_Criteria::EAEDF Criterion} defines the scheduling criterion as Energy-Aware EDF.
The \texttt{Scheduler} consults the information that the criterion class provides to define the appropriate use of lists and operations.

Each criterion class defines the priority of a task, which the scheduler uses to choose a task (\texttt{operator ()}), and other criterion features, such as preemption and timing, for instance.
In this work, we extended the EDF and RM criterions to support energy-aware operation.
We created a flag (\texttt{ENERGY\_AWARE}) that informs the scheduler whether the criterion is energy-aware or not.
As shown in the sequence diagram of Figure~\ref{fig:uml-seq-ea_sched}, the scheduler uses the flag to decide whether it must check for energy availability or not before dispatching a best-effort task.
It is important to highlight that flags are static and constant values\footnote{In C++: \texttt{static const bool ENERGY\_AWARE = true;}} and, due to this reason, the compiler optimizes the if-statements that verify whether the criterion in use is energy-aware or not, reducing the runtime overhead.

\wfig{uml-seq-ea_sched}{UML sequence diagram of the adaptation of the scheduling mechanism of \epos.}{width=0.8\columnwidth}

With this separation of concerns among scheduler, criterion, and thread, it is straightforward to add new scheduling policies into the system.
Moreover, in cases where a scheduling policy requires specific scheduling treatment, a new scheduler may be created by extending the existing schedulers through metaprogramming specialization techniques~\cite{Czarnecki:2000}.


\section{Adaptively sustaining system operation}
\label{sec:adaptation}

% Rechargeable energy sources correctness depend on avoiding failure for running out of energy
Systems featuring rechargeable energy sources have their lifetime bounded to their ability to sustain the energy needed for its operation.
This contrasts, in part, to the traditional concept that the lifetime of a wireless sensor network without renewable energy sources is bounded to the lifetime of the batteries of its nodes.
In the context of real-time systems, the successful operation of a system depends on the compliance to the requirements of its critical parts.
As a consequence, an energy-aware scheduler for real-time systems must somehow guarantee that there will be enough energy to supply the critical parts of the system.

% The case for adapting the system - keep power bellow the discharge rate 
A basic approach for guaranteeing battery lifetime in a system is defining the maximum discharge rate of subsequent iterations ($D_{i+1}$).
This rate is a function of current battery charge ($E_{batt}^i$) and remaining required operation time ($t_r$ is the required time), as shown in Equation~\ref{eq:rate}.
An energy overload takes place when the system demand for energy surpasses this limit.
In this situation, it becomes impossible to keep the energy consumption bellow $D_i$ without, somehow, adapting the system.
Here, a scheduler adjusts periodically the execution rate of best-effort tasks (called ``BET rate'' from now on) to the maximum rate satisfying the computed battery discharge rate.
\begin{equation}
\label{eq:rate} D_{i+1} = \frac{E_{batt}^i}{t_r - t_i}
\end{equation}

% Energy reserve for hard real-time tasks
One assumption of the proposed energy-aware scheduler is that 100\% of the hard real-time tasks execute until the system reaches the targeted lifetime.
Consequently, the scheduler needs to reserve energy for those tasks, what makes all scheduling approaches studied in this work conservative.
This reserve ($E_{res}$), as Equation~\ref{eq:reserve}\footnote{$\Gamma$ is the set of hard real-time tasks; $T_{\Gamma_j}$ is the period of the $jth$ task of $\Gamma$; and $E_{\Gamma_j}$ is the worst-case energy consumption of the $jth$ task of $\Gamma$.} shows, is the sum of the worst-case energy consumption ($E_{\Gamma_j}$) of all instances of hard real-time tasks ($\Gamma$) to be released until the system reaches the targeted lifetime ($t_r$).
\begin{equation}
\label{eq:reserve} E_{res}^i = \sum_{j=0}^{|\Gamma|}{\frac{t_r - t_i}{T_{\Gamma_j}} \cdot E_{\Gamma_j}}
\end{equation}

% Heuristic approaches are proposed to allocate excess energy to best-effort tasks
The following paragraphs describe the four approaches evaluated in this work.
The approaches follow the premises specified on Equations~\ref{eq:rate} and~\ref{eq:reserve}.
This means that all proposed approaches will reserve energy for the hard real-time tasks and allocate excess energy to BET tasks.

% \subsection{BGR -- \bgr}

% Describe the approach
Equation~\ref{eq:bgr} defines the first approach: \bgr~(BGR).
BGR allocates all excess energy (i.e. energy available for BET tasks) to the BET tasks that will run in the next iteration.
At first sight, it might not be a wise decision once the system may run out of excess energy very quickly.
It may not happen, however, in systems where incoming energy exceed consumed energy in an iteration period.
Another drawback of this approach may be its deployment in systems where energy income is sporadic or periodic.
For instance, a photovoltaic system using this approach may perform well during the day but may run out of excess energy during the night due to the absence of incoming energy.
The same may happen with wind turbine systems when a long lull period happens.
\begin{equation}
\label{eq:bgr} BGR_i = \min \left (~1.0~,~\frac{E_{batt}^i - E_{res}^i}{\sum_{j=0}^{|\beta|}{\frac{(t_{(i+1)} - t_i)}{T_{\beta_j}} \cdot E_{\beta_j}}}~\right )
\end{equation}

% TODO: Reasoning: it is highly dependent on energy buffer size (i.e. larger buffers render more excess energy)

% TODO: Graphically show the dependencies using a synthetic example

% \subsection{BWR -- \bwr}

% Describe the approach
Equation~\ref{eq:bwr} presents an approach to prevent the system from running out of excess energy during periods of low energy income and, also, to make a fairer distribution of BET rate along time.
This heuristic, the \bwr~(BWR), divides the excess energy among distinct time windows ($w_s$).
Although ``fairer'', this approach may negatively affect the BET rate depending on the amount of excess energy available to the system, i.e. the size of the energy storage.
It is also sensitive to the relation between the desired lifetime ($t_r$) and time window size ($w_s$). 
\begin{equation}
\label{eq:bwr} BWR_i = \min \left (~1.0~,~\frac{\frac{E_{batt}^i - E_{res}^i}{(t_r - t_i) \div w_s}}{\sum_{j=0}^{|\beta|}{\frac{(t_{(i+1)} - t_i)}{T_{\beta_j}} \cdot E_{\beta_j}}}~\right )
\end{equation}

% TODO: Reasoning: supposed to be fairer but affects significantly the BET rate. Highly dependent on buffer size, and on the relation (lifetime / window size)

% TODO: Graphically show the dependencies using a synthetic example

% \subsection{BWAR -- \bwar}

% Making suppositions about the future energy income
Two other approaches address the enhancement of the performance of the energy allocator to handle sporadic and periodic energy sources.
Both approaches deal with variations in energy income spreading the excess energy evenly across the remaining operation time of the system.
This may prevent the system from passing long periods of time without being able to execute BET tasks.
Another addition to these policies is that both approaches try to predict the short-time future of energy income based on recent history.
This is particularly useful for systems where the storage (i.e. battery) cannot accommodate all the incoming energy, because it helps to reduce the wasted energy.

% Describe the approach
Equation~\ref{eq:bwar} shows the \bwar~(BWAR).
BWAR tries to raise BET rate by predicting that the energy income for the next iteration will be the average of the energy income of the previous iterations.
In the equation, $R$ is the series of incoming energy, $w$ is the number of previous iterations taken into account for the average, and $w_s$ is the length of each iteration.
This approach may perform well for sporadic energy sources and aleatory energy sources, such as wind.
For periodic energy sources such as the solar-based ones, the average may render several errors.
Moreover, the prediction accuracy of this method is highly dependent on the size of the look-back window.
% TODO: plot BWAR prediction error.
As can be seen in Figure~\ref{figtwo:plot_1dayplot_1year} (top), although interferences exist (e.g. clouds blocking sunlight), the tendency of the solar irradiance during a day shows a forth-order shape, raising during the first half of the day and lowering on the second half.
Thus, using recent average may be conservative during the first half of the day, and too aggressive during the second half.
\begin{equation}
\label{eq:bwar} BWAR_i = \min \left (~1.0~,~\frac{\frac{E_{batt}^i - E_{res}^i}{(t_r - t_i) \div w_s} + \frac{\sum_{j=i-(w+1)}^{i-1}{R_j}}{w}}{\sum_{j=0}^{|\beta|}{\frac{(t_{(i+1)} - t_i)}{T_{\beta_j}} \cdot E_{\beta_j}}}~\right )
\end{equation}

% TODO: Reasoning: dependent on the behavior of the input energy. Buffer size has less importance.

% TODO: Graphically show the dependencies using a synthetic example


% \subsection{BWDR -- \bwdr}

% Describe the approach
To cope with the cited limitations of BWAR, \bwdr~(BWDR), shown in Equation~\ref{eq:bwdr}, tries to reduce the prediction errors by taking the energy income derivative into account instead of the recent average.
In the equation, $R_{i-1}$, which is the amount of income energy in the last period, is summed to the expected variation in the energy income for the current period, i.e. its derivative.
In this heuristic, the Newton's difference quotient method (the last fraction), which is a straightforward two-point estimation to compute the slope of a nearby secant line through the previous two points of the energy input curve, numerically approaches this derivative.
\begin{equation}
\label{eq:bwdr} BWDR_i = \min \left (~1.0~,~\frac{\frac{E_{batt}^i - E_{res}^i}{(t_r - t_i) \div w_s} + R_{i-1} + \frac{R_{i-1} - R_{i-2}}{t_{i-1} - t_{i-2}}}{\sum_{j=0}^{|\beta|}{\frac{(t_{(i+1)} - t_i)}{T_{\beta_j}} \cdot E_{\beta_j}}}~\right )
\end{equation}

% TODO: plot BWDR prediction error. (maybe)

% TODO: Reasoning: dependent on the behavior of the input energy. Buffer size has less importance. Compare behavior to BWAR

% TODO: Graphically show the dependencies using a synthetic example


% ------------------------------------------------------------------------------
\section{Case study: scheduling in mobile WSN} \label{case}

The proposed approach was integrated to the power management mechanism of \epos~\cite{Frohlich:IJDSN:2011}.
\epos~is a component-based operating system for embedded applications.
% The approach implementation, following the philosophy behind \epos~power manager, is modeled as a scenario adapter, making it not intrusive to system components.
Also, we ran this implementation in the \emote~platform~\cite{Project:EPOS:2012}, a module for the development of low-power wireless sensor network applications.
Hardware evaluation was performed by analysis of data in an actual implementation.
The scheduling approach was evaluated by means of simulation taking the characterization of the mentioned platform into consideration.
The remaining of this section describes the setup of the evaluation scenario and the obtained results.


\subsection{\emote~Solar Energy Harvesting Circuit} \label{circ_eval}

In order to adequately characterize the developed circuit integrated to the \emote~sensing platform, an outdoor test was performed.
The test period was 62~hours and 45 minutes.
For the test, the \emote~radio was configured to constantly transmit random data.
This $100~\%$ duty cycle was used in order to test critical conditions and also to reduce test duration.
By reducing the duty cycle, it would be necessary to extend the test to several weeks, since the total current consumption when the system is in standby mode is only around $5~\mu A$, i.e., more than 5,000 times smaller than the $29~mA$ it consumes when transmitting data at full power.

%referencia EPOSMote Hardware

Figure~\ref{fig:bat_volt} shows the battery voltage behavior during the test.
In all recharge cycles, the battery has almost reached the fully recharged voltage.
However, there are differences among the peak values, which shows that the maximum voltage reduced from one peak to the previous one.
This fact is explained analyzing the amount of energy delivered by the battery.
Figure~\ref{fig:energy} shows that the system has lost energy after each cycle,
which means that, for a $100\%$ duty cycle, this system would not be self-sufficient.

\fig{0.26}{bat_volt}{Battery voltage behavior.}

The test started on March 16, 2012 at 9:15~pm, i.e. during night, what explains why the first descending slope in Figure~\ref{fig:energy} is shorter than the other two.
Current integration was used in order to calculate the energy expended on slopes.
The first complete slope spent $738.43~mAh$, while the second spent $746.96~mAh$.
The small difference between these values is justified by the difference on the daylight duration, which, although similar, was not perfectly equal in both days.
The analysis of these values is of paramount importance when selecting the capacity of the storage unit.
This extreme case~(i.e. $100\%$ duty cycle) helps to justify the need for an energy-aware scheduler in the system.
The descending slopes must be reduced in order to have a self-sufficient system.
% The energy consumption reduction procedure is done by executing only the critical tasks, which allows the system to work for longer periods.
% Predicting how much energy is going to be delivered by the solar panel is the key knowledge for deciding which best-effort tasks will be executed.  

\fig{0.26}{energy}{Energy evolution.}


% This prediction is possible when analyzing the relation between current delivered by the solar panel and the solar irradiance.
Figure~\ref{fig:i_panel} shows the current delivered by the solar panel, which was calculated by subtracting the system's current~(considered constant at $56~mA$ due to the constant duty cycle) from the input/output battery current.
Figure~\ref{fig:irrad} shows the solar irradiance acquired during the test by a pyranometer placed close to the system's photovoltaic panel at the same inclination~($27~^\circ$).
The current delivered by the solar panel and the solar irradiance were plotted in order to obtain an equation to correlate them.
This curve and its linear approximation are shown in Figure~\ref{fig:i_irrad_final} and Equation~\ref{eq:i_irrad}.
In this equation, $I_{panel}$ is the current delivered by the solar panel in mA and \emph{Irrad} is the solar irradiance in $W/m^{2}$.
This plot is an additional contribution of this work, since it is through this linear equation that the energy-aware task scheduler can map weather forecast to energy input.
This plot will motivate the development of new heuristics for the scheduler which will further improve its efficiency.

\fig{0.26}{i_panel}{Current delivered by the solar panel.}
\fig{0.26}{irrad}{Solar irradiance variation.}
\fig{0.26}{i_irrad_final}{Relation between current and solar irradiance.}

\begin{eqnarray}
I_{panel} = 0.20628 \times Irrad \label{eq:i_irrad}
\end{eqnarray}

Solar irradiance and delivered current present a linear behavior when variations in temperature are not taken into consideration.
% For this first approximation, a linear equation was used.
Figure~\ref{fig:temp} adds temperature variation, which is the main reason for the spread points in Figure~\ref{fig:i_irrad_final}.
A new mathematical model considering temperature variation is being developed in order to design an energy-aware task scheduler with more precise environmental prediction.

\fig{0.26}{temp}{Temperature variation.}


\subsection{Evaluation Scenario}

The application used to evaluate the approach is a mobility-enabled wireless sensor network.
This network runs the Ant-based Dynamic Hop Optimization Protocol~(\adhop) over an IP network using IEEE 802.15.4.
\adhop~is a self-configuring, reactive routing protocol designed with the typical limitations of sensor nodes in mind, energy in particular~\cite{Okazaki:SCPA:2011}.
\adhop's reactive component relies on an \emph{Ant Colony Optimization} algorithm to discover and maintain routes.
Ants are sent out to track routes, leaving a trail of pheromone on their way back.
Routes with a higher pheromone deposit are preferred for data exchange.

\epos~scheduler relies on \epos~power manager to adaptively run the system.
In \epos~adaptive task scheduling model, tasks are classified as hard real-time or best-effort.
In order to guarantee system lifetime, \epos~energy scheduler reserves the amount of energy the system will need to run hard real-time tasks until a desired lifetime is reached.
Best-effort tasks are only allowed to execute when excess energy exists.
In \epos~scheduler, different heuristics can be used to control system quality degradation when best-effort tasks are prevented from executing.
Once specific heuristics for managing energy consumption are not the focus of the present work, only \epos~global energy allocation heuristic was considered~\cite{Hoeller:SMC:2011}.

The objective of this case study is to demonstrate how the employment of the proposed energy input measurement mechanism enhanced system performance.
Thus, \adhop~had to be modified.
\adhop's tasks have then been classified as hard real-time or best-effort.
The main idea behind this setup was to homogenize the battery discharge for every node in the network to enhance the lifetime of the network as a whole.
Considering the radio the most energy-hungry component in a wireless sensing node, we made the design decision of modeling the routing activity of \adhop~as a best-effort task, as shown by the task set at Table~\ref{tab:adhop-taskset}.
The basic node functionality of sensing a value~(task $Sense$) and forwarding it through the radio to the next node~(task $Forward$) where modeled as hard real-time tasks.
The functionality of forwarding other nodes' packets~(and ants) when acting as a ``router'' was modeled as two best-effort tasks, one for monitoring the channel for arriving messages~($LPL$ - Low Power Listen), and another to effectively receive the message and route it to another node~($Route$).

\tab{adhop-taskset}{\adhop~case-study tasks' parameters\protect\footnotemark[3].}

\footnotetext[3]{
% T: task;
Period in $ms$;
WCET: worst-case execution time in $ms$;
WCEC: worst-case energy consumption in $\eta Ah$;
25-days: energy consumption for the targeted lifetime~(25 days) in $mAh$.}
\footnotetext[4]{This is a worst-case scenario as values of ``P'' and, as consequence, ``25-days'', may change due to adaptation.}
\footnotetext[5]{``Route'' is a sporadic task. Once it is a best-effort task in the system, we consider a hypothetic frequency of $2$ Hz~(period of 500 $ms$) to show the impact of routing in the node energy consumption.}

The simulation time was set to 25 days.
By analyzing the task set, it is possible to compute the total energy consumption of hard real-time tasks for the desired lifetime to be of $602~mAh$.
As a consequence, the initial battery charge for the system has to be greater than that to allow the system to reserve energy for the critical part of the application.
The battery capacity specified for this experiment is an of-the-shelf CR-2/$3V$ battery with a total capacity of $850~mAh$
%%%%% the one described in Section~\ref{circ_design}, i.e., a $2,100~mAh$ NiMH battery.
% For the simulation, the voltage model of the battery was defined by the cubic approximation shown in Figure~\ref{fig:bat_volt_charge}, that shows the expected battery charge for a battery voltage level reading.

\fig{.69}{adhop-char-ddr}{Data delivery response to BET rate.}

\fig{.69}{adhop-char-energy}{Average power and energy consumption response to BET rate.}

The simulation was performed in two steps.
In the first step, a simulation using the \textsc{OmNET++} Simulator characterized the application's response to variations in the execution rate of best-effort tasks~(BET rate).
As can be seen in Figures~\ref{fig:adhop-char-ddr} and~\ref{fig:adhop-char-energy}, lower energy consumption at lower BET rates comes at the cost of lower data delivery rate.
Also, it is possible to observe in the graphic that BET rates above $50\%$ have no significant impact on packet delivery.
Thus, it is assumed that BET rate will only be adjusted within the range $[0,50]$ as a means to further save energy.

In the second simulation step, the energy consumption of the system was simulated for 25 days, i.e., the target lifetime.
As can be seen in Figure~\ref{fig:sim-bet_rate}, the employment of the measurement mechanisms proposed here enhanced the BET rate.
This happens because the proposed approach allows for more frequent adaptations that reduces the pessimistic bias of the used scheduler.
It is also possible to observe that the new approach shows a better recover after relatively long periods of low irradiation, such as the one found between days 7 and 11 in Figure~\ref{fig:sim-irrad}.
Short periods of low irradiation, like the one on day 19, are better supported by the new approach.
This can be observed by the slight decrease on BET rate on day 19 for the original method, while the new method is still reaching $50\%$.
Over the 25-days period, $32.93\%$ of the best-effort tasks were executed using the new approach, against $26.81\%$ of BET rate of the former approach.
This meas a total gain of $1.23$ times.


\fig{.69}{sim-bet_rate}{Average BET rate over 25 days of execution using with and without the proposed approach.}

\fig{.69}{sim-irrad}{Daily irradiation levels used for simulation.}



\section{Related Work}
\label{sec:related}

Several works have addressed the deployment of energy harvesters in wireless sensor networks.
Although many of them did not take real-time constraints into account, some can be adapted to support timing requirements.
This section describes the approaches these works used in order to schedule energy usage in energy-harvesting systems.

An important factor for efficiently managing energy in energy-harvesting systems is the choice of an efficient predictor for energy input~\cite{Paradiso:2005}.
% Kansal (Heliomote)
Kansal et al.~\cite{Kansal:2007} introduced a prediction algorithm to support their power management approach for wireless sensor networks.
Their energy prediction model uses an exponentially weighted moving-average filter.
This method assumes that solar irradiation in a given time-slot of the day is similar to the irradiation in the same time-slot of other days.
Their power management approach took into account the predicted amount of energy input to maximize the duty-cycle of the system, subject to energy availability.
% Moser
Moser et al.~\cite{Moser:2007} used a similar approach.
Their work presents a system model, simulated experimental results, and a real implementation of the proposed control law.
The shown experiments, however, focus on a single sensor node, not taking wireless communication into account.
% Recas?
Ali et al.~\cite{Ali:2010} proposed an extension to these models.
Their approach take not only the past days into account, but also the solar irradiation levels of the current day.
This allows for lower prediction errors as, by monitoring the irradiation levels of the current day, it is possible to adjust predicted energy according to the most recent information, i.e., the prediction errors in the most recent time-slots.

A few works analyzed the problem of efficiently scheduling tasks in systems presenting varying energy budgets.
% Rusu:2003
Rusu et al.~\cite{Rusu:2003} presents a multiversion scheduling approach that chooses between the different versions of the same tasks that maximize the system value and prevents battery depletion.
Assuming that batteries may be recharged, they propose a static solution that maximizes value based on worst-case scenarios.
Also, they propose a dynamic scheme that takes advantage of the slack energy generated in the system when worst-case energy consumption does not take place.
% ElGhor:2011
El Ghor et al.~\cite{ELGhor:2011} proposed a modified EDF scheduler that takes energy production, storage, and consumption into account.
They show by simulation that the approach outperforms other algorithms in terms of rate of deadline misses and required size of energy storage.
% Sharma:2010
Sharma et al.~\cite{Sharma:2010} analyze the problem from a different perspective.
They consider that the function of the device in a WSN is solely to sense and to forward data through the wireless interface.
In this context, they propose a set of policies to minimize the size of an outgoing queue subject to energy availability.
They assume that, the smaller the queue size, the larger the amount of data forwarded through the network, and, consequently, the better the system quality.
% Dehghan
% - Still unpublished

From all studied related work, only Kansal et al.~\cite{Kansal:2007} have analyzed the impact of their work in terms network-wide performance.
Other works only analyze the performance of isolated energy-harvesting sensor nodes.
% For this reason, only the approach presented by Kansal and his colleagues have been considered for comparison with the results presented in this paper.

% Adaptive scheduling techniques have long been used to handle processing
% overloads in real-time systems.
% Buttazzo~\cite{Buttazzo:2011} classifies the adaptive scheduling techniques into three categories: job skipping, period adaptation, and service adaptation.
% Altought initially conceived to handle processing overloads, such techniques may also be applied to handle energy overload, once energy consumption decreases as a consequence of reducing the amount of work executed by the system.
% In fact, a few works have already explored this characteristic~\cite{a few
% works}.

% In job skipping techniques, tasks are prevented from executing when the system is overloaded~\cite{Koren:1995,Ramanathan:1995,Ramanathan:1999}.
% Period adaptation techniques alter tasks' frequencies in order to lower system utilization, hence making the system schedulable~\cite{Buttazzo:1998,Kuo:1991,Seto:1998,Abdelzaher:1997}.
% Service adaptation techniques selectively tune system quality set-points in order to be able to comply with system constraints~\cite{Liu:1991,Cucinotta:2010,Yuan:2006,Cornea:2003}.
% Regardless of the technique, systematic ways to adapt the system are deployed in order to control the impact on system quality.

% Some adaptive scheduling approaches have already been deployed in embedded systems and wireless sensor networks to ensure the lifetime of system batteries.
% Not all of them, however, take into account real-time requirements and/or the impact on system quality.

%VADC
% Wanner et al.~\cite{Wanner:2011} proposes a technique called \emph{Variability-Aware Duty Cycling} (\textsc{Vadc}) that adapts duty cycle of wireless sensors by changing task's periods.
% Besides traditional WSN power management techniques, they also take into account hardware power variability, being the duty cycle adapted to observed variations on energy consumption.
% It has a working implementation for \textsc{TinyOS} that includes abstractions for defining adaptable task parameters (period and iterations only) and system models to derive duty cycles able to guarantee a minimum battery lifetime.
% This work relate to ours in the sense that both works adapt the WSN system to guarantee that the energy demands won't ever surpass the energy availability.
% The core difference among them is that while \textsc{Vadc} considers that the energy demand varies due to hardware power variability, we consider that the energy availability varies.
% Moreover, once in our approach energy consumption is monitored at runtime, variations on energy demand are also captured, being the system also able to adapt to such demand variations. 

%Cinder
% Rumble et al.~\cite{Rumble:2009} presented a service adaptation technique based
% on resource reservation, taking energy as a manageable resource.
% The system is called \emph{Cinder}.
% They defined policies to ensure \textit{isolation}, \textit{delegation}, and \textit{subdivision} of allocated resources, doing it through an abstraction they called \emph{capacitor}.
% The system has a ``main'' capacitor (its battery) with a known charge.
% Each task is associated to a capacitor abstraction that is ``charged'' by a higher level capacitor.
% Child tasks also have capacitors, but instead of being ``charged'' by the battery, they drain resources from its parent task.
% Periodically, the system distributes resources (energy) through the capacitor tree.
% This distribution is based on a consumption rate defined as a function of required system lifetime and available energy.
% Cinder was implemented in the \textsc{HiStar} operating system for mobile phones.

%Levels
% Lachenmann et al.~\cite{Lachenmann:2007} proposed \textsc{Levels}, a service adaptation and multi-version scheduling mechanism for wireless sensing systems.
% Their objective is to maximize system quality while respecting a pre-defined requirement of battery duration lifetime.
% They focus on critical application where there is no redundancy and no node may fail before the required operation time.
% In their approach they assume a component-based system and define ``energy levels'' at which each of system components may behave in a different way, consuming less energy and decreasing system quality as the ``levels'' go down.
% At runtime, the system is adapted by setting its energy level to the one that is able to comply with de lifetime requirement given the amount of available energy.
% Levels was implemented in \textsc{TinyOS}.

%ECOSystem
% Ellis et al.~\cite{Zeng:2005} developed the \textsc{ECOSystem}, which is an operating system with support for application adaptation with focus on multimedia systems.
% Similarly to what Cinder does, they defined an abstraction for energy in the system, the \emph{currentcy}.
% In this system, authors expect the energy resource (currentcy) to be allocated in the same way other resources are (e.g., memory, i/o device), i.e., the application should explicitly allocate energy for its execution.
% Each application is allowed to use a pre-defined amount of energy.
% This amount is updated periodically based on the amount of available energy and the required working time.
% \textsc{ECOSystem} was implemented as a modified version of \textsc{Linux}.

% Table~\ref{tab:related_comparisson} shows a comparison of the presented work.
% The next section presents an approach for integrating the desired
% characteristics of the proposed framework (section~\ref{sec:obj}) and exploring
% application-driven adaptation of RT-WSN applications.

%Comparisson
%
% parameters: classe, real-time (soft/hard), mettering/account method, renewable
% energy, system variability (battery + hardware), simulation/real
% implementation/operating system, sincroniza��o, invers�o de prioridade
%
% \tab{related_comparisson}{Comparison of related work.}

% Summary 

% Contributions: (1) common and simple interface (minor), (2)
% Power-management on embedded systems without using any complex
% high-cost methodology.
In this paper we presented an strategy to enable application-driven
power management in deeply embedded systems. In order to achieve this
goal we allowed application programmers to express when certain
components are not being used. This is expressed through a simple
power management interface which allows power mode switching of system
components, subsystems or the system as a whole, making all
combinations of components operating modes feasible. By using the
hierarchical architecture by which system components are organized in
our system, effective power management was achieved for deeply
embedded systems without the need for costly techniques or strategies,
thus incurring in no unnecessary processing or memory overheads.

A case study using a 8-bit microcontroller to monitor temperature in
an indoor ambient showed that almost 40\% of energy could be saved
when using this strategy. % and with minimal application intervention.

% Problems: concurrence. Describe the Thread problem.

% The paper also listed some identified problems on the path for
% power-aware software and hardware components, discussing and
% explaining how some of these problems have been solved in this work
% and how some of them can be solved, and will be, in future work.

% Even so, it still have its usability.



% Bibliography
\bibliographystyle{acmsmall}
\bibliography{paper}

% History dates
\received{October 2012}{}{}



\end{document}
