\hyphenation{EmNets}

%%%%%%%%%%%%%%%%%%%%%%%%%%%%%%%%%%%%%%%%%%%%%%%%%%%%%%%%%%%%%%%%%%%%%%%%%%%%%%%

\begin{abstract}

  Developments in \wsn{} (WSN) hardware have led
  to a diversity of sensing devices, ranging from simple,
  micro-controlled boards to complex, highly integrated
  sensor-transceiver ICs. While this is a desirable effect, it brings
  about a question about sensor application portability. The lack of
  proper abstraction and encapsulation mechanisms at the operating
  system level often forces developers to re-implement their sensing
  applications whenever a different sensor, radio or processor is
  deployed. 
  
  In this paper we introduce a novel strategy for abstracting WSN
  hardware, based on the \emph{\aosd{}}
  method. The strategy was implemented for the \epos{} operating
  system and consists in providing application programmers with an
  \emph{inflated interface} for sensing components --- a high-level
  collection of methods and structures derived from the actual
  components implemented in the system. This interface can latter be
  bound to pre-existing components that are adapted on demand at system
  generation time to fulfill application requirements, thus enabling
  programmers to code portable sensing applications in spite of the
  hardware diversity and without significant overhead.

\end{abstract}

%%%%%%%%%%%%%%%%%%%%%%%%%%%%%%%%%%%%%%%%%%%%%%%%%%%%%%%%%%%%%%%%%%%%%%%%%%%%%%%

\section{Introduction}

\wsn{} (WSN) hardware is, by its very own nature, heterogeneous and
modular.  Application-specific requirements drive the entire hardware
design, from processing capabilities to radio bandwidth and sensor
modules.  Even in a same family of sensor nodes (e.g. the Berkeley
\emph{motes} family \cite{Polastre:2005,Hill:2004}) one can find
architectural differences that cannot be trivially abstracted.  
In this scenario, a sensor application developed for a given platform
will seldom be portable to a different one, unless the run-time support
systems on those platforms deliver mechanisms that abstract and
encapsulate the sensor platform in an adequate manner.
Indeed, a proper abstraction of sensor platforms becomes a key issue in
the face of modern \textsc{SoC} (\emph{system-on-a-chip}) solutions,
which integrate micro-controller, radio and sensor modules on single ICs
that can usually be configured according to user
demands~\cite{Rabaey:2002,Atmel:rf:2003,Hill:2003}.  
Table \ref{tbl:hwcomponents} presents characteristics of some
representative WSN hardware modules.  Architectural differences aside,
sensor modules (e.g. temperature, light, motion sensors) present an even
wider range of variability.  Sensor modules presenting the same
functionality often vary in their access interface, operational
characteristics and parameters.


\begin{table*}[tbh]
\begin{center}
\input{tbl/motes.tbl}
\end{center}
\caption{Typical Wireless Sensor Networks  Hardware}
\label{tbl:hwcomponents}
\end{table*}


%Table~\ref{tbl:sensors} presents these characteristics for a sample
%family of temperature sensors commonly used in sensor nodes.

%\begin{table*}[htb]
%\begin{center}
%\input{tbl/sensor.tbl}
%\end{center}
%\caption{Sensor Module Characteristics}
%\label{tbl:sensors}
%\end{table*}

A properly designed run-time support system could free application
programmers from such architectural dependencies and promote application
portability among different  sensing platforms.  Given a sensing
application implemented for a certain platform, there is no strong
reason why it should not be reused with another platform that fulfills
its requirements (e.g. presence of a certain sensor, non-volatile
memory capabilities, etc.).  For instance, an operating system could
deliver a temperature sensor abstraction that would be instantiated by
applications giving a range and scale (e.g. linear, logarithmic,
Poisson, user-defined, etc). The system would thus ensure proper
behavior independently from the physical sensor that exists in the
platform.

In the following sections we present current strategies for handling
heterogeneity in \emph{\wsn{}} and introduce a novel model based on the
\emph{\aosd{}} (AOSD) methodology \cite{Froehlich:2001}.  AOSD
introduces the concepts of \emph{hardware mediators}, and \emph{system
  abstractions} that implement \emph{scenario independent} system
constructs.  These abstractions are exported to the application through
\emph{inflated interfaces} and adapted according to the applications
needs through \emph{scenario adapters}.  We introduce the model of
\emph{sensing abstractions} used in \epos{}, an experimental
\emph{application oriented} operating system.  We then show that this
model is capable of properly abstracting WSN hardware heterogeneity with
virtually no overhead and producing a very small system footprint.

%%%%%%%%%%%%%%%%%%%%%%%%%%%%%%%%%%%%%%%%%%%%%%%%%%%%%%%%%%%%%%%%%%%%%%%%%%%%%%%

\section{Related Work}

Most of the current effort in development of operating systems and
hardware abstraction for Wireless Sensor Networks is focused on
\emph{\tos{}}~\cite{Hill:2004}.
Initially developed by the University of California at Berkeley and now
an open-source project maintained by several institutions,
it is the most widely used OS for \wsn{}.
\tos{} was written in NesC~\cite{Gay:2003}, 
a high-level programming language that emulates the syntax and
functionality of hardware description languages, and
provides components for communication, thread coordination, and, of
our special interest, hardware abstraction.

Nonetheless, while implementing a complete and functional
{hardware abstraction layer}, the original \tos system presented a
series of structural problems that hindered application
portability. Each hardware platform had a complete and separate HAL
implementation. 
While it is true that architectural
differences between platforms may require separate implementations,
system interfaces should remain uniform whenever
possible. The original \tos{} did not provide a uniform interface for
most high level system abstractions (e.g. sensors, timers, etc.), thus
forcing the application programmer to understand the nuts and bolts of
the underlying hardware platform and compromising application
portability between platforms.

\tos{} tried to solve these problems by introducing a three-tiered hardware
abstraction architecture, comprised by a \emph{Hardware Presentation Layer}, a
\emph{Hardware Adaptation Layer} and a \emph{Hardware Interface Layer}. 
The Hardware
Presentation Layer is placed directly over the underlying hardware, and
\emph{presents} the hardware to the operating system. Components in the
Hardware Presentation Layer are unique for each device they present, but may
share a common structure. The Hardware Adaptation Layer groups the 
hardware-specific components into domain-specific models, such as 
\texttt{Alarm} or \texttt{ADC Channel}. The Hardware Adaptation Layer 
provides the ``best'' possible abstraction in terms of effective resource
usage, but also tries not to hinder application portability. The Hardware
Interface Layer uses the adapted components to implement
platform-independent abstractions. The \tos{} application developer may
choose to use any of the available interface levels, trading off 
application portability and efficient resource usage.

The \emph{Mantis Operating System}~\cite{Abrach:2003}, developed by the
University of Colorado, aims at making the task of programming a sensor
network as close as possible to the one of programming a PC.
Thus, \mos{} uses the classical model of multi-layered operating systems,
witch includes multi-threading, preemptive scheduling, and a network
stack. Hardware is abstracted through a \emph{UNIX-like} API of device
drivers, with a monolithic hardware abstraction layer.

While it us true that the classical OS structure used in \mos{} may
lower the learning curve for novice sensor network developers familiar
with an \emph{UNIX-like} system, it is uncertain whether this model translates
well to such resource-restricted hardware as sensor nodes. The minimal
hardware requirements for \mos{} are small when compared to classical
embedded operating systems \cite{Massa:2003}, they may be too restrictive for
tiny devices.

Higher level abstractions often rely on Virtual Machines that abstract
the physical hardware into an ideal Virtual Architecture. An
underlying translation mechanism ensures correct operation, regardless
of specific physical hardware platform details.  Thus even highly
efficient Virtual Machine implementations, such as \emph{Mat\'{e}}
\cite{Levis:2002} introduce overhead into the system. {Mat\'{e}}
introduces a high-level interface that allows programs to be
replicated throughout the network, reprogramming the nodes in an
energy-efficient way. It also provides a safe execution environment, 
implementing a user/kernel boundary on devices that lack hardware 
protection mechanisms. In spite of its advantages, {Mat\'{e}}
suffers from the problems inherent to virtual machines, which are
especially critical for highly constrained hardware such as sensor
nodes, as practical high level abstractions for Wireless Sensor
Networks hardware must make efficient use of the sensor node's low
memory, processing and energy capabilities.


%%%%%%%%%%%%%%%%%%%%%%%%%%%%%%%%%%%%%%%%%%%%%%%%%%%%%%%%%%%%%%%%%%%%%%%%%%%%%%%

\section{\aosd}

In this section we introduce the concepts of \emph{system
abstractions} and \emph{hardware mediators} in the context of
\emph{Application-Oriented System Design} as efficient, high level,
reusable hardware abstraction components for sensor networks.
\emph{Application-Oriented System Design}~(AOSD)~\cite{Froehlich:2001}
was proposed as a multi-paradigm methodology for software design that
makes use of several programming and software engineering techniques
that can be combined in order to generate run-time support systems
configured and optimized for specific applications. The \textsc{Epos}
operating system was implemented following AOSD techniques, and has
since been successfully used in several
environments~\cite{Froehlich:HPCN:2000,Polpeta:EUC:2004}.

The main goal of the \textsc{Epos} system is to allow application
programmers to write architecture-independent applications, and, through
the application analysis, deliver run-time support that complies all the
resources that specific application needs, and nothing else.  
In order to achieve these goals, \textsc{Epos} relies on
the concepts of \emph{Inflated Interfaces}, \emph{System Abstractions}, 
\emph{Scenario Aspects}, \emph{Configurable Features}, 
and \emph{Hardware Mediators}.
%as defined in \cite{Froehlich:2001}.
\textsc{Epos} makes  use of \emph{Static Meta programming} and 
\emph{Aspect-Oriented Programming} techniques to implement software 
components, thus conferring them a
significant advantage over the classic approaches of VMs and HALs.
From the definition of the scenario in which the component will be
deployed, it is possible to adapt it to perform accordingly without
compromising its interface nor aggregating useless code.

\subsection{Hardware Mediators}



\emph{Hardware mediators} are proposed as
software constructs that mediate the interaction between operating
system components, called \emph{system abstractions}, and hardware
components.  The main idea behind hardware mediators is not building
universal hardware abstraction layers and virtual machines, but
sustaining the \emph{``interface contract''} between system and
machine.

Differently from ordinary \textsc{HAL}s, hardware mediators do not build
a monolithic layer encapsulating the resources available in the hardware
platform. Each hardware component is handled via its own mediator, thus
granting the portability of abstractions that use it without creating
unnecessary dependencies. Hardware mediators are intended to be
mostly meta programmed and therefore dissolve themselves in the
abstractions as soon as the interface contract is met. In other words, a
hardware mediator delivers the functionality of the corresponding
hardware component through a system-oriented interface.

\fig{family}{A family of hardware mediators.}{width=7.6cm}

\begin{figure}[htbp]
\lstset{language=c++,frame=lrtb}
\lstset{basicstyle=\ttfamily}
\lstset{commentstyle=\textit}
\lstinputlisting{fig/avr_gpio.cc}
\caption{The AVR\_GPIO hardware mediator}
\label{fig:hm_gpio}
\end{figure}


Hardware mediators are organized in families whose 
members represent the significant entities in the domain. For
instance, a family of \texttt{CPU} mediators would feature members such
as \texttt{ARM}, \texttt{AVR8}, and \texttt{PPC}. % (figure~\ref{fig:family}). 
A simple hardware mediator for the GPIO bus in the AVR processor
is presented in figure ~\ref{fig:hm_gpio}. This mediator abstracts
the assembly language operations responsible for
reading and writing to IO ports into high level C++ operators
without any overhead.




%%%%%%%%%%%%%%%%%%%%%%%

\subsection{System Abstractions and Inflated Interfaces}

AOSD relies on \emph{Scenario-Independent System Abstractions} to
implement the operating system components. These components define the
system functionalities, and their implementations rely on several
techniques such as \emph{Static Meta programming} and
\emph{Aspect-Oriented Programming},  allowing these abstractions 
to be strongly configurable.

System Abstractions are collected from an \emph{Application-Oriented
Domain Analysis and Decomposition} process. This analysis process is
quite similar to \emph{object-oriented decomposition}. The main
difference is that the Application-Oriented System Design is a
multi-paradigm design methodology, so other entities, such as aspects
and configurable features must come out from this analysis.

Each system abstraction is composed by a set of similar operating
system components. These components are organized according to the
\emph{Family-Based Design} paradigm, and have their \emph{commonalities}
and \emph{variabilities} explored through different class hierarchies.
An \emph{inflated interface} exports the family as though a ``super''
component, that implements all responsibilities assigned to the
family. This component is derived from the interfaces of individual
family members, and realized through their implementations. In \textsc{Epos},
the system framework automatically selects interface realizations,
taking into account the target hardware configuration and a cost model
for components.


\subsection{Scenario Aspects and Configurable Features}

In AOSD, non-functional aspects
and cross-cutting properties are factored out as \emph{scenario aspects}
that can be applied to family members as required. For instance,
families like \texttt{UART} and \texttt{Ethernet} must often operate in
exclusive-access mode.  This could be achieved by applying a
share-control aspect to the families.

\emph{Configurable features}, on the other hand,
designate features of mediators that can be switched
on and off according to the requirements dictated by abstractions.  
A configurable feature is not restricted to a flag indicating whether a
preexisting hardware feature must be activated or not. Usually, it
also incorporates a \emph{Generic Programmed}~\cite{Musser:1989}
implementation of the algorithms and data structures that are
necessary to implement that feature when the hardware itself does not
provide it.  An example of configurable feature is the generation of
CRC codes in an \texttt{Ethernet} mediator.


%%%%%%%%%%%%%%%%%%%%%%%%%%%%%%%%%%%%%%%%%%%%%%%%%%%%%%%%%%%%%%%%%%%%%%%%%%%%%%%

\section{\textsc{Epos} Sensing Components }

The main goal of the \textsc{Epos} system is to allow application
programmers to write architecture-independent applications, and, through
the application analysis, deliver run-time support that complies all the
resources that specific application needs, and nothing else.  In order
to match its goal, \textsc{Epos} relies on the
\emph{\aosd{}}~(AOSD)~\cite{Froehlich:2001}
method to guide the development of system abstractions as families of
software components, each of which implements a
\emph{scenario-independent abstraction} that can latter be adapted to a
given execution scenario with the aid of \emph{scenario adapters}.
Software components are collected in a repository and are exported to
the application programmers via \emph{inflated interfaces}, which hide
the peculiarities of each member in a family as though the whole family
was a component. This strategy, in addition to drastically reducing the
number of exported abstractions, enables programmers to easily express
their application's requirements regarding the operating system.

In order to preserve the portability of its software components,
\textsc{Epos} relies on hardware mediators. In principle, none of
\textsc{Epos}'s abstractions interact directly with the hardware,
utilizing the corresponding hardware mediators instead. In this way, a
context switch done in the realm of a \texttt{Thread} abstraction
concerns mainly the decision of which thread should occupy the
\textsc{CPU} next, leaving the operation of saving and restoring the
\textsc{CPU}'s context to the corresponding mediator. 
%This might seem an
%obvious example, but defining the interface for \texttt{CPU} mediator
%becomes a hard task if one considers the variability of \textsc{CPU}
%cores and the different environments in which they are used (e.g.
%multithread, multitask, multiprocessor, etc).

%Relevant hardware mediators for \textsc{EmNets} in \textsc{Epos}
%include, in addition to the usual modules (e.g. \texttt{CPU},
%\texttt{MMU}, \texttt{Timer}, etc.), \texttt{Radio} and
%\texttt{Sensor}. A more substantial example of System
%Abstractions for these platforms can be found in the
%\texttt{Network} and 
%\texttt{Sentient} family of abstractions. 
%The following sections will describe such abstractions.

%\subsection{\textsc{EmNets} Components}

%Aiming to deliver to the application programmer a way to write their
%applications without caring about which type of sensor the target
%platform is using, the \textsc{Epos} operating system defines a set of
%sensing abstractions and mediators. One important factor that allows
%this design is the use of static metaprogramming and C++ operator
%overriding techniques to allow hardware-independent access to hardware
%registers and buses~\cite{Polpeta:2004}.

\subsection{Sensors and Sentients}



There are several different architectures of sensors in use today, which
makes a hard job to define a common way to access such devices.  These
devices range from simple, digital integrated circuits to complex and
(usually) analog devices.

\figspan{families}{Set of relevant sensing components.}{width=16cm}

Devices in the first category are usually simple, digital integrated
circuits and are generally implemented over a serial bus (e.g. I2C,
SPI). In this kind of sensors, the data acquisition process is initiated
by a specific signal or by read and write operations. Another characteristic
of these sensors is that there is a regular time period in which new
values are sensed. Examples of these sensors are the Texas TMP family of
temperature sensors, most Honeywell Magnetometers, the STM
Accelerometers and most A/D converters.
On the other end, we have very specific and, generally, analog circuits.
Examples of this category of sensors are the Berkley's Mica Sensor Board
light and temperature sensors. These sensors share the same analog
circuit, making its use and management a complex issue. This circuit is
composed by two thermistors and one photo resistor, which are managed
through 3 GPIO pins, and deliver their results to the same A/D
pin. Even being a simple circuit, it has operation and timing
constraints. Another example is the Mica Magnetometer. It is implemented
by a complex analog and digital circuit and in spite of being configured
by a simple I2C potentiometer, it has an analog output and complex
semantics and timing characteristics.

Even with all these differences between sensing hardware, it is possible
to affirm that sensor operation does follow a \emph{regular pattern}.
This pattern is constituted by an ``start sensing'' command, followed by
an optional configuration phase. Once configured, there is a period of
waiting for the sensor's data acquisition phase. When data acquisition
is done, data is finally read.  The sensing routine can then be stopped,
or may be kept in a ``wait acquisition / get data'' loop.

In order to contemplate all possible kinds of sensing hardware,
\textsc{Epos} implements the families of mediators and abstractions
shown in Figure~\ref{fig:families}. The \texttt{Bus} mediator family of
components implements a uniform way to access peripheral devices. The
\texttt{ADC} family of A/D Converters mediators also implements an
uniform access to such devices. These two basic mediators' families are
the key to a modular implementation of the \textsc{Epos}'s sensing
sub-system, once it allows the \texttt{Sensor} family of
mediators to be composed by highly configurable and reusable components.
Some of the implemented mediators for the \texttt{Sensor} family
are also shown in the figure. Every other implementation just needs to
extend the \texttt{Sensor} component and make its hardware
dependent implementation.



These hardware mediators structures allow the design of a higher-level,
architecture-independent family of abstractions, which can be used by
applications without affecting its portability. The \texttt{Sentient}
family is comprised by software components that aims in abstracting the
sensors finality, and not its implementation (which is considered in the
\texttt{Sensor} mediator). Example members of this family are the
\texttt{Temperature\_Sentient} and the \texttt{Light\_Sentient} components. 
These components
implement not only transparent access for sensors, but also some
functionalities such as unit conversion, threshold comparison of
results, and data logging, which are mapped as configurable features or
aspects. 
Once the application uses a member of the \texttt{Sentient} family, it
is up to the \textsc{Epos} framework composition rules to maintain the
sensing sub-system coherence, by granting that, for example, the
\texttt{Temperature\_Sentient} abstraction will use an available
temperature \texttt{Sensor} mediator. This is ensured by the
system configuration and generation process, which takes into account
the application analysis results and the hardware platform
description~\cite{Tondello:2005}.



\section{Evaluation}



In order to test the expressibility, portability and cost of the
\textsc{Epos} sensing subsystem, we implemented a simple sensing
application using three different operating systems: TinyOS, Mantis e
\textsc{Epos}. The application in question implements a loop that
constantly reads data from a sensor and redirects the data read to a
\texttt{UART} interface. Whenever it was necessary to implement
architecture-dependent code, the Berkeley Mica2 mote~\cite{Hill:2000}
was considered as the target platform. In order to better compare the
overhead in the different implementations, the applications used the
least possible operating system resources. Figure~\ref{fig:app} presents
the implementation of this application for the three analyzed
operating systems. 

\begin{figure*}[t!]
\begin{center}
\begin{footnotesize}
%
\lstset{language=c++,frame=lrtb}
\lstset{basicstyle=\ttfamily}
\lstset{commentstyle=\textit}
\begin{minipage}{16.5cm}
\lstinputlisting{fig/app_tos.cc}
\end{minipage}
%
%
    \mbox{
      \subfigure{
      \begin{minipage}{10cm}
        \lstset{language=c++,frame=lrtb}
        \lstset{basicstyle=\ttfamily}
        \lstset{commentstyle=\textit}
        \lstinputlisting{fig/app_mantis.cc}  
      \end{minipage}   
      } \quad
      \subfigure{
      \begin{minipage}{6cm}     
        \lstset{language=c++,frame=lrtb}
        \lstset{basicstyle=\ttfamily}
        \lstset{commentstyle=\textit}
        \lstinputlisting{fig/app_epos.cc}  
      \end{minipage}     
      } 
      }
%
\caption{Sample sensing applications}
\label{fig:app}
\end{footnotesize}
\end{center}
\end{figure*}



The application for TinyOS was written in the \texttt{NesC}
programming language, and its implementation consists simply in
connecting inputs and outputs of interfaces (\texttt{ADC} and
\texttt{UART}, in this particular case). Had it been necessary, the
implementation of algorithms could be done in \texttt{NesC}, with
\emph{C-like} syntax. The application for Mantis was implemented in
\texttt{C}, including the headers defining Mantis's device access and
communication APIs. The application for \textsc{Epos} was implemented
in \texttt{C++}, importing the sensing abstraction and the
\texttt{UART} mediator headers.

In the Mica2 platform, the TinyOS system reflects the hardware design
and exports the temperature sensor as an ADC (in the physical platform 
this sensor is analog and is connected to the micro-controllers ADC). 
This dependency between hardware and operating system will certainly 
bring implications to the application portability when, for example,
it is ported to a platform in which the temperature sensor is digital
and is connected directly to microcontroler IO pins. Even if the application
functionality remains the same, it will have to be altered taking into
account the details of the hardware platform in which it will be executed.

A similar problem occurs in the implementation for the Mantis
operating system. The temperature sensor is read through a device
access function, which takes as a parameter the physical device the
application wishes to read. Evidently the physical sensor model may
vary from platform to platform, and the application will not be
portable between different platforms, even when it maintains the same
functionality. This problem could be partially solved in the Mantis
system by indicating through a series of \texttt{define} statements
that would indicate that, for example, in the Mica platform, the
\texttt{Temperature\_Sensor} symbol denotes \texttt{DEV\_MICA2\_TEMP}.
Nevertheless, this would still be an inefficient and inelegant solution,
as the \texttt{dev\_read} method aggregates code for the reading
of every device available in the platform, even when some of them
are not used by the application.

The implementation for the \textsc{Epos} system does not present
any dependency to the target hardware, except for the requirement of
an available temperature sensor, and is perfectly portable between 
platforms the satisfy this requirement. The selection of hardware
mediators is resolved by the system framework, taking into consideration
the interfaces used by the application and the target platform 
specified by the programmer.

Table~\ref{tbl:overhead} presents the code and data sizes for the
three generated applications. The \textsc{Epos} system presented the
smallest cost in bytes between the three test systems, for both
data memory and code size.



\begin{table}[h]
\begin{center}
\input{tbl/overhead.tbl}
\end{center}
\caption{Generated application sizes}
\label{tbl:overhead}
\end{table}

Evidently, an evaluation based simply in program code and data sizes is
incomplete, but in resource-limited systems such as sensor nodes, these
values are of uttermost importance. Future evaluations of this work will
include performance and energy-consumption measurements.



%%%%%%%%%%%%%%%%%%%%%%%%%%%%%%%%%%%%%%%%%%%%%%%%%%%%%%%%%%%%%%%%%%%%%%%%%%%%%%%

\section{Conclusions}

We discussed the problem of heterogeneity in sensor
networks hardware, and presented a novel technique to handle
application portability in these systems. We used the concepts of
\emph{hardware mediators} and \emph{system abstractions} to model and 
implement the \textsc{Epos} sensing subsystem, which allows sensing
applications to be ported between different sensor networks hardware
platforms.  A sample sensing application was implemented for the
\textsc{Epos} operating system, as well for other available operating
systems for sensor networks. The \textsc{Epos} system presented the
smallest size int terms of code and data size. These results position
\textsc{Epos} as a very viable alternative for operating systems for
sensing devices.

%%%%%%%%%%%%%%%%%%%%%%%%%%%%%%%%%%%%%%%%%%%%%%%%%%%%%%%%%%%%%%%%%%%%%%%%%%%%%%%



%%% Local Variables: 
%%% mode: latex
%%% TeX-master: "heterogeneity_master"
%%% End: 
