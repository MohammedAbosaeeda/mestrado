\section{Implementation in \textsc{EPOS}: a Case Study}\label{sec:epos}

In order to validate the power management strategy for embedded systems
proposed in this paper, which includes an API specification, guidelines
for power management infrastructure implementation through
aspect-programs, and design constraints for the development of
autonomous power management agents, these mechanisms have been
implemented in \textsc{Epos} along with the hypothetical remote
monitoring application described in section~\ref{sec:api}. 

\subsection{\textsc{Epos} Overview}

\textsc{Epos}, the Embedded Parallel Operating System, aims at building
tailor-made execution platforms for specific
applications~\cite{Frohlich:JCS:2008}. It follows the principles of
\emph{Application-driven Embedded System Design}~\cite{Frohlich:2001} to
engineer families of software and hardware components that can be
automatically selected, configured, adapted, and arranged in a component
framework according with the requirements of particular applications.

An application written based on \textsc{Epos} published interfaces can
be submitted to a tool that performs source code analysis to identify
which components are needed to support the application and how these
components are being deployed, thus building an execution scenario for
the application. Alternatively, users can specify execution scenarios
by hand or also review an automatically generated one. A build-up
database, with component descriptions, dependencies, and composition
rules, is subsequently accessed by the tool to proceed component
selection and configuration, as well as software/hardware partitioning
based on the availability of chosen components in each domain. If
multiple components match the selection criteria, then a cost model is
used, along with user specifications for non-functional properties,
such as performance and energy consumption, to choose one of
them\footnote{Design-space exploration is currently being pursued in
  \textsc{Epos} by making the cost model used by the building tool
  dynamic.}.

After being chosen and configured, software components can still
undergo application-specific adaptations while being plugged into a
statically metaprogrammed framework that is subsequently compiled to
yield a run-time support system. This application-specific system can
assume several shapes, from simple libraries to operating system
micro-kernels. On the hardware side, component selection and
configuration yields an architecture description that can be either
realized by discrete components (e.g. microcontrollers) or submitted
to external tools for \textsc{IP} synthesis. An overview of the whole
process can be seen in figure~\ref{fig:epos_tools}.

\wfig{epos_tools}{scale=.75}{Overview of tools involved in
  \textsc{Epos} automatic generation.}


\subsection{Example Application}

The remote sensing application described in section~\ref{sec:api} was
implemented in \textsc{Epos} as excerpted in
figure~\ref{prg:example-source-epos}. When submitted to \textsc{Epos}
tools, the remote sensing program yielded a run-time library that
realizes the required interfaces and a hardware description that could
be matched by virtually any hardware platform in the system build-up
database. We forced the selection of a well-known platform, the Mica2
sensor node~\cite{Hill:2004} by manually binding \texttt{Communicator}
to the \textsc{CC1000} radio on the Mica2 platform and \texttt{Actuator}
to a led. 

In the experiment, energy for the system was delivered by two
high-performance alkaline AA batteries with a total capacity of 58320~J
(5400~mAh at 3~V), in excess of table~\ref{tab:example-energy_estimates}
estimates of what would be necessary to match the intended life-time of
one year (3576~mAh at 3~V). The system was configured with a scheduling
quantum of 15 ms and a battery monitoring period of one day.  Energy
accounting was enabled and produced statistics that were used by the
scheduler on every thread dispatching.

The experiment was profiled during approximately one week using a
digital oscilloscope. From the collected data, we determined the average
energy consumption per hour to be of 5.07 J. We then extrapolated the
total energy consumption for one year to be of 4112~mAh. This
extrapolation projects a system lifetime of 479 days, confirming that
the system will match the expectations in this respect. This experiment
also shows that the energy overhead caused by the implemented power
management mechanisms is largely compensated by the power it saves (as
calculated in section~\ref{sec:api}, running the example application
without any power management would demand 40 times more energy than the
predicted 3576~mAh). It is also important to notice that the additional
536~mAh cannot be entirely accounted to power management. A fraction of
it arises from the additional circuitry needed to couple the key
components considered at design-time, another fraction from the
non-linear discharge nature of the chosen batteries, and yet another can
be accounted to misleading estimates published by manufacturers.

This experiment also allowed us to asses the strategy overhead in terms
of memory and CPU utilization.
Table~\ref{tab:example-overhead-footprint} shows the memory increase
caused by the proposed mechanisms.  The reference system was stripped of
any PM capabilities and than enriched with the PM API, power accounting,
and finally the autonomous manager integrated into the scheduler. The
considerable increase in size for every step is justified by the fact
that they affect all components in the system. The PM API required
versions of mediators that are able to control the power mode of
associated devices plus a global \texttt{System} object and access
control to handle event propagation conflicts as described in
section~\ref{sec:petri_nets}. Accounting enriched components with
counters and the associated maintenance code. The autonomous manager
required battery monitoring plus statistics handling and decision making
support.

\tab{.75\textwidth}{example-overhead-footprint}{Power management memory
  overhead (sizes in bytes).}

In respect to performance, the proposed mechanisms only substantially
affect hardware mediators and the scheduler. Other system components,
although adorned with \texttt{power()} methods and event counters, do
not have their original behavior altered by the aspect programs
responsible for power management and therefore show no performance
loss\footnote{Some event counters are initialized at object construction
  time through modified versions of operator \texttt{new}, but this
  small overhead is usually restricted to start up phases.}.  In order
to precisely asses the overhead caused by the proposed power management
mechanisms, we inserted simple primitives to switch a led on and off
around target methods and obtained the average active period with an
oscilloscope. Table~\ref{tab:example-overhead-performance} shows the
increase in execution time for context switch and I/O operations.�The
calculations performed by the scheduler to keep statistics and decide
whether a best-effort task can be dispatched extended context switch
time by 9$\mu$s (the target platform features an 8-bit AVR running at
8MHz). The increment in the path to I/O devices caused by auto-resume
and accounting was measured to be 60$\mu$s and reveals weakness of the
target architecture to handle the associated arithmetics. Platforms
capable of keeping event counters associated to I/O ports in hardware
could eliminate a reasonable fraction of this overhead.

\tab{.75\textwidth}{example-overhead-performance}{Power management
  performance overhead (times in $\mu$s).}

\subsection{Autonomous Power Manager}

The example application discussed along this article has been conceived
to support the explanation of the proposed power management strategy.
Its implementation described in the previous section also allowed us to
confirm most of the claimed benefits. Nonetheless, it does not feature a
best-effort task that could corroborate the proposed autonomous power
manager design.  Therefore, we extended it with two additional
best-effort threads on a second experiment: thread \texttt{Calibrator}
periodically calibrates the temperature sensor, and thread
\texttt{Display} shows the current temperature on a led display.  Both
threads make use of hardware components available in the original
platform, but with a significant difference: \texttt{Calibrator} uses
the thermometer, which it shares with \texttt{Monitor} and
\texttt{Recovery}, while \texttt{Display} uses dedicated leds. This
distinction is important to drive the power manager through a situation
in which the decision about suspending resources (ADC in the
thermometer) used by a frozen best-effort task (\texttt{Calibrator})
must consider hard real-time tasks demands (\texttt{Monitor} and
\texttt{Recovery}). The threads were created with periods of 100 seconds
and 100 ms, respectively.

% \begin{lstlisting}[language=C++,style=prg]
% Periodic_Thread(&calibrator, BEST_EFFORT, 100000000);
% Periodic_Thread(&display, BEST_EFFORT, 100000);
% \end{lstlisting}

Just like the example application, this second experiment was profiled
during approximately one week using a digital oscilloscope and a new
battery set. The results of the experiment are summarized in
table~\ref{tab:example-energy_measurements}, which presents the system
average energy consumption for five different setups: (a) executing
without the additional threads; (b) executing the \texttt{Calibrator}
thread with hard real-time priority and \texttt{Display} as best-effort;
(c) the reverse situation, \texttt{Display} as hard real-time and
\texttt{Calibrator} as best-effort; (d) with both threads running with
hard-real time priority; and (e) with both threads running in
best-effort priority.

\tab{.75\textwidth}{example-energy_measurements}{Energy consumption
  under different autonomous power manager setups (RT = hard
  real-time / BE = best-effort).}

Setup (a) is equivalent to the example application evaluated in the
previous section and produced equivalent results. Setup (b) is impacted
by the periodic ADC operations performed by \texttt{Calibrator} (100
samplings every 100 seconds). The best-effort thread \texttt{Display}
has virtually no chance to execute in this setup, since the accounting
system quickly feeds the scheduler with information indicating that the
desired lifetime of 365 days cannot be matched under the observed energy
consumption rate. Setup (c) features a similar situation, but with
switched roles. The led display used has a high tool on energy, so the
system reached a lifetime of only 84 days. In this setup,
\texttt{Calibrator} is the thread that is prevented from executing,
however, the ADC it uses (via the thermometer abstraction) is not
implicitly put to sleep, since it is also used by the \texttt{Monitor}
and \texttt{Recovery} hard real-time threads.

For setup (d), both threads have been configured as hard real-time, so
they are always executed. This reduces the system's lifetime to about 76
days.  Setup (e) is the one that best characterizes the proposed
autonomous power manager. Running the additional threads in best-effort
priority enables the scheduler to suppress their execution whenever the
energy budget needed to achieve the specified lifetime is threatened.
This smoothly drives the system toward the desired lifetime, enabling
both threads to run just sporadically when the battery monitor indicates
that there is enough energy.

%%% Local Variables:
%%% mode: latex
%%% TeX-master: "pm"
%%% End:
