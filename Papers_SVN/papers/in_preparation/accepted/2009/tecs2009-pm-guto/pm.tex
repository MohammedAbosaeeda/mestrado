\documentclass{acmtrans2m}
\acmVolume{1}
\acmNumber{1}
\acmYear{1}
\acmMonth{1}

\usepackage[latin1]{inputenc}  
\usepackage[T1]{fontenc}
\usepackage[english]{babel}   
\usepackage{url}

%\usepackage{amssymb}
%\usepackage{amsbsy,natbib}
%\usepackage{amsthm}
%\usepackage{amsfonts,amsmath,bm}

\usepackage[pdftex]{graphicx}

\usepackage{listings}
\lstset{keywordstyle=\bfseries, flexiblecolumns=true}
\lstloadlanguages{C,[ANSI]C++,HTML}
\lstdefinestyle{prg}{
  basicstyle=\small\sffamily,
%  lineskip=-0.2ex,
  showspaces=false
}
% \usepackage{cite}
% \usepackage{afterpage}
% \usepackage{units}

%\makeatletter
%\newif\if@restonecol
%\makeatother
%\let\algorithm\relax
%\let\endalgorithm\relax
%\usepackage[figure,linesnumbered]{algorithm2e}


\newcommand{\fig}[4][h]{
  \begin{figure}
    {\centering{\includegraphics[#3]{fig/#2}}\par}
    \caption{#4}\label{fig:#2}
  \end{figure}
}

\newcommand{\wfig}[4][tb]{
  \begin{figure*}
    {\centering{\includegraphics[#3]{fig/#2}}\par}
    \caption{#4}\label{fig:#2}
  \end{figure*}
}

\usepackage{multirow}
%\setlength{\tabcolsep}{1mm}
\newcommand{\tab}[4][h]{
  \begin{acmtable}{#2}
    {\centering\footnotesize\textsf{\input{fig/#3.tab}}\par}
    \caption{#4}\label{tab:#3}
  \end{acmtable}
}

\newcommand{\wtab}[4][h]{
  \begin{*acmtable}{#2}
    {\centering\footnotesize\textsf{\input{fig/#3.tab}}\par}
    \caption{#4}\label{tab:#3}
  \end{acmtable*}
}

\newcommand{\prg}[4][h]{
  \begin{figure}
    \begin{center}
      \makebox[\width]
 	{\centering\lstinputlisting[language=#2,style=prg]{fig/#3.prg}\par}
      \caption{#4}\label{prg:#3}
    \end{center}
  \end{figure}
}

\pdfpagewidth=8.5in
\pdfpageheight=11in
\sloppy

%%
\markboth{Ant�nio Augusto Fr�hlich}
	{A Comprehensive Approach for Power Management in Embedded Systems}

\title{A Comprehensive Approach for Power Management in Embedded Systems}

\author{ANT�NIO AUGUSTO FR�HLICH \\
	Federal University of Santa Catarina}

\begin{abstract}
% Version submitted on-line
%In this article, power management is addressed in the context of embedded systems from energy-aware design to energy-efficient implementation. A set of mechanisms specifically conceived for this scenario is proposed, including: a power management API defined at the level of user-visible system components, the infrastructure necessary to implement that API, an energy-event propagation mechanism based on Petry Nets and implemented with AOP techniques, and an autonomous power manager. The proposed mechanisms are illustrated and evaluated by realistic embedded systems, thus enabling comparisons with other proposals.

  In this article, power management is addressed in the context of
  embedded systems from energy-aware design to energy-efficient
  implementation. A set of mechanisms specifically conceived for this
  scenario is proposed, including: a power management API defined at the
  level of user-visible system components, the infrastructure necessary
  to implement that API (namely, battery monitoring, accounting,
  auto-suspend, and auto-resume), an energy-event propagation mechanism
  based on \emph{Petry Nets} and implemented with \emph{Aspect-Oriented
    Programming} techniques, and an autonomous power manager build upon
  the proposed API and infrastructure. These mechanisms are illustrated
  and evaluated by a realistic embedded system that is also used to
  sustain comparisons with other proposals at each of the considered
  levels. As a result, this article has its main contribution on the
  introduction of a comprehensive and systematic way to deal with power
  management issues in embedded systems.
\end{abstract}

\category{D.4.7}{Or\-gan\-i\-za\-tion and Design}{Real-time systems
  and embedded systems}

\terms{Design, Management, Experimentation}

\keywords{Embedded systems, power management, energy-aware scheduling}

\begin{document}

\setcounter{page}{111}

\begin{bottomstuff}
  Author's address: Ant�nio Augusto Fr�hlich, Software/Hardware
  Integration Lab, Federal University of Santa Catarina, 88040\-900
  Florian�polis, SC, Brazil, \texttt{guto@lisha.ufsc.br}.
\end{bottomstuff}

\maketitle

% ------------------------------------------------------------------------------
\section{Introduction}
Very-High Level Languages (VHLL), from which \java~and \lua~are examples, is a
kind of programming language which provide developers with features to improve
their productivity\cite{Wilson:1999}.
Productivity improvement is obtained by using constructions with a higher level
of abstraction enabling the developer to express and validate his ideas in a
short period of time (such as object orientation, domain specific constructions
and APIs), and by features that make the occurrence of programming errors less
often reducing the time spend on program debugging (such as automatic memory
management, memory protection, and exceptions).

During the last ten years several initiatives have been taken in order to
enable the use of VHLLs not only in general propose systems scenario as well in
embedded systems scenario fulfilling the time and resource requirements impose
by such systems.
However, in order to be really useful for embedded systems VHLLs must provide
features for interacting with the environment where the embedded system
is inserted on.
Such interaction is usually implemented by using hardware devices.
Sensors and actuators enable the system to interact with the environment.
Transmitters and receivers are used for communicating with other systems.
Timers and alarms are used to implement real-time operations.

The interaction between VHLLs and hardware devices is performed by using the so
called Foreign Function Interface (FFI).
However, a FFI do not specify by itself how to abstract hardware or how
to organize these abstractions.
This work aims to fulfill this gap, introducing a method to interface hardware
devices and applications written using VHLLs in context of embedded systems.
We propose a method to abstract such hardware devices and we show that the
problem of adapting a hardware device to be used for a VHLL can be faced as an
aspect weaving problem, automatically generating the binding between the device
and the language.

The next sections of this paper are organized in the following way: Section
\ref{sec:relat} reviews how VHLLs interact with hardware devices and how
hardware devices can be abstracted and organized.
Section \ref{sec:proposal} introduces the proposed method for
abstracting hardware devices and shows how the adaptation of a hardware device
for a specific VHLL can be solved as an aspect weaving.
Section \ref{sec:eval} presents our cases study as well the obtained results
on evaluating our proposal according to performance, memory consumption,
portability, support to the developer, and reuse.
Our final considerations are presented in Section \ref{sec:conc}.

% ------------------------------------------------------------------------------


\section{Power Management API}\label{sec:api}

% PM at design-time
% PM API in other systems
% - Automatic (ignores designer's knowledge)
% - At the hardware interface level
% - Non portable
% Proposed PM API
% - User-visible abstractions
% - No distinction from software and hardware components
% - Operational modes
%   - Arbitrary x semantic modes
% - Process throttling
%   - DVS

% Design of Energy-aware Embedded Systems
In order to introduce a discussion about power management
\emph{Application Programming Interfaces}~(API), let us first recall how
energy consumption requirements arise during the design of an
energy-aware embedded system and how they are usually captured. In such
systems, designers look for available energy-efficient components and,
eventually, specify new components to be implemented. During this
process, they inherently acquire knowledge about the most adequate
operating strategy for each component and for the system as a whole.
Whenever the identified strategies are associated to modifications in
the energy level of a given component, this can be captured in
traditional design diagrams, such as sequence, activity and timing, or
by specific tools~\cite{Chou:2002}.

% Example
% - Summary
% - Block diagram
Now let us consider the design of a simple application, conceived
specifically to illustrate the translation of energy constraints from
design to implementation. This application realizes a kind of remote
monitoring system, capable of sensing a given property (e.g.
temperature), reporting it to a control center, and reacting by
activating an actuator (e.g. external cooler) whenever it exceeds a
certain limit.  Interaction with the control center is done via a
communicator (e.g.  radio). The system operates on batteries and must
run uninterruptedly for one year. A block diagram of the system is shown
in figure~\ref{fig:example-block_diagram}.

\fig{example-block_diagram}{scale=0.75}{Example monitoring system block
  diagram.}

% Example
% - Description
% - Diagrams
The application is modeled around four tasks whose behavior is depicted
in the sequence diagrams of
figures~\ref{fig:example-sequence_diagram-main} through
\ref{fig:example-sequence_diagram-recovery}: \texttt{Main} allocates
common resources and creates threads to execute the other three tasks;
\texttt{Monitor} is responsible for periodic temperature monitoring
(every second), for reporting the current temperature to the control
center (every 10 seconds), and, in case the temperature threshold is
exceeded, for triggering the emergency handling thread; \texttt{Trigger}
is responsible for triggering the emergency handling thread on command
of the control center; and \texttt{Recovery}, the emergency handling
thread, is in duty of firing an one-shot actuator intended at restoring
the temperature to its normal level. Coordination is ensured by properly
assigning priorities to threads and by the \texttt{emergency}
semaphore. % (see figure~\ref{fig:example-coordination}).

\fig{example-sequence_diagram-main}{scale=0.75}{Main thread sequence
  diagram with power management actions.}
\fig{example-sequence_diagram-monitor}{scale=0.75}{Monitor thread sequence
  diagram with power management actions.}
\fig{example-sequence_diagram-trigger}{scale=0.75}{Trigger thread sequence
  diagram with power management actions.}
\fig{example-sequence_diagram-recovery}{scale=0.75}{Recovery thread sequence
  diagram with power management actions.}

% PM at design-time
% - PM knowledge from designer
% - Expressible in UML
In the sequence diagrams of
figures~\ref{fig:example-sequence_diagram-main}
through~\ref{fig:example-sequence_diagram-recovery}, energy-related
actions captured during design are expressed by messages and remarks.
For instance, the knowledge that the \texttt{Thermometer} component uses
a low-cost thermistor and therefore must perform 64 measurements before
being able to return a temperature value within the desired precision is
expressed as a note associated with the method
(figure~\ref{fig:example-sequence_diagram-monitor}).  In order to be
energetically efficient, the circuitry behind \texttt{Thermometer} (i.e.
ADC and thermistor) should be kept active during all the measurement
cycle, thus avoiding repetitions of the electrical stabilization
phase\footnote{In a typical ADC/thermistor configuration, the electrical
  stabilization phase accounts up to 98\% of the measurement cycle, both
  in terms of time and in terms of energy~\cite{Panasonic:ERTJ:2004}.}.

The diagrams also show power management hints for the
\texttt{Communicator} component (figures
\ref{fig:example-sequence_diagram-monitor},
\ref{fig:example-sequence_diagram-trigger}, and
\ref{fig:example-sequence_diagram-recovery}), which is mostly used to
listen for a message from the control center and thus can be configured
on a listen-only state for most of the time. And, indirectly, also for
the \texttt{CPU} component, which must operate in the maximum frequency
while running the \texttt{Recovery} thread, but can operate in lower
frequencies for the other threads
(figure~\ref{fig:example-sequence_diagram-main}). With this information
in hand, the system can be implemented to be more efficient in terms of
energy, either by the programmer himself or by means of an automatic
power manager.

% Example
% - Energy estimates
% - Power management
Considering the functional properties described so far and the execution
period of each thread, it is possible to estimate duty cycles for each
of the major components in the example system.  This information can
then be combined with energy consumption estimates of individual
components to calculate the power supply required by the system. This
procedure is summarized in table~\ref{tab:example-energy_estimates},
which shows hypothetical energy consumption estimates, duty cycles and
energy consumption for the four major components in the
system\footnote{Estimates were based on the Mica2 sensor
  node~\cite{Hill:2004}}.

\tab{.9\textwidth}{example-energy_estimates}{Example monitoring system energy
  consumption estimates.}

In order to match the requirement of operating uninterruptedly for one
year, the system would demand a battery capable of delivering
approximately 3576~mAh (at 3~V). For comparison, the same system
operating with all components constantly active, that is, without any
power saving strategy, would require about 142 Ah, almost 40 times more.

\subsection{Current APIs}

% Current APIs
% - Low level -> breaks portability
% - OS-driven -> ignores designer's knowledge
Few systems targeting embedded computing can claim to deliver a real
Power Management API. Nevertheless, most systems do deliver mechanisms
that enable programmers to directly access the interface of some
hardware components. These mechanism, though not specifically designed
for power management, can be used for that purpose at the price of
binding the application to hardware details.

% uClinux
% - 
\textsc{$\mu$Clinux}, like many other \textsc{Unix}-like systems, does
not feature a real power management API. Some device drivers provide
power management functions inspired on ACPI. Usually these mechanisms
are intended to be used by the kernel itself, though a few device
drivers export them via the \texttt{/sys} or \texttt{/proc} file
systems, thus enabling applications to directly control the operating
modes of associated devices.

\prg{C}{example-source-uclinux}{\texttt{Thermometer::sample()} method
  implementation for \textsc{$\mu$Clinux}.}

The source code in figure~\ref{prg:example-source-uclinux} is a
user-level implementation of the \texttt{Thermometer::sample()} method
of our example monitoring system. In this implementation, programmers
must explicitly identify the driver responsible for the ADC to which
the thermistor is connected. Besides the overhead of interacting with
the device driver through the \texttt{/sys} file system,
\textsc{$\mu$Clinux} PM API creates undesirable dependencies and would
fail to preserve the application in case the thermistor gets connected
to another ADC channel or in case the ADC in the system gets replaced by
another model.


% TinyOS
% - Low-level, good performance, non-portable
\textsc{TinyOS}, a popular operating system in the wireless sensor
network scene, allows programmers to control the operation of hardware
components through a low-level, architecture-dependent API. Though not
specifically designed for power management purposes, this API ensures
direct access to the hardware and thus can be used in this sense. When
compared to \textsc{$\mu$Clinux}, \textsc{TinyOS} delivers a lighter
mechanism, more adequate for most embedded system, yet suffers from the
same limitations in respect to usability and portability. The use of
\textsc{TinyOS} hardware control API for power management is illustrated
in figure~\ref{prg:example-source-tinyos}, which depicts the
implementation of the \texttt{Trigger} thread of our example.

\prg{C}{example-source-tinyos}{\texttt{Trigger} thread implementation
  for \textsc{TinyOS}.}

% MantisOS
% - 
\textsc{Mantis}, features a \textsc{Posix}-inspired API that abstracts
hardware devices as \textsc{Unix} special files. Differently of
\textsc{$\mu$Clinux} and \textsc{TinyOS}, however, \textsc{Mantis} does
not propose that API to be used for power management purposes: internal
mechanisms automatically deactivate components that have not been used
for a given time, or perform an ``on-act-off'' scheme, thus implementing
a sort of OS-driven power manager. This strategy can be very efficient,
but makes it difficult for programmers to express the knowledge about
energy consumption acquired during the design process.  This is made
evident in the implementation of the \texttt{Thermometer::sample()}
depicted in figure~\ref{prg:example-source-mantis}. Unaware of the
precision required for the temperature variable, \textsc{Mantis} cannot
predict that the ADC is being used in a loop and misses the opportunity
to avoid the repetition of the expensive electrical stabilization phase
of the thermometer operation.

\prg{C}{example-source-mantis}{Thermometer::sample() method implementation
  for \textsc{Mantis}.}

%\clearpage

Some systems assume that architectural dependencies are intrinsic to the
limitations of typical embedded systems, however, this is exactly the
share of the computing systems market that could benefit from a large
diversity of suppliers~\cite{Tennenhouse:2000} and therefore would
profit from quick changes from one architecture to another.  This, in
addition to the fact that current APIs do not efficiently support the
expression of design knowledge during system implementation, led us to
propose a new PM API.


\subsection{Proposed API}

The Power Management API proposed here arose from the observation that
currently available APIs require application programmers to go down to
the hardware whenever they want to manage power, inducing unnecessary
and undesirable architectural dependencies between application and the
hardware platform. In order to overcome these limitations, we believe a
PM API for embedded systems should present the following
characteristics:

\begin{itemize}
\item Enable direct application control over energy-related issues, yet
  not excluding delegation or cooperation with an autonomous power
  manager;

\item Act also at the level of user-visible components, instead of being
  restricted to the level of hardware component interfaces, thus
  promoting portability and usability;

\item Be suitable for both application and system programming, thus
  unifying power management mechanisms and promoting reuse;

\item Include, but not be restricted to, semantic modes, thus enabling
  programmers to easily express power management operations while
  avoiding the limitations of a small, fixed number of operating modes
  (as is the case of ACPI).
\end{itemize}

With these guidelines in mind, we developed a very simple API, which
comprises only two methods, and an extension to the methods responsible
for process creation. They are:

\bigskip

\texttt{Power\_Mode power(void)}

\smallskip

\texttt{void power(Power\_Mode)}

\bigskip

The first method returns the current power mode of the associated object
(i.e., component), while the second allows for mode changes.  Aiming at
enhancing usability, four power modes have been defined with semantics
that must be respected for all components in the system: \emph{off},
\emph{stand-by}, \emph{light} and \emph{full}. Each component is still
free to define additional power modes with other semantics, as long as
the four basic modes are preserved. Enforcing universal semantics for
these power modes enables application programmers to control energy
consumption without having to understand the implementation details of
underlying components (e.g., hardware devices). Allowing for additional
modes, on the other hand, enables programmers to precisely control the
operation of special components, whose operation transcend the
predefined modes.

The introduction of these methods in user-visible components such as
files and sockets certainly requires some sort of propagation mechanism
and could itself introduce undesirable dependencies. We describe a
strategy to implement them using a combination of \emph{Aspect-Oriented
  Programming} techniques and \emph{Hierarchical Petry Nets} later in
section~\ref{sec:infra}. For now, lets concentrate on the
characterization of the API, not the mechanisms behind it.

Table~\ref{tab:api-modes} summarizes the semantics defined for the
four universal operating modes. A component operating in mode
\emph{full} provides all its services with maximum performance,
possibly consuming more energy than in any other mode. Contrarily, a
component in mode \emph{off} does not provide any service and also
does not consume any energy. Switching a component from \emph{off} to
any other power mode is usually an expensive operation, specially for
components with high initialization and/or stabilization times. The
mode \emph{stand-by} is an alternative to \emph{off}: a component in
\emph{stand-by} is also not able to perform any task, yet, bringing it
back to \emph{full} or \emph{light} is expected to be quicker than
from mode \emph{off}. This is usually accomplished by maintaining the
state of the component ``alive'' and thus implies in some energy
consumption.  A component that does not support this mode natively
must opt between remaining active or saving its state, perhaps with
aid from the operating system, and going off.

\tab{.75\textwidth}{api-modes}{Semantic power modes of the proposed PM API.}

Defining the semantics for mode \emph{light} is not so
straightforward. A component in this mode must deliver all its services,
but consuming the minimum amount of energy. This definition brings about
two important considerations. First, if there is a power mode in which
the component is able to deliver all its services with the same quality
as if it was in mode \emph{full}, then this should be mode \emph{full}
instead of \emph{light}, since it would make no sense to operate in a
higher consumption mode without arguable benefits.  Hence, mode
\emph{light} is often attained at the cost of performance (e.g., through
DVS). This, in turn, brings about a second consideration: for a
real-time embedded system, it would be wrong to state that a component
is able to deliver ``all its services'' if the added latency is let to
interfere with the time requirements of applications. Therefore, mode
\emph{light} shall not be implicitly propagated to the CPU
component. Programmers must explicitly state that they agree to slow
down the processor to save energy, or a energy-aware, real-time
scheduler must be deployed~\cite{Wiedenhoft:ETFA:2007}.

Besides the four operating modes with predefined, global semantics, a
component can export additional modes through the API. These modes are
privately defined by the component based on its own peculiarities, thus
requiring the client components to be aware of their semantics in order
to be deployed.  The room for extensions is fundamental for hardware
components with many operating modes, allowing for more refined energy
management policies. For instance, the \emph{listen-only} radio mode in
our example (see figure~\ref{fig:example-sequence_diagram-trigger})
relies on such an extension.

The proposed API also features the concept of a \texttt{System}
pseudo-component, which can be seen as a kind of aggregator for the
actual components selected for a given system instance. The goal of
the \texttt{System} component is to aid programmers to express global
power management actions, such as putting the whole system in a given
operating mode, perhaps after having defined specific modes for
particular components.


\fig{api-hierarchy}{width=\textwidth}{Power Management API
  utilization example.}

Figure~\ref{fig:api-hierarchy} presents all these interaction modes in a UML
communication diagram of a hypothetical system instance. The
application may access a global component (\texttt{System}) that has
knowledge of every other component in the system, triggering a
system-wide power mode change (execution flow 1). The \textsc{API} can
also be accessed to change the operating mode of a group of components
responsible for the implementation of a specific system functionality
(in this example, communication functionality through execution flow
2). The application may also access the hardware directly, using the
API available in the device drivers, such as \textit{Network Interface
Card} (\texttt{NIC}), \texttt{CPU}, \texttt{ADC} (in the figure,
application is accessing the CPU through the execution flow 3). The
\textsc{API} is also used between the system's components, as can be
seen in the figure.

\prg{C++}{example-source-epos}{Example monitoring system
  implementation using the proposed PM API.}

In a system that realizes the proposed API, the monitoring system
introduced earlier could be implemented as show in
figure~\ref{prg:example-source-epos}, a rather direct transcript of the
sequence diagrams of figures~\ref{fig:example-sequence_diagram-main}
through~\ref{fig:example-sequence_diagram-recovery}.


%%% Local Variables:
%%% mode: latex
%%% TeX-master: "pm"
%%% End:


\input{infra}

\section{Autonomous Power Manager}\label{sec:auto}

A considerable fraction of the research effort around power management
at software-level has been dedicated to design and implement
\emph{autonomous power managers} for general-purpose operating systems,
such as \textsc{Windows} and \textsc{Unix}. Today battery-operated
portable computers, including notebooks, PDAs, and high-end cellphones,
can rely on sophisticated management strategies to dynamically control
how the available energy budget is spent by distinct application
processes. Although not directly applicable to the embedded system
realm, those power managers bear concepts that can be promptly reused in
this domain.

As a matter of fact, autonomous power managers grab to a periodically
activated operating system component (e.g. timer, scheduler, or an
specific thread) in order to trigger operation mode changes across
components and thus save energy. For instance, a primitive power manager
could be implemented by simply modifying the operating system scheduler
to put the CPU in standby whenever there are no more tasks to be
executed.  DVS capabilities of underlying hardware can also be easily
exploited by the operating system in order to extend the battery
lifetime at the expense of performance, while battery discharge alarms
can trigger mode changes for peripheral devices~\cite{Aydin:2008}.
Nevertheless, these basic guidelines of power management for personal
computers must be brought to context before they can be deployed in
embedded systems:

\begin{itemize}
\item Embedded systems are often engineered around hardware platforms
  with very limited resources, so the power manager must be designed
  to be as slim as possible, sometimes taking software engineering to
  its limits.

\item Many embedded systems run real-time tasks, therefore a power
  manager for this scenario must be designed in such a way that its own
  execution does not compromise the deadlines of such tasks.
  Furthermore, the decisions taken by an autonomous power manager must
  be in accordance with the requirements of such tasks, since the
  latency of operating mode changes (e.g. waking up a component) may
  impact their deadlines. For a real-time embedded system, having a
  power manager that runs unpredictably might be of consequences similar
  to the infamous garbage collection issues in \textsc{Java}
  systems~\cite{Bacon:2003}.

\item Embedded systems often pay a higher energy bill for peripheral
  devices than for the CPU. Therefore, CPU-centric strategies, such as
  DVS-aware scheduling, must be reviewed to include external devices.
  Thus an active power manager must keep track of peripheral device
  usage and apply some heuristics to change their operating mode along
  the system lifetime.  The decision of which devices will have their
  operating modes changed and when this will occur is mostly based on
  event counters maintained by the power management infrastructure,
  either in hardware or in software.

\item As a matter of fact, critical real-time systems are almost
  always designed considering energy sources that are compatible with
  system demands. Power saving decisions, such as voltage scaling and
  device hibernation, are also made at design-time and thus are also
  taken in consideration while defining the energy budget necessary to
  sustain the system. At first sight, autonomous power management
  might even seem out of scope for critical systems. Nonetheless,
  complex, battery-operated, real-time embedded system, such as
  satellites, autonomous vehicles, and even sensor networks, are often
  modeled around a set of tasks that include both, critical and
  non-critical tasks. A power manager for one such embedded system
  must respect design-time decisions for critical parts while trying
  to optimize energy consumption by non-critical parts.
\end{itemize}

With these premises in mind, the next section briefly surveys the
current scenario for power management in embedded systems.


\subsection{Current Power Managers}

Just like APIs and infrastructures, most of the currently available
embedded system power managers focus on features exported by the
underlying hardware.  \textsc{$\mu$Clinux} captures \textsc{APM},
\textsc{ACPI} or equivalent events to conduct mode transitions for the
CPU and also for devices whose drivers explicitly registered to the
power manager~\cite{Vaddagiri:2004}.

In \textsc{TinyOS}, OS-driven power management is implemented by the
task scheduler, which makes use of the \texttt{StdControl} interface
to start and stop components~\cite{Hill:2000}. When the scheduler
queue is empty, the main processor is put in \emph{sleep} mode.  In
this way, new tasks will only be enqueued during the execution of an
interrupt handler.  This method yields good results for the main
microcontroller, but leaves more aggressive methods, including
starting and stopping peripheral components up to the application.
When compared to \textsc{$\mu$Clinux}, \textsc{TinyOS} delivers a
lighter mechanism, more adequate to embedded systems, yet suffers from
the same limitations with regard to usability and portability.

\textsc{Mantis} uses an \emph{idle} thread as entry point for the
system's power management policies, which put the processor in
\emph{sleep} mode whenever there are no threads waiting to be
executed~\cite{Bhatti:2005}.

\textsc{Grace-OS} is an energy-efficient operating system for mobile
multimedia applications implemented on top of
\textsc{Linux}~\cite{Yuan:2004}. The system combines real-time
scheduling and DVS techniques to dynamically control energy consumption.
The scheduler configures the CPU speed for each task based on a
probabilistic estimation of how many cycles they will need to complete
their computations. Since the systems is targeted at soft real-time
multimedia applications, loosing deadlines due to estimation errors is
tolerated.  \textsc{Grub-PA} follows the same guidelines, but addresses
hard real-time requirements more consistently by imposing DVS
configuration restrictions for this kind of task~\cite{Scordino:2004}.

Niu also proposes an strategy to minimize energy consumption in soft
real-time systems through adjusts in the system QoS
level~\cite{Niu:2005}.  In this proposal, tasks specify CPU QoS
requirements through \texttt{(m,k)} pairs. These pairs are interpreted
by the scheduler as execution constraints, so that a task must meet at
least \texttt{m} deadlines for any \texttt{k} consecutive releases. The
possibility to lose some deadlines enables the scheduler to explore DVS
more efficiently at the cost of preventing its adoption in many (hard
real-time) embedded systems.

Yet in the line of energy savings through adaptive scheduling and QoS,
\textsc{Odyssey} takes the concept of soft real-time to the limit. The
system periodically monitors energy consumption by applications in order
to adjust the level of QoS.  Whenever energy consumption is too high,
the system decreases QoS by selecting lower performance and power
consumption modes.  In this way, system designers are able to specify a
minimum lifetime for the system, which might be achieved by severely
degrading performance~\cite{Flinn:2004}.

\textsc{ECOS} defines a currency, called \emph{currentcy}, that
applications use to \emph{to pay for} system resources~\cite{Zeng:2005}.
The system distributes \emph{currentcies} to tasks periodically
accordingly to an equation that tracks the battery discharge rate as to
ensure a minimum lifetime for the system.  Applications are thus forced
to adapt their execution pace according to their \emph{currentcy}
balances.  This strategy has one major advantage over others discussed
so far in this paper: the \emph{currentcy} concept encompasses not only
the energy spent by the CPU (to adjust DVS configuration), but the
energy spent by the system as a whole, including all peripheral devices.

Harada explores the trade-off between QoS maximization and energy
consumption minimization by allocating processor cycles and defining
operating frequencies with QoS guarantees for two classes of tasks:
real-time (mandatory) and best-effort (optional)~\cite{Harada:2006}.
The division of tasks in two parts, one \emph{mandatory}, that must
always be executed, and another \emph{optional} that is only executed
after ensuring that there are enough resources to execute the mandatory
parts of all tasks is the basic premise behind \emph{Imprecise
  Computation}~\cite{Liu:1994}, which is also one of the foundations of
the power manager proposed in this work.


\subsection{Proposed Power Manager}

From the above discussion about currently available power managers for
embedded system, one can conclude that no single manager consistently
addresses all the points identified earlier in this section: leanness,
real-time conformance, peripheral device control, and design-time
decision awareness. We follow these premises and build on the API
proposed in section~\ref{sec:api} and on the infrastructure presented
in section~\ref{sec:infra} to propose an effective autonomous power
manager for real-time embedded systems.

% - Prepara a proposta
For the envisioned scenarios of battery-operated, real-time, embedded
systems, energy budgets would be defined at design-time based on
critical tasks, while non-critical tasks would be executed on a
best-effort policy, considering not only the availability of time, but
also of energy.  Along with the assumption that an autonomous power
manager cannot interfere with the execution of hard real-time tasks
(i.e., cannot compromise their deadlines), the separation of critical
and non-critical tasks at design-time lead us to the following
scheduling strategy:

% - RT -> EDF, RM, etc, even DVS
% - BE -> only if all RT AND enough energy
% - PM -> run when a BE is canceled to power down devices
\begin{itemize}
\item Hard real-time tasks are handled by the system as mandatory tasks,
  executed independently of the energy available at the moment.  These
  tasks are scheduled according to traditional algorithms such as
  Earliest Deadline First~(EDF) and Rate Monotonic~(RM)~\cite{Liu:1973},
  either in their original shape or extended to support DVS.

\item Best-effort tasks, periodic or not, are assigned lower
  priorities than hard real-time ones and thus are only executed if no
  hard real-time tasks are ready to run.  Furthermore, the decision to
  dispatch a best-effort task must also take in consideration whether
  the remaining energy will be enough to schedule all hard-real time
  tasks.

\item Whenever a best-effort task is prevented from executing due to
  energy limitations, a speculative power manager is activated in order
  to try to change components, including peripheral devices, to less
  energy-demanding operating modes, thus promoting energy savings.
\end{itemize}

% Corolario
With this strategy, the autonomous power manager will only be executed
if energy consumption is detected excessive (i.e. a best-effort task
has been denied execution) and time is available (i.e. a best-effort
task would be executed). Non-interference between power manager and
hard real-time tasks is ensured, in terms of scheduling, by having the
power manager to run in preemptive mode, so that a hard real-time task
would interrupt its execution as soon as it gets ready to run (e.g.
after waiting for the next cycle).

% Needed infrastructure
This scheduling strategy has only small implications in terms of
process management at the operating system level, but require a
comprehensive power management infrastructure, like the one presented
in section~\ref{sec:infra}, in order to be implemented. In particular,
battery monitoring services are needed to support the scheduling
decisions around best-effort tasks and component dependency maps are
needed to avoid power management decisions that could impact the
execution of hard real-time tasks.

% Energy estimation combining battery monitoring and accounting
The battery monitoring service provided by the PM infrastructure can be
combined with the energy accounting service to reduce the costs of
gauging the amount of energy still available to the system. With updated
statistics from the energy accounting infrastructure in hand, the
scheduler can predict battery discharge without having to physically
interact with it, thus sparing the corresponding energy. In this way,
battery monitoring is programmed to take place sporadically based on the
lifetime specified for the system. An additional trigger is bound to the
prediction counter kept by the scheduler, so monitoring also takes place
when power consumption reaches specified thresholds.

% - SO keeps a list of active components
% - PM takes on propagation networks to shut down devices
The operating mode transition networks introduced in
section~\ref{sec:infra} as means to control the propagation of power
management actions from high-level components down to the hardware can
be used by the autonomous power manager to keep track of dependencies
among components. Along with a list of currently active components
maintained by the operating system, these transition networks build
the basis on which peripheral control can be done by the power
manager. For instance, if a task has an open file that is no longer
being used, the power manager could track that component down to a
flash memory and change its operating mode to standby or off.

% - But PM must respect API hints of RT as orders
Nevertheless, the compromise with real-time systems requires our power
manager to take API calls made by hard real-time tasks as ``orders''
instead of ``hints''. We assume that, if a hard real-time task calls the
\texttt{power()} API method on a component to set its operating mode to
\emph{full}, then that component must be kept in that mode even if the
collected statistics indicate that it is no longer being used and thus
would be a good candidate to be shutdown.  Otherwise, the corresponding
task could miss its deadline due to the delay in reactivating that
component.


%%% Local Variables:
%%% mode: latex
%%% TeX-master: "pm"
%%% End:


\input{epos}

\section{Conclusion}
\label{sec:Conclusion}
In this paper, we have shown how to construct a development environment for embedded applications based on specific hardware/software requirements and introduce the automatic exchange of configuration parameters as one anatomic part of fully automated debug.

The integrated development environment provides independence of the physical target platform for development and test. Its an important step, since some embedded systems may not be able to store the extra data needed to support debug. The impact of enable debug information in code size and in the execution time of the real-world application was more than 80\%. Also, developers no longer need to spend time understanding a new development platform whenever some characteristic of the embedded system changes.

The automatic exchange was evaluated using two kinds of test. The fully automated test works with no prior information of the application, but it was possible to generate valid configurations, that could be tested as alternative solutions. In partial automated test all generated configurations were valid and the report was useful to discovery that some parameter values were better then others.

In this sense, was possible to realize that even a small part of the complete automated solution produce answers to help developers find and fix a bug. With only a hundred tries was possible to find error/restriction in the code. Thus, as future work we can integrate the automatic exchange script with a tool that has artificial intelligence, in order to achieve a conscious exchange of type parameters.



\begin{acks}
I would like to thank and acknowledge former LISHA members Arliones
S. Hoeller Jr, Geovani R. Wiedenhoft, Giovani Gracioli and Lucas Wanner
for implementing many of the concepts and ideas presented in this
article.
\end{acks}

\bibliographystyle{acmtrans}
\bibliography{pm,lisha}

\begin{received}
Received November 2009;
Revised May 2010;
Accepted ???.
\end{received}

\end{document}

