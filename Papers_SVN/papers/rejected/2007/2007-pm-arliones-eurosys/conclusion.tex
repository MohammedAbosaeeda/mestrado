% Summary 

% Contributions: (1) common and simple interface (minor), (2)
% Power-management on embedded systems without using any complex
% high-cost methodology.
In this paper we presented an strategy to enable application-driven
power management in deeply embedded systems. In order to achieve this
goal we allowed application programmers to express when certain
components are not being used. This is expressed through a simple
power management interface which allows power mode switching of system
components, subsystems or the system as a whole, making all
combinations of components operating modes feasible. By using the
hierarchical architecture by which system components are organized in
our system, effective power management was achieved for deeply
embedded systems without the need for costly techniques or strategies,
thus incurring in no unnecessary processing or memory overheads.

A case study using a 8-bit microcontroller to monitor temperature in
an indoor ambient showed that almost 40\% of energy could be saved
when using this strategy. % and with minimal application intervention.

% Problems: concurrence. Describe the Thread problem.

% The paper also listed some identified problems on the path for
% power-aware software and hardware components, discussing and
% explaining how some of these problems have been solved in this work
% and how some of them can be solved, and will be, in future work.

% Even so, it still have its usability.

