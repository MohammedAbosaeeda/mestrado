%% bare_jrnl.tex
%% V1.3
%% 2007/01/11
%% by Michael Shell
%% see http://www.michaelshell.org/
%% for current contact information.
%%
%% This is a skeleton file demonstrating the use of IEEEtran.cls
%% (requires IEEEtran.cls version 1.7 or later) with an IEEE journal paper.
%%
%% Support sites:
%% http://www.michaelshell.org/tex/ieeetran/
%% http://www.ctan.org/tex-archive/macros/latex/contrib/IEEEtran/
%% and
%% http://www.ieee.org/



% *** Authors should verify (and, if needed, correct) their LaTeX system  ***
% *** with the testflow diagnostic prior to trusting their LaTeX platform ***
% *** with production work. IEEE's font choices can trigger bugs that do  ***
% *** not appear when using other class files.                            ***
% The testflow support page is at:
% http://www.michaelshell.org/tex/testflow/


%%*************************************************************************
%% Legal Notice:
%% This code is offered as-is without any warranty either expressed or
%% implied; without even the implied warranty of MERCHANTABILITY or
%% FITNESS FOR A PARTICULAR PURPOSE!
%% User assumes all risk.
%% In no event shall IEEE or any contributor to this code be liable for
%% any damages or losses, including, but not limited to, incidental,
%% consequential, or any other damages, resulting from the use or misuse
%% of any information contained here.
%%
%% All comments are the opinions of their respective authors and are not
%% necessarily endorsed by the IEEE.
%%
%% This work is distributed under the LaTeX Project Public License (LPPL)
%% ( http://www.latex-project.org/ ) version 1.3, and may be freely used,
%% distributed and modified. A copy of the LPPL, version 1.3, is included
%% in the base LaTeX documentation of all distributions of LaTeX released
%% 2003/12/01 or later.
%% Retain all contribution notices and credits.
%% ** Modified files should be clearly indicated as such, including  **
%% ** renaming them and changing author support contact information. **
%%
%% File list of work: IEEEtran.cls, IEEEtran_HOWTO.pdf, bare_adv.tex,
%%                    bare_conf.tex, bare_jrnl.tex, bare_jrnl_compsoc.tex
%%*************************************************************************

% Note that the a4paper option is mainly intended so that authors in
% countries using A4 can easily print to A4 and see how their papers will
% look in print - the typesetting of the document will not typically be
% affected with changes in paper size (but the bottom and side margins will).
% Use the testflow package mentioned above to verify correct handling of
% both paper sizes by the user's LaTeX system.
%
% Also note that the "draftcls" or "draftclsnofoot", not "draft", option
% should be used if it is desired that the figures are to be displayed in
% draft mode.
%
%\documentclass[peerreviewca]{IEEEtran}
\documentclass[journal]{IEEEtran}
%
% If IEEEtran.cls has not been installed into the LaTeX system files,
% manually specify the path to it like:
% \documentclass[journal]{../sty/IEEEtran}


% Added to solve the error: "! Package babel Error: You haven't defined the language ENGLISH yet."
% Reference: http://www.michaelshell.org/tex/ieeetran/
\makeatletter
\def\markboth#1#2{\def\leftmark{\@IEEEcompsoconly{\sffamily}\MakeUppercase{\protect#1}}%
\def\rightmark{\@IEEEcompsoconly{\sffamily}\MakeUppercase{\protect#2}}}
\makeatother



% Some very useful LaTeX packages include:
% (uncomment the ones you want to load)

\usepackage[english]{babel} % for multilingual support
% \usepackage{babel} % for multilingual support

\usepackage[utf8]{inputenc} % for use utf8

% *** MISC UTILITY PACKAGES ***
%
%\usepackage{ifpdf}
% Heiko Oberdiek's ifpdf.sty is very useful if you need conditional
% compilation based on whether the output is pdf or dvi.
% usage:
% \ifpdf
%   % pdf code
% \else
%   % dvi code
% \fi
% The latest version of ifpdf.sty can be obtained from:
% http://www.ctan.org/tex-archive/macros/latex/contrib/oberdiek/
% Also, note that IEEEtran.cls V1.7 and later provides a builtin
% \ifCLASSINFOpdf conditional that works the same way.
% When switching from latex to pdflatex and vice-versa, the compiler may
% have to be run twice to clear warning/error messages.






% *** CITATION PACKAGES ***
%
%\usepackage{cite}
% cite.sty was written by Donald Arseneau
% V1.6 and later of IEEEtran pre-defines the format of the cite.sty package
% \cite{} output to follow that of IEEE. Loading the cite package will
% result in citation numbers being automatically sorted and properly
% "compressed/ranged". e.g., [1], [9], [2], [7], [5], [6] without using
% cite.sty will become [1], [2], [5]--[7], [9] using cite.sty. cite.sty's
% \cite will automatically add leading space, if needed. Use cite.sty's
% noadjust option (cite.sty V3.8 and later) if you want to turn this off.
% cite.sty is already installed on most LaTeX systems. Be sure and use
% version 4.0 (2003-05-27) and later if using hyperref.sty. cite.sty does
% not currently provide for hyperlinked citations.
% The latest version can be obtained at:
% http://www.ctan.org/tex-archive/macros/latex/contrib/cite/
% The documentation is contained in the cite.sty file itself.






% *** GRAPHICS RELATED PACKAGES ***
%
\ifCLASSINFOpdf
  \usepackage[pdftex]{graphicx}
  \usepackage[caption=false]{subfig} % Para usar duas ou mais figuras como uma só.
  % declare the path(s) where your graphic files are
  % \graphicspath{{../pdf/}{../jpeg/}}
  % and their extensions so you won't have to specify these with
  % every instance of \includegraphics
  % \DeclareGraphicsExtensions{.pdf,.jpeg,.png}
\else
  % or other class option (dvipsone, dvipdf, if not using dvips). graphicx
  % will default to the driver specified in the system graphics.cfg if no
  % driver is specified.
  % \usepackage[dvips]{graphicx}
  % declare the path(s) where your graphic files are
  % \graphicspath{{../eps/}}
  % and their extensions so you won't have to specify these with
  % every instance of \includegraphics
  % \DeclareGraphicsExtensions{.eps}
\fi
% graphicx was written by David Carlisle and Sebastian Rahtz. It is
% required if you want graphics, photos, etc. graphicx.sty is already
% installed on most LaTeX systems. The latest version and documentation can
% be obtained at:
% http://www.ctan.org/tex-archive/macros/latex/required/graphics/
% Another good source of documentation is "Using Imported Graphics in
% LaTeX2e" by Keith Reckdahl which can be found as epslatex.ps or
% epslatex.pdf at: http://www.ctan.org/tex-archive/info/
%
% latex, and pdflatex in dvi mode, support graphics in encapsulated
% postscript (.eps) format. pdflatex in pdf mode supports graphics
% in .pdf, .jpeg, .png and .mps (metapost) formats. Users should ensure
% that all non-photo figures use a vector format (.eps, .pdf, .mps) and
% not a bitmapped formats (.jpeg, .png). IEEE frowns on bitmapped formats
% which can result in "jaggedy"/blurry rendering of lines and letters as
% well as large increases in file sizes.
%
% You can find documentation about the pdfTeX application at:
% http://www.tug.org/applications/pdftex





% *** MATH PACKAGES ***
%
%\usepackage[cmex10]{amsmath}
% A popular package from the American Mathematical Society that provides
% many useful and powerful commands for dealing with mathematics. If using
% it, be sure to load this package with the cmex10 option to ensure that
% only type 1 fonts will utilized at all point sizes. Without this option,
% it is possible that some math symbols, particularly those within
% footnotes, will be rendered in bitmap form which will result in a
% document that can not be IEEE Xplore compliant!
%
% Also, note that the amsmath package sets \interdisplaylinepenalty to 10000
% thus preventing page breaks from occurring within multiline equations. Use:
%\interdisplaylinepenalty=2500
% after loading amsmath to restore such page breaks as IEEEtran.cls normally
% does. amsmath.sty is already installed on most LaTeX systems. The latest
% version and documentation can be obtained at:
% http://www.ctan.org/tex-archive/macros/latex/required/amslatex/math/





% *** SPECIALIZED LIST PACKAGES ***
%
%\usepackage{algorithmic}
% algorithmic.sty was written by Peter Williams and Rogerio Brito.
% This package provides an algorithmic environment fo describing algorithms.
% You can use the algorithmic environment in-text or within a figure
% environment to provide for a floating algorithm. Do NOT use the algorithm
% floating environment provided by algorithm.sty (by the same authors) or
% algorithm2e.sty (by Christophe Fiorio) as IEEE does not use dedicated
% algorithm float types and packages that provide these will not provide
% correct IEEE style captions. The latest version and documentation of
% algorithmic.sty can be obtained at:
% http://www.ctan.org/tex-archive/macros/latex/contrib/algorithms/
% There is also a support site at:
% http://algorithms.berlios.de/index.html
% Also of interest may be the (relatively newer and more customizable)
% algorithmicx.sty package by Szasz Janos:
% http://www.ctan.org/tex-archive/macros/latex/contrib/algorithmicx/




% *** ALIGNMENT PACKAGES ***
%
\usepackage{threeparttable}
\usepackage{array}
% Frank Mittelbach's and David Carlisle's array.sty patches and improves
% the standard LaTeX2e array and tabular environments to provide better
% appearance and additional user controls. As the default LaTeX2e table
% generation code is lacking to the point of almost being broken with
% respect to the quality of the end results, all users are strongly
% advised to use an enhanced (at the very least that provided by array.sty)
% set of table tools. array.sty is already installed on most systems. The
% latest version and documentation can be obtained at:
% http://www.ctan.org/tex-archive/macros/latex/required/tools/


%\usepackage{mdwmath}
%\usepackage{mdwtab}
% Also highly recommended is Mark Wooding's extremely powerful MDW tools,
% especially mdwmath.sty and mdwtab.sty which are used to format equations
% and tables, respectively. The MDWtools set is already installed on most
% LaTeX systems. The lastest version and documentation is available at:
% http://www.ctan.org/tex-archive/macros/latex/contrib/mdwtools/


% IEEEtran contains the IEEEeqnarray family of commands that can be used to
% generate multiline equations as well as matrices, tables, etc., of high
% quality.


%\usepackage{eqparbox}
% Also of notable interest is Scott Pakin's eqparbox package for creating
% (automatically sized) equal width boxes - aka "natural width parboxes".
% Available at:
% http://www.ctan.org/tex-archive/macros/latex/contrib/eqparbox/





% *** SUBFIGURE PACKAGES ***
%\usepackage[tight,footnotesize]{subfigure}
% subfigure.sty was written by Steven Douglas Cochran. This package makes it
% easy to put subfigures in your figures. e.g., "Figure 1a and 1b". For IEEE
% work, it is a good idea to load it with the tight package option to reduce
% the amount of white space around the subfigures. subfigure.sty is already
% installed on most LaTeX systems. The latest version and documentation can
% be obtained at:
% http://www.ctan.org/tex-archive/obsolete/macros/latex/contrib/subfigure/
% subfigure.sty has been superceeded by subfig.sty.



%\usepackage[caption=false]{caption}
%\usepackage[font=footnotesize]{subfig}
% subfig.sty, also written by Steven Douglas Cochran, is the modern
% replacement for subfigure.sty. However, subfig.sty requires and
% automatically loads Axel Sommerfeldt's caption.sty which will override
% IEEEtran.cls handling of captions and this will result in nonIEEE style
% figure/table captions. To prevent this problem, be sure and preload
% caption.sty with its "caption=false" package option. This is will preserve
% IEEEtran.cls handing of captions. Version 1.3 (2005/06/28) and later
% (recommended due to many improvements over 1.2) of subfig.sty supports
% the caption=false option directly:
%\usepackage[caption=false,font=footnotesize]{subfig}
%
% The latest version and documentation can be obtained at:
% http://www.ctan.org/tex-archive/macros/latex/contrib/subfig/
% The latest version and documentation of caption.sty can be obtained at:
% http://www.ctan.org/tex-archive/macros/latex/contrib/caption/




% *** FLOAT PACKAGES ***
%
%\usepackage{fixltx2e}
% fixltx2e, the successor to the earlier fix2col.sty, was written by
% Frank Mittelbach and David Carlisle. This package corrects a few problems
% in the LaTeX2e kernel, the most notable of which is that in current
% LaTeX2e releases, the ordering of single and double column floats is not
% guaranteed to be preserved. Thus, an unpatched LaTeX2e can allow a
% single column figure to be placed prior to an earlier double column
% figure. The latest version and documentation can be found at:
% http://www.ctan.org/tex-archive/macros/latex/base/



%\usepackage{stfloats}
% stfloats.sty was written by Sigitas Tolusis. This package gives LaTeX2e
% the ability to do double column floats at the bottom of the page as well
% as the top. (e.g., "\begin{figure*}[!b]" is not normally possible in
% LaTeX2e). It also provides a command:
%\fnbelowfloat
% to enable the placement of footnotes below bottom floats (the standard
% LaTeX2e kernel puts them above bottom floats). This is an invasive package
% which rewrites many portions of the LaTeX2e float routines. It may not work
% with other packages that modify the LaTeX2e float routines. The latest
% version and documentation can be obtained at:
% http://www.ctan.org/tex-archive/macros/latex/contrib/sttools/
% Documentation is contained in the stfloats.sty comments as well as in the
% presfull.pdf file. Do not use the stfloats baselinefloat ability as IEEE
% does not allow \baselineskip to stretch. Authors submitting work to the
% IEEE should note that IEEE rarely uses double column equations and
% that authors should try to avoid such use. Do not be tempted to use the
% cuted.sty or midfloat.sty packages (also by Sigitas Tolusis) as IEEE does
% not format its papers in such ways.


%\ifCLASSOPTIONcaptionsoff
%  \usepackage[nomarkers]{endfloat}
% \let\MYoriglatexcaption\caption
% \renewcommand{\caption}[2][\relax]{\MYoriglatexcaption[#2]{#2}}
%\fi
% endfloat.sty was written by James Darrell McCauley and Jeff Goldberg.
% This package may be useful when used in conjunction with IEEEtran.cls'
% captionsoff option. Some IEEE journals/societies require that submissions
% have lists of figures/tables at the end of the paper and that
% figures/tables without any captions are placed on a page by themselves at
% the end of the document. If needed, the draftcls IEEEtran class option or
% \CLASSINPUTbaselinestretch interface can be used to increase the line
% spacing as well. Be sure and use the nomarkers option of endfloat to
% prevent endfloat from "marking" where the figures would have been placed
% in the text. The two hack lines of code above are a slight modification of
% that suggested by in the endfloat docs (section 8.3.1) to ensure that
% the full captions always appear in the list of figures/tables - even if
% the user used the short optional argument of \caption[]{}.
% IEEE papers do not typically make use of \caption[]'s optional argument,
% so this should not be an issue. A similar trick can be used to disable
% captions of packages such as subfig.sty that lack options to turn off
% the subcaptions:
% For subfig.sty:
% \let\MYorigsubfloat\subfloat
% \renewcommand{\subfloat}[2][\relax]{\MYorigsubfloat[]{#2}}
% For subfigure.sty:
% \let\MYorigsubfigure\subfigure
% \renewcommand{\subfigure}[2][\relax]{\MYorigsubfigure[]{#2}}
% However, the above trick will not work if both optional arguments of
% the \subfloat/subfig command are used. Furthermore, there needs to be a
% description of each subfigure *somewhere* and endfloat does not add
% subfigure captions to its list of figures. Thus, the best approach is to
% avoid the use of subfigure captions (many IEEE journals avoid them anyway)
% and instead reference/explain all the subfigures within the main caption.
% The latest version of endfloat.sty and its documentation can obtained at:
% http://www.ctan.org/tex-archive/macros/latex/contrib/endfloat/
%
% The IEEEtran \ifCLASSOPTIONcaptionsoff conditional can also be used
% later in the document, say, to conditionally put the References on a
% page by themselves.





% *** PDF, URL AND HYPERLINK PACKAGES ***
%
\usepackage{url}
% url.sty was written by Donald Arseneau. It provides better support for
% handling and breaking URLs. url.sty is already installed on most LaTeX
% systems. The latest version can be obtained at:
% http://www.ctan.org/tex-archive/macros/latex/contrib/misc/
% Read the url.sty source comments for usage information. Basically,
% \url{my_url_here}.


% *** Do not adjust lengths that control margins, column widths, etc. ***
% *** Do not use packages that alter fonts (such as pslatex).         ***
% There should be no need to do such things with IEEEtran.cls V1.6 and later.
% (Unless specifically asked to do so by the journal or conference you plan
% to submit to, of course. )

% Commands to insert figures ---------------------------------------------------

\newcommand{\fig}[4][ht]{
  \begin{figure}[#1] {\centering\scalebox{#2}{\includegraphics{fig/#3}}\par}
    \caption{#4\label{fig:#3}}
  \end{figure}
}
% fig usage:
% \fig{<scale>}{<file>}{<caption>}
% e.g.: \fig{.4}{uml/uml_comportamental_dia}{Diagramas comportamentais da UML}
% The figure label will be "fig:" plus <file>.
% The figure file must lie in the "fig" directory.

\newcommand{\figtwocolumn}[4][ht]{
  \begin{figure*}[#1] {\centering\scalebox{#2}{\includegraphics{fig/#3}}\par}
    \caption{#4\label{fig:#3}}
  \end{figure*}
}

% Para colocar 2 figuras como uma só - dispostas horizontalmente
\newcommand{\multfigtwoh}[6][htbp]{
\begin{figure*}[#1]
  \centering
  \subfloat[]{\label{fig:#3}\scalebox{#2}{\includegraphics{fig/#3}}}
  \subfloat[]{\label{fig:#4}\scalebox{#2}{\includegraphics{fig/#4}}}
  \caption{#6}
  \label{fig:#5}
\end{figure*}
}
% e.g.
%\multfigtwoh{.65}{fig_plot_time_orig}{fig_plot_time_mod}
%{fig_plot_time_all}
%{Original (a) and modified (b) benchmarks execution time comparison.}

% Para colocar 2 figuras como uma só - dispostas verticalmente
\newcommand{\multfigtwov}[6][htbp]{
\begin{figure}[#1]
  \centering
  \subfloat[]{\label{fig:#3}\scalebox{#2}{\includegraphics{fig/#3}}}\\
  \subfloat[]{\label{fig:#4}\scalebox{#2}{\includegraphics{fig/#4}}}
  \caption{#6}
  \label{fig:#5}
\end{figure}
}
% e.g.
%\multfigtwov{.35}{fig_epos_mem_framework}{fig_epos_mem_framework_spm}
%{fig_epos_mem_framework_all}
%{EPOS memory mapping before (a) and after (b) using the new framework}

\usepackage{multirow}
%\setlength{\tabcolsep}{1mm}
\newcommand{\tab}[4][tb]{
  \begin{table}[tb]
    \caption{#3}\label{tab:#2}
    {\centering\footnotesize\textsf{\input{fig/#2.tab}}\par}
  \end{table}
}

\newcommand{\epos}{\textsc{Epos}}
\newcommand{\emote}{\textsc{EPOSMoteII}}
\newcommand{\adhop}{\textsc{Adhop}}


% correct bad hyphenation here
\hyphenation{op-tical net-works semi-conduc-tor}


\begin{document}

\title{Improving task-scheduling in solar energy harvesting wireless sensor network systems}

%
% paper title
% can use linebreaks \\ within to get better formatting as desired
% \title{Embedded System Energy Scheduler Improvement \\ Through a Solar Energy Harvesting Circuit \\
%\textit{\normalsize{Submitted to ESTIMEDIA12 Special Section}}
% }
%
%
% author names and IEEE memberships
% note positions of commas and nonbreaking spaces ( ~ ) LaTeX will not break
% a structure at a ~ so this keeps an author's name from being broken across
% two lines.
% use \thanks{} to gain access to the first footnote area
% a separate \thanks must be used for each paragraph as LaTeX2e's \thanks
% was not built to handle multiple paragraphs
%

% NOTE TODO Maybe re-enable \thanks latter
\author{
Leonardo~Kessler~Slongo, Arliones~Hoeller~Jr, Antônio~Augusto~Fröhlich and Eduardo Augusto Bezerra\\
\IEEEmembership{Federal University of Santa Catarina}% <-this % stops a space
}

% note the % following the last \IEEEmembership and also \thanks -
% these prevent an unwanted space from occurring between the last author name
% and the end of the author line. i.e., if you had this:
%
% \author{....lastname \thanks{...} \thanks{...} }
%                     ^------------^------------^----Do not want these spaces!
%
% a space would be appended to the last name and could cause every name on that
% line to be shifted left slightly. This is one of those "LaTeX things". For
% instance, "\textbf{A} \textbf{B}" will typeset as "A B" not "AB". To get
% "AB" then you have to do: "\textbf{A}\textbf{B}"
% \thanks is no different in this regard, so shield the last } of each \thanks
% that ends a line with a % and do not let a space in before the next \thanks.
% Spaces after \IEEEmembership other than the last one are OK (and needed) as
% you are supposed to have spaces between the names. For what it is worth,
% this is a minor point as most people would not even notice if the said evil
% space somehow managed to creep in.


% NOTE TODO re-enable latter
% The paper headers
% \markboth{Journal of \LaTeX\ Class Files,~Vol.~6, No.~1, January~2007}%
% {Shell \MakeLowercase{\textit{et al.}}: Bare Demo of IEEEtran.cls for Journals}
% The only time the second header will appear is for the odd numbered pages
% after the title page when using the twoside option.
%
% *** Note that you probably will NOT want to include the author's ***
% *** name in the headers of peer review papers.                   ***
% You can use \ifCLASSOPTIONpeerreview for conditional compilation here if
% you desire.




% If you want to put a publisher's ID mark on the page you can do it like
% this:
%\IEEEpubid{0000--0000/00\$00.00~\copyright~2007 IEEE}
% Remember, if you use this you must call \IEEEpubidadjcol in the second
% column for its text to clear the IEEEpubid mark.



% use for special paper notices
%\IEEEspecialpapernotice{(Invited Paper)}




% make the title area
\maketitle


\begin{abstract}
%\boldmath
This work presents a solar energy harvesting circuit envisioned to extend the lifetime of low power wireless sensing platforms.
The energy harvesting circuit operates solar panels closely to their maximum power point only by precisely matching them to the batteries.
The circuit improves an energy-aware, wireless sensor network system by providing the task scheduler with more frequent and accurate measurements of battery charge.
The paper discusses the circuit design and evaluation, and shows its capability for extending systems' lifetime.
A simulation evaluates the proposed system, showing that the task scheduler was able to run all critical tasks, while also running $1.23$ times more non-critical tasks when compared to the previous approach.
Finally, the outdoor tests allowed an experimental correlation between the current delivered by the solar panels and the solar irradiance, which will be used for including environmental prediction capability on the energy-aware schedulers.
\end{abstract}

% IEEEtran.cls defaults to using nonbold math in the Abstract.
% This preserves the distinction between vectors and scalars. However,
% if the journal you are submitting to favors bold math in the abstract,
% then you can use LaTeX's standard command \boldmath at the very start
% of the abstract to achieve this. Many IEEE journals frown on math
% in the abstract anyway.

% Note that keywords are not normally used for peerreview papers.
\begin{IEEEkeywords}\\
Solar energy harvesting, energy-aware task scheduler, low power wireless platforms, solar panel, wireless sensor networks.
\end{IEEEkeywords}

% = Categories and Subject Descriptors
% I.4 [Image Processing and Computer Vision]: Compression (Coding)
% I.3 [Computer Graphics] Parallel processing
% D.2 [Software Engineering]: Modules and Interfaces
% 
% = General Terms Design
% Algorithms
% Performance
% 
% = Keywords / Index Terms
% Motion Estimation, Components, Designing for interface, Video encoding, H.264




% For peer review papers, you can put extra information on the cover
% page as needed:
% \ifCLASSOPTIONpeerreview
% \begin{center} \bfseries EDICS Category: 3-BBND \end{center}
% \fi
%
% For peerreview papers, this IEEEtran command inserts a page break and
% creates the second title. It will be ignored for other modes.
\IEEEpeerreviewmaketitle

% ------------------------------------------------------------------------------
% ------------------------------------------------------------------------------
\section{Introduction} \label{intro}
% + Introduction
% 
% The very first letter is a 2 line initial drop letter followed
% by the rest of the first word in caps.
%
% form to use if the first word consists of a single letter:
% \IEEEPARstart{A}{demo} file is ....
%
% form to use if you need the single drop letter followed by
% normal text (unknown if ever used by IEEE):
% \IEEEPARstart{A}{}demo file is ....
%
% Some journals put the first two words in caps:
% \IEEEPARstart{T}{his demo} file is ....
%
% Here we have the typical use of a "T" for an initial drop letter
% and "HIS" in caps to complete the first word.
% \IEEEPARstart{T}{his} demo file is intended.

\IEEEPARstart{E}{nergy} consumption is a determining factor when designing wireless sensor networks.
As a consequence, battery lifetime is a limitation on the development of such systems.
Therefore, the idea of extracting energy from the environment has become attractive.
Looking to the energy consumption problem, the intelligent usage of the stored energy contributes to extend the sensor nodes' longevity.
Consequently, energy schedulers have been developed in order to adequately assess the energy consumption and adapt the system accordingly to the available amount of energy.
The purpose of this work is to adapt a solar energy harvesting circuit to supply energy to low power wireless platforms, i.e., those that operate under $50~mW$.
Simultaneously, we aim at improving the performance of the energy-aware task scheduler in wireless sensor network systems by providing fine-grained battery and environmental monitoring.

Among a number of energy sources that have been studied so far, solar has proved to be one of the most effective~\cite{Roundy:2003}.
The solar energy conversion through photovoltaic (PV) cells is better performed at an optimum operating voltage.
Operating a solar panel on this voltage results in transferring to the system the maximum amount of power available.
In this context, \emph{maximum power point tracker circuits} (MPPT) have been proposed.
The drawback is that MPPT circuitry may introduce losses to a solar harvesting system.
Concerning low-power applications, it may be more energy efficient to have a good matching between the solar panel and the energy storage unit~\cite{Raghunathan:2005}.
This well matched system is than able to work close to the maximum power point with less power loss.

In this work, an evaluation of the proposed harvesting circuit is performed in order to show improvements on an energy-aware task scheduler~\cite{Hoeller:SMC:2011}.
It is shown that the combination of the proposed circuit with the cited scheduler not only extended the longevity of the wireless sensor network, but also improved system quality.

The paper is organized as follows:
Section~\ref{fund} presents the fundamentals of solar energy harvesting and energy-aware task scheduler.
Section~\ref{design} discusses the design of the harvesting circuit under the perspective of low power wireless platforms.
Section~\ref{case} presents the evaluation of the harvesting circuit and a case study showing the improvements on system quality.
Finally, section~\ref{concl} closes the paper.

% ------------------------------------------------------------------------------


% ------------------------------------------------------------------------------
% ------------------------------------------------------------------------------
\section{Fundamentals} \label{fund}

Energy-aware task scheduling has been a subject of intense research for the last three decades.
Specifically for wireless sensor network systems, energy-aware issues have been investigated from a design-time perspective, mainly in four fronts:
 low-power hardware design~\cite{Polastre:2005};
 low-power communication protocols~\cite{Huang:2012,Pantazis:2012};
 low-power applications~\cite{Mottola:2011};
 and efficient energy sources~\cite{Sharma:2010}.
A little effort has been dedicated to optimize the system performance through task scheduling while reducing energy cost in wireless sensor networks~\cite{Kulkarni:2011}.

In this paper, we build on previous research~\cite{Hoeller:SMC:2011} in order to identify the benefits of having a more precise way of monitoring the energy source on real-time wireless sensor network systems.
We specifically focus on wireless sensor networks systems that harvest energy from solar irradiation through photovoltaic cells.
Within this context, this section describes a few fundamentals of the used technologies.

\subsection{Solar Energy} \label{solar_usage}

Power management is one of the main research topics in wireless sensor networks.
Batteries, which have a limited energy capacity, are the most common option for powering sensors. % What would be the alternative with ``less limited capacity''?
Sleeping modes and low power communication protocols \cite{Raghunathan:2006} emerged as a solution to extend systems' lifetime.
Although this allows a better distribution of the consumed energy along the network, the idea of energy harvesting has been applied in order to deliver more energy to the nodes.
This implies on further enhancement of system quality, as volume of communication and processing may be raised proportionally to the amount of available energy.

\begin{table}[h]
\renewcommand{\arraystretch}{1.3}
\caption{Power Densities of Harvesting Technologies}
\label{tab:power}
\centering
\begin{tabular}{c||c}
\hline
\bfseries Harvesting technology & \bfseries Power density			\\
\hline\hline
Solar cells (outdoors at noon) 				& $15~mW/cm^{2}$			\\
Piezoelectric (shoe inserts)   				& $330~\mu W/cm^{3}$  	\\
Vibration (small microwave oven) 			& $116~\mu W/cm^{3}$   	\\
Thermoelectric $(10^\circ C$ $gradient)$ 	& $40~\mu W/cm^{3}$    	\\
Acoustic noise (100dB)   					& $330~nW/cm^{3}$ 		\\
\hline
\end{tabular}
\end{table}

%\begin{center}
%\begin{threeparttable}
%\caption{Power Densities of Harvesting Technologies}
%\label{tab:power}
%\begin{tabular}{rlcc}
%\hline
%& $Harvesting technology$ 						& $Power$ $density$	\\
%\hline
%& $Solar$ $cells$ $(outdoors$ $at$ $noon)$ 		& $15mW/cm^{2}$		\\
%& $Piezoelectric$ $(shoe$ $inserts)$   			& $330\mu W/cm^{3}$  	\\
%& $Vibration$ $(small$ $microwave$ $oven)$ 		& $116\mu W/cm^{3}$   	\\
%& $Thermoelectric$ $(10^\circ C$ $gradient)$ 	& $40\mu W/cm^{3}$    	\\
%& $Acoustic$ $noise$ $(100dB)$   				& $330nW/cm^{3}$ 	\\
%\hline
%\end{tabular}
%\end{threeparttable}
%\end{center}

As Table~\ref{tab:power} shows, different harvesting technologies present different power densities~\cite{Roundy:2003}.
Also, it is shown that the solar modality presents the highest power density.
However, many parameters should be taken into account when planning to operate a photovoltaic module, including solar irradiance, temperature variation, mechanical position and the photovoltaic module's electrical characteristics.
Through experimentation,  photovoltaic cells' characteristic power and current curves, like the ones in Figure~\ref{fig:pv_plot}, are obtained under known operating parameters (e.g. temperature of $25~^\circ C$ and an irradiance of $1000~W/m^{2}$).
These curves demonstrate that solar cells are remarkably versatile as they can operate from the state of open circuit -- where, theoretically, there is no current flowing -- to the state of short circuit -- where, theoretically, there is no voltage drop between the positive and negative terminals of the solar cell.
The solar cell power curve is nothing but the product of voltage and current values in \emph{y} and \emph{x} axis.
With this concept in mind, and assuming positive values for voltage and current, the power curve must have at least one maximum value.
In order to extract the maximum power of the environment, a solar panel must operate as close as possible to this maximum power point.
Normally, the maximum power is extracted from the solar panel by applying to it the $V_{MPP}$ (voltage at maximum power point).
% There are several ways of achieving this goal.
Electronic circuits, called \emph{maximum power point trackers} (MPPT), are responsible for ensuring the operation on this point.

\fig{1}{pv_plot}{Typical solar cell voltage-current and voltage-power curves of polycrystalline silicon cells.}


The idea of operating a solar panel as close as possible to its $V_{MPP}$ is not new \cite{Schoeman:1982}.
The microelectronic industry has already developed integrated circuits able to keep a solar panel operating on its $V_{MPP}$.
Most of these ICs, however, are dedicated to high power applications and present high power consumption, making it generally incompatible with low-power energy harvesting systems.
Hence, MPPT circuits for low power applications are also being investigated.
These circuits operate by detecting changes on the solar panel, computing the new $V_{MPP}$ and applying this voltage to the photovoltaic module.
Changes in the behavior of the photovoltaic module happen mainly due to variations in temperature and solar irradiance.
Thus, sensors are needed to monitor state changes.
As these sensors imply in extra load in the system, its use also is often unfeasible in energy-sensitive or low-power systems.

In order to solve this problem, electronic circuits were developed to extract the solar cell information directly from its electrical characteristics.
Methods as Perturb and Observe~\cite{Liu:2004,Tan:2008}, Incremental Conductance~\cite{Jain:2004,Qin:2011}, Fractional Open Circuit Voltage~\cite{Ahmad:2010,Dondi:2008}, Constant Voltage~\cite{Yu:2002} are the most known.
Although these are interesting methods, the solution described here is focused on working as close as possible to the MPP through an ideal match between the battery and the photovoltaic module.
Although it is not the most efficient method, it is the simplest way of operating a solar panel closely to its MPP, and its efficiency have already been proved~\cite{Raghunathan:2005}.
Besides its simplicity, this circuit provides accurate battery information to the sensing platform, what drove a significant improvement in its structure.

\subsection{Energy consumption monitoring} \label{consumption}

Since this work is strongly related with the energy consumption of the sensing platform, it is convenient to elucidate the currently implemented method for estimating its energy consumption and how it can be improved.
The current method, Battery Level Monitoring by Event Accounting~\cite{Hoeller:SMC:2011}, proposes the energy consumption model described by Equations~\ref{eq:en_dev_time} through~\ref{eq:en_tm_ev} to estimate the battery charge.

\begin{eqnarray}
E_{tm}(dev) = (t_{end} - t_{begin}) \times I_{dev,mode} \label{eq:en_dev_time}\\
E_{ev}(dev) = \sum_{event\_counters} E_i * counter \label{eq:en_dev_ev}\\
E_{tot}(dev) = E_{tm}(dev) + E_{ev}(dev) \label{eq:en_tm_ev}
\end{eqnarray}

% In the equations,
% $E_{tm}(dev)$ is the energy consumption in the time-based profile for a specific device,
% $t_{end}$ and $t_{begin}$ denote timestamps,
% and $I$ is the current of a device at a given operating mode.
% $E_{ev}$ denotes the sum of energy ($E_i$) consumed by the observed events ($counters$). $E_{tot}$ is the energy consumption in the combined profile.

The idea is to estimate the energy consumed by the devices in the sensing platform through two different perspectives.
The first one (Equation \ref{eq:en_dev_time}) is dedicated to devices ($dev$) operating with constant current ($I$) over a time period ($t$) and in a determined mode ($mode$).
The second perspective (Equation \ref{eq:en_dev_ev}) is event-based (e.g., sensor sampling), where energy consumed by specific events ($E_i$) are accumulated periodically according to the number of accounted events ($counter$).
Finally, this total energy consumption (Equation \ref{eq:en_tm_ev}) is defined by the sum of $E_{tm}(dev)$ and $E_{ev}(dev)$.

Components' datasheets are the base to estimate the inputs for equations~\ref{eq:en_dev_time} and~\ref{eq:en_dev_ev} (i.e. $I_{dev,mode}$ and $E_i$).
Therefore, those equations render a pessimistic but safe estimation.
In order to avoid underestimations when considering the worst case, the system measures battery voltage periodically to estimate actual battery charge.
The battery energy ($E_{batt}$) is then calculated as shown in Equation~\ref{eq:batt_update}, where $E_{volt}$ is the battery energy estimation based on the battery voltage reading.

\begin{equation}
E_{batt} = max\left(E_{volt} , E_{batt} - \sum_{i = 0}^{\#devs} E_{tot}(i)\right) \label{eq:batt_update}
\end{equation}

%change expression time to time
Measuring the battery voltage through a shunt resistor is the drawback of this method due to its power loss.
This results in sporadic measurements.
Besides providing the sensing platform with an energy harvesting circuit, the work in this paper also contributes to the performance of the scheduling mechanism.
The performance enhancement is achieved by replacing $E_{volt}$ by the energy consumption readings of the battery monitor IC, which are more accurate.
Also, the measurement may be realized with a higher frequency in the new approach, since these measurements are supported by the power coming from the solar panels.
% It is known that there are more efficient circuits using MPPT techniques~\cite{Lopez:2010,Win:2010}, however, none of those works provide a solution with energy harvesting and energy schedulers simultaneously.
% Besides that, the results of this work are meant to be compared with MPPT circuits, in order to investigate for which cases MPPT circuits are more suitable.

% ------------------------------------------------------------------------------


% ------------------------------------------------------------------------------
% ------------------------------------------------------------------------------
\section{Design Considerations} \label{design}

This section explains how the proposed solar harvesting circuit was designed.
It presents a reasoning about technical decisions and modifications implemented on the Heliomote project~\cite{Raghunathan:2005}, in order to support motes based on low power wireless platforms.
Finally, the performance of the harvesting circuit is evaluated through outdoor experiments.

\subsection{Solar Cell and Energy Storage Unit} \label{scell_stunit}

The relation between the solar cell and the energy storage unit is a crucial issue for the proposed harvesting circuit. This highlights the need of working as close as possible to the MPP, in order to extract the maximum amount of energy available. Since the circuit has no maximum power point tracker (the solar cells are directly coupled to the battery), the battery operating voltage should be as close as possible to the $V_{MPP}$ for the selected solar cell.

The two available technologies to store energy, in a rechargeable way, are batteries and super capacitors. Although super capacitors are improving, they are not  recommended for solar harvesting circuits yet due to intrinsic leakage, losses on parasitic paths in the external circuitry \cite{Raghunathan:2005}, and low energy density. Among all the battery technologies available (e.g., Sealed Lead Acid, Nickel Cadmium, Nickel Metal Hydride, Lithium based), the Nickel Metal Hydride (NiMH) was chosen mainly for the low complexity on the recharging circuit and the absence of memory effect. Consequently, the system is powered by two common AA NiMH batteries with a nominal charge capacity of $2100~mA$.  

The PV module is composed by two $4~V-100~mA$ solar cells measuring $60$~x~$60~mm$. The solar cell’s characterization curves were plotted (Figures \ref{fig:cv_cha} and \ref{fig:pv_cha}) based on data collected outdoor, on March 08, 2012 at 1:15 pm, with an average irradiance of $954~W/m^{2}$. The goal of the test was to find out the open circuit voltage ($V_{OC}$), the short circuit current ($I_{SC}$) and the $V_{MPP}$ in a real scenario. Table \ref{tab:solar_cha} shows the obtained values.

\fig{0.25}{cv_cha}{Solar panel characterization carried outside - current-voltage curve}
\fig{0.25}{pv_cha}{Solar panel characterization carried outside - power-voltage curve}

\begin{table}[!t]
\renewcommand{\arraystretch}{1.3}
\caption{Solar Panel Measured Parameters}
\label{tab:solar_cha}
\centering
\begin{tabular}{c||c}
\hline
\bfseries Parameter & \bfseries Measured Value	\\
\hline\hline
$V_{OC}$ 		& $4.15~V$							\\
$I_{SC}$   		& $122.3~mA$  						\\
$V_{MPP}$		& $3.25~V$   						\\
\hline
\end{tabular}
\end{table}
 
\subsection{Adaptation for an Low Power Wireless Platform} \label{adaptation}

In order to understand the adaptations proposed for low power wireless platforms, it is essential to know some of their characteristics.
Table~\ref{tab:mcu_cha} shows three different examples that have motivated the changes applied to Helimote's circuit.
They are typical Platform-in-Package (PiP) which features a $2.4~GHz$ radio frequency transceiver.
As shown in Table~\ref{tab:mcu_cha}, all these PiPs may operate with voltages lower than $3.3~V$.
Therefore, there is no need for the step-up converter used in Heliomote's project.
The new generation of PiPs normally operates at $1.8~V$, having a built-in buck regulator.
Hence, the harvesting circuit efficiency is improved by eliminating the step-up converter.

%\begin{center}
%\begin{threeparttable}
%\caption{Wireless MCU's Electrical Characteristics}
%\label{tab:mcu_cha}
%\begin{tabular}{llll}
%\hline
%Model		& Manufacturer				& Voltage	& Consumption		\\
%\hline
%MC1322	& Freescale Semiconductor	& 2.1V - 3.6V		& 29mA~\tnote{a}	\\
%STM32W	& Texas Instruments			& 2V - 3.6V			& 29mA~\tnote{b}	\\
%CC2530	& STMicroelectronics		& 2.1V - 3.6V		& 31mA~\tnote{c}	\\
%\hline
%\end{tabular}
%\begin{tablenotes}
%\item[a] transmit current, CPU clock at 2 MHz.
%\item[b] transmit current at +3 dBm, CPU idle.
%\item[c] transmit current at 1 dBm, w/CPU.
%\end{tablenotes}
%\end{threeparttable}
%
%\end{center}


\begin{table}[!t]
\renewcommand{\arraystretch}{1.3}
\caption{Wireless MCU's Electrical Characteristics}
\label{tab:mcu_cha}
\centering
\begin{tabular}{l||l||l||l}
\hline
\bfseries Model & \bfseries Manufacturer & \bfseries Voltage	& \bfseries Current* \\
\hline\hline
MC1322	 		& Freescale Semiconductor 	& $2V-3.6V$					& $29mA$				\\
STM32W   		& Texas Instruments			& $2V-3.6V$					& $29mA$				\\
CC2530			& STMicroelectronics		& $2.1V-3.6V$				& $31mA$				\\
\hline
\multicolumn{4}{l}{* current drain in transmit mode.}
\end{tabular}
\end{table}

\subsection{Circuit Design} \label{circ_design}

The harvesting circuit is planned to charge the NiMH batteries until a predefined threshold voltage. This voltage is set by external resistors in a voltage monitor (ICL7665). The threshold voltage selected was based on the charging behavior of the NiMH battery provided by the manufacturer (see Figure \ref{fig:bat_cha}). This battery is considered fully charged when it reaches a voltage around $1.45~V$ ($0.1~C$ curve). The selected threshold voltage was $2.9~V$, since the two AA cells were aranged in-serie. An analog switch short circuits the solar panel to ground when this value is reached. The overvoltage signal provided by the ICL7665 controls the analog switch. This prevents battery overcharges. A normally opened (MAX4614) analog switch were used instead of a normally closed one. Hence, if the initial battery voltage is bellow $2~V$ it is charged by the solar panel instead of being short circuited to the ground as in the case of using a normally closed one. 

The circuit must also take care of the undercharge situation. There are two different reasons to avoid that. First, there is a minimum voltage threshold from which batteries do not operate properly. In this case, it is recommended to disconnect the load in order to avoid battery degradation. The second reason is that the input voltage of the wireless MCU's should not be lower than $2~V$. Thus, the under voltage pin of the ICL7665 is connected to a transistor. Then, when the battery voltage drops below $2.1~V$ the transistor is opened, disconnecting the mote. A simplified block diagram of the harvesting circuit is shown in Figure \ref{fig:circ_diagram}.

Besides the already mentioned features, the circuit has a battery monitor (DS2438). This IC communicates trhough a protocol called 1-Wire. It provides the battery's information (battery voltage, battery current, remaining capacity and temperature) that may be used by a energy-aware task scheduler. The DS2438 also has an integrated current accumulator, which informs the total current going in and out of the battery.

\fig{0.26}{bat_cha}{Charging curves for AA Nickel Metal Hidryde battery \cite{Panasonic:2000}.}
\fig{0.45}{circ_diagram}{Simplified block diagram of a sensing platform solar harvesting circuit.}


% ------------------------------------------------------------------------------
% ------------------------------------------------------------------------------
\section{Case study: scheduling in mobile WSN} \label{case}

The proposed approach was integrated to the power management mechanism of \epos~\cite{Frohlich:IJDSN:2011}.
\epos~is a component-based operating system for embedded applications.
% The approach implementation, following the philosophy behind \epos~power manager, is modeled as a scenario adapter, making it not intrusive to system components.
Also, we ran this implementation in the \emote~platform~\cite{Project:EPOS:2012}, a module for the development of low-power wireless sensor network applications.
Hardware evaluation was performed by analysis of data in an actual implementation.
The scheduling approach was evaluated by means of simulation taking the characterization of the mentioned platform into consideration.
The remaining of this section describes the setup of the evaluation scenario and the obtained results.


\subsection{\emote~Solar Energy Harvesting Circuit} \label{circ_eval}

In order to adequately characterize the developed circuit integrated to the \emote~sensing platform, an outdoor test was performed.
The test period was 62~hours and 45 minutes.
For the test, the \emote~radio was configured to constantly transmit random data.
This $100~\%$ duty cycle was used in order to test critical conditions and also to reduce test duration.
By reducing the duty cycle, it would be necessary to extend the test to several weeks, since the total current consumption when the system is in standby mode is only around $5~\mu A$, i.e., more than 5,000 times smaller than the $29~mA$ it consumes when transmitting data at full power.

%referencia EPOSMote Hardware

Figure~\ref{fig:bat_volt} shows the battery voltage behavior during the test.
In all recharge cycles, the battery has almost reached the fully recharged voltage.
However, there are differences among the peak values, which shows that the maximum voltage reduced from one peak to the previous one.
This fact is explained analyzing the amount of energy delivered by the battery.
Figure~\ref{fig:energy} shows that the system has lost energy after each cycle,
which means that, for a $100\%$ duty cycle, this system would not be self-sufficient.

\fig{0.26}{bat_volt}{Battery voltage behavior.}

The test started on March 16, 2012 at 9:15~pm, i.e. during night, what explains why the first descending slope in Figure~\ref{fig:energy} is shorter than the other two.
Current integration was used in order to calculate the energy expended on slopes.
The first complete slope spent $738.43~mAh$, while the second spent $746.96~mAh$.
The small difference between these values is justified by the difference on the daylight duration, which, although similar, was not perfectly equal in both days.
The analysis of these values is of paramount importance when selecting the capacity of the storage unit.
This extreme case~(i.e. $100\%$ duty cycle) helps to justify the need for an energy-aware scheduler in the system.
The descending slopes must be reduced in order to have a self-sufficient system.
% The energy consumption reduction procedure is done by executing only the critical tasks, which allows the system to work for longer periods.
% Predicting how much energy is going to be delivered by the solar panel is the key knowledge for deciding which best-effort tasks will be executed.  

\fig{0.26}{energy}{Energy evolution.}


% This prediction is possible when analyzing the relation between current delivered by the solar panel and the solar irradiance.
Figure~\ref{fig:i_panel} shows the current delivered by the solar panel, which was calculated by subtracting the system's current~(considered constant at $56~mA$ due to the constant duty cycle) from the input/output battery current.
Figure~\ref{fig:irrad} shows the solar irradiance acquired during the test by a pyranometer placed close to the system's photovoltaic panel at the same inclination~($27~^\circ$).
The current delivered by the solar panel and the solar irradiance were plotted in order to obtain an equation to correlate them.
This curve and its linear approximation are shown in Figure~\ref{fig:i_irrad_final} and Equation~\ref{eq:i_irrad}.
In this equation, $I_{panel}$ is the current delivered by the solar panel in mA and \emph{Irrad} is the solar irradiance in $W/m^{2}$.
This plot is an additional contribution of this work, since it is through this linear equation that the energy-aware task scheduler can map weather forecast to energy input.
This plot will motivate the development of new heuristics for the scheduler which will further improve its efficiency.

\fig{0.26}{i_panel}{Current delivered by the solar panel.}
\fig{0.26}{irrad}{Solar irradiance variation.}
\fig{0.26}{i_irrad_final}{Relation between current and solar irradiance.}

\begin{eqnarray}
I_{panel} = 0.20628 \times Irrad \label{eq:i_irrad}
\end{eqnarray}

Solar irradiance and delivered current present a linear behavior when variations in temperature are not taken into consideration.
% For this first approximation, a linear equation was used.
Figure~\ref{fig:temp} adds temperature variation, which is the main reason for the spread points in Figure~\ref{fig:i_irrad_final}.
A new mathematical model considering temperature variation is being developed in order to design an energy-aware task scheduler with more precise environmental prediction.

\fig{0.26}{temp}{Temperature variation.}


\subsection{Evaluation Scenario}

The application used to evaluate the approach is a mobility-enabled wireless sensor network.
This network runs the Ant-based Dynamic Hop Optimization Protocol~(\adhop) over an IP network using IEEE 802.15.4.
\adhop~is a self-configuring, reactive routing protocol designed with the typical limitations of sensor nodes in mind, energy in particular~\cite{Okazaki:SCPA:2011}.
\adhop's reactive component relies on an \emph{Ant Colony Optimization} algorithm to discover and maintain routes.
Ants are sent out to track routes, leaving a trail of pheromone on their way back.
Routes with a higher pheromone deposit are preferred for data exchange.

\epos~scheduler relies on \epos~power manager to adaptively run the system.
In \epos~adaptive task scheduling model, tasks are classified as hard real-time or best-effort.
In order to guarantee system lifetime, \epos~energy scheduler reserves the amount of energy the system will need to run hard real-time tasks until a desired lifetime is reached.
Best-effort tasks are only allowed to execute when excess energy exists.
In \epos~scheduler, different heuristics can be used to control system quality degradation when best-effort tasks are prevented from executing.
Once specific heuristics for managing energy consumption are not the focus of the present work, only \epos~global energy allocation heuristic was considered~\cite{Hoeller:SMC:2011}.

The objective of this case study is to demonstrate how the employment of the proposed energy input measurement mechanism enhanced system performance.
Thus, \adhop~had to be modified.
\adhop's tasks have then been classified as hard real-time or best-effort.
The main idea behind this setup was to homogenize the battery discharge for every node in the network to enhance the lifetime of the network as a whole.
Considering the radio the most energy-hungry component in a wireless sensing node, we made the design decision of modeling the routing activity of \adhop~as a best-effort task, as shown by the task set at Table~\ref{tab:adhop-taskset}.
The basic node functionality of sensing a value~(task $Sense$) and forwarding it through the radio to the next node~(task $Forward$) where modeled as hard real-time tasks.
The functionality of forwarding other nodes' packets~(and ants) when acting as a ``router'' was modeled as two best-effort tasks, one for monitoring the channel for arriving messages~($LPL$ - Low Power Listen), and another to effectively receive the message and route it to another node~($Route$).

\tab{adhop-taskset}{\adhop~case-study tasks' parameters\protect\footnotemark[3].}

\footnotetext[3]{
% T: task;
Period in $ms$;
WCET: worst-case execution time in $ms$;
WCEC: worst-case energy consumption in $\eta Ah$;
25-days: energy consumption for the targeted lifetime~(25 days) in $mAh$.}
\footnotetext[4]{This is a worst-case scenario as values of ``P'' and, as consequence, ``25-days'', may change due to adaptation.}
\footnotetext[5]{``Route'' is a sporadic task. Once it is a best-effort task in the system, we consider a hypothetic frequency of $2$ Hz~(period of 500 $ms$) to show the impact of routing in the node energy consumption.}

The simulation time was set to 25 days.
By analyzing the task set, it is possible to compute the total energy consumption of hard real-time tasks for the desired lifetime to be of $602~mAh$.
As a consequence, the initial battery charge for the system has to be greater than that to allow the system to reserve energy for the critical part of the application.
The battery capacity specified for this experiment is an of-the-shelf CR-2/$3V$ battery with a total capacity of $850~mAh$
%%%%% the one described in Section~\ref{circ_design}, i.e., a $2,100~mAh$ NiMH battery.
% For the simulation, the voltage model of the battery was defined by the cubic approximation shown in Figure~\ref{fig:bat_volt_charge}, that shows the expected battery charge for a battery voltage level reading.

\fig{.69}{adhop-char-ddr}{Data delivery response to BET rate.}

\fig{.69}{adhop-char-energy}{Average power and energy consumption response to BET rate.}

The simulation was performed in two steps.
In the first step, a simulation using the \textsc{OmNET++} Simulator characterized the application's response to variations in the execution rate of best-effort tasks~(BET rate).
As can be seen in Figures~\ref{fig:adhop-char-ddr} and~\ref{fig:adhop-char-energy}, lower energy consumption at lower BET rates comes at the cost of lower data delivery rate.
Also, it is possible to observe in the graphic that BET rates above $50\%$ have no significant impact on packet delivery.
Thus, it is assumed that BET rate will only be adjusted within the range $[0,50]$ as a means to further save energy.

In the second simulation step, the energy consumption of the system was simulated for 25 days, i.e., the target lifetime.
As can be seen in Figure~\ref{fig:sim-bet_rate}, the employment of the measurement mechanisms proposed here enhanced the BET rate.
This happens because the proposed approach allows for more frequent adaptations that reduces the pessimistic bias of the used scheduler.
It is also possible to observe that the new approach shows a better recover after relatively long periods of low irradiation, such as the one found between days 7 and 11 in Figure~\ref{fig:sim-irrad}.
Short periods of low irradiation, like the one on day 19, are better supported by the new approach.
This can be observed by the slight decrease on BET rate on day 19 for the original method, while the new method is still reaching $50\%$.
Over the 25-days period, $32.93\%$ of the best-effort tasks were executed using the new approach, against $26.81\%$ of BET rate of the former approach.
This meas a total gain of $1.23$ times.


\fig{.69}{sim-bet_rate}{Average BET rate over 25 days of execution using with and without the proposed approach.}

\fig{.69}{sim-irrad}{Daily irradiation levels used for simulation.}



% ------------------------------------------------------------------------------

% ------------------------------------------------------------------------------
\section{Conclusão} \label{concl}

Após realizarmos os testes concluímos a viabilidade da implementação do protocolo PTP para realizar a sincronização dos tempos de relógio em um sistema operacional embarcado, pois conseguimos manter o {\it offset} próximo a 0 segundo. Isso nos fornece uma base para trabalharmos a implementação com o intuito de obter um {\it offset} na faixa de sub-milissegundos. Obtendo essa precisão conseguiríamos garantir a aplicação, por exemplo, da implementação para sistemas de sensoriamento oceânico, como é abordado em \cite{DelRio2012}. Onde é feita uma abordagem sobre a implementação do protocolo PTP para distribuir os tempos de relógio em uma rede Ethernet de Sensoriamento Marinho(MSN). O fato da necessidade de um protocolo deste tipo para tal escopo se dá pelo fato de sinais GPS não estarem disponíveis devido à atenuação da água no fundo do mar e à requisitos de sincronização de instrumentos marítimos, tais como sismógrafos.




% ------------------------------------------------------------------------------

% if have a single appendix:
%\appendix[Proof of the Zonklar Equations]
% or
%\appendix  % for no appendix heading
% do not use \section anymore after \appendix, only \section*
% is possibly needed

% use appendices with more than one appendix
% then use \section to start each appendix
% you must declare a \section before using any
% \subsection or using \label (\appendices by itself
% starts a section numbered zero.)
%


% \appendices
% \section{Proof of the First Zonklar Equation}
% Appendix one text goes here.
% 
% % you can choose not to have a title for an appendix
% % if you want by leaving the argument blank
% \section{}
% Appendix two text goes here.


% use section* for acknowledgement
%\section*{Acknowledgment}
%We would like to thank Ronaldo Husemann and Prof. Valter Roesler for the
%prolific discussions about motion estimation optimization.
%This work was partially funded by FINEP grant INF/FINEP 01.08.0287.00-REDEH264,
%project Rede H.264 SBTVD.
% TODO colocar número do projeto.
% PROJETO: INF/FINEP 01.08.0287.00-REDEH264
% NÚMERO:
% 06341 - X



% Can use something like this to put references on a page
% by themselves when using endfloat and the captionsoff option.
\ifCLASSOPTIONcaptionsoff
  \newpage
\fi



% trigger a \newpage just before the given reference
% number - used to balance the columns on the last page
% adjust value as needed - may need to be readjusted if
% the document is modified later
%\IEEEtriggeratref{8}
% The "triggered" command can be changed if desired:
%\IEEEtriggercmd{\enlargethispage{-5in}}

% references section

% can use a bibliography generated by BibTeX as a .bbl file
% BibTeX documentation can be easily obtained at:
% http://www.ctan.org/tex-archive/biblio/bibtex/contrib/doc/
% The IEEEtran BibTeX style support page is at:
% http://www.michaelshell.org/tex/ieeetran/bibtex/
%\bibliographystyle{IEEEtran}
% argument is your BibTeX string definitions and bibliography database(s)
%\bibliography{IEEEabrv,../bib/paper}
%
% <OR> manually copy in the resultant .bbl file
% set second argument of \begin to the number of references
% (used to reserve space for the reference number labels box)
% References
\bibliographystyle{IEEEtran}
\bibliography{solar}


\end{document}
