\section{Desenvolvimento do Hardware} 

\fig{Job1.jpg}{A PCB final do projeto}{width=0.8\columnwidth}

O \emote~apresenta mais funcionalidades do que as requisitadas pelo projeto, porém não apresenta as características físicas requisitadas pelo mesmo, sendo assim foi necessário um redesenho do hardware do \emote~que atende-se não apenas as funcionalidades como também apresenta-se tamanho e peso coerentes com o esperado do produto.

O hardware do \emote~foi redesenhado apresentando as mesmas características de sistemas do \emote~padrão, porém apresentando apenas os dispositivos externos necessários. Optou-se também por utilizar uma bateria CR2450 em função de seu tamanho reduzido e formato propício a ser utilizado em um hardware que necessita apresentar tamanho reduzido.

O processo de redesenho do \emote~foi simples uma vez que tem-se acesso ao desenho original do \emote~e esse apresenta todas as ligações entre processador e os dispositivos externos. Assim é apenas utilizar o desenho original retirando os dispositivos que não são utilizados e rearranjando as trilhas de modo que a placa fique a menor possível.
