\documentclass[a4paper,10pt,english]{article}
\usepackage[latin1]{inputenc}
\usepackage{babel}
\usepackage{times}
\usepackage[margin=3cm,noheadfoot]{geometry}
   
\title{Operating Systems: are we finally ready to move\\
  forward after 30 years of stagnation?}

\author{
  \begin{tabular}[t]{c@{\extracolsep{2cm}}c}
    Ant�nio Augusto Fr�hlich & Wolfgang Schr�der-Preikschat\\
    \small{GMD-FIRST}&\small{University of Magdeburg}\\[-0.5ex]
    \small{Kekul�stra�e 7}&\small{Universit�tsplatz 2}\\[-0.5ex]
    \small{12489 Berlin, Germany}&\small{39106 Magdeburg, Germany}\\[-0.5ex]
    \small\texttt{guto@first.gmd.de}&\small\texttt{wosch@ivs.cs.uni-magdeburg.de}
  \end{tabular}
} 

\date{}

\begin{document}

\maketitle
\thispagestyle{empty}
\pagestyle{empty}

\section*{\protect\centering Abstract}

If there are areas of computer science that were left behind by the
market, operating systems is certainly one of them. The products
currently available in the area, namely Microsoft \textsc{Windows} and
\textsc{Unix}, owe their designs to projects such as
\textsc{The}~\cite{Dijkstra:1968} and
\textsc{Multics}~\cite{Organick:1972}, which are at least 30 years old.
In the meanwhile, several new "hacks" have been proposed, but very few
systems brought about revolutionary designs. The partitioning of the
problem domain in abstractions like \texttt{process} and \texttt{file}
is so old that some people regard it as a "canonical" one. The scene is
even worse if one takes in consideration that only 2\% of the
microprocessors produced in the year 2000 targeted the interactive
market~\cite{Tennenhouse:2000}, for which those operating systems have
been designed. The software industry spent 30 years saying that there
are not many things to be improved in the operating system
area~\cite{Pike:2000}, but left 98\% of market without a choice!

Innumerable software engineering techniques have been proposed for the
development of applicative software, many of them are now mature and
widely used in production. Nevertheless, not many operating system
developers have tried to deploy them, or to adapt them, to the
development of their products. We believe that several of these
techniques can be successfully deployed in the area of operating systems
with minor adjusts, as long as we reformulate our view of the
corresponding domain. Of course we are dealing with a very special
field. A field where words such as asynchronism and determinism have
very special meanings. A field pressed to squeeze the last bit of
performance and to blow overhead away without relaxing on correctness.
Anyway, by now we should be ready to make it up with software
engineering and move forward with operating system design.

We developed a novel operating system design method that addresses many
of the questions raised above. Deeply influenced by
\emph{object-oriented design}, but also by \emph{family-based design},
\emph{collaboration-based design}, \emph{aspect-oriented programming}
and \emph{generative programming}, our method enables the development of
operating systems as an assemblage of reusable and adaptable components.
Our method produces systems that can be tailored to fulfill the
requirements of any particular application, without disregarding any of
the quality metrics compulsory to the field. Indeed, applications play
such a major role that we decided to name the method
\emph{application-oriented system design}.

In summary, our method conducts the partitioning of the problem domain
in \emph{scenario-independent, appli\-ca\-tion-ready abstractions} that
will shape the components of the resulting system. Because these
abstractions know very little about the execution scenario they will
join, they can be adapted to join several scenarios. When performing a
scenario, abstractions are wrapped by \emph{scenario-adapters}, which
know details of both scenario and abstractions. In order to avoid
overloading users with system-level decisions, all implementations of an
abstraction are made visible through the same \emph{inflated interface}.
If the system is proper designed, tools can automatically select the
best implementation via syntactical analysis of the application's source
code.

This combination of objects, collaborations, and aspects may sound
low-performing. However, our method proposes abstractions to be arranged
in a \emph{statically metaprogrammed component framework}, so that
compositions are carried out during compilation, resulting in virtually
no run-time overhead. Such frameworks define the relationships between
abstractions in terms of scenario-adapters, hence capturing a reusable
system architecture. If the target application does not need this or
that abstraction, the corresponding scenario-adapters will contribute
for a scenario free of them. In this way, each application gets exactly
the operating system it needs. Yet, system developers do not need to
redesign or reimplement the system several times, they can play with
components, adapters and frameworks to deliver a vast range of
application-oriented systems.
 
We deployed this design method in the \textsc{Embedded Parallel
  Operating System} (EPOS) project under development at
GMD-FIRST~\cite{Froehlich:sbac:1999}. The current implementation targets
parallel applications running in a cluster of PCs interconnected through
a \textsc{Myrinet} high-speed network. Although the number of components
in the repository is still small, and the tools are relatively
primitive, EPOS first results are very encouraging, not only in terms of
performance, but of quality in general. As far as we are concerned,
application-to-application communication over \textsc{Myrinet} in EPOS
has the best performance ever reported~\cite{Froehlich:hpcn:2000}.
Besides continuing the development for the cluster domain, we will soon
deploy application-oriented system design to the embedded systems
domain.

\bibliographystyle{plain}
\bibliography{se,os,guto}

\end{document}

%%% Local Variables: 
%%% mode: latex
%%% TeX-master: t
%%% End: 
