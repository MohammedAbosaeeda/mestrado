\documentclass[12pt]{article}
\usepackage{sbc-template}

\usepackage{graphicx,url}
\usepackage[portuges]{babel} % for multilingual support
\usepackage[utf8]{inputenc}   % for Latin languages

\usepackage{appendix}
\usepackage{listings}
\lstset{keywordstyle=\bfseries, flexiblecolumns=true}
\lstloadlanguages{C,[ANSI]C++,HTML}
\lstdefinestyle{prg}{
  basicstyle=\small\sffamily,
%  lineskip=-0.2ex,
  showspaces=false
}

\newcommand{\fig}[4][ht]{
  \begin{figure}[#1]
    {\centering{\includegraphics[#4]{fig/#2}}\par}
    \caption{#3}
    \label{fig:#2}
  \end{figure}
}

\newcommand{\prg}[4][h]{
  \begin{figure}
    \begin{center}
      \makebox[\width]
    {\centering\lstinputlisting[language=#2,style=prg]{fig/#3.prg}\par}
      \caption{#4}\label{prg:#3}
    \end{center}
  \end{figure}
}

\usepackage{multirow}
%\setlength{\tabcolsep}{1mm}
\newcommand{\tab}[4][h]{
  \begin{table}
    {\centering\footnotesize\textsf{\input{fig/#2.tab}}\par}
    \caption{#3}\label{tab:#2}
  \end{table}
}

\newcommand{\lisha}{
\textsc{Lisha}
}
\newcommand{\epos}{
\textsc{Epos}
}
\newcommand{\emote}{
\textsc{EposMote}
}

\newcommand{\rssf}{
redes de sensores sem-fio
}

\newcommand{\Rssf}{
Redes de sensores sem-fio
}

\sloppy

\title{Redes de sensores sem-fio sob a perspectiva do EPOS\\
\normalsize{- - Minicurso - -}}

\author{Arliones Hoeller Jr \and Antônio Augusto Fröhlich}

\address{Universidade Federal de Santa Catarina\\
Laboratório de Integração Software/Hardware\\
Cx. Postal 476, 88040-900, Florianópolis-SC, Brasil\\
\{arliones,guto\}@lisha.ufsc.br
}

\begin{document}

\maketitle

\begin{abstract}
\Rssf e outros sistemas pervasivos são cada vez mais comuns em nosso dia a dia.
Esta tecnologia traz, junto das novas possibilidades de aplicações, um grande
conjunto de desafios, o que deu margem ao surgimento de novas soluções de
integração de componentes de processamento (CPU) e rádio, novos mecanismos de
comunicação, variadas implementações de controle de acesso ao meio (MAC), sempre
atendendo às rígidas restrições em termos de capacidade de processamento,
potência de transmissão e, de modo especial, consumo de energia. Para tratar
tudo isso, uma série de novos sistemas operacionais focou no domínio de \rssf,
abstraindo as funcionalidades de sensoriamento e comunicação, com o objetivo de
agilizar o processo de desenvolvimento de aplicações. Este minicurso visa
apresentar as principais características das \rssf focando em temas importantes
a serem considerados durante o desenvolvimento de aplicações, utilizando-se para
isso dos trabalhos desenvolvidos para implementação do
\epos~\cite{Froehlich:2001} e do \emote~\cite{Project:Emote:2010}.
\end{abstract}

\section{\Rssf}
\label{sec:rssf}

%introdução
Avanços recentes nos projetos de dispositivos eletrônicos e a miniaturização
levou ao surgimento de um novo conjunto de aplicações para computadores na forma
de microsensores sem-fio de baixa potência. Estes microsensores são equipados
com dispositivos de sensoriamento analógicos ou digitais (e.g., temperatura,
campo magnético, som), um processador digital, um transceptor de comunicação
sem-fio (e.g., rádio de baixa potência, infra-vermelho) e um módulo de
alimentação (e.g., bateria, célula foto-sensível). Cada sensor, individualmente,
é capaz de obter uma visão local de seu ambiente e de coordenar e se comunicar
com outros sensores para criar uma visão global do objeto de estudo alvo da
aplicação.

\fig{wsn}{\Rssf.}{width=\columnwidth}

A ideia de uma rede auto-gerenciada composta por dispositivos autônomos que
coletam e enviam dados através de um enlace sem-fio traz à tona uma série de
novos desafios ao projeto das plataformas (hardware). Para que não sejam
intrusivos e operem autonomamente por longos períodos de tempo, os nodos
sensores precisam ser pequenos e consumir pouca energia. Além disso, nodos
sensores precisam ser modulares e permitir uso de diferentes tipos de
dispositivos sensores para permitir que uma única plataforma possa ser empregada
nos diversos tipos de aplicações, podendo assim serem adaptados de acordo com as
especificidades de cada aplicação. De modo similar, o hardware de comunicação
deve permitir ampla configuração do canal de dados, possibilitando que
diferentes aplicações se beneficiem de diferentes esquemas de modulação ou de
controle de acesso ao meio (MAC). Com o aumento da complexidade das tecnologias
de \rssf a necessidade de software para suporte à execução composto por sistemas
operacionais e componentes abstratos de alto nível (e.g., middleware) se torna
essencial.

Neste contexto, desde o ano de 2003 o \lisha (Laboratório de Integração
Software/Hardware) vem trabalhando com \rssf. Trabalhos desenvolvidos no \lisha
no escopo do Projeto EPOS~\cite{Project:Emote:2010} desenvolveram suporte de
sistema operacional para a abstração dos mecanismos de:
\begin{itemize}
    \item aquisição de dados~\cite{Wanner:ETFA:2006};
    \item comunicação~\cite{Wanner:IESS:2007};
    \item gerência do consumo de energia~\cite{Hoeller:WSO:2006};
    \item outros serviços comuns de sistema operacional como alocação de
    memória e escalonamento de tarefas~\cite{Marcondes:ETFA:2006};     
\end{itemize}

Mais recentemente, o \lisha passou a dedicar esforços no desenvolvimento de
plataformas próprias de \rssf, o que culminou com o projeto dos módulos de
sensoriamento \emote I, baseado em uma arquitetura \textsc{Avr} de 8 bits, em
2009 e \emote II, baseado em uma arquitetura \textsc{Arm7} de 32 bits, em 2010.

Este curso prevê uma revisão dos aspectos das tecnologias de \rssf, buscando
descrever as tecnologias e apresentar como estas foram tratadas tanto na
implementação do sistema operacional \epos quanto no projeto dos módulos de
sensoriamento \emote I e II. Para isso, esta primeira seção apresenta uma
contextualização da tecnologia seguido de exemplos de aplicações das \rssf. A
seção~\ref{sec:modulos} apresenta requisitos comumente procurados em módulos de
\rssf e apresenta alternativas de módulos disponíveis. A seção~\ref{sec:sensores}
apresenta tipos de sensores, métodos de interfaceamento e técnicas de aquisição
de dados. A seção~\ref{sec:MAC} revisa técnicas de controle de acesso ao meio,
apresenta topologias e descreve alguns dos MACs mais utilizados em \rssf. A
seção~\ref{sec:so} apresenta características de sistemas operacionais para
\rssf. A seção~\ref{sec:exercicios} lista alguns exercícios para utilização do
\epos na plataforma \emote.

%aplicações
\subsection{As aplicações de \rssf}

Novas tecnologias, sistemas ou plataformas surgem, quase que exclusivamente, por
um único motivo: atender a uma determinada demanda, ou seja, uma aplicação! No
caso das \rssf não poderia ser diferente. A criação das \rssf foi motivada por
aplicações militares como vigilância de campos de batalha e são hoje utilizados
em diversas áreas de aplicações civis e industriais, incluindo monitoramento e
controle de processos industriais, supervisão de maquinário, monitoramento de
ambientes ou habitats, automação doméstica, entre outras.

Esta variada gama de aplicações implica na existência de grande variação no
conjunto de requisitos que uma \rssf precisa atender, incluindo requisitos
possivelmente conflitantes, ou seja, características das quais um determinado
tipo de aplicação se beneficia, eventualmente, pode tornar o uso da tecnologia
proibitiva a outra aplicação. Para dar conta desta variabilidade, \rssf precisam
ser amplamente configuráveis. E, para ser configurável, um sistema de \rssf
precisa levar em consideração as características específicas dos diferentes
tipos de aplicações existentes.

\fig{mottola_taxonomy}{Uma taxonomia para aplicações de
\rssf~\cite{Mottola:2010}}{width=\columnwidth}

\tab{mottola_classification}{Exemplos de aplicações classificadas segundo a
taxonomia proposta por Mottola~\cite{Mottola:2010}.}

Motolla e Picco~\cite{Mottola:2010} publicaram um excelente estudo em que,
dentre outras coisas, classificaram diferentes aplicações de \rssf com o
objetivo de destacar as diferenças entre estas aplicações. A
Figura~\ref{fig:mottola_taxonomy} apresenta uma taxonomia pela qual as
características das aplicações podem ser classificadas. Neste trabalho, os
autores ainda analisam uma série de aplicações publicadas em canais científicos,
classificando-as segundo a taxonomia proposta. Alguns exemplos destas
classificações estão na Tabela~\ref{tab:mottola_classification}. Cada uma das
características da taxonomia por eles proposta podem ser assim interpretadas:
\begin{itemize}
    \item \textit{Objetivo:} dependendo do objetivo das redes, isto é, apenas
    monitorar um objeto de estudo ou, além de monitorar, também atuar sobre este
    objeto, a topologia empregada pode sofrer modificações, especialmente devido
    à inclusão, neste último tipo de objetivo, de módulos atuadores capazes de
    interagir com o objeto de estudo.
    \item \textit{Padrão de interação:} é, normalmente, dependente do
    \textit{objetivo}. ``Muitos-para-um'', mais comumente empregado, é utilizado
    quando dados de diversos nodos são coletados por um nodo central.
    Comunicação ``um-para-muitos'' é normalmente utilizada para envio de
    comandos de configuração nas redes, e ``muitos-para-muitos'' é mais comum em
    situações onde há múltiplos concentradores de informação, o que geralmente
    ocorre na presença de atuadores.
    \item \textit{Mobilidade:} em configurações ``estáticas'' nenhum elemento da
    rede se move após a implantação. Em configurações com ``nodos móveis''
    alguns elementos das redes são móveis, como, por exemplo, em aplicações de
    monitoramento de habitats em que alguns nodos estão implantados em animais.
    Já em configurações com ``concentradores móveis'', normalmente, são
    indiferentes quanto à mobilidade dos demais nodos já que, neste caso, a
    diferença básica reside no fato de que a coleta de dados é realizada de modo
    oportunístico quando os concentradores se aproximam dos demais nodos.
    \item \textit{Espaço:} diz respeito à semântica dos dados recolhidos.
    Configurações ``globais'' são aquelas em que dados de sensores individuais
    não são úteis, sendo úteis apenas as informações extraídas da análise das
    medições realizadas por todos os sensores em uma rede. Já configurações
    ``regionais'' são aquelas o objeto de interesse está localizado em uma
    região limitada, sendo, neste caso, leituras individuais relevantes.
    \item \textit{Tempo:} do ponto de vista temporal a operação dos nodos em uma
    \rssf pode ser classificada em ``periódica'', quando nodos realizam
    leituras de seus sensores e enviam os dados para processamento na rede
    periodicamente, ou ``orientada a eventos'', quando nodos sensores permanecem
    em modo quiescente observando os dados lidos por seus sensores e enviam
    dados na rede apenas quando um determinado evento é detectado (e.g., o
    valor lido em um determinado sensor ultrapassa um limite pré-definido).
\end{itemize}

\section{Módulos de sensoriamento}
\label{sec:modulos}

%introdução
Este capítulo aborda a arquitetura dos módulos de sensoriamento. Será discutida
a arquitetura básica, comumente composta de um conjunto de sensores, um
processador e um transceptor de rádio~\cite{Barr:2002,Pottie:2000}. Serão
apresentadas abordagens comerciais recentes de integração destes componentes
em \emph{single-package} ou em \emph{single-die}. Também serão discutidos os
requisitos buscados em um módulo de sensoriamento em termos de dimensões,
consumo de energia, modularidade e adaptabilidade do canal de comunicação.

%arquitetura dos módulos de sensoriamento (sensores + processador + transceptor)
\subsection{Arquitetura de módulos de sensoriamento}

Em\rssf, diversos nodos sensores, compostos por um conjunto de sensores
analógicos e digitais, um microcontrolador, um transceptor sem-fios e bateria,
coordenam-se e trocam informações de maneira a prover uma visão global de um
dado objeto de estudo. Cada nodo individual possui capacidade limitada, mas a
comunicação e processamento cooperativo na rede permitem a obtenção de dados
mais precisos. Com base na pesquisa e aplicações atuais, e possível definir que
a arquitetura básica de um nodo de sensor, composta por um microcontrolador e a
um transceptor sem-fios deve~\cite{Wanner:MSC:2006}:
\begin{itemize}
    \item Ter dimensões físicas reduzidas.\\
    Para poderem ser instalados de maneira não intrusiva, os nodos sensores
    devem ter dimensões reduzidas. Dado o constante avanço das técnicas de
    miniaturização de hardware, o tamanho dos componentes eletrônicos utilizados
    nos nodos tende a diminuir constantemente. Entretanto, a miniaturização dos
    nodos sensores pode estar limitada ao tamanho da fonte de energia (seja na
    forma de baterias ou dispositivos para captura de energia ambiente).

    \item Ser capaz de operar por um longo tempo com quantidade limitada de
    energia.\\
    A necessidade de operação autônoma de um nodo sensor, e a capacidade
    limitada de energia disponível ao mesmo, fazem com que o baixo consumo de
    energia seja um fator determinante no projeto de hardware. Sendo assim, o
    projeto de um nodo sensor priorizará componentes de baixa potência e com
    suporte a gerencia do consumo de energia (e.g., microcontroladores,
    transceptores de baixa potência) em detrimento de componentes direcionados a
    alta capacidade de processamento, desempenho ou potência.

    \item Ter um projeto modular, permitindo a conexão com sensores específicos
    para diferentes aplicações.\\
    Os serviços de uma rede de sensores tendem a ser específicos, e utilizar
    somente o hardware necessário aos requisitos de cada aplicação. Desta forma,
    é importante que o projeto seja modular, e permita a remoção e inclusão de
    sensores conforme as necessidades da aplicação.

    \item Permitir a mais ampla configuração possível do canal de transmissão de
    dados.\\
    O transceptor de dados sem-fios é, em geral, o componente com maior consumo
    de energia em um nodo sensor. Desta forma, é importante que este transceptor
    passe a maior parte do tempo desligado. Por outro lado, aplicações
    específicas terão padrões de comunicação específicos, e poderão se beneficiar
    de diferentes técnicas de modulação de dados e controle de acesso ao meio,
    que permitam o controle do consumo de energia sem comprometer a comunicação
    de dados. Desta forma, o transceptor deve permitir a maior configuração do
    canal de dados possível.

\end{itemize}

%abordagens de integração
Para atender a esta série de requisitos, módulos de sensoriamento focaram no uso
de componentes de baixa potência de pequeno tamanho. É o caso, por exemplo, da
família módulos de sensoriamento Mica da Crossbow, Inc.~\cite{Crossbow:MTS:2005}.
Estes \emph{motes} (como o da Figura~\ref{fig:mica_mote}) são baseados em
processadores \textsc{Avr}, da Atmel, Inc., que são processadores RISC de baixa
potência. Além dos processadores \textsc{Avr}, estes dispositivos apresentam
variados modelos de transceptores de rádio, como o caso do CC1000 no Mica2 e do
CC2400 no MicaZ, ambos transceptores do fabricante Chipcom, Inc. O CC1000 é um
dispositivo de rádio que opera com modulação FM (\textit{Frequency Modulation})
nas faixas de frequência 315, 433, 868 e 915 MHz. Este dispositivo implementa
apenas a camada física de um transceptor FM, sendo que a camada de enlace (MAC e
LLC) são implementadas em software. A popularização do uso deste tipo de
transceptor deu origem a uma série de protocolos de MAC criados para dar melhor
suporte a certas categorias de aplicação. Já o CC2400 implementa as camadas
física e de enlace (MAC) do padrão IEEE 802.15.4. Este padrão, embora diminua a
flexibilidade no uso da camada física, permitiu o desenvolvimento de abstrações
de mais alto nível estáveis, já que os desenvolvedores passaram a trabalhar
sobre versões estáveis das camadas 1 (física) e 2 (enlace) da pilha de
comunicação. É exemplo destes desenvolvimentos o consórcio $ZigBee^{TM}$, que
implementa uma série de protocolos das camadas de 3 (rede) e 4 (transporte) para
uso em dispositivos das chamadas PAN (\textit{Personal Area Network}). Embora
descrever a tecnologia $ZigBee^{TM}$ não seja objetivo deste minicurso, os
diferentes MACs utilizados em \rssf, incluindo o IEEE 802.15.4, são melhores
descritos na seção~\ref{sec:MAC}.

\fig{mica_mote}{Mica2 mote da Crossbow Inc.}{scale=.7}

\tab{micaz_x_zigbit_x_mc13224v}{Comparação de características do MicaZ, ZigBit e
MC13224V. Todos os três apresentam um rádio compatível com IEEE 802.15.4.}

Para atender as demandas por maior poder de processamento e maior potência de
rádio em plataformas com menores dimensões e menor consumo de energia, os
projetos modulares caminharam para a integração de microcontrolador, transceptor
e outros componentes como antena e reguladores de tensão em uma abordagem
\emph{single-package}. Como exemplo pode ser citado o projeto $ZigBit^{TM}$ da
MeshNetics~\cite{Meshnetics:ZigBit:2007}. Buscando um desempenho ainda maior,
abordagens \emph{single-die} permitem a integração de microcontrolador, rádio, e
diversos outros dispositivos em um único circuito integrado, como, por exemplo,
os dispositivos da família MC1322X da Freescale, ou os dispositivos STM32W da
STMicroelectronics. A Tabela~\ref{tab:micaz_x_zigbit_x_mc13224v} mostra como
dispositivos que representam cada uma das abordagens apresentam melhor
desempenho junto com melhores características de tamanho e consumo de energia.

%requisitos buscados em um módulo de sensoriamento
\subsection{A arquitetura do \emote}
\label{sec:emote_arch}

O projeto \emote tem por objetivo desenvolver uma família de módulos de
sensoriamento que permita ampla configurabilidade tanto da plataforma quanto do
ambiente de software (sistema operacional). O \emote, apresentado na
Figura~\ref{fig:emote-modulos}, foi concebido com um projeto modular, sendo
previstos três módulos, entre os quais foram estabelecidas interfaces padrão,
permitindo o uso intercambiável de diferentes versões dos módulos. A
Figura~\ref{fig:emote2-block_diagram} apresenta os três tipos de módulos, que
são os seguintes:
\fig{emote-modulos}{O \emote ao lado de uma moeda de uma libra.}{scale=.15}
\begin{itemize}
    \item \textbf{Módulo de Base:} o módulo base incorpora as funcionalidades de
    processamento e de comunicação. O projeto \emote desenvolveu duas versões
    deste módulo, uma utilizando o \emph{single-package} $ZigBit^{TM}$ e outra
    utilizando o \textsc{SoC} (\emph{single-die}) MC13224V, da Freescale.
    Características específicas de cada versão do módulo base são apresentadas
    abaixo. Este módulo deve implementar detalhes de do suporte a estes
    dispositivos, como a regulação de tensão e o dimensionamento da antena, além
    rotear os pinos dos dispositivos de modo a manter o padrão das interfaces de
    alimentação e de entrada e saída.
    \item \textbf{Módulo de Entrada e Saída:} no módulo de entrada e saída devem
    ser implementadas as interfaces necessárias de entrada e saída, podendo um
    novo módulo destes ser desenvolvido para cada aplicação que se pretende
    desenvolver, permitindo o emprego dos sensores ou atuadores desejados para
    uma aplicação específica. O projeto \emote desenvolveu um módulo de entrada
    e saída ao qual deu o nome de \textit{start-up board}. Esta placa incorpora
    uma interface USB, sensor de temperatura, acelerômetro de 3 eixos, alguns
    LEDs e botões~\cite{Project:Emote:2010}.
    \item \textbf{Módulo de Alimentação:} de modo a permitir o emprego de
    diferentes fontes de alimentação, uma interface de alimentação foi
    implementada. Módulos que se conectam a esta interface podem ser tão simples
    como uma bateria alcalina AA, ou tão complexas quanto um sistema com bateria
    de lítio recarregável ou com painéis solares. A interface de alimentação
    ainda disponibiliza uma interface I2C, permitindo que o módulo de
    alimentação se comunique com o de processamento. O projeto \emote ainda não
    desenvolveu nenhum módulo de alimentação específico, mas trabalhos em
    andamento estão explorando tecnologias de captação de energia utilizando
    esta interface.
\end{itemize}

\fig{emote2-block_diagram}{Arquitetura básica do \emote}{scale=.45}

As interfaces padrão definidas pelo projeto \emote para interconexão dos
módulos desenvolvidos são as seguintes:
\begin{itemize}
    \item \textbf{Interface de Entrada e Saída:} 32 pinos, sendo 2 para
    alimentação e outros 30 que podem ser utilizados ou como GPIO, ou com
    funções específicas que inclui ADC, UART e SPI.
    \item \textbf{Interface de Alimentação:} disponibiliza pinos aos quais devem
    ser conectados o terra e alimentação do módulo de alimentação. O módulo de
    processamento devolve ao módulo de alimentação o sinal com tensão regulada.
    Também existem os 2 pinos empregados na comunicação I2C.
\end{itemize}

\section{Sensores e aquisição de dados}
\label{sec:sensores}

%introdução
Um sensor é um dispositivo que responde a estímulos físicos (e.g., luz,
temperatura, pressão, campo magnético ou movimento) e transmite um impulso
resultante. Um sensor normalmente interage com um sistema digital, provendo
informações sobre o mundo analógico. Essa informação pode ser fornecida através
de um sinal analógico, que é convertido para valores digitais através de um
conversor analógico-digital externo ao sensor, ou através de uma interface
digital que converte internamente os sinais analógicos. Interfaces de dados de
sensores podem variar amplamente mesmo dentro da mesma classe de
sensores~\cite{Wanner:MSC:2006}.

%tipos de sensores
\subsection{Tipos de sensores}

Devido ao pequeno porte e às suas limitações em termos de disponibilidade de
energia, módulos de \rssf normalmente empregam sensores eletrônicos ou óticos
com relativa eficiência energética. É o caso de, por exemplo, termistores para
medir temperatura, foto-didos para medir luminosidade ou acelerômetros
para medir deslocamento.

De maneira a possibilitar um melhor entendimento dos desafios envolvidos no
projeto e implementação de um sistema de aquisição de dados de sensores, esta
seção apresenta alguns dispositivos sensores usados em nodos de sensor
contemporâneos. Esta não é uma lista exaustiva, mas deve prover um caso geral de
uso de dispositivos sensores em \rssf.

\subsubsection{Termistores}

Um termistor é um resistor cuja resistência varia com mudanças de temperatura. A
equação de Steinhart-Hart é uma aproximação de terceira ordem amplamente
utilizada para determinar a curva de resposta de um termistor:
\begin{equation}
\label{eq:steinhart_hart}
T = \frac{1}{a + b.ln{R_t} + c.(ln{R_t})^3}
\end{equation}
onde $a$, $b$, e $c$ são parâmetros Steinhart-Hart específicos para cada
termistor, $T$ é a temperatura Kelvin, e $R_t$ é a resistência em Ohms
apresentada pelo termistor na temperatura atual. Um termistor normalmente é
ligado a um conversor analógico-digital através de um circuito divisor de tensão
simples. A estimativa de temperatura baseada em leituras do ADC pode depender do
cálculo em tempo de execução de funções de aproximação complexas (e.g., a
equação Steinhart-Hart), ou pode fazer uso de tabelas de conversão previamente
calculadas. Termistores diferentes podem ter constantes de tempo e precisão
diferentes, bem como diferentes constantes de dissipação de energia.

\subsubsection{Sensores digitais de temperatura}

A família SHT1x~\cite{Sensirion:SHT1x:2005} de sensores de umidade e temperatura provê
um exemplo de sensores digitais de temperatura usados em \rssf. O sensor é
fabricado pela Sensirion, e provê leituras digitais calibradas por uma interface
SPI de dois fios. Um microcontrolador pode e ler dados do sensor enviando um
comando específico solicitando uma medida temperatura para o sensor pela
interface SPI. Ao completar da leitura, o sensor envia um sinal de dados
prontos. O microcontrolador pode, então, ler os dados acompanhados de um código
CRC para validação. Depois dessa transmissão, o sensor entra em modo inativo. Os
coeficientes de calibração são programados na memória interna do sensor, e são
usados internamente durante as leituras para calibrar os sinais lidos. Dados
sensoriais providos pelo SHT1x podem ser convertidos para valores de temperatura
através de uma e função linear. Um registrador de status provê uma interface de
detecção de baixa tensão, configura a resolução da leitura (por exemplo, 8 bits,
16 bits) e controla um aquecedor interno.

\subsubsection{Foto-resistores e foto-diodos}

Um foto-resistor é um componente eletrônico cuja resistência diminui com o
aumento da intensidade de luz incidente. A maioria dos foto-resistores são
implementados através de células de sulfeto de cádmio, que usam a habilidade
desse material de variar sua resistência (por exemplo, apresentando $2k\Omega$
em condições de baixa luminosidade e $500\Omega$ quando exposto à luz). Essas
células também são capazes de reagir a uma ampla gama de frequências, incluindo
infravermelho, luz visível ultravioleta. Como no caso dos termistores,
foto-resistores normalmente fazem interface com um conversor analógico-digital
através de um circuito divisor de tensão.

Um foto-diodo é um semicondutor que responde a estímulo óptico. Foto-diodos
operam pela absorção de fótons que geram uma variação na corrente que flui
através deles. Um circuito RC (resistor-capacitor) auxiliar tem sua fonte de
corrente afetada por este sensor, o que permite a detecção da presença ou
ausência de quantidades diminutas de luz através de medidas de tensão no
resistor desse circuito auxiliar.

\subsubsection{Sensores digitais de luminosidade}

O sensor TSL2550~\cite{TAOS:TSL:2005}, fabricado pela TAOS, provê um exemplo de
sensor digital de luminosidade usado em \rssf. Ele combina dois foto-diodos e um
conversor analógico-digital de 12 bits num circuito integrado para prover
medidas de luz com uma sensitividade parecida com a do olho humano. Um dos
foto-diodos é sensível a luz visível e infravermelha, enquanto o segundo
foto-diodo é sensível primariamente a luz infravermelha.

Dados de sensoriamento são lidos através de uma interface de dois fios SMBus. Um
registrador de controle gere o dispositivo, e dois registradores de ADC
armazenam os dados de para leitura. As saídas dos dois canais ADC podem ser
usadas em uma função linear para se obter um valor que aproxima a resposta do
olho humano na unidade comumente utilizada de Lux.

\subsubsection{Magnetômetros}

Magnetômetros são sensores capazes de medir força e/ou direção de campos
magnéticos. Magnetômetros podem ser divididos em magnetômetros escalares, que
medem a força total do campo magnético ao qual eles são expostos, e
magnetômetros de vetor, que têm a capacidade de medir a componente do campo
magnético em determinada direção. Um conjunto de magnetômetros de vetor podem
ser combinados para permitir a definição de força, declinação e inclinação de um
campo magnético. A família Honeywell HMC100x~\cite{Honeywell:HMC:2005} de sensores
magnetoresistivos são dispositivos de pontes resistivas simples que requerem
apenas uma fonte de tensão para medir campos a magnéticos. Esses dispositivos
são capazes de medir qualquer campo magnético ambiente ou aplicado em um eixo
sensível. A tensão de saída do sensor é linear em relação ao campo magnético
aplicado.

\subsubsection{Acelerômetros}

Acelerômetros são usados para medir mudanças na velocidade. Na sua forma mais
simples, um acelerômetro é composto por uma massa suspensa e um dispositivo
sensível à deflexão. A família Analog Devices
ADXL345BCCZ-RL7~\cite{Analog:ADXL345:2009} provê um exemplo de acelerômetros com
2 eixos e baixo consumo de energia. O sensor provê tanto saídas analógicas
quanto digitais. O valor de saída do sensor é linear em relação à aceleração
aplicada, e calibração para baixas gravidades podem usar o campo gravitacional
da Terra como referência.

%aquisição de dados
\subsection{Calibração de sensores}

Dados lidos diretamente de sensores precisam ser calibrados para garantir um
determinado nível de qualidade. Imprecisões nas medições podem surgir por
diversas razões~\cite{Albertazzi:2008}:
\begin{itemize}
    \item \textbf{Definição do mensurando:} diz respeito ao conhecimento
    disponível acerca da grandeza física que é objeto de estudo.
    \item \textbf{Procedimento de medição:}  diz respeito aos procedimentos
    utilizados para realizar a medição. Neste caso se aplica, por exemplo, a
    precisão dos modelos matemáticos sendo utilizados para a medição de uma
    determinada grandeza.
    \item \textbf{Condições ambientais:} diz respeito a condições que podem
    interferir no resultado de uma medição, como por exemplo, interferência
    eletromagnética ou de pressão atmosférica sobre o sistema de medição.
    \item \textbf{Sistema de medição:} diz respeito a interferência gerada no
    resultado de uma medição pelos equipamentos utilizados no processo. Esta
    interferência pode vir, por exemplo, da incerteza do sensor utilizado ou
    erros de arredondamento na conversão analógico-digital.
    \item \textbf{Operador:} diz respeito a erros que possam ser ocasionados
    pela ação humana quando o procedimento de medição depende da ação de um
    operador. Normalmente, este fator tem pouco efeito em aplicações de \rssf,
    já que estas aplicações tendem a operar de modo autônomo.
\end{itemize}

Estes fatores agem sobre o sistema formado pelo mensurando (objeto de estudo) e
pelo sistema de medição que, no caso das \rssf, é formado pelo sensor escolhido,
pelo conversor analógico-digital ou pelo microcontrolador no caso de um sensor
digital. O que resulta destas interferências é uma variação no valor indicado.
Para tornar as leituras mais confiáveis é necessário, de algum modo, compensar o
efeito destas interferências.

As interferências geradas por estes agentes no processo de medição podem ser
agrupados de modo a formar duas variáveis de erro: o erro sistemático e o erro
aleatório. O erro sistemático é aquela parcela do erro que se mantém constante
entre várias medições. Para permitir uma correção das medidas, é possível
definir uma estimativa do erro sistemático chamada de \emph{Tendência}. A
\emph{Tendência} pode ser determinada através de medições sucessivas de um
mensurando com valores conhecidos. Por exemplo, para calibrar um sensor de
temperatura, pode-se realizar leituras sucessivas deste sensor em um ambiente
com temperatura constante conhecida. A \emph{Tendência} é dada, neste caso, pela
diferença entre a média das leituras realizadas e o valor do mensurando. Este
procedimento é conhecido calibração do sensor.

Não há meio de se estimar o valor exato do erro aleatório, mas é possível,
através de tratamento estatísticos de uma série de amostras realizadas sobre um
mensurando com valores conhecidos, determinar a repetitividade do sistema de
medição e, consequentemente, seu erro máximo. Do ponto de vista de \rssf é
importante analisar o erro máximo do sistema definido para verificar sua
adequação à aplicação que se pretende desenvolver. Outra característica
importante do erro aleatório é que a soma dos erros aleatórios de medidas
sucessivas de um mesmo mensurando tende a se anular no infinito. Logo, a média
de sucessivas medições pode diminuir o efeito deste erro, gerando resultados
mais confiáveis.

%modelos de interfaceamento destes sensores com processadores
% \subsection{Interfaces de senores}
% interfaces analógicas e digitais

%fusão de dados
%\subsection{Fusão de dados}

\subsection{Sensores no \emote}

Como apresentado na Seção~\ref{sec:emote_arch}, o \emote apresenta uma interface
padrão para um módulo de entrada e saída que deve ser utilizada para conectar os
sensores a serem utilizados pela aplicação em questão. No atual estágio de
desenvolvimento, o \emote conta com a \emph{startup board}, um módulo de entrada
e saída que, além de componentes como interface USB, botões e LEDs, conta com um
sensor de temperatura do tipo termistor, modelo
ERT-J1VG103FA~\cite{Panasonic:ERTJ:2004}, e um acelerômetro digital de 3 eixos,
modelo ADXL345BCCZ-RL7~\cite{Analog:ADXL345:2009}. A fim de exemplificar o
processo de calibração de um sensor, vamos aqui demonstrar como foi realizada a
calibração do sensor de temperatura da \emph{startup board} do \emote.

O mediador\footnote{Mediador é o artefato de software que realiza interface do
sistema operacional com o hardware no Projeto \epos} do sensor de temperatura foi
implementado utilizando a Equação de Steinhart-Hart~(\ref{eq:steinhart_hart}).
Os valores dos parâmetros de Steinhard-Hart podem ser
calculados~\cite{Steinhart:1968} a partir de dados fornecidos pelo fabricante
do termistor, e são substituídos na Equação~\ref{eq:steinhart_hart} de modo a
montar o modelo matemático do sensor em uso:
\begin{eqnarray}
%T = \frac{1}{a + b.ln{R} + c.(ln{R})^3}\\
\label{eq:stnhrt_a}a = 1,0750492 \times 10^{-3}\\
\label{eq:stnhrt_b}b = 0.27028218 \times 10^{-3}\\
\label{eq:stnhrt_c}c = 0.14524838 \times 10^{-6}
\end{eqnarray}

Para efetuar este cálculo, é necessário conhecer a resistência apresentada pelo
termistor. Como o termistor está conectado ao ADC por um circuito divisor de
tensão a resistência pode ser calculada da seguinte forma:
\begin{eqnarray}
V_s = I.(R+R_t)\\
V_o = I.R\\
com\;I\;constante\;no\;divisor,\;logo,\; \frac{V_s}{R+R_t} =
\frac{V_o}{R}\\ \label{eq:r_t_value}isolando\;R_t,\; R_t = R.\left ( \frac{V_s}{V_o} - 1 \right)
\end{eqnarray}
onde $V_s$ é a tensão de entrada do circuito de divisão de tensão, $I$ é a
corrente que passa pelas resistências (igual tanto sobre o resistor do divisor
de tensão quanto sobre o termistor), $R$ é a resistência conhecida do resistor
do divisor de tensão, $V_o$ é a tensão de saída do divisor, que é lido pelo
conversor analógico-digital, e $R_t$ é o valor atual da resistência do
termistor, que se deseja obter.

Ao final, substituindo o valor da resistência atual do termistor definido pela
Equação~(\ref{eq:r_t_value}), e os parâmetros de Steinhart-Hart definidos
em~(\ref{eq:stnhrt_a}),~(\ref{eq:stnhrt_b}) e~(\ref{eq:stnhrt_c}) na
Equação~(\ref{eq:steinhart_hart}) (Steinhart-Hart), e ainda, sabendo que o valor
do resistor utilizado no circuito divisor de tensão é de $10\;k\Omega$,
podemos chegar à equação para o cálculo da temperatura amostrada pelo termistor
estudado.
% Primeiro, vamos simplificar o termo que calcula o logaritmo natural da
% resistência do termistor:
% \begin{eqnarray}
% ln\;R_t = ln \left [ R \times \left ( \frac{V_s}{V_o} - 1 \right ) \right ]\\
% ln\;R_t =\;ln\;R + ln \left( \frac{V_s}{V_o} - 1 \right)\\
% com\;R\;conhecido,\;logo,\;ln\;R_t\;=\;ln\;10^4 + ln \left( \frac{V_s}{V_o} - 1
% \right)\\ ln\;R_t =\;2,302585093 + ln \left( \frac{V_s}{V_o} - 1 \right)\\
% considerando,\; L_{R_t} = ln\;R_t\\
% \nonumber temos,\\
% T^{-1} = \left[
%   1,0750492 +
%   (0.27028218\,\times\,L_{R_t}) +
%   \left(0.14524838\times10^{-3} \times L_{R_t}^3\right)
% \right]\times10^{-3}
% T = \left\{
% \frac{1}{
% a +
% (b \times L_{R_t}) +
% [c \times (L_{R_t})^3]
% }
% \right\}
% \end{eqnarray}

A calibração do sensor de temperatura foi realizado por comparação com outro
sensor de temperatura, modelo SHT-11~\cite{Sensirion:SHT1x:2005}, previamente
calibrado. Ambos os sensores foram conectados a um \emote, e os dois \emote
foram colocados em uma caixa hermeticamente fechada. Os \emote utilizados
tiveram seus timers configurados igualmente, e foram conectados a uma mesma
fonte de alimentação. Eles foram programados para realizar 1000 medidas de
temperatura espaçadas de 10 segundos cada e armazená-las em sua memória interna.
Ao final das medições, as leituras foram enviadas por rádio para uma
estação-base, de onde a tendência do sistema foi extraída. A correção, definida
como o inverso da tendência ($C\,=\,-Td$), foi então inserida no sistema,
ficando a equação final de temperatura do termistor assim definida:
\begin{equation}
T = \left\{
\frac{1}{
a +
(b \times ln\,R_t) +
[c \times (ln\,R_t)^3]
}
\right\}
+ C
\end{equation}
com a indicação $T$ expressa em Kelvin. A implementação deste modelo, na
linguagem \textsc{C++}, para o sistema operacional \epos, é apresentada na
Figura~\ref{prg:steinhart_hart_epos}.

É importante destacar que as arquiteturas utilizadas no \emote, \textsc{Avr} e
\textsc{Arm7-TDMI-S}, não apresentam unidade de ponto flutuante (FPU), logo, o
sistema carece de software adicional para executar as operações em ponto fixo,
chegando a resultados semelhantes, porém com muito mais demanda de
processamento. Por isso, é comum utilizar, em algumas aplicações, versões de
mediadores para o termistor que utilizem tabelas pré-calculadas de conversão das
leituras do ADC para um valor equivalente de temperatura, ao custo, neste caso,
de um consumo extra de espaço de armazenamento. De qualquer modo, o processo de
calibração descrito acima continua sendo necessário, neste caso, para a
construção da tabela que ficará armazenada na memória do nodo.

\prg{C}{steinhart_hart_epos}{Código para implementação do termistor calibrado
em C++ para o \epos.}

\subsection{Abstrações de sensores no \epos}
\label{sec:epos_sensing}

Do ponto de vista do programador da aplicação, abstrair questões específicas
como o procedimento de calibração de sensores descrito acima é de grande
importância. Neste contexto, o \epos fornece suporte de sensoriamento às
aplicações através de uma interface de software/hardware que abstrai famílias de
sensores de forma uniforme~\cite{Wanner:ETFA:2006}. O sistema define classes de
dispositivos baseado em sua finalidade (e.g. medir aceleração ou temperatura), e
estabelece um substrato comum para cada classe. Para cada dispositivo são
armazenadas propriedades e parâmetros operacionais, de maneira similar ao TEDS
(\textit{Transducer Electronic Data Sheet}) do padrão IEEE 1451. Uma camada fina
de software adapta dispositivos individuais (e.g., a converte leituras de ADC em
valores contextualizados, aplica as correções) para adequá-lo às características
mínimas da sua classe de sensores. Desta forma, um termistor simples é exportado
para a aplicação exatamente do mesmo modo que um sensor de temperatura digital
complexo~\cite{Wanner:JCC:2008}.

\fig{sensor_classes}{Diagrama de classes das abstrações de
sensoriamento do \epos.}{scale=.1}

A Figura~\ref{fig:sensor_classes} apresenta um diagrama de classes com as
abstrações de sensoriamento do \epos. No subsistema de sensoriamento do \epos,
métodos comuns a todos dispositivos de sensoriamento são definidos pela
interface \texttt{Sensor\_Common}. O método \texttt{get()} provê leituras para
um único sensor em um único canal (i.e. habilita o dispositivo, espera os dados
estarem disponíveis, lê o sensor, desabilita o dispositivo e retorna a leitura
convertida em unidades físicas previamente determinadas). Os métodos
\texttt{enable()}, \texttt{disable()}, \texttt{data\_ready()} e
\texttt{get\_raw()} permitem que o sistema operacional e as aplicações realizem
controle de grão fino sobre leituras de sensores (e.g., realizar leituras
sequenciais, obter dados não convertidos de sensores). O método
\texttt{convert(int v)} pode ser utilizado para converter valores não
processados de sensores em unidades científicas ou de engenharia. O método
\texttt{calibrate()} executa calibragem específica para cada sensor.

Cada família de sensor pode especializar a interface \texttt{Sensor\_Common}
para abstrair adequadamente características específicas da família. A família
\texttt{Magnetometer} pode adicionar, por exemplo, métodos para realizar
leituras em diferentes eixos de sensibilidade. A família \texttt{Thermistor},
por outro lado, provavelmente não precisará estender a interface comum. Cada
família também define uma estrutura \texttt{Descriptor} específica, que define
campos como precisão, dados para calibração e unidades físicas. Cada dispositivo
sensor implementa uma das interfaces definidas, e preenche a estrutura
\texttt{Descriptor} da família com valores específicos do sensor. Valores padrão
de configuração para cada dispositivo (e.g., frequência, ganho, etc.) são
armazenados em uma estrutura de \emph{traits de configuração}.

Sempre que o sistema operacional ou uma aplicação precisam fazer referência a um
dispositivo de sensoriamento, estes podem utilizar um \emph{dispositivo
específico} e realizar operações específicas do dispositivo, ou utilizar a
\emph{classe do dispositivo}, e restringir-se às operações definidas por aquela
classe. Uma realização utilizando metaprogramação estática da classe do
dispositivo agrega os dispositivos disponíveis em uma configuração do sistema.

\tab{sensing_footprint}{Tamanhos de código e dados de componentes de
sensoriamento (em bytes).}

\tab{sampling_rates}{Taxas de amostragem dos sistemas de sensoriamento (em Hz).}

A Tabela~\ref{tab:sensing_footprint} apresenta características de tamanho de
código e dados do subsistema de sensoriamento do \epos em comparação com as estruturas
equivalentes dos sistemas TinyOS e MANTIS OS. A Tabela~\ref{tab:sampling_rates}
mostra as taxas máximas de amostragem possíveis no \epos, também em comparação
com os outros sistemas. O baixo sobrecusto e alta taxa de amostragem no \epos
são resultado direto do projeto do sistema, que minimiza dependências entre
componentes de sensoriamento e o resto do sistema. No \epos, um componente que
abstrai um sensor analógico normalmente depende apenas do conversor
analógico-digital da plataforma e do seu subsistema de I/O, que são abstraídos
por operadores metaprogramados \textit{inline}. Estes mecanismos envolvidos no
desenvolvimento do \epos serão melhores descritos na Seção~\ref{sec:epos}.

\section{Controle de acesso ao meio (MAC)}
\label{sec:MAC}

%introdução
Rádios de banda estreita suportam, tipicamente, esquemas simples de modulação,
deixando a cargo do software o controle da comunicação, o que incorre em maior
custo de processamento. Por outro lado, rádios de banda larga empregam técnicas
de modulação sofisticadas, como \emph{Direct-Sequence Spread Spectrum} (DSSS) e
\emph{Phase Shift Keying} (PSK), que são mais resistentes a ruído e
interferências, mas apresentam pouca flexibilidade e impõem sobrecustos em
termos de consumo de energia. Neste contexto, rádios de baixa potência se
apresentam como uma alternativa viável para comunicação sem-fio para sistemas
embarcados como \rssf.

Nesta seção faremos uma breve revisão de técnicas de controle de acesso ao meio
para redes sem-fio. Também serão apresentadas alternativas existentes de
implementações de MACs específicos para aplicações de \rssf.

%MACs para RSSF
\subsection{Protocolos de acesso ao meio}

\Rssf apresentam um canal de comunicação único e compartilhado em que duas ou
mais transmissões podem ocorrer simultaneamente, gerando interferência e
invalidando a comunicação. Para viabilizar a comunicação neste meio, um
\emph{protocolo de múltiplo acesso} se faz necessário. Este tipo de protocolo
implementa um algoritmo distribuído de controle de acesso ao meio (MAC -
\textit{Medium Access Control}) qeu determina como as estações compartilha o
canal, ou seja, determinam quando uma estação pode iniciar uma transmissão.

Abordagens de MAC para controle de acesso múltiplo podem, geralmente, ser
classificados em três classes:
\begin{itemize}
  \item \textbf{Particionamento de canal:} estas abordagens dividem o canal
  disponível e aloca uma parte do canal para uso exclusivo de um determinado
  nodo. Exemplos deste tipo de abordagem é o FDMA, que divide o canal em faixas
  de frequência, e o TDMA, que divide o canal em fatias de tempo.
  \item \textbf{Acesso aleatório:} estas abordagens permitem a ocorrência de
  colisões. Diferentes abordagens constituem diferentes modelos pelos quais é
  possível, ou não, identificar, recuperar, evitar, ou eliminar colisões.
  \item \textbf{Passagem de permissão:} nestes protocolos, o compartilhamento do
  canal é estritamente coordenado para evitar colisões. Normalmente implementam
  técnicas de passagem de ficha (\textit{token-passing}).
\end{itemize}

É comum que implementações de protocolos de controle de acesso ao meio utilizem
mais de uma destas técnicas em conjunto, permitindo uma melhor exploração do
canal disponível e oferecendo diferentes características de transmissão de dados
às aplicações que o utilizam.

\subsubsection{TDMA - Time Division Multiple Access}

No TDMA, dispositivos compartilham o mesmo canal de frequência pela divisão do
sinal em fatias de tempo (\emph{time slots}). Neste protocolo, cada nodo de uma
rede transmite, exclusivamente, dentro da fatia que foi previamente alocada a
ele. Esta tecnologia é utilizada, por exemplo, em sistemas 2G de telefonia
celular digital, como no GSM. O emprego de \emph{time slots} também é bastante
explorada em \rssf, especialmente em aplicações que querem evitar contenção de
dados e colisões. Uma abordagem TDMA é utilizada pelo MAC do padrão IEEE
802.15.4 para em seu período livre de contenção (CFP - \textit{Contention-Free
Period}).

Uma característica peculiar das abordagens TDMA diz respeito a necessidade de
manter a sincronização dos nodos na rede de modo a garantir que todos concordam
com as bordas das fatias de tempo do protocolo. No padrão IEEE 802.15.4 esta
sincronização é realizada através de uma técnica chamada \emph{beaconing}, em
que um nodo ``mestre'' emite, periodicamente, \emph{beacons}, pacotes especiais
que contêm a configuração do canal (e.g., quais fatias de tempo estão alocadas
para quais nodos), e também são utilizados como baliza para determinar a borda
de início de um período de transmissão no protocolo.

\subsubsection{FDMA - Frequency Division Multiple Access}

No FDMA, nodos compartilham um canal dividindo o espectro do canal em bandas de
frequência. Através desta divisão, cada banda de frequência pode ser alocado a
um único nodo, ou a um conjunto diferente de nodos, eliminando ou, ao menos,
diminuindo a ocorrência de contenção. A divisão do espectro de frequência em
vários canais é bastante utilizado em redes sem-fio, inclusive sendo seu uso
previsto nos padrões IEEE 802.11 (WiFi) e IEEE 802.15.4.

Um dos principais problemas relacionados ao FDMA é o \textit{crosstalk}. Este
fenômeno ocorre em algumas situações em que o sinal transmitido em uma banda de
frequência interfere no sinal de uma banda de frequência adjacente. Este
problema é geralmente causado por erros ou na emissão do sinal no transmissor ou
na filtragem no receptor, e geralmente é minimizado com o emprego de
dispositivos moduladores e demoduladores de melhor qualidade.

\subsubsection{ALOHA e \textit{slotted} ALOHA}

O protocolo ALOHA (também referenciado como ALOHA Puro, ou \textit{Pure ALOHA})
foi implementado como protocolo de controle de acesso ao meio da ALOHAnet, que
foi a primeira demonstração de uma rede sem-fio de dados~\cite{Abramson:1970}. O
protocolo original (Puro) é bem simples: se um nodo tem dados a enviar, ele
envia; se há uma colisão, o nodo tenta novamente no futuro. No ALOHA, portanto,
não há verificação de sinal ocupado. Outro problema do protocolo advém da
determinação de quanto tempo um nodo deve esperar antes de efetuar uma
retransmissão, ficando a qualidade do canal de comunicação dependente da
eficiência do algoritmo de \textit{backoff} utilizado.

Para reduzir o volume de colisões do ALOHA há uma variação chamada
\textit{slotted} ALOHA. Nesta variação, o tempo é dividido em fatias de tamanho
igual, que também deve equivaler ao tempo máximo de transmissão de um pacote na
rede. Nodos que desejam transmitir iniciam transmissão apenas no início de cada
fatia de tempo, garantindo que apenas ocorrerão colisões caso dois nodos tenham
dados prontos para enviar no início da fatia de tempo. Se um nodo inicia a
transmissão de uma pacote sozinho no início de uma fatia de tempo é garantido
que terminará a transmissão sem colisão. Como no ALOHA Puro, a eficiência do
algoritmo de \textit{backoff} continua sendo de crucial importância para impedir
que colisões ocorram com muita frequência.


\subsection{CSMA - Carrier Sense Multiple Access, e variações CD e CA}

O CSMA é um protocolo MAC probabilístico. Este protocolo verifica, sempre, o
estado da rede no momento do envio de um pacote, ou seja, ele apenas transmite
se o meio estiver livre. Caso o dispositivo encontre o meio livre, ele transmite
seu pacote com uma probabilidade $p$, caso esteja ocupado, ele aguara por um
período de tempo e tenta novamente. No momento da transmissão, ``transmitir com
uma probabilidade $p$'' significa que nem sempre que o protocolo encontrar o
meio livre ele transmitirá o pacote. Esta medida visa a redução de colisões, já
que mais de um nodo podem decidir transmitir em um mesmo instante, sentindo,
simultaneamente, o meio livre. Na versão CSMA $p$-persistente, $p$ define a
probabilidade de o protocolo utilizar o meio caso o detecte livre. No caso
especial CSMA $1$-persistente, o MAC vai transmitir pacotes sempre que encontrar
o meio livre ($1$, neste caso, equivale a uma probabilidade de 100\%).

Duas modificações do CSMA são largamente utilizadas em redes padronizadas: o
CSMA/CD (CSMA with Collision Detection - Detecção de Colisão) e o CSMA/CA (CSMA
with Collision Avoidance - Prevenção de colisão). O CSMA/CD é utilizado no
padrão IEEE 802.3. Ao detectar uma colisão, o MAC CSMA/CD pode, eventualmente,
tentar se recuperar da colisão ou, caso não seja possível recuperar, reiniciar a
transmissão. Este MAC é bastante utilizado em redes cabeadas. Já o CSMA/CA é
utilizado em redes sem-fio, incluindo os padrões IEEE 802.11 (WiFi) e IEEE
802.15.4.

No algoritmo CSMA/CA (\textit{Carrier-Sense Multiple Access with Collision
Avoidance}) uma estação que deseja realizar uma transmissão verifica o meio para
determinar se está livre. Caso o meio estiver livre, transmite-se o quadro. Do
contrário, a estação aguarda um intervalo de tempo aleatório antes de verificar
o meio novamente. O CSMA/CA busca solucionar os problemas clássicos de
transmissão sem-fio conhecidos como ``problema da estação escondida'' e
``problema da estação exposta''. Basicamente, o CSMA/CA utiliza sinalizações RTS
(\textit{Ready to Send}) e CTS (\textit{Clear to Send}) para verificar a
existência de atividade de rádio no canal em uso na região do receptor.

%revisão de métodos de MAC
\subsection{MACs para \rssf}

Um protocolo de controle de acesso ao meio (MAC) decide quando um nodo de rede
pode acessar o meio, tentando garantir que nodos não interfiram nas transmissões
uns dos outros. No contexto de \rssf, protocolos MAC são ainda responsáveis por
implementar o uso eficiente do rádio, que é frequentemente o componente mais
crítico em termos de consumo de energia. Um MAC neste cenário normalmente
considera métricas tradicionais de rede como latência, vazão e disponibilidade
menos importantes que o baixo consumo de energia. Com isso, os principais fontes
de sobrecusto em comunicação via rádio (i.e., escuta ociosa, colisões, escuta
desnecessária e flutuações no tráfego) definem metas seguidas por, praticamente,
todos os MACs neste contexto~\cite{Langendoen:2005}.

O \emph{B-MAC} é um protocolo MAC para \rssf com \textit{carrier sense}, ou
seja, um protocolo que observa o uso do meio antes de utilizá-lo para envio de
dados~\cite{Polastre:2004}. Ele provê uma interface que permite reconfiguração
online, o que permite que os serviços de rede ajustem seus mecanismos. Estas
reconfigurações incluem aspectos como ligar e desligar o uso de CCA
(\textit{Clear Channel Assessment}), o envio de mensagens de reconhecimento
(ACKs), ajuste do tamanho do preâmbulo e do intervalo de escuta. Uma limitação
do B-MAC é que um receptor tem que esperar até que o preâmbulo seja
completamente transmitido para iniciar a troca de dados, mesmo se o receptor já
estiver acordado no início da transmissão. Além deste atraso, isto ainda implica
num problema conhecido como escuta desnecessária (\textit{overhearing}), onde
receptores permanecem acordados (e, portanto, consumindo energia) até o final do
preâmbulo para, apenas após este momento, descobrir que o pacote não era
endereçado a eles. Estas limitações são resolvidas pelo \emph{X-MAC}, que usa
preâmbulos curtos dentro dos quais o endereço do receptor está
``escondido''~\cite{Buettner:2006}. Assim, um receptor pode identificar se um
pacote é destinado a ele antes de receber o pacote inteiro, podendo ou
simplesmente desligar o rádio, caso não seja o destinatário, ou enviar um ACK ao
transmissor, avisando-o de que este pode parar de enviar o preâmbulo e iniciar o
envio do pacote de dados. Como ambos os protocolos são baseados no CSMA eles
sofrem do problema da estação escondida.

\emph{S-MAC} é um protocolo MAC para \rssf também é baseado no
CSMA~\cite{Ye:2002}, mas ele utiliza um mecanismo de RTS/CTS para evitar o
problema da estação escondida. Nodos vizinhos trocam informação de sincronização
para que acordem simultaneamente para se comunicar. Uma grande limitação do
S-MAC é que ele não permite nenhum tipo de configuração, nem estática, nem
dinâmica, apresentando um período de atividade (\textit{duty cycle}) fixo que
pode terminar por desperdiçar energia (\textit{idle listening}). O \emph{T-MAC},
que é uma versão melhorada do S-MAC, trata deste problema e adapta,
dinamicamente, seu \textit{duty cycle} através de um mecanismo refinado baseado
em \textit{timeouts}.
%Although the RTS/CTS mechanism solves the hidden terminal problem, it
%introduces the external terminal problem, to which both these protocols are
%exposed.
Nestes protocolos, a troca de informação necessária para manter os nodos
sincronizados produz um certo sobrecusto.

O \emph{Z-MAC} é um protocolo híbrido, que combina TDMA e CSMA~\cite{Rhee:2008}.
Ele usa um escalonamento TDMA, mas permite que nodos disputem as fatias alocadas
a outros nodos utilizando CSMA. O protocolo dá aos nodos aos quais as fatias de
tempo foram alocadas exclusividade para iniciar transmissões logo no início do
período alocado. Caso o nodo dono da fatia não inicia sua transmissão, outros
nodos começam a disputar o uso do meio através de um CSMA. O Z-MAC prevê o
cálculo e atribuição das fatias de tempo no momento da implantação da rede, o
que limita sua adaptabilidade.

O MAC do padrão \emph{IEEE 802.15.4} controla o acesso dos protocolos de mais
alto nível à camada física por dois modos distintos~\cite{IEEE:2006}.
Em um modo de operação básico, ele utiliza CSMA/CA e pacotes de reconhecimento
(ACK) para tratar colisões. Em outro modo, conhecido como modo com
\textit{beacons}, o MAC utiliza quadros chamados \textit{beacons} para
sincronizar os dispositivos na rede. Neste modo, o período de atividade da rede
é divido em duas partes: um com contenção (CAP - \textit{Contention Access
Period}) e um livre de contenção (CFP - \textit{Contention Free Period}). No
CAP, os nodos operam normalmente no CSMA/CA, com possibilidade de colisões
durante a contenção. No CFP, contudo, o protocolo divide o meio em fatias de
tempo (GTS - \textit{Guaranteed Time Slots}), que são alocadas a nodos que
podem, então, se comunicar sem a possibilidade de colisões. O frame de
\textit{beacon} é transmitido periodicamente pelo nodo coordenador da rede e,
além de servir como baliza para sincronização da rede, transporta informações de
configuração como, por exemplo, a tabela de alocação dos GTSs, informação sobre
dados pendentes para leitura, que são utilizados na sincronização de dados da
rede, e a definição do período de atividade da rede, o que permite que
dispositivos entrem em modo de baixo consumo de energia, desligando seus rádios,
durante a fase de inatividade. O modo \textit{beacon} permite uma melhor
sincronização entre os dispositivos, baixando o consumo de energia, mas ao preço
de uma menor vazão.

\subsection{C-MAC}
\label{sec:cmac}

o \emph{C-MAC} (\textit{Configurable MAC}) é um protocolo MAC altamente
configurável para \rssf implementado como um arcabouço de estratégias de
controle de acesso ao meio que podem ser combinadas para produzir protocolos
específicos a uma determinada aplicação~\cite{Wanner:IESS:2007}. Ele permite que
programadores de aplicação configure diversos parâmetros de comunicação (e.g.,
sincronização, contenção, detecção de erros, reconhecimento, empacotamento, etc)
para ajustar o protocolo especificamente para a aplicação sendo desenvolvida.
Embora altamente configurável, as instâncias do C-MAC configuradas para operar
conforme o B-MAC produziram melhores resultados que a implementação original
deste protocolo em termos de tamanho de código e dados, desempenho e eficiência
na utilização da rede. Isto se deve às técnicas de metaprogramação estática
usadas na implementação do C-MAC em C++, que permitem ao compilador gerar uma
série de otimizações reduzindo, especialmente, sobrecusto introduzido por
chamadas de funções e por polimorfismo.

A versão original do C-MAC, no entanto, definiu os elementos configuráveis do
protocolo de modo relativamente grosseiro. Por exemplo, sincronização foi
modelada como um único e grande componente, que precisava ser reimplementado
para todo novo protocolo, mesmo sabendo que aspectos como geração de preâmbulo e
sincronização de relógio são comuns a, praticamente, qualquer protocolo. Uma
revisão do projeto foi então realizada para refinar os componentes do C-MAC
~\cite{Steiner:ICUMT:2010}. Nesta revisão do C-MAC foi realizada uma decomposição dos
protocolos MAC tradicionais para obter uma máquina de estados generalizada para
cada um das três principais categorias de MAC~\cite{Klues:2007}: \emph{channel
pooling}, \emph{scheduled contention} e TDMA. As
Figuras~\ref{fig:channel_polling},~\ref{fig:scheduled_contention}
e~\ref{fig:tdma} apresentam as máquinas de estados desenvolvidas. É importante
notar que estas máquinas de estado incluem funcionalidades exclusivas de alguns
tipos de MAC. Dependendo da configuração adotada, estados das máquinas de
estados podem ser suprimidos. Como exemplo, pode-se citar o estado \texttt{TX
ACK PREAMBLE} na máquina de estados da Figura~\ref{fig:channel_polling}, que é
utilizada pelo X-MAC; 

\fig{channel_polling}{Máquina de estados para protocolos baseados em
\emph{channel polling}.}{scale=.6}

\fig{scheduled_contention}{Máquina de estados para protocolos baseados em
\emph{scheduled contention}.}{scale=.6}

\fig{tdma}{Máquina de estados para protocolos baseados em TDMA.}{scale=.6}

Uma análise cuidadosa destas máquinas de estado levou à definição da máquina de
estados para o C-MAC (Figura~\ref{fig:cmac}). Cada estado representa um
microcomponente que pode ter diferentes implementações. Estes microcomponentes
junto das transições de estados podem ser combinados para produzir protocolos
específicos para cada aplicação. A implementação desta máquina de estados
utiliza técnicas de metaprogramação estática (templates C++) para remover
estados que não fazem parte de uma determinada instância do protocolo. Quando um
estado é removido da máquina de estados, as entradas do estado removido são
encaminhadas diretamente ao(s) próximo(s) estados conectado(s) a ele, mantendo a
semântica original das transições. Além de ser capaz de acomodar protocolos
representativos em qualquer das três categorias estudadas, a máquina de estados
do C-MAC ainda suporta protocolos híbridos, como Z-MAC e IEEE 802.15.4.

\fig{cmac}{Máquina de estados do C-MAC.}{scale=.6}

Na máquina de estados apresentada na Figura~\ref{fig:cmac} existem quatro
``macro estados'': \texttt{SYNCHRONOUS SYNC}, \texttt{ASYNCHRONOUS SYNC},
\texttt{RX CONTENTION} e \texttt{TX CONTENTION}. Estes estados, na verdade, são
abstrações de outras máquinas de estados específicas para estas funções
complexas executadas pelo protocolo, e são arpesentados na
Figura~\ref{fig:synchronous_sync},~\ref{fig:asynchronous_sync},~\ref{fig:cmac_tx_contention}
e,~\ref{fig:cmac_rx_contention}.

\fig{synchronous_sync}{Máquina de estados \texttt{SYNCHRONOUS SYNC}.}{scale=.6}
\fig{asynchronous_sync}{Máquina de estados \texttt{ASYNCHRONOUS SYNC}.}{scale=.6}
\fig{cmac_tx_contention}{Máquina de estados \texttt{TX CONTENTION}.}{scale=.6}
\fig{cmac_rx_contention}{Máquina de estados \texttt{RX CONTENTION}.}{scale=.6}

Através desta máquina de estados é possível prover um maior número de pontos
configuráveis num framework com um alto nível de reuso. Os principais pontos
configuráveis do C-MAC incluem:
\begin{itemize}
  \item \textbf{Configuração da camada física:} através da disponibilização de
  funcionalidades que adaptam o hardware abaixo do MAC para atender às
  funcionalidades incluídas na instância do C-MAC em uso (e.g., frequência,
  potência de transmissão, taxa de transmissão);
  \item \textbf{Sincronização e organização:} permite configurar os dispositivos
  para enviar ou receber dados de sincronização para organizar a rede e
  sincronizar dados ou os períodos de atividade (\textit{duty cycle});
  \item \textbf{Mecanismo de prevenção de colisões (colision avoidance):} define
  os mecanismos de contenção usados para prevenir colisões. Pode incluir um
  algoritmo CSMA-CA, a troca de pacotes de controle de contenção (RTS/CTS) ou
  uma combinação dos dois;
  \item \textbf{Mecanismo de reconhecimento:} a troca de pacotes ``ACK'' para
  determinar o sucesso de uma transmissão, incluindo reconhecimento de pacotes
  de dados e de preâmbulo.
  \item \textbf{Tratamento de erros e segurança:} determinar que mecanismos
  serão usados para garantir a consistência dos dados (e.g., CRC, checksum) e
  segurança dos dados.
\end{itemize}

Testes realizados com o C-MAC apresentaram desempenho ligeiramente superior a
outros protocolos configurados de modo equivalente, porém apresentou um menor
tamanho~\cite{Wanner:IESS:2007}. Esta vantagem, contudo, é aumentada pelo
sistema de configuração do C-MAC, que permite a criação de protocolos
específicos para aplicações, que terminam por ter apenas os sobrecustos
estritamente necessários.

\section{Sistemas Operacionais para \Rssf}
\label{sec:so}

%introdução
Esta seção apresenta uma revisão de características necessárias aos sistemas
operacionais de \rssf. Ao final será apresentado o \epos, um sistema operacional
desenvolvido no \lisha para plataformas embarcadas que implementa muitas das
funcionalidades utilizadas em \rssf.

Em uma \rssf, requisitos específicos das aplicações em desenvolvimento guiam
todo o projeto de hardware, incluindo a capacidade de processamento, largura de
banda e taxas de transmissão do rádio, e módulos de sensores, requerendo que o
projeto seja modular. Estes requisitos, contudo, leva a uma grande variedade de
componentes de hardware, tornando os dispositivos de \rssf não apenas modulares,
mas heterogêneos. Neste cenário, uma aplicação de sensoriamento desenvolvida
para uma dada plataforma dificilmente será portável a uma plataforma diferente,
a não ser que o sistema de suporte de tempo de execução nestas plataformas
provejam mecanismos que abstraiam e encapsulem a plataforma de sensoriamento de
modo adequado. Ao mesmo tempo, os recursos limitados tipicamente encontrados no
hardware para \rssf forçam qualquer sistema para estes dispositivos a serem
eficientes e não usar recursos em excesso.

A necessidade por conectividade, abstração de hardware e gerenciamento de
recursos limitados torna imperativo o suporte de um sistema operacional para
aplicações de redes de sensores. Considerando a pesquisa, tecnologia e
aplicações atuais é enumera-se uma série de requisitos para sistemas
operacionais de \rssf~\cite{Wanner:JCC:2008}:
\begin{enumerate}

  \item \textbf{Prover funcionalidade básica de sistema operacional:} Para não
  restringir funcionalidade e portabilidade das aplicações, um sistema
  operacional para \rssf deve prover serviços tradicionais de sistema
  operacional como: abstração de hardware, gerenciamento de processos
  (geralmente seguindo o prisma ``monotarefa, multi-thread''), serviços de
  temporização e gerenciamento de memória.

  \item \textbf{Prover mecanismos eficientes para gerenciar consumo de energia:}
  Gerência de energia dos nodos de sensoriamento é um fator determinante da vida
  útil de uma \rssf. O sistema de suporte em tempo de execução para aplicações
  de redes de sensores deve prover mecanismos de gerência de energia para as
  aplicações, assim como usar o mínimo de energia possível para prover seus
  serviços.

  \item \textbf{Prover mecanismos de reprogramação em campo:} Dado que os nodos
  de \rssf podem estar localizados em regiões inóspitas e que requisitos e
  parâmetros das aplicações podem mudar com o tempo, reprogramação em campo
  através da rede de comunicação é um serviço importante neste tipo de sistema.
  Um sistema operacional para \rssf deve idealmente prover, para aplicações já
  implantadas, mecanismos de reprogramação total ou parcial em campo.

  \item \textbf{Abstrair hardware de sensores heterogêneos de modo uniforme:}
  Como já dito acima, os requisitos específicos de uma aplicação de \rssf faz
  do seu hardware não apenas modular, mas também heterogêneo, fazendo com que
  seja difícil portar uma aplicação de sensoriamento de uma plataforma para
  outra distinta. Além das diferenças arquiteturais, os próprios sensores
  (e.g., temperatura, luz, sensores de movimento) apresentam uma variabilidade
  ainda maior. Módulos de sensores apresentando a mesma funcionalidade
  frequentemente variam sua interface de acesso, características operacionais e
  parâmetros. Um sistema de suporte de tempo de execução apropriadamente
  projetado pode liberar os programadores das aplicações destas dependências
  arquiteturais e promover a portabilidade das aplicações a diferentes
  plataformas de sensoriamento.

  \item \textbf{Prover uma pilha de protocolos de comunicação configurável:}
  Dados os requisitos específicos de comunicação que diferentes aplicações
  apresentam, o hardware de comunicação para \rssf deve apresentar um certo
  nível de configurabilidade. O sistema operacional deve prover meios para
  configurar a pilha de protocolos de comunicação, a partir dos protocolos de
  controle de acesso ao meio (MAC), garantindo assim que aplicações possam
  explorar as características do dispositivo de comunicação empregado do modo
  que melhor lhe convir.

  \item \textbf{Operar com recursos limitados:} Como nodos de \rssf precisam
  consumir pouca energia, o projeto do hardware destes nodos deve trocar, sempre
  que possível, sua capacidade de computação por mais baixa potência. Assim, os
  nodos terão recursos de processamento e memória limitados. Um sistema
  operacional para \rssf deve entregar os serviços requisitados pela aplicação
  sem utilizar uma quantidade elevada dos recursos computacionais disponíveis.

\end{enumerate}

%outros sistemas . . .

Sistemas operacionais típicos para sistemas embarcados, como VxWorks, QNX, OS-9,
WinCE e $\mu$Clinux, provêm um ambiente de programação similar àqueles
existentes em computadores tradicionais, normalmente através de serviços
compatíveis com POSIX. Muitos destes sistemas operacionais provêm e requerem
suporte em hardware para proteção de memória. Embora estes sistemas sejam
adequados para outros aplicações embarcadas complexas como telefones celulares,
set-top-boxes, seus requisitos em termos de capacidade de memória e de
processamento torna impossível seu uso em \rssf. Vários sistemas foram
projetados especialmente para estas redes, incluindo MagnetOS~\cite{Barr:2002},
Contiki~\cite{Dunkels:2004} e AmbientRT~\cite{Hofmeijer:2004}. Os mais
proeminentes são, contudo, os sistemas TinyOS~\cite{Hill:2000}, MANTIS
OS~\cite{Abrach:2003} e SOS~\cite{Han:2005}.

O TinyOS é um sistema operacional baseado em eventos. O sistema é organizado em
uma coleção de componentes. Cada configuração do TinyOS é composta por uma
aplicação e os serviços de sistema operacional por ela requiridos, consistindo
de um escalonador e um grafo de componentes. Cada componente é composto por
comandos, tratadores de eventos, tarefas e quadro de execução. Cada componente
ainda declara os comandos aos quais responde e os eventos que ele sinaliza.
Comandos são chamadas de métodos não bloqueantes e são tipicamente usados para
iniciar requisições de software e hardware e, condicionalmente, iniciar tarefas.
tratadores de eventos são usados para tratar interrupções de hardware e podem
chamar comandos ou disparar tarefas.

O sistema possui um modelo de concorrência simplificado, baseado num modelo em
que tarefas executam até completar, sendo apenas preemptadas apenas por
interrupções. Este modelo traz consequências tanto positivas quanto negativas.
em um modelo tradicional baseado em threads, onde cada thread tem sua própria
pilha, cada thread precisa reservar espaço na memória, já limitada, do nodo para
seu contexto de execução. Dependendo da arquitetura, troca de contexto pode ser
uma operação extensa. Restringindo este modelo, o TinyOS reduz grande parte
deste sobrecusto, mas também perde a maior parte das características de um
modelo multi-thread. Esta restrição de concorrência pode ainda inibir a
capacidade do sistema em tratar restrições de tempo-real. O TinyOS não provê
mecanismos de alocação dinâmica de memória. Serviços de temporização são
providos por uma interface de Timer. O modelo de componentes do TinyOS, junto de
seu sistema simplificado de concorrência, permite ao sistema operar em
plataformas com menos de 1 kilobyte de memória RAM.

Gerenciamento de energia no TinyOS é implementado pelo escalonador de tarefas,
que faz uso da interface \texttt{StdControl} para iniciar e para componentes.
Quando a fila do escalonador está vazia, o processador principal é colocado em
um modo \textit{sleep}, de baixa potência. Deste modo, novas tarefas somente
serão disparadas na execução de um tratador de interrupção, que ``acorda'' o
processador. Este método permite bons resultados para o microcontrolador
principal, mas deixa métodos mais agressivos de gerência de energia a cargo da
aplicação.

O TinyOS apresenta uma arquitetura de abstração de hardware composta de três
camadas: \textit{Hardware Presentation Layer} (HPL), \textit{Hardware Adaptation
Layer} (HAL) e \textit{Hardware Interface Layer} (HIL)~\cite{Handziski:2004}. A
HPL agrupa componentes de dispositivos específicos em modelos específicos de
cada domínio, como \textit{Alarm} ou \textit{ADC Channel}. A HAL provê a melhor
abstração possível em termos de uso efetivo de recursos, mas ainda tenta não
inibir a portabilidade da aplicação. A HIL usa componentes adaptados para
implementar abstrações independentes de plataforma. O desenvolvedor de
aplicações para o TinyOS pode escolher usar qualquer um dos níveis de abstração
disponíveis, trocando portabilidade da aplicação por uso eficiente dos recursos.

A pilha de comunicação do TinyOS é baseada no protocolo controle de acesso ao
meio B-MAC~\cite{Polastre:2004}. O protocolo é implementa em duas camadas: controle de
hardware (LLHC - \textit{low-level hardware control}) e lógica do protocolo (PL
- \textit{protocol logic}). A camada de controle de hardware permite
configuração estática e dinâmica dos parâmetros básicos de comunicação (e.g.,
frequência, potência de transmissão). O sistema também permite algum nível de
configuração da camada da lógica do protocolo (e.g., \textit{duty cycle},
algoritmo de detecção de canal livre, uso de reconhecimento).

O MANTIS OS (Multimodal networks of in situ sensors)~\cite{Abrach:2003} é um sistema
operacional multi-thread que utiliza uma API (\textit{Application Programming
Interface}) inspirada em POSIX e adaptada às necessidades e restrições das
\rssf. A API é preservada entre diferentes plataformas e o kernel do sistema é
composto de um escalonador e drivers de dispositivos. Uma pilha de comunicação e
um servidor de comandos são disponibilizados a serviços de nível de usuário.

O escalonador do MANTIS OS provê um subconjunto do pacote POSIX threads, com
escalonamento round-robin baseado em prioridades. O sistema suporta alocação de
memória na estática e dinâmica para as threads. O escalonador é acionado
periodicamente por um timer, através de operações em semáforos. Uma thread
\textit{idle} de baixa prioridade é utilizada como ponto de entrada para as
políticas de gerência de energia do sistema, que põem o processador em modo
\textit{sleep} sempre que não há threads aptas para execução. Serviços de
temporização e de sincronização são providos através de interfaces similares às
definidas pelo padrão POSIX. O mecanismo de escalonamento complexo utilizado no
MANTIS OS acarreta um sobrecusto maior que o acarretado por modelos mais
simples, baseados em eventos. Logo, o sistema necessita de mais memória, tanto
RAM quanto de código, que, por exemplo, o TinyOS. O sistema ainda é, contudo,
adequado para uso em protótipos atuais de \rssf.

O MANTIS usa uma camada de abstração de hardware (HAL - \textit{Hardware
Abstraction Layer}), com as funções \texttt{dev\_read()}, \texttt{dev\_write()},
\texttt{dev\_mode()} e \texttt{dev\_ioctl()}. Cada função toma um dispositivo
como parâmetro e uma tabela de funções redireciona chamadas gerais aos drivers
dos dispositivos. A lista de parâmetros para as funções \texttt{dev\_mode()} e
\texttt{dev\_ioctl()} são específicas para cada dispositivo e não há uma
abstração unificada para os sensores (cada driver de dispositivo apresenta
semânticas específicas).

O sistema provê uma interface de comunicação unificada através de threads em
nível de usuário. Há um formato unificado de pacote para diferentes interfaces
de comunicação (e.g., RS-232, USB, rádio). Esta camada de comunicação gerencia a
sincronização de pacotes e \textit{buferização}. Sob a API de comunicação, o
MANTIS OS usa drivers de dispositivos tradicionais. As aparentes vantagens de um
ponto de entrada único para comunicação é diminuída devido à semântica e
parâmetros específicos dos métodos de comunicação para cada interface.

O SOS~\cite{Han:2005} é um sistema operacional dinamicamente reconfigurável para
\rssf. O kernel do sistema inclui serviços de passagem de mensagens, alocação dinâmica
de memória e carga dinâmica de módulos. O SOS é organizado em uma série de
módulos binários que implementam tarefas específicas. Estes componentes são
comparáveis em funcionalidade aos componentes do TinyOS. Uma aplicação é
composta por uma série de módulos que interagem entre si, apresentando tanto
interface por chamada de métodos quanto por passagem de mensagens. Passagem de
mensagens é assíncrona e coordenada por um escalonador que usa uma fila ordenada
por prioridades. Chamadas diretas a funções são utilizadas para operações
síncronas entre os módulos. O sistema integra alocação dinâmica de memória e
\textit{garbage collection}. Como no TinyOS e no MANTIS OS, SOS coloca o
processador em modo \textit{sleep} sempre que não há mensagens para escalonar. O
modelo de reconfiguração dinâmica do SOS implica em sobrecustos
consideravelmente maiores quando comparado com outros sistemas. Contudo, este
sobrecusto ainda é aceitável aplicações de \rssf]~\cite{Han:2005}.

SOS provê serviços para, dinamicamente, incluir, atualizar e remover módulos em
programas previamente implantados. O sistema divide a memória de programa em
páginas, e mantém estruturas de estado e contexto na memória RAM para cada
módulo.

O sistema usa os mecanismos de módulos de kernel carregáveis nas abstrações de
hardware de sensoriamento. Através desta arquitetura, drivers de dispositivos
podem registrar seus serviços e associá-los a um nome, permitindo às aplicações
acessar componentes através destes nomes. Por exemplo, um driver de um sensor
analógico pode se associar a um canal do ADC (Analog-to-Digital Converter) e
registrar um sensor de um tipo específico (e.g., PHOTO - luminosidade, TEMP -
temperatura, etc). Quando a aplicação requer dados de, por exemplo, PHOTO, o
kernel usa o driver registrado para obter a leitura apropriada. Esta abstração
semântica das leituras de sensores promove portabilidade das aplicações.
Contudo, como o sistema operacional precisa manter uma tabela de ponteiros para
funções indexada por nomes, o registro de drivers implica em algum sobrecusto de
memória.


\subsection{\epos - Embedded Parallel Operating System}
\label{sec:epos}

Requisitos de sistema para \rssf incluem funcionalidades básicas de sistema
operacional (e.g., gerência de memória, escalonamento), gerência de energia,
mecanismos de reprogramação dinâmica, abstração dos dispositivos de
sensoriamento (sensores) e uma pilha de comunicação configurável. A capacidade
limitada do hardware de \rssf requer que estes sistemas operem com recursos
limitados, o que faz do uso e adaptação de sistemas operacionais tradicionais
impossível. Vários projetos de
pesquisa~\cite{Abrach:2003, Barr:2002, Dunkels:2004, Han:2005, Hill:2000,
Hofmeijer:2004} focaram na solução do problema de suporte de sistema para redes
de sensores. A maioria deles, contudo, falhou no tratamento de dois requisitos:
configuração transparente do canal de comunicação e abstração eficiente e
unificada da camada de abstração de sensores.

O \epos (Embedded Parallel Operating
System)~\cite{Froehlich:2001,Marcondes:ETFA:2006} é um arcabouço baseado em
componentes para a geração de ambientes dedicados de suporte de tempo de
execução. O arcabouço do \epos permite que programadores desenvolvam aplicações
independentes de plataforma, e ferramentas de análise permitem que componentes
sejam adaptados automaticamente para atender a requisitos destas aplicações
particulares. Por definição, uma instância do sistema agrega todo suporte
necessário para a sua aplicação dedicada, e nada mais.

O projeto modular do \epos foi guiado pela metodologia Projeto de Sistemas
Embarcados Guiado pela Aplicação (ADESD - \textit{Application-Driven Embedded
System Design}). A ADESD baseia-se nas conhecidas estratégias de decomposição de
domínio por trás do Projeto Baseado em Famílias (FBD - \textit{Family-Based
Design}) e Orientação a Objetos (OO) como, por exemplo, análise de atributos
comuns e variabilidade, para adicionar o conceito de identificação e separação
de aspectos ainda nos estágios iniciais do projeto~\cite{Froehlich:2001}. Deste
modo, a ADESD guia a engenharia de domínio para famílias de componentes, das
quais dependências de cenários de execução são fatoradas na forma de aspectos e
relacionamentos externos são capturados em um arcabouço de componentes. Esta
estratégia de engenharia de domínio trata consistentemente algumas das questões
mais relevantes do desenvolvimento de software baseado em componentes:
reuso, gerenciamento da complexidade e composição.

\fig{aosd}{Processo de decomposição de domínio da ADESD.}{width=.9\columnwidth}

A Figura~\ref{fig:aosd} mostra o processo de decomposição de domínio no projeto
de sistemas embarcados guiado pela aplicação. Abstrações são identificadas a
partir do domínio do problema e organizadas em famílias, de acordo com suas
características comuns. Dependências de cenário são modeladas como aspectos que
podem ser aplicados através de adaptadores de cenário. Famílias de abstrações
são visíveis para aplicações através de interfaces infladas, que exportam seus
membros como um único ``super-componente''. As arquiteturas de sistema são
capturadas em arcabouços de componentes, que são definidos em termos de aspectos
de cenário.

Famílias de abstrações no \epos representam abstrações tradicionais de sistemas
operacionais e implementam serviços como gerenciamento de memória e de
processos, sincronização de processos, gerenciamento de tempo e comunicação.
Abstrações são projetadas e implementadas de modo independente de seus cenários
de execução e arquiteturas. Todas unidades de hardware dependentes de
arquitetura são abstraídas como mediadores de hardware, que exportam, através de
interfaces independentes de plataforma, a funcionalidade necessária às
abstrações. Devido ao uso de metaprogramação estática e \textit{function
inlining}, mediadores de hardware implementam suas funcionalidades sem formar
uma camada de abstração de hardware (HAL - \textit{Hardware Abstraction Layer})
tradicional. Através do uso de mediadores de hardware, as abstrações do \epos
atingiram um nível de reuso que permite, por exemplo, a mesma família de
abstrações de threads ser utilizada em ambientes tanto monotarefa quanto
multitarefa, como parte de um $\mu$kernel ou meramente ligada à aplicação,
tanto em microcontroladores de 8-bits quanto em um processador de 64-bits.

No \epos, processos são gerenciados pelas abstrações \texttt{Thread} e
\texttt{Task}. Cada thread armazena seu contexto em sua própria pilha. A
abstração de contexto defini o conjunto de dados que precisa ser armazenado para
um fluxo de execução e, deste modo, cada arquitetura define seu próprio
contexto.

Tempo é tratado pela família de abstrações \texttt{Timepiece}. Estas abstrações
são suportadas através dos mediadores \texttt{Timer}, Timestamp Counter
(\texttt{TSC}) e Real-Time Clock (\texttt{RTC}). A abstração \texttt{Clock} é
responsável por um controle estrito de tempo e está disponíveis em sistemas que
possuam um dispositivo de relógio de tempo real (RTC). A abstração
\texttt{Alarm} pode ser utilizada para gerar eventos que acordem uma thread ou
chamem uma função. Alarmes têm ainda um evento mestre de altíssima prioridade
que está associado com um período de tempo pré-definido. Este evento mestre é
utilizado para acionar o algoritmo de escalonamento do sistema quando o
\textit{quantum} de escalonamento é atingido, nos casos em que uma configuração
com um escalonador ativo é utilizada. Finalmente, a abstração
\texttt{Chronometer} é utilizada para realizar operações de medição de tempo.

A família de abstrações \texttt{Synchronizer} provê mecanismos que garantem a
consistência de dados em ambientes com processos concorrentes. O membro
\texttt{Mutex} implementa um mecanismo de exclusão mútua que entrega duas
operações atômicas: \texttt{lock} e \texttt{unlock}. O membro \texttt{Semaphore}
implementa, como o próprio nome diz, um semáforo, que é uma variável inteira
cujo valor apenas pode ser manipulado indiretamente através das operações
atômicas \texttt{p} e \texttt{v}. O membro \texttt{Condition} realiza uma
abstração de sistema inspirada no conceito de variável de condição, que permite
a uma thread esperar que um predicado se torne válido.

No \epos, detalhes referentes a proteção e tradução de espaços de endereçamento,
assim como alocação de memória, são abstraídas através da família de mediadores
de hardware \texttt{MMU} (\textit{Memory Management Unit}). A abstração
\texttt{Address Space} é um contêiner para ``fatias'' de memória física chamados
de \texttt{Segments}. Ele não implementa tarefas de proteção, tradução ou
alocação de endereços de memória, deixando este papel para o mediador da MMU. O
membro \texttt{Flat Address Space} define um modelo de memória no qual endereços
lógicos e físicos coincidem, eliminando a necessidade de hardware para MMU. Em
plataformas que não dispõem de MMU, um mediador para MMU simplesmente mantém o
contrato de interface com o \texttt{Flat Address Space}, realizando
implementações vazias de métodos quando necessário. Métodos relativos a alocação
de memória operam sobre bytes de modo similar ao que é feito pela função de
alocação de memória da libc.

Controle de entrada e saída (I/O) de dispositivos periféricos é disponibilizado
no \epos pelo mediador de hardware correspondente. O mediador \texttt{Machine}
armazena as regiões de I/O e trata o registro dinâmico de interrupções. O
mediador \texttt{IC} (\textit{Interrupt Controller}) trata a ativação ou
desativação de interrupções individuais. Para lidar com as diferentes
interrupções existentes em diferentes plataformas e contextos, \epos atribui um
nome e uma sintaxe independente de plataforma a interrupções pertinentes ao
sistema (e.g., interrupção de timer, interrupção de conversão completa no ADC).


\subsection{O \epos para \rssf}

Aplicações de \rssf apresentam requisitos específicos que vão além dos atendidos
pelos serviços tradicionais de sistemas operacionais. Estes incluem
gerenciamento de energia eficiente, reprogramação em campo, abstração uniforme
de sensores e serviços de comunicação configuráveis. O \epos foi estendido de
modo a atender estes requisitos extras~\cite{Wanner:JCC:2008}.

O \epos provê serviços de gerência de energia dirigida pela aplicação que
permite um consumo consciente de energia em sistemas profundamente embarcados
sem comprometer a portabilidade da aplicação e sem gerar sobrecustos excessivos.
O objetivo do subsistema de gerenciamento de energia do \epos é permitir que
aplicações expressem quando determinados componentes de software não estão sendo
utilizados, permitindo que o sistema migre dispositivos (hardware) associados a
estes componentes de software para modos de operação que consumam menos energia.
Disto emergiram várias questões que dizem respeito a diferenças arquiteturais
entre dispositivos distintos e ao acesso concorrente aos recursos de hardware
por diferentes componentes de software. Para tratar destas questões, foram
concebidas (1) uma interface genérica para gerenciamento de energia, (2) um
mecanismo de propagação de mensagens e (3) um modelo de formalização das trocas
entre modos de operação~\cite{Hoeller:DIPES:2006}.

Nesta estratégia de gerenciamento de energia, espera-se que o programador da
aplicação especifique, em seu código-fonte, quando certos componentes não serão
utilizados. Assim, uma API uniforme que permite gerenciamento de energia foi
definida. Esta interface permite a interação entre aplicação e sistema, entre
componentes do sistema e dispositivos de hardware, além de permitir que a
aplicação acesse diretamente o hardware se assim desejar o programador. Para que
o programador da aplicação não necessite ``acordar'' explicitamente os
componentes do sistema sempre que eles forem utilizados, o mecanismo de
gerenciamento de energia intercepta acessos aos componentes de hardware
desativados, retornando estes componentes a seu estado operacional anterior
sempre que estes são utilizados.

\fig{api-hierarchy}{Hierarquia da API de gerenciamento de energia do
\epos.}{width=.9\columnwidth}

A aplicação pode, como demostrado na Figura~\ref{fig:api-hierarchy}, acessar um
componente global (\texttt{System}) que possui referências para todos os outros
componentes no sistema, disparando ações globais de troca de modo de operação no
sistema. A aplicação ainda pode utilizar a interface de gerenciamento de
energia através de subsistemas (e.g., comunicação, processamento,
sensoriamento). Deste modo, mensagens são propagadas apenas para componentes
usados na implementação de cada subsistema. A aplicação pode ainda acessar o
hardware diretamente, utilizando a API disponível nos drivers de dispositivos
(e.g., \textit{Network Interface Card} (NIC), CPU, Thermistor). A mesma API
também é utilizada para troca de mensagens de gerenciamento de energia entre
componentes do sistema.

Para garantir a portabilidade da aplicação e para facilitar o desenvolvimento da
aplicação API de gerenciamento de energia foi definida com um conjunto mínimo de
métodos e um conjunto de modos de operação universais, que possuem uma semântica
unificada para todo o sistema. Portabilidade é alcançada pelo fato de a
aplicação não necessitar implementar procedimentos específicos para cada
dispositivo de modo a trocar seu modo de operação. Estes procedimentos são
abstraídos pela API. A facilidade de uso vem do fato de que o programador da
aplicação não necessita analisar documentação específica do hardware para
identificar os modos de operação disponíveis, os procedimentos para utilizar
estes modos e as consequências destas mudanças.

Estudos de caso~\cite{Hoeller:DIPES:2006} demonstram que é possível alcançar uma
economia significante de energia com mínima intervenção pela aplicação. Esta
estrutura de gerenciamento de energia do \epos é também utilizado por um
gerenciador de energia ativo, que executa periodicamente ou quando não á threads
a serem escalonadas e, oportunisticamente, altera o modo de operação de
componentes e dispositivos para diminuir o consumo de energia do sistema. Este
gerenciador de energia verifica o período de inatividade de cada componente e
emprega heurísticas configuráveis que decidem os momentos apropriados para
trocar seus modo de operação. Em sua versão mais simples, o gerenciador de
energia põe todos os componentes ociosos por um determinado tempo em um modo de
baixo consumo.

\fig{update}{\textsc{Elus} - o mecanismo de reprogramação em campo do
\epos.}{width=.9\columnwidth}

Reprogramação em campo é realizada no \epos através de um mecanismo de indireção
similar ao RPC (\textit{Remote Procedure Calls}), chamado
\textsc{Elus} (\epos Live Update System)~\cite{Gracioli:HotSWUp:2008}. Nesta
infraestrutura, apresentada na Figura~\ref{fig:update}, a invocação de um método
de um componente por uma aplicação cliente passa por um \texttt{Proxy}, que
envia uma mensagem a um \texttt{Agent}. Após a execução do método, uma mensagem
com o valor de retorno é devolvido à aplicação. Nesta estrutura, um nível de
indireção é criado entre as chamadas de métodos pela aplicação e o sistema,
fazendo com que o componente \texttt{Agent} seja o único ciente da posição do
componente do sistema na memória. O \texttt{Agent} controla o acesso aos métodos
dos componentes através de um sincronizador (\texttt{Semaphore}), não permitindo
chamadas a um componente que esteja passando por uma atualização. Uma thread do
sistema é responsável por receber requisições de atualização e o novo código
para os componentes. Estas requisições é enviada ao agente, que sobrescreve o
código antigo para incluir o novo. A infraestrutura de reprogramação é
transparente à aplicação é pode ser ``desligada'', eliminando as indireções e o
ocasional sobrecusto que surge tanto em volume de código e dados (memória) e
atrasos de processamento.

O \epos ainda provê suporte específico para aplicações de sensoriamento,
definido interfaces software/hardware capazes de abstrair famílias de
dispositivos sensores de modo uniforme~\cite{Wanner:ETFA:2006}. O subsistema de
sensoriamento do \epos já foi discutido na Seção~\ref{sec:epos_sensing} deste
texto.

A infraestrutura de comunicação do \epos para \rssf é implementada pelo
protocolo C-MAC, \textit{Configurable MAC}, que provê suporte a comunicação de
baixo nível (MAC - \textit{Medium Access Control})~\cite{Wanner:IESS:2007}. O
C-MAC já foi discutido na Seção~\ref{sec:cmac} deste texto.

% Summary 

% Contributions: (1) common and simple interface (minor), (2)
% Power-management on embedded systems without using any complex
% high-cost methodology.
In this paper we presented an strategy to enable application-driven
power management in deeply embedded systems. In order to achieve this
goal we allowed application programmers to express when certain
components are not being used. This is expressed through a simple
power management interface which allows power mode switching of system
components, subsystems or the system as a whole, making all
combinations of components operating modes feasible. By using the
hierarchical architecture by which system components are organized in
our system, effective power management was achieved for deeply
embedded systems without the need for costly techniques or strategies,
thus incurring in no unnecessary processing or memory overheads.

A case study using a 8-bit microcontroller to monitor temperature in
an indoor ambient showed that almost 40\% of energy could be saved
when using this strategy. % and with minimal application intervention.

% Problems: concurrence. Describe the Thread problem.

% The paper also listed some identified problems on the path for
% power-aware software and hardware components, discussing and
% explaining how some of these problems have been solved in this work
% and how some of them can be solved, and will be, in future work.

% Even so, it still have its usability.



\section*{Agradecimentos}
Os autores gostariam de agradecer a todos os colaboradores do LISHA que
contribuíram para o desenvolvimento do \epos e do \emote ao longo de quase 10
anos de trabalho.

\bibliographystyle{sbc}
\bibliography{lisha,wsn}

\appendix
\section*{Anexo: Exercícios práticos}
\label{sec:exercicios}

O \epos suporta uma variedade de arquiteturas, variando de 8 a 32 bits.
Atualmente, estas arquiteturas são:
\begin{itemize}
  \item IA32: single and multi-core;
  \item AVR8: atmega16, atmega128, atmega1281, and at90can128 microcontrollers;
  \item PPC32: including ml310;
  \item MIPS32: including plasma soft-core;
  \item ARM7: incluindo o MC13224V (Freescale) e o AT91SAM7 (Atmel).
\end{itemize}

O \epos é software aberto, livre para uso não comercial. Mais detalhes sobre a
licença do \epos podem ser encontrados no sítio do
projeto~\cite{Project:EPOS:2010}.

Esta seção apresenta os passos básicos para obter o \epos, instalar seu ambiente
de desenvolvimento, compilá-lo e executá-lo em um \emote.

\subsection*{Baixando e instalando o ambiente de desenvolvimento}

O \epos foi concebido prevendo seu desenvolvimento utilizando o ambiente
GNU/Linux. Assim, o \epos é um sistema cross-compilado no GNU/Linux para várias
plataformas alvo, utilizando o \textit{GNU Compiler Collection} (GCC).
Versões binárias deste compilador para as plataformas que o \epos suporta estão
disponíveis para serem baixadas no sítio do \epos~\cite{Project:EPOS:2010}. O
primeiro passo é baixar o compilador e instalá-lo conforme as instruções
constantes no sítio.

O \epos também está disponível para baixar no sítio do projeto. Após baixar o
sistema, é necessário descompactá-lo e configurar as seguintes variáveis de
ambiente em seu sistema:

\begin{verbatim}
export EPOS=/path/to/epos
export PATH=$PATH:$EPOS/bin
\end{verbatim}

\subsubsection*{Configurando o \epos}

A configuração do \epos é realizada através da definição de variáveis no arquivo
\texttt{\$EPOS/makedefs}, com as seguintes opções:
\begin{itemize}
  \item \textbf{MODE:} configura a arquitetura do sistema operacional:
  \begin{itemize}
    \item \textit{Library:} sistema é ligado à aplicação;
    \item \textit{Builtin:} sistema e aplicação estão no mesmo espaço de
    endereçamento;
    \item \textit{Kernel:} sistema e aplicação estão em espaços de endereçamento
    diferentes com uma camada de chamadas de sistema (\textit{System Call})
    entre eles.
  \end{itemize}
  \item \textbf{ARCH:} configura a arquitetura do processador na plataforma
  alvo:
  \begin{itemize}
    \item \textit{ARCH\_IA32:} Intel x86 32-bits Architecture;
    \item \textit{ARCH\_AVR8:} Atmel AVR 8-bits Architecture;
    \item \textit{ARCH\_PPC32:} IBM PowerPC 32-bits Architecture;
    \item \textit{ARCH\_MIPS32:} MIPS 32-bits Architecture;
  \end{itemize}
  \item \textbf{MACH:} configura a máquina para a qual o sistema será gerado:
  \begin{itemize}
    \item \textit{MACH\_PC:} computadores pessoais;
    \item \textit{MACH\_ATMEGA16:} microcontrolador Atmel ATMega16;
    \item \textit{MACH\_ATMEGA128:} microcontrolador Atmel ATMega128;
    \item \textit{MACH\_ATMEGA1281:} microcontrolador Atmel ATMega1281;
    \item \textit{MACH\_AT90CAN128:} microcontrolador Atmel AT90CAN128;
    \item \textit{MACH\_ML310:} Xilinx ML310 and ML403 Evaluation Boards;
    \item \textit{MACH\_PLASMA:} microprocessador Plasma;
  \end{itemize}
\end{itemize}


\subsection*{Compilando o \epos}

Com o \epos configurado, para compilar o sistema basta executar um comando
\texttt{make all} dentro do diretório em que o sistema foi instalado
(\texttt{\$EPOS}). Devido a algumas variações nas distribuições GNU/Linux
utilizadas, alguns problemas podem surgir durante este processo. Soluções para
estes problemas estão documentados no guia de usuário do \epos, disponível no
sítio do projeto~\cite{EPOS:UserGuide:2010}.


\subsection*{Compilando uma aplicação para o \epos}

Para compilar aplicações para o \epos, o sistema provê uma ferramenta
específica: o \texttt{eposcc}. Esta ferramenta analisa a aplicação, busca a
configuração do sistema e realiza os comandos de compilação adequados utilizando
o GCC. O primeiro passo é compilar a aplicação através do seguinte comando:

\begin{verbatim}
$ eposcc app/your_app.cc -o app/your_app.o
\end{verbatim}

Feito isso, basta ligar o arquivo objeto da aplicação com o sistema operacional,
pelo seguinte comando:

\begin{verbatim}
$ eposcc -o app/your_app app/your_app.o
\end{verbatim}

Agora, o arquivo \texttt{app/your\_app} contém o binário de sua aplicação já
ligado ao sistema operacional. Para ter uma imagem binária pronta para ser
carregada em seu hardware, basta utilizar a ferramenta \texttt{eposmkbi}, que é
responsável por gerar uma imagem completa do sistema com as aplicações para
execução na plataforma alvo. Para isso, basta executar o seguinte comando:

\begin{verbatim}
$ eposmkbi img/your_app.img app/your_app
\end{verbatim}

Caso a imagem gerada para o sistema seja multitarefa, e você tenha compilado
mais de uma aplicação para o sistema, basta listar todas as aplicações geradas
no comando de construção da imagem de boot, deste modo:

\begin{verbatim}
$  eposmkbi img/your_app.img app/your_app1 app/your_app2...
\end{verbatim}

\subsection*{Hello World!}

Para simplificar o processo de desenvolvimento da aplicação, o \texttt{makefile}
do \epos está configurado para construir automaticamente o sistema para uma
aplicação específica que pode ser informada pela linha de comando da seguinte
forma:

\begin{verbatim}
$  make APPLICATION=helloworld
\end{verbatim}

Deste modo, o \texttt{makefile} construirá o \epos e, em seguida, compilará a
aplicação informada e preparará a imagem de boot pronta para o sistema. Por
exemplo, para criar uma aplicação ``Hello World!'' no \epos, basta criar um
arquivo, por exemplo, \texttt{helloworld.cc} no diretório \texttt{\$EPOS/app}.
Deste modo, a execução do comando acima resultará na imagem do sistema com a
aplicação informada, pronta para ser carregada na plataforma alvo.

\subsection*{Programando no \epos}

A API de programação do \epos é composta de dois tipos de componentes de
software: abstrações e mediadores. As abstrações são classes C++ independentes
de plataforma com interface e comportamento bem definidos. Suporte específico
para plataformas é implementado através de mediadores de hardware, que são,
funcionalmente, equivalentes aos \emph{device drivers} do Unix, mas sem utilizar
uma estrutura tradicional de HAL (\emph{Hardware Abstraction Layer}). Ao invés
disto, eles mantém um contrato de interface entre as abstrações e dispositivos
de hardware através de técnicas de metaprogramação estática (C++ templates), o
que implica na dissolução do código dos mediadores nas abstrações em tempo de
compilação, eliminando sobrecustos de chamada de função. O \epos ainda oferece
uma série de componentes ``utilitários'', como implementações de estruturas de
dados (lista, fila, mapa, etc).

A documentação da API do \epos é mantida atualizada no sítio do
projeto~\cite{EPOS:UserGuide:2010}.



\end{document}
