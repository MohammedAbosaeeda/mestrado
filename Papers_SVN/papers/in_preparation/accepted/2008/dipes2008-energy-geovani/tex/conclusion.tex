\section{Conclusion}
\label{sc:conclusao}

%- Retomar os objetivos deste trabalho. Retomar as implica��es e caracter�sticas. Atender o tempo de dura��o da bateria e os \deadlines{} das partes obrigat�rias.

%Este trabalho prop�s uma abordagem que explora a energia como um par�metro de \qos{} em sistemas embarcados m�veis, para atender o tempo de dura��o do sistema com a execu��o e garantia dos \deadlines{} de, no m�nimo, tarefas essenciais, definidas previamente pelo programador da aplica��o. Essa abordagem foi inspirada pelos conceitos de tarefas imprecisas, que s�o tarefas que podem ser divididas em parte obrigat�ria e parte opcional. Neste artigo, equa��es em tempo de projeto foram apresentadas com o objetivo do programador da aplica��o verificar se o conjunto de tarefas utilizado ser� escalon�vel no nosso algoritmo com rela��o aos dois par�metros desejados (tempo e energia). Em tempo de execu��o, nosso escalonador baseado no algoritmo \textsc{EDF} garante os \deadlines{} das partes obrigat�rias e com a realimenta��o da equa��o de energia verifica se o tempo de dura��o do sistema exigido ser� alcan�ado. Caso algum dos par�metros desejados seja comprometido, as partes opcionais ser�o descartadas, diminuindo os n�veis da \QoS{}.  Um prot�tipo foi implementado no \epos{}, que permitiu que um gerente de energia fosse executado nos per�odos ociosos em que as partes opcionais foram impedidas de executar, consumindo menos energia e aumentando o tempo de dura��o da bateria.

This work proposed an approach to exploit energy as a \qos{} parameter in order
to guarantee that battery lifetime can last time desired by mobile embedded
system and yet preserve the deadlines of hard real-time tasks.
Our approach was inspired by imprecise tasks concepts, according to tasks
can be divided into mandatory and optional parts. In this article, equations at
project-time were presented with objective the of application programmer to
check if a set of tasks will be schedulable in our algorithm in relation to two
parameters desired, i.e., time and energy. At execution-time, our scheduler
based on \textsc{EDF} algorithm ensures the mandatory subtasks deadlines and
recalculates the equation of energy in order to check if the required battery 
lifetime will be met. The optional subtasks are prevented from executing, i.e,
decreasing \qos{} levels if any required parameter will not be met. A prototype
was developed in \epos{}, which allowed the execution of a power manager in idle
periods created by non-execution of the optional subtasks, thus reducing 
energy consumption by stopping or slowing down system components during these
idle periods.

