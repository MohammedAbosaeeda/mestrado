\section{A aplicação de detecção de queda}

A Figura~\ref{fig:Planta_baixa} apresenta uma visão geral da aplicação proposta.
Em um determinado local, diversas pessoas (usuários) utilizam pulseiras capazes
de detectar uma eventual queda. Ao detectar uma determinada queda, a pulseira
deve emitir um sinal sonoro e aguardar alguns segundos (e.g., 5 segundos) para
que o usuário, eventualmente, sinalize, pressionando um botão, a ocorrência de
um falso positivo, ou seja, de que a detecção foi indevida. Não ocorrendo a
sinalização no tempo esperado, a pulseira deve emitir um sinal (pacote de dados)
que deve chegar até um gateway capaz de remeter esta informação a um servidor,
onde um sistema de monitoramento se encarregaria de providenciar atendimento
adequado à vítima da queda. Neste trabalho, focaremos no fluxo da informação
sobre a queda da pulseira até o sistema de monitoramento. Não aboradamos, ainda,
integração com serviços de emergência ou de comunicação extra para providenciar
a prestação de socorro.

\fig{Planta_baixa}{Planta baixa mostrando o funcionamento da
aplicação}{width=0.8\columnwidth}

Os requisitos da aplicação são apresentados na Tabela~\ref{tab:requisitos}.
Neste ponto ressalta-se a importância do uso de uma plataforma completa. Muitos
dos requisitos da aplicação são resolvidos praticamente de imediato com a
seleção do hardware a ser utilizado e com a configuração adequada do sistema
operacional.

\tab{requisitos}{Tabela de requisitos da aplicação de detecção de quedas.}

O hardware de sensoriamento precisava possuir um tamanho reduzido e um formato
que pudesse ser usado por uma pessoa, optou-se por uma pulseira, alem dos
requisitos de forma precisava da funçao de monitor de queda, a qual foi
realizada usando um acelerômetro de três eixos, uma interface com o usuário que
consistia de um botão e de um LED que seriam ativado em caso de falso positivo,
e por final a capacidade de enviar os dados sem fio a outros dispositivos.

O hardware de recepção precisava ser capas de receber o sinal de queda enviado e
enviar esse sinal para a internet de modo a ser visualizado pela pessoa
responável por monitorar os dispositivos.

Com base nas informações supra citadas inicio-se o projeto utilizando o hardware
do \emote~para o desenvolvimento das aplicações de software, após a validação do
software em cima do \emote~houve a necessidade do desenho de um hardware
reduzido visto que o hardware de \emote~possui características que não eram
úteis ao projeto e e demasiado grande para a finalidade desejada do projeto,
sendo assim se projetou um hardware derivado do antigo de modo a não ser
necessário refazer o software ja validado e para que esse hardware atende-se as
necessidades de projeto tanto de tamanho quanto de funcionalidade.
