\documentclass[english,10pt]{letter}
\usepackage[T1]{fontenc}
\usepackage[latin1]{inputenc}
\usepackage{babel}
\usepackage[top=.3in,bottom=.3in,left=0.75in,right=0.75in,noheadfoot]{geometry}
\usepackage{graphicx}

\address{\sffamily
\begin{tabular}{p{11cm}r}
Leonardo Kessler Slongo									& Prof.~Dr.~Eduardo Augusto Bezerra \\
M.Sc. Arliones~Hoeller~Jr								& Federal University of Santa Catarina \\                 			
Prof.~Dr.~Ant\^{o}nio Augusto Fr\"{o}lich				& Embedded Systems Group (GSE) \\   			
Federal University of Santa Catarina      				& Florian\'{o}polis, 88040-900, Brazil \\  			
Software/Hardware Integration Lab (LISHA)				& Phone: +55-48-3721-2358 \\		
Florian\'{o}polis, 88040-900, PO Box 476, Brazil 		& E-Mail: eduardo.bezerra@eel.ufsc.br\\		
Phone: +55-48-3721-9516                     			&\\		
E-Mail: \{leokessler,arliones,guto\}@lisha.ufsc.br		&\\      
\\
\multicolumn{2}{r}{\today}
\end{tabular}
}

\date{}

\signature{
  \begin{center}
   %\raisebox{-.9cm}[0pt][0pt]{\includegraphics[scale=1]{}\\
%    \hrulefill\\
    The Authors
  \end{center}
}

\begin{document}

\begin{letter}{IEEE Sensors Journal \\
    Editor-in-Chief\\
    Professor Krikor B. Ozanyan}
\opening{Dear Prof. Ozanyan,}
\setlength{\parskip}{1.5ex}

We would like to submit the attached manuscript, "Design considerations for the integration of energy harvesting systems with energy-aware task schedulers", for consideration for possible publication in IEEE Sensors Journal.

The main contributions of the paper are:

\begin{itemize}
	\item The adaptation of a solar energy harvesting circuit for low power wireless sensing platforms, improving its efficiency.
	\item The test of the mentioned circuit outdoor, proving its capability for both harvesting energy and providing to an energy-aware task scheduler more frequent and accurate measurements of battery charge.
	\item The energy harvesting system simulation, showing that the task scheduler was able to run all critical tasks, while also running 1.23 times more non-critical tasks when compared with the previous approach. 
	\item An experimental correlation between the current delivered by the solar panels and the solar irradiance, which will be used for including environmental prediction capability on the energy-aware task scheduler.
\end{itemize}

Regarding the potential referees, follows the list:\\

\textbf{Mani B Srivastava}\\
EE Department, UCLA, MC \#951594, 6730-E Boelter Hall, Los Angeles, CA 90095-1594 - USA - Phone: +1-310-267-2098 Fax: +1-310-825-8282 Email: mbs@ucla.edu\\
\textbf{Luca Benini}\\
Dipartimento di Elettronica, Informatica e Sistemistica - Facolt\`a di Ingegneria - Universit\`a di Bologna Viale Risorgimento 2, 40136 Bologna - Italy - Phone: +39-051-2093782 Fax: +39-051-2093073 Email: lbenini@deis.unibo.it\\
\textbf{Manel Gasulla}\\
Instrumentation, Sensors and Interfaces Group EETAC, Universitat Polit\`ecnica de Catalunya, Barcelona~Tech, c/Esteve Terradas, 7, Room 108, 08860, Castelldefels (Barcelona) - Catalonia - Phone: +34-93-413-7092 Email: manel.gasulla@upc.edu\\
\textbf{Ricardo Bianchini}\\
Department of Computer Science, Rutgers University, 110 Frelinghuysen Road, Piscataway, NJ 08854-8019 - USA - Phone: +1-732-445-7951 / +1-732-445-2001 Ext 7951 Email: ricardob@cs.rutgers.edu\\
\textbf{Frank Bellosa}\\
Karlsruher Institut f\"ur Technologie (KIT), Lehrstuhl Systemarchitektur, Am Fasanengarten 5, Geb. 50.34, 76128 Karlsruhe, Room 158 - Germany - Phone: +49-721-608-43834 Fax: +49-721-608-47664 Email: bellosa@kit.edu\\

We would like to confirm that this article is a new and first time submission. It has not been published or accepted for publication and it is not under consideration at another journal.

\closing{Sincerely,}

\end{letter}

\end{document}
