%===============================================================================
% this is a sample file on how to format your SBAC-PAD paper
%
% the SBAC-PAD organising committee
% version 1.0 - 17/jun/99
%===============================================================================
\documentclass{SBACpaper}

\title{How to format your SBAC-PAD paper}
\author{The SBAC-PAD Organising Committee\inst1\thanks{Many hard-working people.}}
\institute{Institute of Informatics\\Federal University of Rio Grande do Sul\\Porto Alegre --- Brazil\\Comments and suggestions to: {\tt sbac99@inf.ufrgs.br}\\\vspace{3mm}Version 1.0 - 17/jun/99}

\begin{document}

\maketitle

\begin{abstract}
	This document contains the description of the formatting standard for the production of SBAC-PAD papers. Please read it carefully if you are not using any of the provided \LaTeX\ typesetting styles.
\end{abstract}

\begin{keywords}
	SBAC-PAD, \TeX, \LaTeX.
\end{keywords}

\section{Introduction}
	This document describes the formatting standards for the production of SBAC-PAD papers. It is probably of no use to you if you are using \LaTeX\ and the SBAC-PAD document class {\tt SBACpaper.cls} (with which this document has been produced), unless you want to check out the source file to see how commands are used. If this is your case, relax and let \LaTeX\ do the hard work. If not, we still strongly recommend that you consider using it. The class file, as well as other information related to SBAC-PAD, can be obtained from the Internet at {\tt http://www.inf.ufrgs.br/{\~\space}sbac99}.

\section{How to format your paper}

\subsection{Page layout}
	SBAC-PAD papers must be formatted in two-column style, with 8.5cm-wide columns and 0.8cm of inter-column space. Please keep the text centered on the (physical) page (it will depend on whether you are using A4 or Letter size paper). For the text use Times-Roman or a similar font at 10pt.

\subsection{The title information}
	The title must be centred on the top of the first page, and typeset with 25pt font. Below the title come the authors' names, typeset with 12pt font, separated by commas. Each author should be indentified, by means of a superscript number, with an institution. The institutions appear below the authors' names, typeset with 9pt font.

\subsection{Abstract and keywords}
	The abstract should appear on the left column of the first page, typeset with 9pt font. The first line contains the text ``{\em Abstract---}'' alone, in italic shape. The abstract text should be typeset in bold face.

	Keywords are formatted similarly to the abstract, except that they should start on the same line as the text ``{\em Keywords---}'', as you can see on this page.

\subsection{Sectioning}
	SBAC-PAD papers should have at most three sectioning levels. The first-level sections have a centred title, typeset with small caps and standard point size (10pt), numbered with Roman digits. Second-level titles are typeset in italic shape, left-indented and numbered with capital letters. Notice that second level numbering does {\em not\/} include the first level algarism. Third-level titles are typeset with regular font size and shape, and take the second level numbering as a prefix. Please check out the examples in this document, like the third-level section that comes now.

\subsubsection{Acknowledgements and references}
	These two special sections are formatted just like regular first-level sections, only they are not numbered. You can check them out at the end of this document.

\section{Tables and figures}
	Tables and figures must agree to the following rules.

\subsection{Tables}
	Table captions must be placed before the table itself, like table~\ref{table} on this document. The caption is separated in two centred lines, the first one containing just the word ``TABLE'', typeset in capitals, followed by the table number in Roman digits, and the second one with the caption text, typeset in small caps, 9pt font.

\begin{table}[htb]
	\begin{center}
		\caption{A simple table}
		\label{table}
		\begin{tabular}{c|c}
			\hline
			Name & Age \\
			\hline\hline
			John & 30 \\
			Mary & 20 \\
			Joseph & 15 \\
			\hline
		\end{tabular}
	\end{center}
\end{table}

\subsection{Figures}
	Figure captions come after the figure, being composed of the abbreviation ``Fig.'', followed by the figure number, and the caption text, all typeset with regular 9pt font, as you can see in figure~\ref{figure}.

\begin{figure}[htb]
	\begin{center}
		\begin{tabular}{|||c|||}
			\hline\hline\hline
			\\
			Hello, world! \\
			\\
			\hline\hline\hline
		\end{tabular}
		\caption{A dummy figure.}
		\label{figure}
	\end{center}
\end{figure}

\section{Conclusion}
	Well, this was not supposed to be a long text, so we've already come to the end. We hope we could express ourselves in a clear way, and that you do use the information in this document to submit papers to SBAC-PAD. If you have any comments or suggestions to enhance this document, please send a message to {\tt sbac99@inf.ufrgs.br}, and we'll be pleased to try to implement it.

\section*{Acknowledgements}
	We would like to thank everyone who helped us in producing this document, as well as all the people involved in the organisation of SBAC-PAD.

\begin{thebibliography}{WAR98}

\bibitem[SBA99]{sbac99} The SBAC-PAD'99 Homepage. Available at {\em
  http://www.inf.ufrgs.br/{\~\space}sbac99}.

\bibitem[BOD95]{boden:Myrinet} BODEN, N. et al. {M}yrinet: a gigabit-per-second
  local-area network. {\bf IEEE Micro}, Los Alamitos, v.15, n.1, p.29--36,
  Feb.~1995.

\bibitem[DIL95]{dillon:HomogeneousHeterogeneousNetworks} DILLON, E.; SANTOS,
  C.~Gamboa dos; GUYARD, J. Homogeneous and heterogeneous networks of
  workstations: message passing overhead. In: MPI DEVELOPERS CONFERENCE '95,
  1995, Notre-Dame, IN. {\bf Proceedings{\ldots}} 1995.

\bibitem[GEI94]{geist:PVM} GEIST, Al et al. {\bf {PVM}}: parallel virtual
  machine. Cambridge, MA: MIT Press, 1994.

\bibitem[GRO96]{gropp:HighPerformancePortableImplementation} GROPP, W.; LUSK,
  E.; DOSS, N.; SKJELLUM, A. A high-performance, portable implementation of the
  {MPI} message passing interface standard. {\bf Parallel Computing}, v.22,
  n.6, p.789--828, Sep.~1996.

\bibitem[HIP97]{tavangarian:AdvancedWorkstationCluster} HIPPER, G.;
  TAVANGARIAN, D. Advanced workstation cluster architectures for parallel
  computing. {\bf Journal of Systems Architecture}, Amsterdam, v.44, n.3/4,
  p.207--226, Dec.~1997.

\bibitem[IEE95]{ieee:FastEthernet} INSTITUTE OF ELECTRICAL AND ELECTRONIC
  ENGINEERS. {\bf Local and metropolitan area networks-supplement---media
  access control ({MAC}) parameters, physical layer, medium attachment units
  and repeater for 100{M}b/s operation, type 100{BASE}-{T} (clauses 21--30)},
  IEEE 802.3u-1995. New York, NY, 1995.

\bibitem[LAU97]{lauria:MPIFM} LAURIA, Mario; CHIEN, Andrew. {MPI}-{FM}: high
  performance {MPI} on workstation clusters. {\bf Journal of Parallel and
  Distributed Computing}, Orlando, FL, v.40, n.1, p.4--18, Jan.~1997.

\bibitem[{MPI}94]{mpiforum:MPI} {MPI FORUM}. {\bf The {MPI} message passing
  interface standard}. Knoxville: University of Tennessee, 1994.

\bibitem[PRY98]{prylli:BIP} PRYLLY, Loic; TOURANCHEAU, Bernard. {BIP}: a new
  protocol designed for high performance networking on myrinet. In:
  IPPS/SPDP'98 WORKSHOPS, 10., 1998. {\bf Proceedings{\ldots}} Springer, 1998.
  p.472--485. (Lecture Notes in Computer Science, v.1388).

\bibitem[WAR98]{thomas:PULC} WARSCHKO, Thomas~M.; BLUM, Joachim~M.; TICHY,
  Walter~F. \mbox{PULC}: {P}ara{S}tation user-level communication: design and
  overview. In: IPPS/SPDP'98 WORKSHOPS, 10., 1998. {\bf Proceedings{\ldots}}
  Springer, 1998. p.498--509. (Lecture Notes in Computer Science, v.1388).

\end{thebibliography}

\end{document}
