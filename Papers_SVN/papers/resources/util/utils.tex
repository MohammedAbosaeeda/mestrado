%%%%%%%%%%%%%%%%%%%%%%%%%%%%%%%%%%%%%%%%%%%%%%%%%%%%%%%%%%%%%%%%%%%%%%%%%%%%%%%%%%%%%%%%%%%%%%%%%%%%%%%%%%%%%
% Figure macros
%
% - All commands assume figures are in fig/ inside the root directory
% - Commands use the file name (without extension) as label unless the label is explicitly given as argument
%%%%%%%%%%%%%%%%%%%%%%%%%%%%%%%%%%%%%%%%%%%%%%%%%%%%%%%%%%%%%%%%%%%%%%%%%%%%%%%%%%%%%%%%%%%%%%%%%%%%%%%%%%%%%

% Inserts a single-column figure
% Args:
%   - placement (optional)
%   - scale factor
%   - file name (without extension)
%   - caption
% Usage:
% \figSC[h]{.33}{fig_adesd}{An overview of ADESD}
\newcommand{\figSC}[4][htbp]{
  \begin{figure}[#1] 
    \centering
    \scalebox{#2}{\includegraphics{fig/#3}}
    \caption{#4}
    \label{#3}
  \end{figure}
}

% Inserts a single-column figure. The figure is rotated by x degrees
% Args:
%   - placement (optional)
%   - scale factor
%   - file name (without extension)
%   - caption
%   - rotation 
% Usage:
% \figSC[h]{.33}{fig_adesd}{An overview of ADESD}{45}
\newcommand{\figSCR}[5][htbp]{
  \begin{figure}[#1]{\centering\scalebox{#2}{\includegraphics[angle=#5]{fig/#3}}\par}
    \caption{#4\label{#3}}
  \end{figure}
}

% Inserts a two-column figure. For single-column documents, this command has the same effect as \figSC
% Args:
%   - placement (optional)
%   - scale factor
%   - file name (without extension)
%   - caption
% Usage:
% \figTC[h]{.33}{fig_adesd}{An overview of ADESD}{45}
\newcommand{\figTC}[4][htbp]{
  \begin{figure*}[#1] 
    \centering
    \scalebox{#2}{\includegraphics{fig/#3}}
    \caption{#4}
    \label{#3}
  \end{figure*}
}

% Inserts a two-column figure composed by two subfigures. Subfigures are placed horizontally.
% Args:
%   - placement (optional)
%   - scale factor (applied to both figures)
%   - file name (without extension) for first figure
%   - file name (without extension) for second figure
%   - label (individual figures can still be referenced using the file names)
%   - caption
% Usage:
% \multfigH[h]{.35}{fig_epos_before}{fig_epos_after}{fig_epos_changes}{EPOS before (a) and after (b) modifications.}
\newcommand{\multfigtH}[6][htbp]{
\begin{figure*}[#1]
  \centering
  \subfloat[]{\label{#3}\scalebox{#2}{\includegraphics{fig/#3}}}
  \subfloat[]{\label{#4}\scalebox{#2}{\includegraphics{fig/#4}}}
  \caption{#6}
  \label{#5}
\end{figure*}
}

% Inserts a single-column figure composed by two subfigures. Subfigures are placed vertically.
% Args:
%   - placement (optional)
%   - scale factor (applied to both figures)
%   - file name (without extension) for first figure
%   - file name (without extension) for second figure
%   - label (individual figures can still be referenced using the file names)
%   - caption
% Usage:
% \multfigV[h]{.35}{fig_epos_before}{fig_epos_after}{fig_epos_changes}{EPOS before (a) and after (b) modifications.}
\newcommand{\multfigtV}[6][htbp]{
\begin{figure}[#1]
  \centering
  \subfloat[]{\label{#3}\scalebox{#2}{\includegraphics{fig/#3}}}\\
  \subfloat[]{\label{#4}\scalebox{#2}{\includegraphics{fig/#4}}}
  \caption{#6}
  \label{#5}
\end{figure}
}


% Inserts a placeholder for a future figure. 
% Args:
%   - placement (optional)
%   - figure width
%   - figure height
%   - label
%   - caption
% Usage:
% \figEMPTY[h]{3in}{4in}{fig_epos}{EPOS overview}
\newcommand{\figEMPTY}[5][htbp]{
  \begin{figure}[#1]
    \fbox{\begin{minipage}{#2}\hfill\vspace{#3}\end{minipage}}
    \centering
     \label{#4}
    \caption{#5}
  \end{figure}
}

%Two column version of \figEMPTY
\newcommand{\figEMPTYTC}[5][htbp]{
  \begin{figure*}[#1]
    \fbox{\begin{minipage}{#2}\hfill\vspace{#3}\end{minipage}}
    \centering
     \label{#4}
    \caption{#5}
  \end{figure*}
}

%%%%%%%%%%%%%%%%%%%%%%%%%%%%%%%%%%%%%%%%%%%%%%%%%%%%%%%%%%%%%%%%%%%%%%%%%%%%%%%%%%%%%%%%%%%%%%%%%%%%%%%%%%%%%%%%%%%%%
% Table macros
%
% - All commands assume tables are .tex files in tab/ inside the root directory
%      - the .tex file is supposed to contain only the definition of a single 'tabular' environment with the table data
% - All commands assume figures are in fig/ inside the root directory
% - Commands use the file name (without extension) as label unless the label is explicitly given as argument
%%%%%%%%%%%%%%%%%%%%%%%%%%%%%%%%%%%%%%%%%%%%%%%%%%%%%%%%%%%%%%%%%%%%%%%%%%%%%%%%%%%%%%%%%%%%%%%%%%%%%%%%%%%%%%%%%%%%%%

%Forces a font size for all tables. 
\newcommand{\generictablefontsize}{\footnotesize}


% Inserts a single-column table. 
% Args:
%   - placement (optional)
%   - file name (without extension)
%   - caption
% Usage:
% \tabSC[h]{tab_results}{Evaluation results}
\newcommand{\tabSC}[3][htbp]{
  \begin{table}[#1]
    \generictablefontsize
    \centering
    \input{tab/#2}
    \caption{#3}
    \label{#2}
  \end{table}
}

%Two column version of \tabSC
\newcommand{\tabTC}[3][htbp]{
  \begin{table*}[#1]
    \generictablefontsize
    \centering
    \input{tab/#2}
    \caption{#3}
    \label{#2}
  \end{table*}
}

% Inserts a single-column table using the ACM table macro.
% Args:
%   - placement (optional)
%   - file name (without extension)
%   - caption
% Usage:
% \tabTBL[h]{tab_results}{Evaluation results}
\newcommand{\tabTBL}[3][htbp]{
  \begin{table}[#1]
    \tbl{#3\label{#2}}{%
      \input{tab/#2}
    }  
  \end{table}
}

% Inserts a single-column table using an alternative ACM table macro.
% Args:
%   - placement (optional)
%   - file name (without extension)
%   - table size
%   - caption
% Usage:
% \tabACMtable[h]{tab_results}{.75\textwidth}{Evaluation results}
\newcommand{\tabACMtable}[4][htbp]{
  \begin{acmtable}{#3}[#1]
    \generictablefontsize
    \centering
    \input{tab/#2}
    \caption{#4}
    \label{#2}
  \end{acmtable}
}

% Inserts a 'figure' environment containing a table side-by-side with a figure.
% The figure file name is used as label
% Args:
%   - placement (optional)
%   - table file name (without extension)
%   - figure file name (without extension)
%   - figure scale factor (without extension)
%   - caption
% Usage:
% \tabSCxXxXfig[h]{tab_result}{fig_result_plot}{.75}{Performance data}
\newcommand{\tabSCxXxXfig}[5][htbp]{
\begin{figure}[#1]
  \centering
  \begin{minipage}[c]{0.4\textwidth}
    \generictablefontsize
    \centering
    \input{tab/#2}
  \end{minipage}
  \hspace{0.5cm}
  \begin{minipage}[c]{0.4\textwidth}
    \centering\scalebox{#4}{\includegraphics{fig/#3}}
  \end{minipage}
  \caption{figure}{#5\label{#3}}
\end{figure}
}

% Same as tabSCxXxXfig, but the figure is placed bellow the table.
\newcommand{\tabSCxXxXfigV}[5][htbp]{
\begin{figure}[#1]
  \centering
  \begin{minipage}[c]{\textwidth}
    \generictablefontsize
    \centering
    \input{tab/#2}
  \end{minipage}
  \\
  \begin{minipage}[c]{\textwidth}
    \centering\scalebox{#4}{\includegraphics{fig/#3}}
  \end{minipage}
  \caption{figure}{#5\label{#3}}
\end{figure}
}

% Same as tabSCxXxXfig, but uses the ACM table environment
\newcommand{\tabTBLxXxXfig}[5][htbp]{
\begin{figure}[#1]
  \centering
  \begin{minipage}[c]{0.4\textwidth}
    \centering
    %\tbl{#3\label{#2}}{
    \tblnocaption{*}{
      \input{tab/#2}
    }
  \end{minipage}
  \hspace{0.5cm}
  \begin{minipage}[c]{0.4\textwidth}
    \centering\scalebox{#4}{\includegraphics{fig/#3}}
  \end{minipage}
  \captionof{figure}{#5\label{#3}}
\end{figure}
}

% Same as tabTBLxXxXfig, but the figure is placed bellow the table.
\newcommand{\tabTBLxXxXfigV}[5][htbp]{
\begin{figure}[#1]
  \centering
  \begin{minipage}[c]{0.4\textwidth}
    \centering
    %\tbl{#3\label{#2}}{
    \tblnocaption{*}{
      \input{tab/#2}
    }
  \end{minipage}
  \\
  \begin{minipage}[c]{0.4\textwidth}
    \centering\scalebox{#4}{\includegraphics{fig/#3}}
  \end{minipage}
  \captionof{figure}{#5\label{#3}}
\end{figure}
}

