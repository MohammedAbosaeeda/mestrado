% $Id: etfa2006.tex,v 1.1 2005/10/12 14:45:41 cibrario Exp $
%
% Derived from  Author: vasques/sauter, Date: 2000/05/24 15:20:59,
% Revision: 1.4
%
% and updated for better adherence to typesetting guidelines on A4
% paper for ETFA 2006

\documentclass[a4paper,10pt,twocolumn]{article}
\usepackage{etfa2006}
\usepackage{times}

%-------------------------------------------------------------------------
% take the % away on next line to produce the final camera-ready version
%\pagestyle{empty}

%-------------------------------------------------------------------------
\begin{document}

\title{ETFA-2006 Author Guidelines
       for A4 ({\boldmath $21\!\!\times\!\!29.7$ cm}) Proceedings Manuscripts}

\author{First Author\\
Name of Institution 1\\
First line of Institution 1 address\\ Second line of Institution 1 address\\
FirstAuthor@institution1.edu\\
% For a paper whose authors are all at the same institution,
% omit the following lines up until the closing ``}''.
% Additional authors and addresses can be added with ``\and'',
% just like the second author.
\and
Second Author\\
Name of Institution 2\\
First line of Institution 2 address\\ Second line of Institution 2 address\\
SecondAuthor@institution2.edu\\
}

\maketitle
\thispagestyle{empty}

\begin{abstract}
  The ABSTRACT is to be in fully-justified italicized text, at the top
  of the left-hand column, below the author and affiliation
  information. Use the word "Abstract" as the title, in 12-point
  Times, boldface type, centered relative to the column, initially
  capitalized. The abstract is to be in 10-point, single-spaced type.
  The text of the abstract should not be more than 16 lines long (this
  abstract is exactly sixteen lines long).  Abstracts should be
  followed by one blank line (using the abstract style) if the text
  begins with a level 1 heading (like here) or two blank lines if the
  text begins with the normal style.

  These last five lines are just to fill out the abstract text to
  exactly 16 lines. These last five lines are just to fill out the
  abstract text to exactly 16 lines. These last five lines are just to
  fill out the abstract text to exactly 16 lines. These last five
  lines are just to fill out the abstract text...
\end{abstract}


%-------------------------------------------------------------------------
\Section{Introduction}

Please follow the steps outlined below when submitting your final
manuscript to the ETFA workshop.

%-------------------------------------------------------------------------
\Section{Instructions}

Please read the following carefully.

%-------------------------------------------------------------------------
\SubSection{Language}

All manuscripts must be in English. The use of an automatic spell-checker
is advised.

%-------------------------------------------------------------------------
\SubSection{Submitting your paper}

The final manuscript of your paper must be submitted through the
electronic submission systems on the conference web page
(http://www.action-m/etfa2006), in pdf format, before June 30th, 
2006.  Please verify if it prints properly on A4 printer paper.

The final manuscript must be accompanied by the registration form, the
signed IEEE copyright release form and also by a proof of the
registration fee payment.  Detailed registration instructions will be
available at the workshop web page.


%-------------------------------------------------------------------------
\SubSection{Margins and page numbering}

All printed material, including text, illustrations, and charts, must
be kept within a print area 17 cm wide by 24.2 cm high.  Do not write
or print anything outside the print area. Do not include any page
numbering in your manuscript.


%------------------------------------------------------------------------
\SubSection{Length of your paper}

The length of your paper is limited to 10 two-column pages. For any
extra proceeding pages please contact the ETFA 2006 team
(milena@action-m.com).


%------------------------------------------------------------------------
\SubSection{Formatting your paper}

All text must be in a two-column format. The total allowable width of
the text area is 17 cm wide by 24.2 cm high.  Columns are to be 8.1 cm
wide, with 0.8 cm space between them.  The main title (on the first
page) should begin 3 cm from the top edge of the page.  The second and
following pages should begin 2.5 cm from the top edge.  On all pages,
the bottom margin should be, at least, 3 cm from the bottom edge of
the page.


%-------------------------------------------------------------------------
\SubSection{Type-style and fonts}

Wherever Times is specified, Times Roman may also be used. If neither is
available on your word processor, please use the font closest in
appearance to Times that you have access to.

MAIN TITLE. The title should be in Times 14-point, boldface type and,
at most, 2 lines long.  Capitalize the first letter of nouns,
pronouns, verbs, adjectives, and adverbs; do not capitalize articles,
co-ordinate conjunctions, or prepositions (unless the title begins with
such a word).  Leave two blank lines after the title.

AUTHOR NAME(s) and AFFILIATION(s) are to be centered beneath the title
and printed in Times 12-point, non-boldface type.  This information is
to be followed by two blank lines.

The ABSTRACT and MAIN TEXT are to be in a two-column format.

MAIN TEXT. Type main text in 10-point Times, single-spaced. Do NOT use
double-spacing.  All paragraphs should be indented 0.42 cm (1 pica).
Make sure your text is fully justified --- that is, flush left and flush
right.  Please do not place any additional blank lines between
paragraphs. Figure and table captions should be 10-point Helvetica
boldface type as in
%
\begin{figure}[h]
   \caption{Example of caption.}
\end{figure}

\noindent Long captions should be set as in
%
\begin{figure}[h]
   \caption{Example of long caption. It is
     aligned on both sides and indented with an
     additional margin on both sides of 0.42 cm (1~pica).}
\end{figure}

\noindent Initially capitalize only the first word of section titles
and first-, second-, and third-order headings.

FIRST-ORDER HEADINGS should be Times 12-point boldface, initially
capitalized, flush left, with one blank line before, and one blank
line after.

SECOND-ORDER HEADINGS should be Times 10-point boldface, initially
capitalized, flush left, with one blank line before. If you require a
third-order heading (we discourage it), use 10-point Times, boldface,
initially capitalized, flush left, preceded by one blank line,
followed by a period and your text on the same line.


%-------------------------------------------------------------------------
\SubSection{Footnotes}

Please use footnotes sparingly and place them at the bottom of the
column on the page on which they are referenced. Use Times 8-point
type, single-spaced.

% Please place a \newpage command in the last page of the paper,
% preferably on a \SubSection or paragraph boundary, in order to
% balance the left and right columns.
\newpage


%-------------------------------------------------------------------------
\SubSection{Illustrations and graphs}

All illustrations and graphs should be centered and must be in place
in the manuscript. If you must use photographs, they must be scanned
and pasted onto your manuscript. Please supply the best quality
photographs and illustrations possible.


%-------------------------------------------------------------------------
\SubSection{References}

List and number all bibliographical references in 9-point Times,
single-spaced, at the end of your paper. When referenced in the text,
enclose the citation number in square brackets, for
example~\cite{ex1}.  Where appropriate, include the name(s) of editors
of referenced books.


%-------------------------------------------------------------------------
\nocite{ex1,ex2, ex3, ex4}
\bibliographystyle{etfa2006}
\bibliography{etfa2006}

\end{document}
