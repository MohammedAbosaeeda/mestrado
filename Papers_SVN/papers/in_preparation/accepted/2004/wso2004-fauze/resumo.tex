\documentclass[portuguese,a4paper,12pt]{article}

\usepackage{sbc2003}
\usepackage{graphicx,url}
\usepackage{epsfig}

\usepackage[brazil]{babel}   
\usepackage[latin1]{inputenc}  

\usepackage[tmargin=3.0cm, bmargin=2.5cm, lmargin=2.5cm, rmargin=2.5cm,nohead]{geometry} 

%%%%%%%%%%%%%%%%%%%%%%%%%%%%%%%%%%%%%%%%%%%%%%%%%%%%%%%%%%%%%%%%%%%%%%%%%%%%%%%%%%%%%%

\usepackage{listings}
\lstset{keywordstyle=\bfseries, flexiblecolumns=true}
\lstloadlanguages{[ANSI]C++,HTML}
\lstdefinelanguage{XML} {
  keywords={xml,version,DOCTYPE,SYSTEM,EPOSConfig,family,member,name,type,
  default,pos,pre}}
\lstdefinestyle{prg} {basicstyle=\small\sffamily, lineskip=-0.2ex}
\lstdefinestyle{prgbox} {basicstyle=\small\sffamily lineskip=-0.2ex}
\lstdefinestyle{inlineprg} {basicstyle=\small\sffamily}

\newcommand{\fig}[3][htbp]{
  \begin{figure}[#1] {\centering{\includegraphics{fig/#2}}\par}
    \caption{#3\label{fig:#2}}
  \end{figure}
}

\newcommand{\prg}[4][h]{
  \begin{figure}[#1]
    \vspace{\parskip}
    \framebox[\textwidth][c]{
      \lstinputlisting[language=#2,style=prg,
                       basicstyle=\fontfamily{pcr}\fontseries{m}\selectfont\footnotesize
                      ]{prg/#3.prg}}
    \vspace{0.3\parskip}
    \caption{#4\label{prg:#3}}
  \end{figure}
}

\newcommand{\bigprg}[4][h]{
  \begin{figure}[#1]
    \vspace{\parskip}
    \framebox[\textwidth][c]{
      \lstinputlisting[language=#2,style=prg,
                       basicstyle=\fontfamily{pcr}\fontseries{m}\selectfont\footnotesize
                      ]{prg/#3.prg}}
    \vspace{0.3\parskip}
    \caption{#4\label{prg:#3}}
  \end{figure}
}

\newcommand{\prgbox}[3][h]{
  \begin{figure}[#1]
  \vspace{\parskip}
  \makebox[\textwidth][c]{
    \lstinputlisting[language=#2,style=prgbox
                     basicstyle=\fontfamily{pcr}\fontseries{m}\selectfont\footnotesize
                    ]{prg/#3.prg}}
  \vspace{0.3\parskip}
  \end{figure}
}

\newcommand{\inlineprg}[2][C++]{
  \vspace{\parskip}
  \noindent\makebox[\textwidth][c]{
    \begin{lstlisting}[language=#1,style=inlineprg]{}
      ^^J #2
    \end{lstlisting}}
  \vspace{0.3\parskip}
}

\def\textprg{\lstinline[language=C++,style=inlineprg]}


%%%%%%%%%%%%%%%%%%%%%%%%%%%%%%%%%%%%%%%%%%%%%%%%%%%%%%%%%%%%%%%%%%%%%%%%%%%%%%%%%%%%%%

\sloppy

\title{Portabilidade em Sistemas Operacionais Baseados em Componentes de Software}

\author{Fauze Val�rio Polpeta\inst{1}, Ant�nio Augusto Fr�hlich\inst{1}\\
        Arliones Stevert Hoeller J�nior\inst{1}, Tiago Stein D�Agostini\inst{1}}

\address{Laborat�rio de Integra��o Software e Hardware -- 
         Universidade Federal de Santa Catarina\\
         PO Box 476\\
         88049-900 Florian�polis - SC, Brazil\\
         \texttt{\{fauze,guto,arliones,tiago\}@lisha.ufsc.br}\\
         \texttt{http://www.lisha.ufsc.br/$\sim$\{fauze,guto,arliones,tiago\}}
}

\begin{document}

\maketitle

\begin{abstract}
In this article we elaborate on portability in component-based
operating systems, focusing in the \emph{hardware mediator} construct
proposed by Fr�hlich in the \emph{Application-Oriented System Design}
method. Differently from hardware abstraction layers and virtual
machines, hardware mediators have the ability to establish an
interface contract between the hardware and the operating system
components and yet incur in very little overhead when comparing to
traditional portability approaches.
  
The use of hardware mediators in the \textsc{Epos} system corroborates
the portability claims associated to the techniques explained in this
article, for it enabled \textsc{Epos} to be easily ported across very
distinct architectures, such as the \texttt{IA-32} and the
\texttt{H8}, without any modification in its software components.
\end{abstract}

\begin{resumo}
Este artigo apresenta um estudo realizado sobre portabilidade em
sistemas operacionais baseados em componentes de software. Tem como
principal foco \emph{mediadores de hardware}, um artefato de software
proposto por Fr�hlich em sua metodologia \emph{Application Oriented
System Design}. Diferentemente de camadas de abstra��o de hardware e
m�quinas virtuais, mediadores de hardware permitem que seja
estabelecido um contrato de interface entre o hardware e os
componentes do sistema operacional, garantindo a indep�ncia
de plataforma sem \emph{overhead} significativo quando comparado aos
mecanismos tradicionais de portabilidade.

O uso de mediadores de hardware no sistema \textsc{Epos} permite
comprovar o grau de portabilidade associado �s t�cnicas apresentadas
neste artigo, visto que este sistema foi facilmente portado para
arquiteturas distintas como \texttt{IA-32} e \texttt{H8} sem que
qualquer modifica��o nos componentes do sistema operacional fosse
realizada.
\end{resumo}

\end{document}
