\section{Conclusão}

Este trabalho apresentou as principais características e funcionalidades da
ferramenta de análise de tempo de execução de pior caso (WCET) \boundt, da
Tidorum Ltd. Uma análise geral da ferramenta foi realizada seguida da
apresentação de um exemplo disponível junto à documentação da ferramenta.

No contexto da disciplina DAS9007 - Sistemas de Tempo Real do Programa de
Pós-Graduação em Engenharia de Automação e Sistemas (PGEAS) da UFSC, um desafio
foi proposto em que alunos, divididos em grupos, deveriam estudar ferramentas de
estimativa de WCET e propôr aplicações-desafio aos demais grupos. Cada grupo,
então, deveria analisar as aplicações propostas com as ferramentas que
estudaram.

Neste trabalho, foi posposta uma aplicação baseada em alguns dos programas
utilizados pelo benchmark de WCET do MRTC, da Universidade de Mälardalen,
Suécia. A análise dos demais desafios propostos também foi realizada utilizando
o Bound-T, o que revelou uma limitação da ferramenta no que diz respeito a
integração com código mais complexo presente em bibliotecas do sistema (e.g.,
$libm$ e $libc$). A ferramenta também apresentou alguns bugs. Alguns destes bugs
foram, de algum modo, contornados. Outros problemas, contudo, inviabilizaram o
uso da ferramenta para a análise de dois dos desafios propostos (Transformada
Discreta de Fourrier e Codec ADPCM).

Como resultado do estudo, percebemos que as ferramentas disponíveis atualmente,
de modo geral, não apresentam um nível de maturidade no qual sua aplicação a
qualquer programa seja possível. Mesmo nas ferramentas comerciais mais
utilizadas (categoria na qual o \boundt~se insere), uma série de anotações
precisam ser feitas para viabilizar a análise. Embora muitas destas limitações
existam devido a questões teóricas relacionadas a problemas não computáveis
(e.g., \textit{The Halting Problem}), muitas limitações devem-se a bugs ou
incompletudes. Em alguns casos, chegar a limites razoáveis para estas anotações
pode ser uma tarefa bastante difícil, o que leva, frequentemente, ao uso de
margens seguras, fazendo da estimativa de WCET um valor muito além do valor
real, tornando sistemas de tempo-real superdimensionados.

Como trabalho futuro, pretende-se utilizar o \boundt~para permitir a análise de
WCET de aplicações utilizando o sistema operacional
\textsc{Epos}~\cite{Marcondes:WSO:2009}. Análises iniciais já apresentaram
diversos desafios nesta tarefa, como na interação entre o sistema operacional e
periféricos e na implementação de funções clássicas de sistemas operacionais
como sincronizadores com espera ociosa e troca de contexto. Espera-se, através
deste estudo, viabilizar uma estrutura de anotações que torne viável a análise
do WCET do código do sistema operacional \textsc{Epos}, viabilizando assim a
análise de suas aplicações.
