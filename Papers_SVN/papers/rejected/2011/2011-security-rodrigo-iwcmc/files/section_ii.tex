\section{What is just wrong with TCP/IP}
\label{sec:tcp_ip}

The TCP/IP reference model is the foundation of the Internet today.
Composed of two main protocols, \emph{Transmission Control Protocol} (TCP) runs over \emph{Internet Protocol} (IP), and provides a connection-oriented service between hosts in order to guarantee data delivery in the correct sequence.
When TCP/IP was designed, security was not a primary concern.
However, since its inception, the lack of trusted communication in TCP/IP has become more of a problem.
The "patches" on the Internet architecture show that the initial project no longer fits the current needs in the network~\cite{Doraswamy:1999}\cite{Perkins:1997}.
Furthermore, the current Internet architecture already has many problems still unsolved in terms of security, scalability, and mobility, which impedes the fulfillment of the requirements of new applications and services.

IP addressing is identified as one of the main challenges for the Future Internet.
There are a number of structural principles that are at odds with the current requirements of the network.
Among the main associated problems, the address shortage and the overloaded semantics of IP address are the major concern in the current Internet architecture.
Solutions are adopted to reduce the shortage of addresses through mechanisms such as \emph{Network Address Translation} (NAT).
The use of NAT allows multiple devices to access the network simultaneously using a single valid address for the Internet.
However, this technique contradicts some of the Internet architecture requirements, once an IP address used by a connected device is no longer global.
Therefore, it becomes necessary to insert in the network an intermediary contact element, thus breaking the end-to-end principle, thereby also breaking the IP semantics.

One of the main problems also related to IP addressing is the overloading semantics of addresses, which can be both an identifier and a node locator.
Due to this overloading, support for mobile nodes and ad hoc networks has become a challenge for the Internet \cite{Meisel:2010}.
Another aspect related to the IP semantics is the naming of service entities.
Both IP address and DNS names are linked to preexisting structures.
As a result, the use of DNS and IP to name services implies failures in the sense that they are associated with a machine rather than with a denomination.
Therefore, the main consequence of changing a service from a machine is that the service name may no longer be valid.
Moreover, it is difficult to replicate data and services in the network, because the names rather than just identifying the service, also identify the location and the associated domain.

Differently, NDN aims at creating a flexible Internet architecture by naming data, and routing data by names in order to resolve the usual problems in the current Internet architecture.
Since IP complicates the replication of data and services, and performs open loop data delivery, transport protocols have been needed to provide unicast traffic balance.
In NDN, each data is uniquely named, differently from IP, which usually enforces the single-path forwarding constraints in the routing.
NDN names are opaque to the network, thus allowing each application to choose the naming scheme that fits its needs.
Therefore, NDN allows any node to freely use all of its connectivity to solicit or distribute data, and also removes the information asymmetries that give disproportionate control over routes.
As a result, usual addressing problems from the IP architecture can be eliminated.
In NDN, there is no address exhaustion problem since the naming scheme is unbounded.
The hosts do not need to expose their address in order to offer content, thus avoiding NAT traversal problem.
Finally, address assignment and management is no longer required in local networks, which is especially empowering for sensor networks.

Users, service providers, industry, and application developers have expressed increasing concern about security aspects \cite{Rexford:2010}.
The forms of attack are becoming increasingly sophisticated and adaptable to defense system evolutions.
Nevertheless, the current Internet architecture does not provide any mechanism to limit the behavior of malicious stations and protect the right stations.
Furthermore, the current architecture does not provide any protection against attacks on network elements themselves.
A major cause that motivates all the current security problems is the lack of trusted communication in the design of network architecture.
Security must be built into the architecture, but it is actually an afterthought in the current Internet architecture.
The current security approach is a model of armoring the channel between two IP addresses, but it rarely meets the end-to-end security needs of data producers and consumers.
And the premise of not being able to change the core of the network hinders the implementation of security mechanisms.
Most of routing protocols for the current Internet do not use security mechanisms, while strong authentication and monitoring systems are very distant from what is needed in terms of delay, scalability, among others.

In NDN, all data is secured end-to-end and the naming scheme provides essential context for trusted communication by enabling the development of more robust applications.
Moreover, NDN routing can be done in a similar fashion to IP routing.
However, instead of announcing IP prefixes, an NDN router announces name prefixes.
Signing all data and routing control messages prevents them from being spoofed or tampered with.
Furthermore, the fact that NDN messages can only talk about data makes it difficult to send malicious packets to a particular target.
