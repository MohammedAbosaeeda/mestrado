
\section{Conclusion}
\label{sec:conclusion}

% Summarize paper
This paper presented the evaluation of four heuristic approaches for energy allocation in real-time wireless sensor network systems that harvest energy from the environment.
A specific task model is assumed in which system tasks may be classified as critical or noncritical.
Critical parts are hard real-time tasks and must be assured energy and processing time for execution.
Noncritical parts are best-effort tasks and have they execution granted when there is enough energy for doing so.
The four heuristics were evaluated in a realistic simulation environment considering a dense, mobile, wireless sensor network application.
Simulations show that all heuristics respect the real-time constraints of the system.
The efficiency metric used for evaluation was the rate of best-effort tasks executed by the system.

% Summarize contribution results
Results show that two of the heuristics, \bwr~(BWR) and \bwar~(BWAR), behave in much better way when compared with others.
BWR shows a highly stable behavior at the cost of higher demand of battery size.
BWAR, although showing a large operating amplitude, was able to execute more best-effort tasks than the others.
Also, during the whole year, the rate calculated by BWAR was never zero.

% On-going/future work
Ongoing work is evaluating the same approaches using wind turbines instead of solar panels as the energy source.
It is expected that the behavior of the schedulers will be considerably distinct from the ones shown here because of the chaotic behavior of wind speed.
