% ------------------------------------------------------------------------------
\section{Introduction} \label{intro}
% + Introduction
% 
% The very first letter is a 2 line initial drop letter followed
% by the rest of the first word in caps.
%
% form to use if the first word consists of a single letter:
% \IEEEPARstart{A}{demo} file is ....
%
% form to use if you need the single drop letter followed by
% normal text (unknown if ever used by IEEE):
% \IEEEPARstart{A}{}demo file is ....
%
% Some journals put the first two words in caps:
% \IEEEPARstart{T}{his demo} file is ....
%
% Here we have the typical use of a "T" for an initial drop letter
% and "HIS" in caps to complete the first word.
% \IEEEPARstart{T}{his} demo file is intended.

\IEEEPARstart{E}{nergy} consumption is a determining factor when designing wireless sensor networks.
As a consequence, battery lifetime is a limitation on the development of such systems.
Therefore, the idea of extracting energy from the environment has become attractive.
Looking to the energy consumption problem, the intelligent usage of the stored energy contributes to extend the sensor nodes' longevity.
Consequently, energy schedulers have been developed in order to adequately assess the energy consumption and adapt the system accordingly to the available amount of energy.
The purpose of this work is to adapt a solar energy harvesting circuit to supply energy to low power wireless platforms, i.e., those that operate under $50~mW$.
Simultaneously, we aim at improving the performance of the energy-aware task scheduler in wireless sensor network systems by providing fine-grained battery and environmental monitoring.

Among a number of energy sources that have been studied so far, solar has proved to be one of the most effective~\cite{Roundy:2003}.
The solar energy conversion through photovoltaic (PV) cells is better performed at an optimum operating voltage.
Operating a solar panel on this voltage results in transferring to the system the maximum amount of power available.
In this context, \emph{maximum power point tracker circuits} (MPPT) have been proposed.
The drawback is that MPPT circuitry may introduce losses to a solar harvesting system.
Concerning low-power applications, it may be more energy efficient to have a good matching between the solar panel and the energy storage unit~\cite{Raghunathan:2005}.
This well matched system is than able to work close to the maximum power point with less power loss.

In this work, an evaluation of the proposed harvesting circuit is performed in order to show improvements on an energy-aware task scheduler~\cite{Hoeller:SMC:2011}.
It is shown that the combination of the proposed circuit with the cited scheduler not only extended the longevity of the wireless sensor network, but also improved system quality.

The paper is organized as follows:
Section~\ref{fund} presents the fundamentals of solar energy harvesting and energy-aware task scheduler.
Section~\ref{design} discusses the design of the harvesting circuit under the perspective of low power wireless platforms.
Section~\ref{case} presents the evaluation of the harvesting circuit and a case study showing the improvements on system quality.
Finally, section~\ref{concl} closes the paper.

% ------------------------------------------------------------------------------
