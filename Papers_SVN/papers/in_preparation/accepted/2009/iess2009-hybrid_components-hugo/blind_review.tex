\documentclass[11pt]{article}

\setlength\topmargin{0in}
\setlength\headheight{0in}
\setlength\headsep{0in}
\setlength\textheight{9.0in}
\setlength\textwidth{6.5in}
\setlength\oddsidemargin{0in}
\setlength\evensidemargin{0in}
\setlength\parindent{0.25in}
\setlength\parskip{0.0in}

\usepackage{graphicx,url}
\usepackage[latin1]{inputenc}

\let\algorithm\relax
\let\endalgorithm\relax
\usepackage[figure,linesnumbered]{algorithm2e}

\usepackage{listings}
\lstset{keywordstyle=\bfseries, flexiblecolumns=true,showstringspaces=false,breaklines=true, frame=single,numbers=left}   \lstloadlanguages{[ANSI]C++,HTML}
\lstdefinestyle{prg} {basicstyle=\tiny, lineskip=-0.2ex, showspaces=false}

\newcommand{\prg}[3][tbp]{
\begin{figure}[#1]
    \lstinputlisting[language=C++,style=prg]{fig/#2.cc}
  \caption{#3\label{prg:#2}}
\end{figure}
}

\newcommand{\fig}[4][ht]{
  \begin{figure}[#1] {\centering\scalebox{#2}{\includegraphics{fig/#3}}\par}
    \caption{#4\label{fig:#3}}
  \end{figure}
}
\sloppy

\title{A Hybrid Hardware and Software Component Architecture for Embedded System Design}

\author{Blind-Review}
\date{}

\begin{document}

\maketitle

\begin{abstract}
Embedded systems are increasing in complexity, while several metrics such as time-to-market, reliability, safety and performance should be considered during the design of such systems. Frequently, the design of such systems imposes an integrated hardware/software design to cope with such metrics. In this sense, a component-based design methodology with components that can freely migrate through hardware and software domain benefits the design process of such systems. Moreover, a design based on higher-level abstraction enables a better design space exploration between several hardware and software compositions. We define hybrid hardware and software components as a development artifact that can be deployed by different combinations of hardware and software elements. In this paper, we will present an architecture for developing such components in order to construct a repository of components that can migrate between the hardware and software domains to meet the design system requirements.
\end{abstract}

\section{Introduction}

% Embedded systems are getting complex
Several challenges arise on the design and implementation of current embedded systems. The applications themselves are becoming increasingly complex as the advances of the semiconductor industry enabled more sophisticated use of computational resources on a spread of market appliances. If at one side the applications are becoming more complex, on the other side the pressure of the market for rapid development of those systems makes the task of designing them a challenge.  

% Design is getting harder -> MPSoC is often required
The constraints imposed to such systems, in terms of functionality, performance, energy consumption, cost, reliability and time-to-market are getting tighter. Therefore, the task of designing such systems is becoming increasingly important and difficult at the same time~\cite{Pop:2005}. Moreover, those systems could require an integrated hardware and software design that can be realized by a myriad of distinct computational architectures, ranging from simple 8-bit microcontrollers, digital signal processors (\textsc{DSP}), programmable logic devices (\textsc{FPGA}) to dedicated chips (\textsc{ASIC}) that provides the system functionality.
%
% Component based design...
In order to cope with these challenges, several methodologies were proposed by the hardware and software co-design community over the last decade. One approach to deal with these challenges is based on the concept of build a system based on the assembly of pre-validated components, like the \emph{Platform-based design}~\cite{Vincentelli:2001}. However, designing such reusable artifacts to meet the requirements of several distinct applications should be as challenging as well~\cite{Vincentelli:2004}.
%
% Usually HW/SW partition is done early when requirements are not quite know
The partition of the system between hardware and software also plays a key role in the design process. Usually, this mapping of system functionality into hardware implementation and software implementation is done in the initial phases of the specification of the system, enabling the development and implementation of the hardware and software occur concurrently. This approach however, is not ideal, as a mistake on this beginning phase of the project could lead to a re-engineering of the system, which can sometimes be too costly.

% Migration of components between hw/sw
Our proposal to deal with these challenges is to use refined engineering techniques to build a repository of components that are flexible enough to provide components free of implementation domain. In this scenario, embedded systems could be built on such components that can be migrated to hardware or software domains without major redesigns to the system, according to the requirements of the application.
%
% AOSD -> foundation for this scenario (Not talking directly to AOSD because the blind review)
To enable the construction of those flexible components, a set of engineering techniques was used. Domain Engineering was used to identify a set of representative entities within a domain. Such entities are modeled using Object-oriented design, Family-based design, and Aspect-orientation. A framework models the composition rules of such components, using advance techniques such as generative programming to ensure a low overhead to the composed system.

% Describe structure of the paper
The next section will present the related work on hardware and software co-design. In section~\ref{sec:hc} the proposed architecture of hybrid hardware and software components is laid out.  Three components built with this architecture are described and evaluated in section ~\ref{sec:case_studies}, followed by the conclusion of this paper.

\section{Related Work}
\label{sec:related_work}

% ES Design
Several methodologies propose the integration of tools and design phases of embedded systems, to promote a rapid-prototyping and design of such systems. 
% Metropolis
Metropolis~\cite{Balarin:2003} proposes the use of a unified framework, based on a metamodel with formal semantics that developers can use to capture designs, and an environment to support the simulation, formal analysis and synthesis of complex electronic systems, providing an adequate support to the design chain.

% Ptolemy 
The Ptolemy project~\cite{Eker:2003} focuses on the modeling design of heterogeneous systems, as mostly modern embedded computing systems are heterogeneous in the sense of being composed of subsystems with very different characteristics among their interactions as synchronous or asynchronous calls, buffered or unbuffered, etc. To deal with such heterogeneity, Ptolemy proposes a model structure and semantic framework that support several models of computations, such as \emph{Communicating Sequential Processes}, \emph{Continuous Time}, \emph{Discrete Events}, \emph{Process Network}, and \emph{Synchronous Dataflow}.

% PeaCE
While most of existent hardware-software co-design tools focus mainly on the hw-sw co-simulation to build a virtual prototyping environment for performing software design and system verification, PeaCE~\cite{Ha:2007} appear as an extension to Ptolemy to provide a full-fledged co-design environment from functional simulation to system synthesis. It is targeted for multimedia applications with real-time constraints, specifying the system behavior with a heterogeneous composition of three models of computations and exploiting features of formal models maximally during the design process.

% MPSoC Component Based Desing (Jerraya / Ces�rio) - OK
The use of a component-based design approach for multiprocessor SoC platforms are presented by~\cite{Cesario:2002}. This work proposes a unified methodology for automatic integration of heterogeneous pre-designed components effectively. A design flow called ROSES~\cite{Dziri:2004}, uses this methodology to generate hardware, software, and functional interface sub-systems automatically starting from a system architectural model.

% Unified Inter-Communication Architecture for Systems-on-Chip (J.C. Lopez)
Another approach to deal with the component communication on multiprocessors SoC is based on the distributed system paradigm to provide a unified abstraction for both hardware and software components~\cite{Rincon:2007} that is deeply inspired by the concepts of communication objects standards such as CORBA. This approach uses the generation of a proxy-skeleton scheme to provide transparent communication architecture of the components in both domains (hardware and software).

% Hybrid Threads
HThreads~\cite{Anderson:2006}, focus on specifying and unifying the programming model for hybrid CPU/FPGA systems, under the umbrella of multithreading programming. In this sense, they provide what they call \emph{hardware thread interface} (HWTI) which supports the generalized pthreads API semantics, allowing for the passing of abstract data types between hardware and software. This approach enabled the migration of threads to the hardware domain, to be implemented as hardware accelerators. The HWTI interface provides access to the same system calls available to software threads, a globally distributed memory to support pointers, a generalized function call model including recursion, local variable declaration, dynamic memory allocation, and a remote procedural call model that enables hardware threads access to any library function~\cite{Anderson:2007}. 

%%%%%%%%%%%%%%%%%%%%%%%%%%%%%%%%%%%%%%%%%%%%%%%%%%%%%%%%%%%%
%2 - Componentes H�bridos
\section{Hybrid Hw/Sw Components}
\label{sec:hc}

%O que s�o
Hybrid Hw/Sw components can be realized as a mixture of hardware and
software implementation that can vary from a component that realizes all your
functionality in hardware to an implementation fully realized in software. 
The figure~\ref{fig:hc_generic} depicts this concept, illustrating a full hardware 
implementation (A), a full software implementation (C) and a mixture of both (B). 

\fig{.8}{hc_generic}{Hybrid components.}

%por que existem
In this way, a system composed by such kind of components can adapt to its requirements
according to the actual implementation selected to realize a specific interface, used by 
the application. For instance, a mobile application that requires an efficient use of energy
could select an implementation that optimize such a metric to the detriment of others (i.e. cost), while
applications that have an unlimited source of power could select implementations that benefit
other metrics (i.e. performance, costs).

%como s�o realizados ????
To illustrate such components, consider a very common component of most embedded system: 
a task scheduler. Such a component is mainly represented by a queue of elements that are 
ready to receive a resource from the system, usually the CPU, an ordering algorithm to establish 
the order in which the elements on the queue will receive the resource, and a timer responsible 
for managing the amount of time in which each element will receive of the resource (\emph{quantum}). 
A hybrid hw/sw component could arise by pushing those elements to hardware or software domains in several combinations. As an example, the queue and the ordering algorithm could be realized in hardware to improve performance,  at the expense of cost. The realization of the time management by the scheduler in hardware could also reduce the occurrence of interruptions of the CPU (to deal with time ticks that will not cause a rescheduling) that could lead to decreased energy consumption for instance.

%%%%%%%%%%%%%%%%%%%%%%%%%%%%%%%%%%%%%%%%%%%%%%%%%%%%%%%%%%%%
%3 - Projetando Componentes H�bridos Reutiliz�veis
\section*{Design reusable hybrid hw/sw components}
\label{sec:design}
%Texto atual da AOSD

Most of the methodologies in the design of embedded systems focus the 
design of each system independently. Although most of them consider 
the use and selection of pre-existent components already in the initial phases 
of the design process, most of them do not address how to guide the system 
development process to yield components that can be effectively reused on 
further projects. In fact, the construction of components that can be extremely 
reusable is one of the most challenging issues in 
\emph{Platform-based design}~\cite{Vincentelli:2004}.
%
% AOSD concepts  - Domain Engineering
Our proposal rests on the foundation of refined software engineering techniques to 
overcome such challenges and bring not only a flexible interface of components 
that can freely migrate between hardware and software domains, but also foster 
the reuse of the captured knowledge from previous projects in the form of 
reusable components.

To achieve such a degree of flexibility, it is essential to use a domain engineering 
methodology that elaborates on the well-known domain decomposition strategies, 
allied with Family-Based Design~(FBD) and Object-Orientation~(OO). In such 
an approach, the use of \emph{commonality} and \emph{variability} analysis captures 
the usage variations of the elements of the domain, than can be further factored 
out as aspects. In this sense, the use of such techniques guides the domain 
engineering towards families of components, of which execution scenario 
dependencies are factored out as "aspects" and external relationships are captured 
in a component framework, addressing consistently some of the most relevant issues 
in component-based design, such as reusability, complexity management and composability. 

% fig -> generic components-based system / AOSD domain  decomposition. ?
\fig{.4}{aosd_full}{Overview of domain decomposition.}

Figure~\ref{fig:aosd_full} illustrate the main elements of %AOSD blind review.
domain decomposition, with domain entities being captured as
abstractions that are organized in families and exported to users
through comprehensive interfaces. Abstractions designate scenario
independent components, since scenario dependencies are captured as
aspects during design. Subsequent factorization captures configurable
features as constructs that can be reused throughout the family.
Relationships between families of abstractions delineate a component
framework. Each of these elements is subsequently modeled according
to the guidelines of Object-Oriented Design~(OOD).

% Hw mediators as portability artifacts
The portability of such components, and thus of applications that use them, across distinct 
hardware platforms is achieved by means of a construct called \emph{``hardware mediator''}, 
which defines a hardware/software interface contract between higher-level 
components and the hardware. %~(\cite{Polpeta:2004}). BlindReview
Hardware mediators are meant to be implemented using Generative
Programming techniques~\cite{Czarnecki:2000} and, instead of building an
ordinary \emph{Hardware Abstraction Layer}~(HAL), implicitly adapt
existing hardware components to match the required interface by adding
software to client components. For example, the hardware mediator for a
hardware component that already presents the desired interface would be
eliminated totally during the system generation process; while the
hardware mediator for a hardware component that does not provide all the
desired functionality could exceed the role of interface and include
software elements to complement the hardware functionality.

Indirectly, the concept of hardware mediator defines a kind of
\emph{hybrid hardware/software component}, since different mediator
implementations can exist for the same hardware component, each designed
around a particular set of goals such as performance and energy
efficiency.  If the hardware platform itself can be synthesized---as is
the case with IP-based platforms---then the notion of a hybrid component
becomes even more appealing, since some hardware mediators could exist
in different pre-validated combinations of hardware and software.

% The role of domain engineering for reusable artifacts (no matter if sw or hw)
In fact, the flexibility that underlies the hardware mediator concepts is yielded 
from the domain decomposition processes that established a model that 
represents elements of the domain (concepts) and was not driven by a specific 
implementation of these concepts (no matter if they are hardware or software). In other 
words, this means that the interface provided by these components is free 
of implementation domain, and thus can be realized either as hardware or 
as software.

%%%%%%%%%%%%%%%%%%%%%%%%%%%%%%%%%%%%%%%%%%%%%%%%%%%%%%%%%%%%
\section*{Hybrid Hw/Sw Component Architecture} 
% Caracteriza��o
% princ�pios de projeto/implementa��o.

% Structure of hardware / software components
%Besides the internal design of hardware components it's free to assume any 
%structure as it can always be adapted by means of a hardware mediator, 
%the design of such implementations following some guidelines that can 
%benefit the system, as it avoids the implicit overhead of adaptations on its 
%mediator. 

%As such, a generic structure to implement the hardware 
%realizations of hybrid components is depicted on figure~\ref{fig:hw_structure}.
% fig -> generic component on HW
%\fig{.75}{hw_structure}{Structure of HW implementation of a hybrid component.}

%This structure is strongly based on the object-orientation design, with objects that 
%encapsulate data (attributes) accessed through its operations (methods). The 
%structure is divided into the following modules:  a \emph{interface} based on a set of 
%registers exported to the interconnection infrastructure (i.e. bus), a \emph{internal 
%memory} used to hold internal data of the instantiated objects provided by the 
%component, a \emph{resource allocator} used to manage the memory used by the 
%objects,  a set of \emph{modules} responsible for implementing each operation 
%provided by the component, and a \emph{controller} that performs the bridge between 
%the \emph{interface} and the internal structures of the component.

% Aspect around componentization for both domain - maintaing communication semantics
%Furthermore, 
In order to provide the seamless migration of the components between 
both implementation domains, not only should the interface be able to be realized 
in both domains, but also behave equally in both domains, avoiding 
the refactoring of the clients that use them. Analyzing how client components 
interact with their providers, we observed three distinct behaviors patterns:

\begin{description}
% Synchronous 
\item [Synchronous:] observed in components with sequential objects that
 only perform tasks when their methods are explicitly invoked; client
 components are blocked on the method call until service is completed. Such behavior 
 is intrinsic to software components, and can be preserved in hardware by means of its 
 hardware mediator that can block client requests until the service is completed. 
 The figure~\ref{fig:activity_sync} illustrates an UML activity diagram of such behavior 
 when the component is implemented on the hardware domain. The client requests 
 the service to the component, which is executed while the client stands polling a 
 register to be notified upon the finish of the service (busy waiting), or suspend 
 itself until the hardware interrupts the CPU to resume the suspended 
 client (idle waiting).

\fig{.65}{activity_sync}{UML activity diagram of Synchronous Components.}

% Asynchrounous
\item [Asynchronous:] observed in components around active objects that perform 
tasks when their methods are explicitly invoked, but do not block the execution of 
the client component; some sort of callback mechanism is used to notify the client 
about service completion. Typical examples for this class of hybrid components 
are I/O related subsystems, such as file systems and communication systems. 
The figure ~\ref{fig:activity_async}  illustrates an UML activity diagram of such behavior. 
The client register a callback function if this is not already set (i.e. at initialization of 
the component) and then requests the service. Once the service is accepted by the 
component (i.e. the component is not servicing another request) the client continues 
it execution, while the component executes the service. When the requested 
service is finished, the component will call the registered call back function, which 
can be achieved by a simple call if the component is implemented in software or 
through an interrupt if the component is in software.

\fig{.65}{activity_async}{UML activity diagram of Asynchronous Components.}

% Autonomous
\item [Autonomous:] components implemented as active objects that perform tasks 
independently of clients; the services provided by the component are either ubiquitous 
or generate events for clients.  Its behavior is depicted in figure~\ref{fig:activity_autonomous}, 
by a loop of service execution and event generation activities that could be 
interrupted by external events. In this scenario, moving a hybrid component from software 
to hardware is feasible as long as the triggering events can be forward to the hardware 
component. The other way around this is usually accomplished by having the 
hardware to generate interrupts to notify other components about general system 
status changes that might result from autonomous activities.

\fig{.65}{activity_autonomous}{UML activity diagram of Autonomous Components.}

\end{description}

The following section present three case studies that were designed according to 
the proposed architecture of components, and represent these three behavior patterns.

\section{Case studies}
\label{sec:case_studies}

% Several components was built to explore the several aspect of the architecture
To evaluate the proposed hybrid hw/sw component architecture, three components
were developed in both implementation domains, each one representing a specific 
behavior. A \emph{Semaphore} component, that behaves as a synchronous 
component, a \emph{Scheduler} that behaves as an autonomous component and an
\emph{Alarm} component that behaves as an asynchronous component. The following
sections describe the implementation of those components.

\subsection{Semaphore}
\label{sec:cs_semaphore}

% Describe the component interface
A semaphore is a synchronization tool represented by an integer variable
that can be accessed only by two \emph{atomic} operations: \texttt{p}
(from the Dutch \emph{proberen}, to test) and \texttt{v} (from Dutch
\emph{verhogen}, to increment). 
%
%The figure~\ref{fig:semaphore_class_diagram} illustrate
%the \emph{Synchronization} family of components, of which the \emph{Semaphore}
%component belongs.
%\fig{.4}{semaphore_class_diagram}{Synchronization family of components.}
%
% implementantion in software
The software implementation of the component is realized by an object that
aggregates the semaphore variable and a list of blocked threads that are waiting
for the resource guarded by the semaphore abstraction. To guarantee the atomicity
of its methods, the software implementation of the semaphore components
uses the bus locking mechanisms of the underlying architecture, and when such a
feature is not available, the atomicity is provided by masking the occurrence of 
interrupts.

% implementantion in hardware
The hardware implementation of the semaphore component, pushes each
semaphore variable to a hardware implementation, and also manipulates the
blocked threads queue on hardware. In this sense, four commands
are implemented by the controller: \texttt{Create} and \texttt{Destroy}, responsible
for allocation and deallocation of the internal resources (memory for the variable
and the queue) and the other two traditional methods of semaphores \texttt{P} and 
\texttt{V}. For every \texttt{P} operation, the address of the caller of the method is passed
through the input registers to the hardware, and if the caller has to be blocked its
address is automatically inserted on the respective queue, and signalized by the
status register. Once the resource becomes available (through a \texttt{V} operation)
the address of the blocked thread waiting for the resource is removed from
the queue, and put in the output registers. The need to resume the thread that is
addressed on the output register is signalized by the status register.

\subsection{Scheduler}
\label{sec:cs_scheduler}

% Describe the component interface
The scheduler is responsible for organizing and defining the order that elements access
a resource, when such a resource is shared among several elements. The most common
use of a scheduler is to establish the order that tasks or process (elements) gain access
to use the CPU to run (resource). Figure~\ref{fig:scheduler_class_diagram} depict the design
of the process management family of components, where the \texttt{Scheduler} hybrid 
hw/sw components arise. 

\fig{0.55}{scheduler_class_diagram}{Process Management family of components}

The \texttt{Scheduler} provides the basic implementation of
methods to manipulate the queue of elements that are ready to use the resource managed
by the scheduler, such as \texttt{insert()}, \texttt{remove()}, \texttt{resume()} and \texttt{suspend()}.
A deeper explanation around the whole \emph{Process management} family of components
is beyond the scope of this paper. Let's focus here only on the component \texttt{Scheduler}
that was implemented as a hybrid component. Such a component implements the fundamental
structure of a scheduler, which consists of a queue of ready elements and time management
mechanisms.

% implementantion in software
The implementation of the software scheduler follows the traditional design of lists. Such a list implementation could be configured to be realized as a conventional ordering list of its elements, as well as a relative list, where each element stores its ordering parameter relative to its predecessor. In this sense each element will hold the difference of its ordering parameter from the previous element, and so and on. In such kinds of implementations, it is particularly interesting when the scheduling policy has dynamic priority increases over time, such as the EDF policy, for example. In such a policy, as the absolute deadline is always a crescent value, the use of a conventional ordering, using the absolute deadline should lead to an overflow of the variable as the execution time is always growing (which can occur in a few hours on 8 bits microcontrollers). Instead of this, the use of a relative queue insures that the deadline is always stored relatively close to the current time, and in this way, the variable will not overflow.

% implementantion in hardware
The implementation of the scheduler in hardware has an internal memory that implements an ordered list. One module (\texttt{Controller}) is responsible for interpreting all the data received by the interface of the component in hardware and then, activates the process responsible for implementing the functionality requested by the user (through the \texttt{command} interface register). This implementation, as the software counterpart, realizes the insertion of its elements already in order, that is, the queue is always maintained in an orderly fashion, following the information that the \texttt{SchedulingCriteria} provides. In the memory of the component, a double-linked list is implemented.

It is worth highlighting two aspects of the implementation of this component regarding its implementation on hardware, especially for programmable logic devices. Both of these aspects are related to constraints in terms of the resources of such  devices. Ideally, the hardware scheduler should exploit maximally the inherent parallelism of the hardware resources. However, such resources are very expensive, especially when the internal resources are used to implement several parallel bit comparators in order to search for elements in the queue, as well as, to find the insertion position of an element in queue.

Moreover, the use of 32 bits pointers, to reference the elements stored on the list (in this case \texttt{Threads}) becomes extremely costly, for implementing the comparators to search for those elements. On the other hand, the maximum number of tasks that a system will execute in an embedded system is usually known at design time, and for that reason, the resources usage of this component could be optimized implementing a mapping between the system pointer (32 bits) and an internal representation that uses only the number of bits necessary, taking into account, the maximum number of tasks running on the system.

\subsection{Alarm}
\label{sec:cs_alarm}

% Describe the component interface
The \texttt{Alarm} component is responsible for providing the abstraction of an event generator to the system. This component behaves asynchronously, as its service (the event generation) occurs asynchronous from its request (Alarm instantiation). The figure~\ref{fig:alarm_class_diagram} illustrate the design of the \texttt{Alarm} component. This component provides three types of event generation: call a function implemented by the user, resumes a blocked thread, or releases a semaphore (by calling it's \texttt{v()} operation). These events are supported by the \texttt{Handler} interface.

\fig{0.65}{alarm_class_diagram}{Alarm component design.}

% implementantion in software
The software implementation of the \texttt{Alarm} component is implemented sharing a \texttt{Timer} used to manage the passage of time, and a relative queue of event requests. This queue is organized relative to the number of ticks missing for the occurrence of the event. At each interrupt of the \texttt{Timer}, the number of ticks is updated in the queue, and when this number is less than zero, the handler of the event is invoked.
%
% implementantion in hardware
Its implementation on hardware implement dedicated counters for each supported \texttt{Alarm} component, implementing parallel countdown counters that generate interrupts to signalize events. The internal memory of the component is used to store a reference for each \texttt{Alarm} handler that is passed to the interrupt handler of the component to generate the respective event.

\subsection{Results}
\label{sec:results}

% Describe the experimentation platform
An experimentation platform was used to develop, debug, and evaluate those components. It was used the \textsc{Xilinx} development board \texttt{ML-403}, which has a \textsc{Virtex 4} FPGA, that enable the instantiation of the hardware components on the configurable logic, while running the software components on its embedded \textsc{PowerPC}. 
%The interconnection of the hardware components with the \textsc{PowerPC} was done using the \textsc{Ibm CoreConnect} bus, generated by the development tools provided by \textsc{Xilinx}.
The FPGA used on the platform was the \texttt{XC4VFX12}, which provides 5,412 slices of logic blocks for the implementation of the hardware accelerators.

\fig{0.55}{hw_area}{Logic usage  and performance of hybrid hw/sw components}

%  Show the results  & Describe how the results was gather
In order to evaluate the hybrid component implementations, the "The Dinning Philosophers" application was implemented using the three components. The alarm was used to wake-up the philosophers when their thinking period expired. The figure~\ref{fig:hw_area} shows the area consumed by the hardware implementation of the components, and the execution time of some methods of the components on both domains. The consumed logic of the hardware implementations is compared with the \emph{Plasma} processor, an open implementation of the MIPS architecture. The execution times show a acceleration of hardware implementations for the Scheduler and the Alarm component, while the Semaphore component did not gain that much performance from the hardware implementation, mainly because the evaluated application did not push the usage of semaphores queues, whereas a hardware implementation could effectively bring benefits.

\section{Conclusions}
\label{sec:conclusions}

This paper presented the guidelines to build an architecture of hybrid hw/sw components. It highlighted the importance of the use of an adequate engineering technique in order to design components that are flexible enough to migrate from hardware to software, and vice-versa. Three hybrid hw/sw components that represent all the possible communication behavior of such components was developed, and several experiments were done building a benchmarking application using different combinations of hardware and software implementation of those components. Further research is directed to the migration of hybrid hw/sw components during runtime.

\bibliographystyle{plain}
\bibliography{dl}

\end{document}
