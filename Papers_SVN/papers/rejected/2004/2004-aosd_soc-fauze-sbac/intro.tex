Embedded systems are becoming more and more complex, yet, there is no
room for development strategies that incur in extended time-to-market in
this extremely competitive sector.  In this context, \emph{Programmable
  Logic Devices}~(PLD) define a compromise between system complexity and
development costs. Furthermore, recent advances in programmable logic
devices are enabling developers to integrate complex designs into a
single silicon pastille. These \emph{Systems-On-Chip}~(\textsc{SoC})
have several advantages over traditional circuit board implementations,
including the level of integration, power consumption, maintainability
and most other development metrics.

Nevertheless, getting a \textsc{SoC} out off a \emph{Field-Programmable
  Gate Array}~(FPGA) is not a trivial task and requires an intricate
engineering process. Several strategies have been proposed for this
purpose, having in common a methodology to devise components, usually
referred to as \emph{Intellectual Property}~(\textsc{IP}), and to
assembly these components together in a functioning system. The
reusability of such soft \textsc{IP}s can be considerably increased by
isolating their core functionalities from interfacing aspects, thus
enabling the creation of libraries of pre-verified {IP}s that can be
integrated as a \textsc{SoC} with the use of additional ``glue logic''.

In order to automate the development of \textsc{SoC}s, much effort has
been payed to supporting tools that assist designers in selecting and
configuring the most adequate \textsc{IP}s and also in generating the
necessary glue logic. These tools usually presuppose an standardized
interconnect, such as \textsc{Wishbone}~\cite{Wishbone},
\textsc{Amba}~\cite{Amba}  or \textsc{Coreconnect}~\cite{Coreconnect}, 
and thus can be very effective. For instance, the
\textsc{Coral}~\cite{Bergamaschi:2000} tool from IBM uses the concept
of \emph{Virtual Design}~\cite{Bergamaschi:2000a} to provide the
designer with simplified \textsc{IP} descriptions that hide many
implementation details.

Some \emph{embedded systems} can be completely implemented in hardware
using this approach, but the more complex the application, the greater
is the probability it will need some kind of \emph{run-time support
  system} and an \emph{application program}. This is, after all, the
reason why so many groups are concentrating efforts to develop processor
soft cores~\cite{openrisc,leon,etc}. Nevertheless, run-time support
systems are often neglected by currently available \textsc{SoC}
development methodologies and tools, being mostly restricted to simple
processor scheduling routines and the definition of a
\textsc{software/hardware interface}~\cite{SOPCBuilder, ISE, acad�mico}.
The gap between software and hardware gets even bigger when we recall
that one of the primary goals of an operating system is to grant the
portability of applications, since ordinary operating systems cannot go
with the dynamics of \textsc{SoC}s.

In this paper, we consider the deployment of
\textsc{Application-Oriented System Design}~(AOSD)~\cite{Froehlich:2001}, 
a design methodology originally proposed for run-time support systems,
to guide the complete development of embedded systems, including
software and hardware components as well as their integration as a
\textsc{SoC}. The results obtained so far are encouraging and will be
discussed in details in the subsequent sections.

