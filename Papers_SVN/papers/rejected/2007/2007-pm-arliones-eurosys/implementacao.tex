\note{Abertura do cap�tulo.}

This section will present the power manager's implementation. This
includes the message propagation mechanism and the definition of a
descriptive language to represent the operating modes nets. This
language will be used to enable automatic generation of the operating
mode switching methods. Therefore, the experimental environment, which
is the \textsc{Epos}~\cite{Marcondes:ETFA:2006} operating system, is
briefly described.


\subsection{Static resolution of the Operating Modes Nets}

\note{As redes s�o legais mas n�o precisam ser interpretadas
  on-the-fly.}

The Operating Mode Nets presented at section~\ref{sc:migration_nets}
offer to this proposal a way to specify the operating mode switching
procedures. Although there is a lot of mathematical analysis models to
interpret these nets at execution time, these models demand for
processing and memory capabilities well beyond of embedded devices'
capacities. However, once in this proposal the nets and the
application are known at system generation time, online interpretation
of these nets is not needed, and can be eliminated. The elimination is
done by analyzing the nets at compilation time, thus generating the
operating mode switching methods accordingly to the system
configuration.

\note{Ent�o definimos uma linguagem descritiva das redes que,
  interpretada, gera os m�todos 'power'.}

An open-source graphical Petri Net editor called
Pipe2~\cite{Akharware:2005} was used to generate the operation mode
nets. This tool was then modified to export the operation mode nets as
a \textsc{Xml} based language which feeds an analysis tool with
information about the operating mode transition procedures. At system
generation time, this tool uses system information about which
components are in use (\textsc{Epos} provides this information after
an application analisys~\cite{Tondello:2005}) to generate the
operating mode switching procedures which will be needed. These
procedures are aggregated by an aspect, which is applied to the
components which power consumption must be handled. This aspect
aggregate some data to control the power consumption of the components
it is applied to and, also, adds code to garantee that components will
be in an operating mode in which the requested operation can be
executed. Information about in which operating mode each system
interface method works was included in \textsc{Epos}' information
database.

\begin{figure*}[t]
\begin{center}
\begin{footnotesize}

\lstset{language=c++,frame=lrtb}
\lstset{basicstyle=\ttfamily}
\lstset{commentstyle=\textit}

\subfigure[Power\_Manager aspect declaration]{
  \begin{minipage}{7cm}
    \lstinputlisting{prg/power_manager.h.prg}
  \end{minipage}
}
\subfigure[Automatically generated \texttt{power} method]{
  \begin{minipage}{7cm}
    \lstinputlisting{prg/power_manager_power.cc.prg}
  \end{minipage}
}
\subfigure[Wrapped interface method]{
  \begin{minipage}{7cm}
    \lstinputlisting{prg/power_manager_get.cc.prg}
  \end{minipage}
}
\subfigure[Wrapped interface method]{
  \begin{minipage}{7cm}
    \lstinputlisting{prg/power_manager_put.cc.prg}
  \end{minipage}
}


\caption{Fragment of the Power\_Manager aspect}
\label{prg:power_manager}
\end{footnotesize}
\end{center}
\end{figure*}



The aspect was implemented as a \emph{scenario adapter}, i.e., a
static meta-programmed construct (C++ template class) which extends
the components (C++ classes) over which it will be applied,
aggregating data by declaring its variables, and modifying code by
using polymorfism. As template dependencies are solved at compilation
time in C++, the polymorphism haven't aggregated aditional overhead to
the system. This happens because, once power management is switched on
for a certain component, the system configuration prevents the
application from calling the original version of the component,
incurring in the existence of only one used method.

Figure~\ref{prg:power_manager}(a) shows a source code fragment of the
Power\_Manager aspect. There are two configurable features for this
manager: \emph{shared} and \emph{instances}. \emph{shared} is used to
inform that there is the possibility of the managed component be
shared. When this happens, the power manager tracks the component's
users, preventing itself from switching off in use components.
\emph{instances} enables a funcionality which keeps track of
components instances. It is used to allow message propagation from
higher level components. These features are configured by the system
or by the application programmer, and the aspect is specialized to
operate accordingly to this configuration.
Figures~\ref{prg:power_manager}(a), \ref{prg:power_manager}(b), and
\ref{prg:power_manager}(c), shows the methods which were automatically
generated for a UART components, i.e., the operating mode switching
procedure (\emph{power}) and the wrapping methods for the UART's
interface methods \emph{get} and \emph{put}.


\subsection{Ambiente Experimental}

\note{Prot�tipo da proposta com EPOS para ATMegaS e XScale(?). Suporte
  para perif�ricos.}
To test this proposal, this power manager was adapted to operate over
the \textsc{Epos} operating system~\cite{Froehlich:2001}.
Implementations of this system for \textsc{AVR} microcontrolers were
used. Also, support for other devices featured by some used platforms
featured was developed in cooperation with a parallel work which
explored operating system support for wireless sensor
networks~\cite{Wanner:ETFA:2006}. Among these devices are
communication devices (\textsc{Uart} and radio \textsc{MACs}) and
sensing devices (thermistores, photo-diodes, accelerometers,
anolog-digital conversors and high-level abstractions for sensing).


\subsubsection{EPOS - Embedded Parallel Operating System}

\note{Come�ou com a tese do Guto. Extens�es para projeto de hardware e
  particionamento autom�tico.}

The \textsc{Epos} operating system was proposed by Fr�hlich in his PhD
thesis as a prototype to prove the concepts behind his
Application-Oriented System Design methodology
(\textsc{Aosd})~\cite{Froehlich:2001}. This methodology uses several
advanced software engineering and programming techniques that,
combined, enable the generation of optimized operating systems for
dedicated applications. Since its creation, \textsc{Epos} have been
used as platform to validate and extend \textsc{Aosd}'s concepts and,
recently, has also been successfully used as a hardware design
methodology~\cite{Polpeta:ETFA:2005}. This methology has evoluted and
\textsc{Epos} is becoming a complete solution for software/hardware
co-design of embedded systems~\cite{Cancian:2006}.

\note{EPOS permite o desenvolvimento de aplica��es port�veis. Sistema
  otimizado para aplica��o dedicada. Para isso usa AOSD (Interfaces,
  Mediadores, Abstra��es).}
The main goals of the \texttt{EPOS} system are to allow application
programmers to write architecture-independent applications, and,
through the application analysis, to deliver a run-time software
support for such applications which complies all resources that a
specific application needs, and nothing else. In order to achieve
these goals, \texttt{EPOS} relies on AOSD's concepts of \emph{Inflated
  Interfaces}, \emph{Hardware Mediators} and \emph{System
  Abstractions}.


\note{Pra que servem Mediadores, Abstra��es e Interfaces?}
\emph{Hardware mediators} are software constructs that mediate the
interaction between operating system components, called \emph{System
  Abstractions}, and hardware components. This mediation is done
through the definition of a rigid interface for each group (family) of
hardware components. This interface is called \emph{Inflated
  Interface}, and it is responsible for defining the main difference
between hardware mediators and \textsc{HAL}s. Instead of building a
monolithic layer encapsulating the resources available in the hardware
platform, each hardware mediator abstracts the correspondent hardware
component functionalities and deliver these functionalities to the
operating system through the inflated interface. As hardware mediators
are intended to be mostly meta-programmed, their code is dissolved in
the abstractions as the ``interface contract'' is met, generating
virtually no overhead to the system~\cite{Polpeta:2004}.

\note{Componentes organizados em fam�lias. Regras de composi��o
  garantem coer�ncia dos designs.}

Each system abstraction or hardware mediator is composed by a set of
similar operating system components. These components are organized
according to the \emph{Family-Based Design} paradigm, and have their
\emph{commonalities} and \emph{variabilities} explored through
different class hierarchies. For instance, the \texttt{Thread} family
of system abstractions is comprised by several different thread
implementations (e.g., \texttt{Exclusive\_Thread},
\texttt{RT\_Thread}), and the \texttt{Scheduler} family is comprised
by several schedulers (e.g., \texttt{FCFS\_Scheduler},
\texttt{EDF\_Scheduler}, \texttt{RM\_Scheduler}). Composition rules
help a graphical tool to suggest for the application programmer a
running environment for each application. An example of composition
rule would be the mandatory use of the \texttt{RT\_Thread} member when
a real-time scheduler (e.g., Erliest Deadline First (EDF) and Rate
Monotonic (RM)) is chosen.


\note{Engenharia de dom�nio. Scenario Adapters e Configurable
  Features.}

System Abstractions and Hardware Mediators are intended to be
collected from an \emph{Application-Oriented Domain Analysis and
  Decomposition} process. This analysis process is quite similar to
\emph{object-oriented decomposition}. The main difference is that the
Application-Oriented System Design is a multi-paradigm design
methodology, so other entities, such as \emph{aspects} and
\emph{configurable features}, must come out from this analysis. The
use of \emph{configurable features} and \emph{scenario aspects},
allied to advanced programming techniques, such as \emph{Static
  Meta-programming} and \emph{Aspect-Oriented Programming}, deliver to
the application programmer a widely configurable and adaptive system.

