% Programming Languages have a main role in computational systems development.
% Among them, the
Managed Programming Languages (MPLs), from which
\java~and \lua~are examples, provide developers with features to improve their
productivity such as object-orientation, domain-specific constructs, and APIs,
which allows expressing his ideas in a high level of abstraction.
Furthermore, features as
automatically memory management, memory protection, and
exception handling, avoid
errors such as memory leak, access to dangling pointers, and
unpredictable behavior.

Several initiatives have been taken on the last decade in order to enable the
use of MPLs
% not only for general propose systems as well
for embedded systems,
fulfilling time and resource usage requirements impose by these systems.
However, in order to be useful in embedded systems, MPLs must provide
features for interacting with the environment where the embedded system is
deployed.
Such interaction is usually implemented by using hardware devices, such as
sensors, actuators, transmitters, receivers, timers and alarms.
The interface between hardware devices and MPL is commonly implemented by hand,
which decreases the productivity of developers and can
easily lead to programming errors on the resultant binding code.

% nota pode reduzir e jogar a questão do automatico mais pra cima
This paper presents a method to abstract hardware devices in order to be used by
applications written using MPLs in the embedded systems scenario.
Hardware mediators abstract and organize hardware devices
in a suitable manner for embedded systems fulfilling time and resource
consumption requirements.
By isolating hardware mediators from the specificities of an MPL, the problem of
adapting a hardware device to work with a new MPL can be faced as an aspect
weaving problem which can be automatically solved by using a proper tool.

We have evaluated the proposed method for the MPLs \java~and \lua~
% among three cases study encompassing,
%video encoding, serial communication, and temperature sensing.
in a case study encompassing H.264 video encoding.
The obtained results corroborate the suitability of the proposed method to adapt
a hardware device to a new MPL automatically while fulfilling
requirements of performance and memory consumption.

