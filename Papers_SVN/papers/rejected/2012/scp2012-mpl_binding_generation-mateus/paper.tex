\documentclass[final,1p,times]{elsarticle}
% \documentclass[final,3p,times,twocolumn]{elsarticle}

\usepackage[english]{babel}	% for multilingual support

\usepackage[utf8]{inputenc} % for use utf8
\usepackage{graphicx}

\usepackage[caption=false]{subfig} % Para usar duas ou mais figuras como uma só.

\usepackage{algorithmic}
\usepackage{algorithm}

\usepackage{verbatim} % Para usar comentários em bloco.


% ------------------------------------------------------------------------------

% Command to use code as figure ------------------------------------------------
\usepackage{listings}
\lstset{keywordstyle=\bfseries, flexiblecolumns=true}
\lstloadlanguages{[ANSI]C++,HTML}
\lstdefinestyle{prg} {basicstyle=\small\sffamily, lineskip=-0.2ex, showspaces=false}

% header C
\newcommand{\headerc}[3][tbp]{
 \begin{figure}[#1]
     \lstinputlisting[language=C,style=prg]{fig/#2.h}
   \caption{#3\label{headerc:#2}}
 \end{figure}
}

% C
\newcommand{\progc}[3][tbp]{
 \begin{figure}[#1]
     \lstinputlisting[language=C,style=prg]{fig/#2.c}
   \caption{#3\label{progc:#2}}
 \end{figure}
}

% header C++
\newcommand{\headercpp}[3][tbp]{
 \begin{figure}[#1]
     \lstinputlisting[language=C++,style=prg]{fig/#2.hh}
   \caption{#3\label{headercpp:#2}}
 \end{figure}
}


% C++
\newcommand{\progcpp}[3][tbp]{
 \begin{figure}[#1]
     \lstinputlisting[language=C++,style=prg]{fig/#2.cc}
   \caption{#3\label{progcpp:#2}}
 \end{figure}
}

% Java
\newcommand{\progjava}[3][tbp]{
 \begin{figure}[#1]
     \lstinputlisting[language=Java,style=prg]{fig/#2.java}
   \caption{#3\label{progjava:#2}}
 \end{figure}
}


% Para colocar 2 programas como um só - dispostos horizontalmente
\newcommand{\multprogjavatwoh}[5][htbp]{
\begin{figure*}[#1]
  \centering

\subfloat[]{\label{fig:#2}\lstinputlisting[language=Java,style=prg
]{fig/#2.java}}

\subfloat[]{\label{fig:#3}\lstinputlisting[language=Pascal,
style=prg]{fig/#3.lua}}
  \caption{#5}
  \label{fig:#4}
\end{figure*}
}
% e.g.
%\multfigtwoh{fig_plot_time_orig}{fig_plot_time_mod}
%{fig_plot_time_all}
%{Original (a) and modified (b) benchmarks execution time comparison.}


% Lua
\newcommand{\proglua}[3][tbp]{
 \begin{figure}[#1]
     \lstinputlisting[language=Pascal,style=prg]{fig/#2.lua}
   \caption{#3\label{proglua:#2}}
 \end{figure}
}

% KCL
\newcommand{\progkcl}[3][tbp]{
 \begin{figure}[#1]
     \lstinputlisting[language=C++,style=prg]{fig/#2.kcl}
   \caption{#3\label{progkcl:#2}}
 \end{figure}
}

% OCL
\newcommand{\progocl}[3][tbp]{
 \begin{figure}[#1]
     \lstinputlisting[language=C++,style=prg]{fig/#2.ocl}
   \caption{#3\label{progocl:#2}}
 \end{figure}
}

% OIL
\newcommand{\progoil}[3][tbp]{
 \begin{figure}[#1]
     \lstinputlisting[language=C++,style=prg]{fig/#2.oil}
   \caption{#3\label{progoil:#2}}
 \end{figure}
}

% ------------------------------------------------------------------------------

% Commands to insert figures ---------------------------------------------------
\newcommand{\figu}[4][ht]{
  \begin{figure}[#1] {\centering\scalebox{#2}{\includegraphics{fig/#3}}\par}
    \caption{#4\label{fig:#3}}
  \end{figure}
}

\newcommand{\fig}[4][ht]{
  \begin{figure}[#1] {\centering\scalebox{#2}{\includegraphics{fig/#3}}\par}
    \caption{#4\label{fig:#3}}
  \end{figure}
}
% fig usage:
% \fig{<scale>}{<file>}{<caption>}
% e.g.: \fig{.4}{uml/uml_comportamental_dia}{Diagramas comportamentais da UML}
% The figure label will be "fig:" plus <file>.
% The figure file must lie in the "fig" directory.

\newcommand{\figtwocolumn}[4][ht]{
  \begin{figure*}[#1] {\centering\scalebox{#2}{\includegraphics{fig/#3}}\par}
    \caption{#4\label{fig:#3}}
  \end{figure*}
}

\newcommand{\figb}[4][hb]{
  \begin{figure}[#1] {\centering\scalebox{#2}{\includegraphics{fig/#3}}\par}
    \caption{#4\label{fig:#3}}
  \end{figure}
}

% Para colocar 2 figuras como uma só - dispostas horizontalmente
\newcommand{\multfigtwoh}[6][htbp]{
\begin{figure*}[#1]
  \centering
  \subfloat[]{\label{fig:#3}\scalebox{#2}{\includegraphics{fig/#3}}}
  \subfloat[]{\label{fig:#4}\scalebox{#2}{\includegraphics{fig/#4}}}
  \caption{#6}
  \label{fig:#5}
\end{figure*}
}
% e.g.
%\multfigtwoh{.65}{fig_plot_time_orig}{fig_plot_time_mod}
%{fig_plot_time_all}
%{Original (a) and modified (b) benchmarks execution time comparison.}

% Para colocar 2 figuras como uma só - dispostas verticalmente
\newcommand{\multfigtwov}[6][htbp]{
\begin{figure}[#1]
  \centering
  \subfloat[]{\label{fig:#3}\scalebox{#2}{\includegraphics{fig/#3}}}\\
  \subfloat[]{\label{fig:#4}\scalebox{#2}{\includegraphics{fig/#4}}}
  \caption{#6}
  \label{fig:#5}
\end{figure}
}
% e.g.
%\multfigtwov{.35}{fig_epos_mem_framework}{fig_epos_mem_framework_spm}
%{fig_epos_mem_framework_all}
%{EPOS memory mapping before (a) and after (b) using the new framework}


% Alguns outros comandos:
\newcommand{\lua}{\textsc{Lua}}
\newcommand{\java}{\textsc{Java}}
\newcommand{\python}{\textsc{Python}}
\newcommand{\ruby}{\textsc{Ruby}}


% ------------------------------------------------------------------------------
\journal{Science of computer programming}

\begin{document}

\begin{frontmatter}

% Títulos já usados em papers publicados sobre o mesmo tema
% + Método para Abstração de Componentes de Hardware para Sistemas Embarcados (2012:Dissertação)
% + Abstracting hardware devices to embedded Java applications (2011:IADIS)
% + Abstraindo dispositivos de hardware para aplicações Java embarcadas (2011:SBESC)
% título em inglês para o IEEE: Interfacing Hardware Devices to Embedded Java
% + Interfacing Operating Systems Components with Embedded Java Applications (2009:I2TS)

% \title{MPL binding generation as an AOP problem}
% \title{Efficient generation of wrappers for hardware devices in the embedded
% system scenario}
% \title{MPL Binding Generation for Embedded Systems as Aspect Weaving}
% \title{Managed Programming Languages Binding Generation for Embedded Systems Components}
\title{Binding Embedded System Components in Managed Programming Languages}

\author{Mateus Krepsky Ludwich and Antônio Augusto Fröhlich}

\address{Laboratory for Software and Hardware Integration (LISHA)\\
Federal University of Santa Catarina (UFSC)\\
Florianópolis, Brazil\\
\{mateus,guto\}@lisha.ufsc.br
}


\begin{abstract}
\documentclass[10pt,english]{article}
\usepackage[T1]{fontenc}
\usepackage[latin1]{inputenc}
\usepackage{geometry}
\usepackage{babel}

\geometry{a4paper,tmargin=3cm,bmargin=3cm,lmargin=3cm,rmargin=3cm}

\begin{document}

\title{Scenario Adapters: Efficiently Adapting Components\\[0.2cm]
	\normalsize{(extended abstract)}}
\author{Ant�nio Augusto Fr�hlich\\[0.2cm]
	\small{GMD FIRST}\\[-0.5ex]
	\small{Kekul�stra�e 7}\\[-0.5ex]
	\small{12489 Berlin, Germany}\\[-0.5ex]
        \small\tt{guto@first.gmd.de}}
\date{}
\maketitle
 
\sloppy

\section{Introduction}

Object-orientation together with component-based software engineering is enabling the long dreamed production of software as an assemblage of ordinary components. Similarly to other industries, software developers can now reuse components, reducing costs and accelerating production. Moreover, in comparison to other assembly lines, for instance to the largely acclaimed automotive industry, component-based software engineering shows an expressive advantage: software components, besides being reused, can easily be adapted to match particular requirements. This, for the car industry, would mean having a single engine that could be adapted either to propel a limousine or a small city car.

Conceiving a software system as an assemblage of adaptable components, however, brings about new challenges. What granularity components will have, how they will be configured and adapted, how they will be put together, how to grant that an assemblage of components yields to a system that matches the requisites of a given set of applications, are just some of the important questions that have to be answered in order to make this strategy really effective.

This extended abstract focuses on the question about component adaptation. It first presents the motivation for component adaptation and then describes the scenario adapter mechanism. Scenario adapters can be used to efficiently adapt an existing component to join a specific execution scenario. The full paper, besides more detailed discussion about the theme, describes the implementation of scenario adapters as C++ templates and features a case study about \textsc{Epos}, a highly configurable operating system that employs the scenario adapter mechanism described here.

\section{Adaptable Components}

Component-based systems are usually born with a big question about component granularity. The question is so crucial because it directly impacts performance, configurability and maintainability. On the one hand, a system made up of a large amount of fine components will certainly achieve better performance than one made up of a couple of coarse components, because each coarse component brings along functionality that will not always be used. The component functionality that is not used turns into overhead for the applications. On the other hand, a set of fine grain components is harder to configure and to maintain.

If supporting high performance applications is a goal for the component-based system, the configuration and maintenance difficulties is a price that has to be paid. Nevertheless, the number of components and the complexity of each component may be reduced if components are designed to be independent of some execution aspects and if there is a way to adapt them to execute in a given scenario. In this way, instead of implementing several versions of a component to run in different scenarios, one may implement a single scenario-independent component and later adapt it to the respective execution scenario. This does not only bring benefits due to concern separation, but also because the strategy used to adapt a component will likely be reused to adapt other components to the same execution scenario, thus reducing the total amount of code. Aspects such as synchronization, remote invocation, protection and architectural characteristics, can usually be avoided from a component's realization~\cite{Froehlich:ehpc:1999}.

A process abstraction is a good example for an adaptable component. Instead of implementing versions of it to run on a several architectures, to run locally or remotely, to support single or multithreading, to run in single or multiprocessors, to run in safe or unsafe environments,  it is possible to work out a single implementation that regards only aspects that are intrinsic to the process abstraction. This single implementation can later be adapted to a specific execution environment. 

The approach of writing pieces of software that are independent of certain aspects and later adapting them to a given scenario is often referred as \textit{Aspect Oriented Programming} (AOP)~\cite{Kiczales:ecoop:1997}. However, AOP usually implies in describing aspects in aspect-oriented languages and also demands for tools (\textit{weavers}) to combine aspects and aspect-independent components. Weaving a component to adapt it to an execution scenario may be an efficient alternative, but it raises the problem of granting that the wove component preserves its original semantics. Therefore, we propose another approach, based on statically metaprogrammed  scenario adapters.

\section{Scenario Adapters}

Scenario adapters are mechanism to efficiently adapt existing aspect-independent components to specific execution scenarios. In order to achieve the efficiency demanded by high performance systems, scenario adapters are statically metaprogrammed~\cite{Veldhuizen:1995}, so that component adaptations occur at compile time, before system execution.

Contrary to what it may look, adapting components at compile time, at least in regard to high performance execution scenarios, usually does not imply in flexibility loss. Once adapted to a scenario, a component is not expected to move to another scenario, although the abstractions it represents may freely move. Recurring to the process example again, it is the process itself that can migrate and not the respective system component implementation. The process component may be adapted and assembled into several systems. The system that will run on a single processor node will have the respectively adapted component, just as the multiprocessor node system. When a process migrates from a single processor node to a multiprocessor one, the component itself is not migrated, but its state and behavior.
 
Another advantage of an statically metaprogrammed scenario adapter is that it can easily be optimized by the compiler. Segments of the metaprogram that are not referred are normally not included in the compilation output. Therefore, a component adapted to support remote invocation, but that is not actually remotely invoked, will not include the respective code. Besides, when a component that completely fulfills the execution scenario requirements exists, the metaprogram does not generate any code.

We implemented scenario adapters in the realm of the \textsc{Epos} project~\cite{Froehlich:ooosw:1999}, which aims to deliver, whenever possible automatically, a highly performance operating system to each application. Aimed are parallel and embedded applications. \textsc{Epos} scenario adapters are C++ templates that jointly define a framework in which application-ready, aspect-independent components can be composed and adapted. Current adaptations include multithreading synchronization, remote invocation and access control.

Several application-oriented systems have already been generated and evaluated. These systems showed no overhead due to the use of scenario adapters, what demonstrate the efficiency of scenario adapters. We are now extending the set of \textsc{Epos} scenario adapters and also the set of system components in order to better validate and evaluate the project.

\nocite{Cordsen:iwooos:1991}
\bibliographystyle{plain}
\bibliography{cluster_computing,operating_systems,communication,processors,parallel_machines,software_engineering,guto}

\end{document}
\end{abstract}

\begin{keyword}
Binding Generation, Aspect-Oriented Programming, Embedded Systems, Foreign Function Interface

\end{keyword}

\end{frontmatter}


%-------------------------------------------------------------------------------
% Section <Introduction>
% ------------------------------------------------------------------------------
\section{Introduction}
Very-High Level Languages (VHLL), from which \java~and \lua~are examples, is a
kind of programming language which provide developers with features to improve
their productivity\cite{Wilson:1999}.
Productivity improvement is obtained by using constructions with a higher level
of abstraction enabling the developer to express and validate his ideas in a
short period of time (such as object orientation, domain specific constructions
and APIs), and by features that make the occurrence of programming errors less
often reducing the time spend on program debugging (such as automatic memory
management, memory protection, and exceptions).

During the last ten years several initiatives have been taken in order to
enable the use of VHLLs not only in general propose systems scenario as well in
embedded systems scenario fulfilling the time and resource requirements impose
by such systems.
However, in order to be really useful for embedded systems VHLLs must provide
features for interacting with the environment where the embedded system
is inserted on.
Such interaction is usually implemented by using hardware devices.
Sensors and actuators enable the system to interact with the environment.
Transmitters and receivers are used for communicating with other systems.
Timers and alarms are used to implement real-time operations.

The interaction between VHLLs and hardware devices is performed by using the so
called Foreign Function Interface (FFI).
However, a FFI do not specify by itself how to abstract hardware or how
to organize these abstractions.
This work aims to fulfill this gap, introducing a method to interface hardware
devices and applications written using VHLLs in context of embedded systems.
We propose a method to abstract such hardware devices and we show that the
problem of adapting a hardware device to be used for a VHLL can be faced as an
aspect weaving problem, automatically generating the binding between the device
and the language.

The next sections of this paper are organized in the following way: Section
\ref{sec:relat} reviews how VHLLs interact with hardware devices and how
hardware devices can be abstracted and organized.
Section \ref{sec:proposal} introduces the proposed method for
abstracting hardware devices and shows how the adaptation of a hardware device
for a specific VHLL can be solved as an aspect weaving.
Section \ref{sec:eval} presents our cases study as well the obtained results
on evaluating our proposal according to performance, memory consumption,
portability, support to the developer, and reuse.
Our final considerations are presented in Section \ref{sec:conc}.

% ------------------------------------------------------------------------------


% ------------------------------------------------------------------------------
% Section <SOTA>
% ------------------------------------------------------------------------------
\section{Related work} \label{sec:relat}
This section reviews how VHLL interact to hardware devices, and how hardware
devices are abstracted and organized to be used by software.

\subsection{FFIs}
Automatic memory management is one of the main features provided by VHLLs.
Allocated objects are automatically freed after they are no longer needed,
freeing developers from writing code to deallocate objects and eliminating the
risk of memory leak.
On the other hand, because VHLLs use automatic memory management the object
address is only know by the runtime support system (e.g. Virtual Machine) of
the VHLLs.
Therefore, VHLLs are unable to directly control memory mapped devices in the
way languages such as C and C++ do (by using the concept of \emph{pointer}).
VHLLs also do not provide the concept of \emph{inline assembly}, used to
control hardware devices by using dedicated I/O assembly instructions.
In other to solve these limitations, VHLLs use a mechanism called Foreign
Function Interface (FFI).

FFI is a mechanism which allows for programs written in one language to use
constructions of a program written in another.
The language that defines the FFI is called \emph{host language} and the
language that has theirs constructions used is called \emph{guest language}.
The host language is usually a higher level language (i.e. VHLL) and the guest
language is usually a lower level language, such as C and C++.

The FFI mechanism has a high importance to VHLLs, not only new programs use FFI
but also some of the main libraries of the VHLL are implemented by using FFI.
This is the case of the Java Standard Platform (JSE), which uses FFI to
implement packages such as \emph{java.io}, \emph{java.net} and \emph{java.awt}.

However a FFI by itself does not guide the developers on the abstraction of
hardware devices, it just provides for the developers a way to access
constructions of other languages capable of access hardware devices directly.
Besides that a manual utilization of a FFI could be difficult since the
developer must take into consideration the semantics of two distinct languages
(the host and the guest language).
Also the manual utilization of a FFI can be error prone.
The following code illustrates the implementation of a sum method using the
Java FFI KNI \cite{_k_2002}.

\begin{verbatim}
KNIEXPORT KNI_RETURNTYPE_VOID
Java_simplemath_Adder_sum() {
    jint a = KNI_GetParameterAsInt(1);
    jint b = KNI_GetParameterAsInt(2);
    KNI_ReturnInt(a + b);
}
\end{verbatim}

In order to obtain the sum arguments, the developer should remember the order
of the formal parameter declaration, represented by the indexes ``1'' and ``2''
passed to FFI functions.

\subsection{HAL}
% Encapsula todos os recursos disponíveis em uma plataforma
% Interdependência desnecessária entre dispositivos
% Hardware abstraído por meio de serviços
% Todos os serviços devem estar presentes em uma dada plataforma independente da
% aplicação utilizá-los ou não
% 
% = Exemplo Linux
% 
% = Exemplo micro Linux
% 
% = Exemplo eCos
\emph{Hardware Abstraction Layer} (HAL) is an approach to abstract
hardware specificities, providing for the operating system (OS) abstract
hardware devices.
The hardware devices abstraction is performed by services such as interruption
handling, reinitialization handling, DMA transferences, timers control, and
synchronization between multiprocessors \cite{Tanenbaum:2007:MOS:1410217}.

In order to port a HAL to a new hardware platform it is necessary to implement
all services provided by such HAL.
However, many services provided by a HAL could not be used by an application
and providing all these services generates unnecessary memory and performance
overhead.

An example of overhead caused by the way HAL are designed and implemented is
the \textsc{uCLinux}, an OS which target embedded systems and
relies on a HAL. % CITE uCLinux.
In such case, \textsc{uCLinux} inherits from \textsc{Linux} several of its
features, among them, the file system abstraction.
Such fact impacts not only on the OS size, as well in all system
initialization infrastructure and application loading \cite{Polpeta:2004}.

Anther example occurs in the RedHat's \textsc{eCos} OS.
% cite ECOS
Although, the HAL used by \textsc{eCos} is based on software components it is
not generated according to the application thus, it can carry unnecessary code
for the system.
In the case of SOCs generated from the micro-architecture LEON2, for example,
the \textsc{eCos} system assumes the existence of an UART device what is not
necessarily true \cite{Polpeta:2004}.

\subsection{Hardware Mediators} \label{sec:adesc_n_hwmediators}
Methodologies based on domain engineering for developing components have been
used as an alternative to the monolithic HALs approach.
Such methodologies have been used for abstracting hardware devices without
generating unnecessary interdependencies between the devices been abstracted.
That is the case of the Application-Driven Embedded System Design (ADESD)
methodology \cite{Froehlich:2001}.

The \emph{Embedded Parallel Operating System} (EPOS), represents the case study
of the ADESD application on the domain of operating systems.
At EPOS the abstractions obtained from domain decomposition are abstraction
of OS, such as threads, semaphores, abstractions for communication such as
network, channels, and other.

In order to provide system abstractions with the needed hardware support ADESD
defines the concept of \emph{hardware mediator}.
Hardware mediators sustain an \emph{interface contract} between system
abstractions and the machine, allowing for such abstractions to be
machine-independent \cite{Polpeta:2004}.
% Este contrato de interface além de ocultar especificidades de dispositivos de
% hardware de fabricantes ou plataformas distintos, pode prover em software
% funcionalidades que não estejam originalmente presentes no dispositivo de
% hardware utilizado.
% Por exemplo, um mediador de Interface de Rede \textit{NIC} pode prover um
% método
% para geração de \emph{CRC}, um código utilizado para verificar a integridade
% na
% transmissão de quadros em uma rede \emph{ethernet}, independentemente do
% dispositivo de hardware em questão possuir ou não tal funcionalidade.
% Ou seja, a abstração que utilizará o mediador da \textit{NIC} poderá sempre
% assumir
% que o método para calculo de CRC existe e ficará a cargo da implementação do
% mediador delegar este calculo para hardware, caso exista esta funcionalidade
% no
% dispositivo de hardware em questão ou, caso contrário, de implementar esta
% funcionalidade em software.

There is a hardware mediator for each hardware device present in the machine.
Such fine grain control avoids the false dependency between devices that occurs
on the monolithic HALs.
The Figure \ref{fig:hardware_mediator-gs} shows hardware mediators, one for
each hardware device.
The figure also depicts another hardware mediator aspect, it is designed to be
overhead free.
By using metaprogramming techniques and function’s inlining is possible to
dissolve mediators among the abstractions that use it, which avoids time
overhead in the use of mediators eliminating method call overhead.

\figtwocolumn{.7}{hardware_mediator-gs}{Hardware Mediators.}


% ------------------------------------------------------------------------------



% ------------------------------------------------------------------------------
% Section <DERCS>
% ------------------------------------------------------------------------------
\section{DERCS} \label{dercs}
% + Intro
Our method uses an intermediate representation to describe hardware mediators and
FFI aspects.
This section describes the meta-model we have used and
Section \ref{sec:proposal} describes the method we have proposed.

% NOTE: Maybe move
% + Why we have used DERCS
% and
% + How we have used DERCS
% To here.

% + DERCS def
Model-Driven Engineering (MDE) proposes the development of complete
computational systems from high level specifications which are transformed,
in one of more phases, in order to generate the final system. % cite MDE
In such context, the Distributed Embedded Real-Time Compact Specification
(DERCS) is a specification/meta-model used to represent platform independent
models.
Such models, according to the
Aspect-oriented Model-Driven Engineering for Real-Time systems (AMoDE-RT)
methodology, are used together with the description of the target platform and
mapping rules to generate the final system, which encompass software and hardware
elements \cite{Wehrmeister:2009}.

% + DERCS and AMoDE-RT
According to the AMoDE-RT methodology,
a DERCS model is generated from UML's class and sequence diagrams, and
from diagrams that specify non-functional elements as aspects.
Therefore, the DERCS meta-model defines
structural elements, behavioral elements, and aspect-oriented elements,
encompassing distinct visions into a single model.

% + DERCS's Structural and Behavioral Elements
% Figures \ref{fig:structural_elements} and \ref{fig:behavioral_elements}
% show, respectively, the structural and behavioral elements defined by DERCS.
Among the structural elements of DERCS are classes, attributes,
methods, and parameters.
The behavioral elements determine the behavior of a method, detailing which
messages such method can send to objects, which are its
local variables, actions, etc.
The structural and behavioral elements of DERCS have semantics similar to the
semantics of object-oriented programming languages.
Details concerning structural and behavioral elements of DERCS can be found at
\cite{Wehrmeister:2009}.

% \figtwocolumn{1}{structural_elements}{DERCS structural elements. Adapted from \cite{Wehrmeister:2009}.}

% \figtwocolumn{1}{behavioral_elements}{DERCS behavioral elements. Adapted from \cite{Wehrmeister:2009}.}

% + Aspect-Oriented Elements
Figure \ref{fig:ao_elements} shows the aspect-oriented elements
defined by DERCS.
Among the aspect-oriented elements are aspects, pointcuts,
structural adaptations, and behavioral adaptations.
Those elements correspond to the concepts of Aspect-Oriented Programming (AOP),
as defined by \cite{Kiczales:1997} whereas
aspect adaptations (\emph{AspectAdaptation})
represent the concept of \emph{advices}.
An aspect adaptation can be structural (\emph{StructuralAdaptation}) or
behavioral (\emph{BehavioralAdaptation}).
Structural adaptations modify the structure of the elements of the model, for
example, adding methods to a class or parameters to a method.
On the other hand, behavioral adaptations modify the behavior of the elements of
the model, for example, executing tasks before or after the execution of a
method behavior, or completely changing a method's behavior.

\figtwocolumn{1}{ao_elements}{DERCS aspect-oriented elements. Adapted from \cite{Wehrmeister:2009}.}

% + For what DERCS was used
The aspect-oriented elements of DERCS meta-model can be used to modify the
interface of a component in order to adapt such component according to what is
required by its client.
As next section shows, we have explored this feature of DERCS in order to
adapt a hardware mediator interface for what is expected by a specify FFI.

% \clearpage

% ------------------------------------------------------------------------------


% ------------------------------------------------------------------------------
% Section <proposal>
\section{The Proposed Scheduling Strategy}
\label{sc:proposta:proposta}

%- Deixar claro: caso seja constatado a falta de energia para atender o tempo de dura��o do sistema, a proposta � n�o executar as partes opcionais, somente as partes obrigat�rias. No momento em que a energia torna-se suficiente, as partes opcionais voltam a ser executadas no devido tempo. Explicar a figura~\ref{fig:energiaTempo}.

%O nosso escalonador baseado no algoritmo \textsc{EDF} garante a execu��o das partes obrigat�rias com os seus respectivos \deadlines{} atendidos sem levar em considera��o o n�vel de energia do sistema. A execu��o das partes opcionais, entretanto, n�o � garantida. Nesta proposta, as partes opcionais s�o executadas somente se os \deadlines{} das partes obrigat�rias e o tempo de dura��o da bateria desejado s�o sustentados. A figura~\ref{fig:energiaTempo} representa as tarefas que atendem ao par�metro de energia (tempo de dura��o do sistema) e as tarefas que atendem ao par�metro do tempo (\deadline{} das partes obrigat�rias). A intersec��o dessas representa��es indica as tarefas que podem ser executadas e que ser�o atendidas em rela��o aos dois par�metros desejados (energia e tempo). As tarefas fora dessa intersec��o n�o s�o escalon�veis neste algoritmo. 

Our scheduler, based on \textsc{EDF}, guarantees the execution of mandatory 
subtasks with their deadlines respectively met, independently of 
%the without taking into account 
the system energy level. However, the optional subtasks execution is not
guaranteed. The optional subtasks are executed only if the
mandatory subtasks deadlines and the system's batteries lifetime desired by
application are met. Figure~\ref{fig:energiaTempo} presents the tasks that meet
the energy parameter, i.e., system's batteries lifetime and the tasks that meet
the time parameter, i.e., the mandatory subtasks deadlines. Intersection of
these representations indicates the tasks that can be executed and that will be
met in relation to two parameters desired, i.e., time and energy. Tasks outside
that intersection are not schedulable in this algorithm.


\begin{figure}[!ht]
\centering
     {\includegraphics[scale=0.5]{figuras/energy.eps}}
     \caption{Intersection between time and energy.}
%     \caption{Intersec��o entre a energia e o tempo.}
     \label{fig:energiaTempo}
\end{figure}


%- Objetivo n�o � apenas economizar energia, mas otimizar a funcionalidade do sistema. Executar partes opcionais... 

%O objetivo deste escalonador n�o � apenas economizar a energia consumida no sistema, pois, caso contr�rio, a t�cnica seria simplesmente nunca executar as partes opcionais. A partir disso, o objetivo � atender o tempo especificado pela aplica��o com a execu��o dentro dos \deadlines{} das partes obrigat�rias e com a execu��o do m�ximo poss�vel das partes opcionais, ou seja, otimizar a funcionalidade do sistema. 

The objective of this scheduler is not only save the energy consumed in the
system --- otherwise, the technique would simply never execute the optional
subtasks --- but to meet the battery lifetime specified by
the application and to meet the mandatory subtasks deadlines with the execution 
of the maximum possible of the optional subtasks, thus optimizing the
application functionality.


%- Formalizar o algoritmo de escalonamento. (diagrama de sequencia seria interessante)

%O algoritmo do escalonador proposto neste trabalho � apresentado na figura~\ref{fig:algoritmo}. Neste algoritmo, $\pi$ � o intervalo entre medi��es da carga da bateria que pode ser especificado pelo programador da aplica��o e que deve levar em considera��o que cada medi��o tamb�m consome energia para ser realizada. Esse intervalo depende do estado de energia da bateria constatado na �ltima medi��o. Caso a �ltima medi��o constate que existe energia suficiente e que ultrapasse um determinado \thr{}, o valor do intervalo ser� maior, pois o sistema n�o necessita que sejam realizadas medi��es freq�entes. Entretanto, caso a �ltima medi��o constate que a energia existente n�o � suficiente para atender o tempo de dura��o especificado, as medi��es devem ser mais freq�entes, pois tarefas opcionais est�o sendo descartadas.


Figure ~\ref{fig:algoritmo} presents proposed scheduler algorithm. 
In proposed scheduler, the subtasks are treated as tasks in terms of scheduling.
$\pi$ is the interval among battery charge measurements
that can be specified by the application programmer and must take into consideration
that each measurement consumes energy to be performed. This interval depends on
the battery power state found in the last measurement. If the last measurement
finds that there is sufficient energy and that exceed a certain threshold, the
interval will be greater, because the system does not need to be made frequent
measurements. However, the measurements must be more frequent if the last 
measurement finds that energy is not enough to meet battery lifetime specified 
because optional subtasks are being discarded. 


%\begin{center}  
%$D$ = $\left [ \frac{n-1}{n+1}N\sqrt{2} \right]$
%\end{center}

\begin {scriptsize}
\begin {center}
 \rule[0.1ex]{35em}{0.2ex}
\end {center}

%\noindent 1: {\textbf{A cada tarefa que entra no estado de \textit{Pronto}:}}\\
\noindent 1: {\textbf{For} every task that enters in \textit{READY} state:}\\
%2:\indent calcula o novo \deadline{} absoluto de acordo com o tempo decorrido\\
2:\indent Determine the new absolute deadline in accordance with the elapsed time\\
%3:\indent calcula a prioridade baseada no \deadline{} absoluto \\
3:\indent Determine the priority based on absolute deadline \\
%4:\indent adiciona na fila de acordo com a prioridade calculada\\
4:\indent Add to the queue according to calculated priority\\
5:\\
%\noindent 6: {\textbf{A cada $\pi$ unidades de tempo:}}
\noindent 6: {\textbf{For} each $\pi$ time units:}
%\hfill{/* $\pi$ especificado pelo programador e depende do estado de energia */}\\
\hfill{/* $\pi$ specified by the programmer and it depends on the energy state */}\\
%7:\indent afere a bateria\\
7:\indent Measure the battery\\
%8:\indent verifica se existe energia suficiente para atender o tempo desejado pela aplica��o\\
8:\indent Check if there is enough energy to meet the time required by application\\
9:\\
%\noindent 10: {\textbf{A cada reescalonamento:}}\\
\noindent 10: {\textbf{For} each rescheduling:}\\
%11:\indent seleciona na fila a tarefa com estado \textit{Pronto} de mais alta prioridade \\
11:\indent Select the highest priority task in \textit{READY} state \\
%12:\indent \textbf{SE}, tarefa � \emph{hard} de tempo real,\\
12:\indent \textbf{if}, task is hard real-time, \textbf{then}\\
%13:\indent \hspace{2em} executa a tarefa selecionada\\
13:\indent \hspace{2em} Execute the selected task\\
%14:\indent \textbf{SENAO}, 
14:\indent \textbf{else}, 
%\hfill{/* tarefa � melhor esfor�o */}\\
\hfill{/* task is ``best effort'' */}\\
%15:\indent \hspace{2em} \textbf{SE}, existe energia suficiente para atender o tempo de dura��o requerido,\\
15:\indent \hspace{2em} \textbf{if}, there is enough energy to meet the system lifetime required, \textbf{then}\\
%16:\indent \hspace{4em} executa a tarefa selecionada\\
16:\indent \hspace{4em} Execute the selected task\\
%17:\indent \hspace{2em} \textbf{SENAO}, 
17:\indent \hspace{2em} \textbf{else}, 
%\hfill{/* bateria n�o possui energia suficiente */}\\
\hfill{/* Battery does not have sufficient energy */}\\
%18:\indent \hspace{4em} executa a ger�ncia de energia\\
18:\indent \hspace{4em} Use power management techniques\\
19:\\

\begin {center}
 \rule[0.1ex]{35em}{0.2ex}
\end {center}

\begin {figure}[!h] 
\centering 
%\caption {Algoritmo do escalonador proposto.}
\caption {Proposed scheduler algorithm.}
\label {fig:algoritmo}
\end {figure}
\end {scriptsize}

%\begin{figure*}[!ht]
\linespread{1}
\begin{center}
\begin{footnotesize}

\lstset{language=c++,frame=lrtb}
\lstset{basicstyle=\ttfamily}
\lstset{commentstyle=\textit}

  \begin{minipage}{14cm}
    \lstinputlisting{figuras/algoritmoEscalonador.tex}
  \end{minipage}

\caption{Algoritmo do escalonador proposto.}
\label{fig:algoritmo}
\end{footnotesize}
\end{center}
\end{figure*}



%- A formaliza��o matem�tica do escalonador pode ser observada com rela��o ao tempo e � energia, respectivamente, na figura~\ref{fig:formalizacaoTempo} e na figura~\ref{fig:formalizacaoEnergia}.

%N�s apresentamos algumas equa��es que verificam em tempo de projeto e em tempo de execu��o se as tarefas s�o escalon�veis no nosso algoritmo. As equa��es em tempo de projeto s�o descritas com maiores detalhes na se��o~\ref{sc:proposta:proposta:projeto}. A se��o~\ref{sc:proposta:proposta:execucao} apresenta as equa��es em tempo de execu��o.

% Coment�rio Lucas: novamente os sub-indices!

We present some equations that verify at project-time and at execution-time if
the tasks are schedulable in our algorithm.
Section~\ref{sc:proposta:proposta:projeto} describes the equations at 
project-time with more detail. Section~\ref{sc:proposta:proposta:execucao} 
presents the the equations at execution-time.


\subsection{Equations at Project-Time}
\label{sc:proposta:proposta:projeto}

%Como o escalonador proposto � baseado no algoritmo \textsc{EDF}, � poss�vel seguir a mesma l�gica para o c�lculo da escalonabilidade das tarefas em tempo de projeto com algumas adapta��es. Supondo que o sistema de tempo real considerado possua $n$ tarefas peri�dicas e independentes, {\Large $\tau$} = $\{\tau_0,\tau_1,...,\tau_{n-1}\}$, sendo $\forall\tau_i$, $D_i=P_i$ . No modelo da computa��o imprecisa, cada $\tau_i$ � dividida em parte obrigat�ria e parte opcional com tempos de execu��es nos piores casos, respectivamente, de $\mu_i$ e $\theta_i$. Com isso, o tempo total de execu��o de $\tau_i$ no  pior caso � $C_i = \mu_i + \theta_i$ . 

The proposed scheduler is based on the \textsc{EDF} algorithm, thus it is
possible to follow the same logic to calculate the tasks schedulability at
project-time with a few adjustments. Suppose that the real-time system
considered has $n$ periodic and independent tasks, {\Large $\tau$} =
$\{\tau_0,\tau_1,...,\tau_{n-1}\}$, where $\forall\tau_i$, $D_i=P_i$ . In the
imprecise computation model, each $\tau_i$ is divided into mandatory and
optional subtasks with execution times in the worst cases of $\mu_i$ and
$\theta_i$, respectively. Therefore, the total execution time of $\tau_i$, in
the worst case, is $C_i = \mu_i + \theta_i$ .


%Para atender o nosso objetivo em rela��o aos \deadlines{} das partes obrigat�rias, a equa��o (\ref{eq:escalonador:m}) deve ser respeitada

In order to guarantee that no mandatory subtasks deadlines will be lost, 
equation (\ref{eq:escalonador:m}) must be respected

\begin{equation}
\sum_{i=1}^n \left (\frac{\mu_i}{D_i} \right) + \sigma \le \omega
\label{eq:escalonador:m}
\end{equation}

%\[ \sum_{i=1}^n \left (\frac{\mu_i}{D_i} \right) + \sigma \le \omega \]

%onde $\omega = 1$ para um sistema com mono-processador, e $\sigma$ representa o pior caso de interfer�ncias, que inclui: tempo gasto no sistema operacional, nas trocas de contexto, no pr�prio algoritmo de escalonamento, entre outros. A equa��o (\ref{eq:escalonador:m}) deve ser atendida para as tarefas serem escalon�veis em rela��o aos \deadlines{} das partes obrigat�rias, caso contr�rio ($\sum_{i=1}^n \left (\frac{\mu_i}{D_i} \right) + \sigma > \omega$), o processador estar� sobrecarregado. 

\noindent where $\omega = 1$ for a system with a single-processor and $\sigma$
represents the interference in the worst cases, which includes: time spent in
the operating system, context switch, scheduler algorithm.
Equation (\ref{eq:escalonador:m}) must be met in order for the tasks to be
schedulable in relation to mandatory subtasks deadlines, otherwise
($\sum_{i=1}^n \left (\frac{\mu_i}{D_i} \right) + \sigma > \omega$), the
processor is overloaded.


%Com a inclus�o do tempo de execu��o da parte opcional na equa��o (\ref{eq:escalonador:m}), podemos determinar se as tarefas, como um todo, ser�o executadas (parte obrigat�ria e parte opcional). Entretanto, � importante observar que isso n�o � um requisito fundamental no nosso algoritmo e ser� relevante, apenas, quando a equa��o (\ref{eq:escalonador:m}) � v�lida, caso contr�rio, as tarefas j� n�o seriam escalon�veis. 

With the inclusion of the optional subtask execution time in equation
(\ref{eq:escalonador:m}), we can determine if the tasks as a whole will be
executed, mandatory and optional subtasks. However, it is important to note that
equation (\ref{eq:escalonador:mo}) is not a obligatory requirement in our 
algorithm and only will be relevant when
equation (\ref{eq:escalonador:m}) is true, otherwise, the tasks are not
schedulable.


\begin{equation}
\sum_{i=1}^n \left (\frac{\mu_i + \theta_i}{D_i} \right) + \sigma \le \omega
\label{eq:escalonador:mo}
\end{equation}


%\[ \sum_{i=1}^n \left (\frac{\mu_i + \theta_i}{D_i} \right) + \sigma \le \omega \]

%Quando a equa��o (\ref{eq:escalonador:mo}) for respeitada, as tarefas s�o 100\% escalon�veis (parte obrigat�ria e parte opcional) em rela��o aos seus \deadlines{}. Caso contr�rio, uma certa fra��o $\chi$ das partes opcionais � descartada. A equa��o (\ref{eq:escalonador:x}) apresenta como encontrar a fra��o $\chi$. 

Mandatory and optional subtasks are schedulable in relation to
their deadlines when equation (\ref{eq:escalonador:mo}) is respected.
Otherwise, a certain fraction $\chi$ of optional subtasks is discarded. Equation
(\ref{eq:escalonador:x}) presents how to find the fraction $\chi$.


\begin{equation}
\chi = \frac{\sum_{i=1}^n \left (\frac{\mu_i + \theta_i}{D_i} \right)
+ \sigma - \omega}{\sum_{i=1}^n \left (\frac{\theta_i}{D_i}
\right)}
\label{eq:escalonador:x}
\end{equation}

%\[ \chi = \frac{\sum_{i=1}^n \left (\frac{\mu_i + \theta_i}{D_i} \right)
%+ \sigma - \omega}{\sum_{i=1}^n \left (\frac{\theta_i}{D_i}
%\right)} \] 

%\[ \chi = \frac{\left (\sum_{i=1}^n \left (\frac{\mu_i + \theta_i}{D_i} \right)
%+ \sigma - \omega \right ) \times 100}{\sum_{i=1}^n \left (\frac{\theta_i}{D_i}
%\right)} \] 

%O objetivo em rela��o � energia pode ser alcan�ado seguindo o mesmo tipo de racioc�nio l�gico que foi realizado at� o presente momento, mas tendo em vista a taxa do consumo de energia das tarefas. O consumo de energia de $\tau_i$ no pior caso, $E_i$, � dado pela soma dos consumos de energia da parte obrigat�ria e da parte opcional nos piores casos, respectivamente, $E_{\mu i}$ e $E_{\theta i}$ , ($E_i = E_{\mu i} + E_{\theta i}$). O n�mero m�ximo poss�vel de execu��es, $\eta_i$, de $\tau_i$ no tempo requerido pela aplica��o, $T_t$, � dado pela divis�o entre o tempo requerido e o intervalo de execu��o de $\tau_i$, ou seja, $\eta_i = \frac{T_t}{D_i}$ .

The energy-related objective can be achieved by following the same kind
of logic presented thus far, but taking into account the tasks' energy
consumption rate. The $\tau_i$ energy consumption in the wort case, $E_i$, is
given by the sum of the energy consumption in the mandatory and optional
subtasks worst cases times $E_{\mu i}$ e $E_{\theta i}$, respectively, ($E_i =
E_{\mu i} + E_{\theta i}$). We suppose that, as worst cases times, the worst
cases energy consumptions are previously known by the application developer.
These values can be obtained by energy profiling or another techniques. 
The maximum number of possible executions $\eta_i$ of $\tau_i$ in the time
required by application $T_t$ is given by division between the time required and
the execution interval of $\tau_i$, i.e., $\eta_i = \frac{T_t}{P_i}$ . $T_t$ is
given by the application developer based on the battery' capacity.


%Com o intuito de atender, no m�nimo, as partes obrigat�rias das tarefas, temos a equa��o (\ref{eq:escalonador:en}) que indica se o conjunto das tarefas ser� escalon�vel em rela��o � energia. 

In order to meet at least the mandatory parts of the tasks, we have equation
(\ref{eq:escalonador:en}) which indicates if the set of tasks will be
schedulable with respect to energy.

\begin{equation}
\sum_{i=1}^n \left (\frac{E_{\mu i} \times \eta_i}{E_t} \right) + \epsilon \le 1
\label{eq:escalonador:en}
\end{equation}


%\[ \sum_{i=1}^n \left (\frac{E_{\mu i} \times \eta_i}{E_t} \right) + \epsilon \le
%1 \]

%Onde $E_t$ � a energia total do sistema (especifica��o da bateria), ou seja, a capacidade da bateria, $\epsilon$ representa o pior caso do consumo de energia de diferentes fatores, como, a energia consumida pelo sistema operacional, pelas trocas de contexto, pelo pr�prio algoritmo de escalonamento, entre outros. A capacidade do sistema em rela��o � energia � definida como 1, ou seja, 100\%. Substituindo $\eta_i$ de $\tau_i$ na equa��o (\ref{eq:escalonador:en}) temos a equa��o (\ref{eq:escalonador:e}). 

\noindent where $E_t$ is the total energy of the system (battery specification), i.e.,
battery' capacity, $\epsilon$ represents energy consumption in the worst case of
different factors such as the energy consumed by the operating system, the
context switch, the scheduler algorithm itself. The battery's capacity is set to
1, i.e., 100 \%. Substituting $\eta_i$ $\tau_i$ in the equation
(\ref{eq:escalonador:en}) we have equation (\ref{eq:escalonador:e}).


%$\varrho = 1$

\begin{equation}
\sum_{i=1}^n \left (\frac{E_{\mu i} \times T_t}{P_i \times E_t} \right) + \epsilon \le 1
\label{eq:escalonador:e}
\end{equation}


%\[ \sum_{i=1}^n \left (\frac{E_{\mu i} \times T_t}{D_i \times E_t} \right) + \epsilon \le
%1 \]

%As tarefas s�o escalon�veis em rela��o � energia no nosso algoritmo se a equa��o (\ref{eq:escalonador:e}) for atendida. Caso contr�rio ($\sum_{i=1}^n \left (\frac{E_{\mu i} \times T_t}{D_i \times E_t} \right) + \epsilon > 1$), o sistema n�o atender� ao tempo de dura��o requerido pela aplica��o para esse conjunto de tarefas.

The tasks are schedulable in relation to energy in our algorithm if equation
(\ref{eq:escalonador:e}) is respected. Otherwise ($\sum_{i=1}^n \left
(\frac{E_{\mu i} \times T_t}{P_i \times E_t} \right) + \epsilon > 1$), the
system will not meet the battery lifetime required by application for this set
of tasks.

%A inclus�o da energia consumida pelas partes opcionais no pior caso na equa��o (\ref{eq:escalonador:e}) possibilita que verifiquemos se as tarefas, como um todo (parte obrigat�ria e parte opcional), s�o executadas. Como j� explicado anteriormente, isso n�o � um requisito obrigat�rio e a equa��o (\ref{eq:escalonador:eo}) s� deve ser calculada se a equa��o (\ref{eq:escalonador:e}) � respeitada (partes obrigat�rias atendidas).

The inclusion of the energy consumed by optional subtasks in
equation (\ref{eq:escalonador:e}) allows us to check if the tasks as a whole
will be executed, mandatory and optional subtasks. As discussed previously, this
is not an obligatory requirement and equation (\ref{eq:escalonador:eo}) only
should be calculated if the equation (\ref{eq:escalonador:e}) is respected,
i.e., mandatory subtasks met.


\begin{equation}
\sum_{i=1}^n \left (\frac{\left (E_{\mu i} + E_{\theta i} \right ) \times T_t}{P_i
\times E_t} \right) + \epsilon \le 1
\label{eq:escalonador:eo}
\end{equation}


%\[ \sum_{i=1}^n \left (\frac{\left (E_{\mu i} + E_{\theta i} \right ) \times T_t}{D_i
%\times E_t} \right) + \epsilon \le 1 \]

%Caso a equa��o (\ref{eq:escalonador:eo}) seja respeitada, todas as partes obrigat�rias e opcionais das tarefas s�o executadas em rela��o � energia do sistema. Caso contr�rio, uma determinada fra��o $\gamma$ das partes opcionais n�o ser� executada, pois o sistema n�o atenderia ao tempo de dura��o desejado pela aplica��o. A equa��o (\ref{eq:escalonador:y}) fornece a fra��o de partes opcionais descartadas em rela��o � energia.

All mandatory and optional parts of the tasks are executed in relation to system
energy if equation (\ref{eq:escalonador:eo}) is respected. Otherwise, a
certain fraction $\gamma$ of optional subtasks will not be executed because the
system would not meet the battery lifetime specified by the application. Equation
(\ref{eq:escalonador:y}) provides a fraction $\gamma$ of optional subtasks
discarded in relation to energy.

\begin{equation}
\gamma = \frac{\sum_{i=1}^n \left (\frac{ \left (E_{\mu i} + E_{\theta i} \right ) \times T_t}{P_i \times E_t} \right) + \epsilon - 1 }{\sum_{i=1}^n \left (\frac{E_{\theta i} \times T_t}{P_i \times E_t} \right)}
\label{eq:escalonador:y}
\end{equation}


%\[ \gamma = \frac{\sum_{i=1}^n \left (\frac{ \left (E_{\mu i} + E_{\theta i} \right ) \times T_t}{D_i \times E_t} \right) + \epsilon - 1 }{\sum_{i=1}^n \left (\frac{E_{\theta i} \times T_t}{D_i \times E_t} \right)} \] 

%\[ \gamma = \frac{\left (\sum_{i=1}^n \left (\frac{ \left (E_{\mu i} + E_{\theta i} \right ) \times T_t}{D_i \times E_t} \right) + \epsilon - 1 \right ) \times 100}{\sum_{i=1}^n \left (\frac{E_{\theta i} \times T_t}{D_i \times E_t} \right)} \] 

%Neste algoritmo, o objetivo � atender os dois par�metros em rela��o ao tempo e � energia, respectivamente, os \deadlines{} das partes obrigat�rias e o tempo de dura��o da bateria especificado pela aplica��o. Com isso, (\ref{eq:escalonador:te}) � a equa��o completa do nosso escalonador que deve ser verdadeira para as tarefas serem escalon�veis.

In this algorithm, the objective is to meet the two parameters in relation
to time and energy, i.e., the mandatory subtasks deadlines and battery lifetime
specified by the application, respectively. Thus, (\ref{eq:escalonador:te}) is the
full equation of our scheduler that must be true in order to tasks will be
schedulable.


\begin{equation}
\left [ \sum_{i=1}^n \left (\frac{\mu_i}{D_i} \right) + \sigma \le \omega
\right ]  \wedge  \left [ \sum_{i=1}^n \left (\frac{E_{\mu i} \times T_t}{P_i \times E_t}
\right) + \epsilon \le 1 \right ]
\label{eq:escalonador:te}
\end{equation}


%\[ \left [ \sum_{i=1}^n \left (\frac{\mu_i}{D_i} \right) + \sigma \le \omega
%\right ]  \wedge  \left [ \sum_{i=1}^n \left (\frac{E_{\mu i} \times T_t}{D_i \times E_t}
%\right) + \epsilon \le 1 \right ] \]

%Caso a equa��o (\ref{eq:escalonador:te}) seja respeitada, as partes obrigat�rias das tarefas tem as execu��es garantidas no nosso escalonador em rela��o aos dois par�metros desejados neste trabalho. A fra��o m�xima $\lambda$ poss�vel de tarefas opcionais perdidas em rela��o aos dois par�metros pode ser obtida atrav�s da equa��o (\ref{eq:escalonador:l}).

The mandatory subtasks have their executions guaranteed in our scheduler in
relation to time and energy if equation (\ref{eq:escalonador:te}) is
respected. The maximum fraction $\lambda$ possible of optional subtasks lost in
relation to time and energy can be obtained by the equation
(\ref{eq:escalonador:l}).


\begin{equation}
\lambda = \max \left ( \chi , \gamma  \right)
\label{eq:escalonador:l}
\end{equation}


%\[ \lambda = \max \left ( \chi , \gamma  \right)\]

%Cada $\tau_i$ nesta abordagem com os dois par�metros (tempo e energia) � caracterizado neste modelo por oito par�metros, $(P_i, D_i, C_i, \mu_i, \theta_i, E_i, E_{\mu i}, E_{\theta i})$, mencionados anteriormente. 


Each $\tau_i$ in this approach with two parameters, time and energy, is
characterized by eight parameters $(P_i, D_i, C_i, \mu_i, \theta_i, E_i, E_{\mu
i}, E_{\theta i})$ mentioned earlier.



%\input{figuras/tempo.tex}
%\input{figuras/energia.tex}


%\subsection{Estimativa do Tempo de Dura��o do Sistema}
%\label{sc:proposta:estimativa}

\subsection{Equation at Execution-Time}
\label{sc:proposta:proposta:execucao}

%- Proposta para a estimativa do tempo restante de dura��o da bateria. Utilizo uma t�cnica para estimar o tempo restante da bateria. Estimativa do tempo restante na figura~\ref{fig:energiaTempoRestante}.

%Com objetivo de prover \qos{} em termos de energia e aproveitar melhor os recursos com a utiliza��o das partes opcionais � necess�rio verificar periodicamente, em tempo de execu��o, se o tempo de dura��o do sistema requerido pela aplica��o, $T_{t \kappa}$, no instante $\kappa$ pode ser alcan�ado. Para isso, $T_{t \kappa}$ � recalculado no instante $\kappa$ de acordo com o tempo decorrido. A energia total do sistema (carga da bateria), $E_{t \kappa}$, tamb�m, deve ser  recalculada no instante $\kappa$ . As plataformas dos sistemas embarcados, normalmente, prov�m mecanismos para obter a carga da bateria. Os novos valores podem realimentar a equa��o (\ref{eq:escalonador:ek}) com o intuito de verificar se $T_{t \kappa}$ pode ser atendido. 


In order to provide \qos{} in terms of energy and make better use the resources
with the optional subtasks execution it is necessary periodically to check at
execution-time if the battery lifetime specified by the application $T_{t \kappa}$
in the instant $\kappa$ can be achieved. Therefore, $T_{t \kappa}$ is
recalculated in the instant $\kappa$ according to the elapsed time. The total
energy of the system (battery charge) $E_{t \kappa}$ also must be recalculated
in the instant $\kappa$. The embedded systems platforms usually provide
mechanisms to get the battery charge. Equation (\ref{eq:escalonador:ek}) can
be recalculated with the new values in order to check if $T_{t \kappa}$ can be
met in the instant $\kappa$.


\begin{equation}
\sum_{i=1}^n \left (\frac{E_{\mu i} \times T_{t \kappa}}{P_i \times E_{t \kappa}}
\right) + \epsilon \le 1
\label{eq:escalonador:ek}
\end{equation}


%\[ \sum_{i=1}^n \left (\frac{E_{\mu i} \times T_{t \kappa}}{D_i \times E_{t \kappa}}
%\right) + \epsilon \le 1 \]

%Caso a equa��o (\ref{eq:escalonador:ek}) seja atendida, todas as partes das tarefas s�o executadas (partes obrigat�rias e partes opcionais), pois essa equa��o indica que existe energia suficiente para atender $T_{t \kappa}$. Caso contr�rio, as partes opcionais ser�o descartadas. O escalonador chama um gerente do consumo de energia no tempo em que as partes opcionais estariam em execu��o, aproveitando o tempo ocioso do sistema para economizar energia. Quando for constatado que a equa��o (\ref{eq:escalonador:ek}) volta a ser verdadeira, as partes opcionais das tarefas voltam a ser executadas.


All mandatory and optional parts of the tasks are executed in relation to energy
if the equation (\ref{eq:escalonador:ek}) is respected because this equation
indicates there is sufficient energy to meet $T_{t \kappa}$. Otherwise, some
optional subtasks will be discarded. The scheduler calls a power manager in
the time that the optional subtasks would be in execution. Thus, it takes the
idle time of the system in order to save energy. The optional subtasks return to
be executed when it is observed that the equation (\ref{eq:escalonador:ek}) 
returns to be true.













%Um monitor l� a carga da bateria
%periodicamente e estima o tempo de dura��o da bateria considerando que a
%velocidade de descarga � constante. Apesar desta t�cnica apresentar uma baixa
%precis�o, uma vez que a velocidade de descarga n�o � constante em todos os
%casos, ela � uma solu��o com um baixo \ov{} para a aplica��o. 
%A figura~\ref{fig:formalizacaoEstimativaEnergia} apresenta a formaliza��o 
%matem�tica da estimativa de energia para o teste em tempo de execu��o.



%\begin{figure*}[!ht]
\linespread{1}
\begin{center}
\begin{footnotesize}

\lstset{language=c++,frame=lrtb}
\lstset{basicstyle=\ttfamily}
\lstset{commentstyle=\textit}

%  \begin{minipage}{13cm}
%    \lstinputlisting{figuras/formalizacaoTempo.tex}
%  \end{minipage}

  \begin{minipage}{14cm}
    \lstinputlisting{figuras/formalizacaoEstimativaEnergia.tex}
  \end{minipage}

\caption{Formaliza��o da estimativa de eneriga em tempo de execu��o.}
\label{fig:formalizacaoEstimativaEnergia}
\end{footnotesize}
\end{center}
\end{figure*}



%Ser� que precisa a figura~\ref{fig:energiaTempoRestante}.

%\begin{figure}[!ht]
%\centering
%     {\includegraphics[scale=0.5]{figuras/energiaTempoRestante.eps}}
%     \caption{Tempo restante}
%     \label{fig:energiaTempoRestante}
%\end{figure}


% ------------------------------------------------------------------------------
% Section <Evaluation>
% ------------------------------------------------------------------------------
\section{Cases study} \label{sec:eval}
This section presents the evaluation of the proposal to interface hardware
devices and embedded VHLL, introduced at Section \ref{sec:proposal}, on the
context of embedded \java~and embedded \lua.
In order to evaluated our proposal three cases study are presented: serial
communication, distributed motion estimation, and temperature sensing.
The proposal was evaluate according to the following metrics: performance,
memory consumption, platform portability, developer support, and reuse of
native code adapters among distinct FFIs.

\subsection{Serial Communication}
% NOTA:  Acho que é interessante citar os meus papers anteriores IADIS AC 2011.
% Algo como, resultados iniciais mostrados para KESO em \cite...
The first case study evaluated as a synthetic application which uses the
\emph{Universal Asynchronous Receiver Transmitter} (UART) hardware device for
serial communication.
Such application, was evaluated in \java, using the KESO FFI, for the
architectures IA32, AVR8, and PPC32.
The Figure \ref{progjava:uart_app} shows the \java~source of the application,
which uses the UART hardware mediator in order to write characters on a serial
device.

\progjava{uart_app}{UART application.}

The \java~ class \emph{UART}, generated as one of the outputs of EBG, is the
\java~counterpart of the UART hardware mediator and has only native methods
without implementation.
The other output of EBG is the native code adapter already tailored to be
integrated with the KESO JVM.
This integration process is show in details in our previous work
\cite{Ludwich:IADIS_AC:2011}.
At such work the UART application using our approach is compared to the a
equivalent application written using the \emph{Java Standard Edition} platform,
which uses the \emph{Java Native Interface} (JNI) as FFI, running on Linux.
Our approach is around 38X faster than using JNI and Linux.

\subsection{Distributed Motion Estimation}
% Dizer que nosso trabalho anterior mostra a versão em Java feita a mão?
% Dizer que neste trabalho a versão Java foi automaticamente gerada e mostrar a
% versão em Lua.
The second case study for the proposed method is a real application which uses
a component to perform \emph{Motion Estimation} (ME) for H.264 video encoding.
In this case study we have generated native code adapters for \java~and \lua.
Such component for ME computation was developed for the project the Brazilian
project ``Rede H.264'', which aims to develop standards and products for the
Brazilian Digital Television \cite{RedeH264:2009}.

Motion Estimation is a technique used to explore the similarity between
neighboring frames in a video sequence, thus enabling them to be differentially
encoded, improving the compress ratio of the generated bitstream
\cite{citeulike:1269699}.
ME is a significant stage for H.264 encoding, since it consumes around 90\% of
the total time of the encoding process \cite{XiangLi:2004}.
In order to improve the performance of ME, the component uses a data
partitioning strategy where the motion estimation for each partition of the
picture is performed in parallel in a specific functional unit, such as a core
of a multicore processor.

% O DMEC é implementado como um componente em C++ e é exportado para a
% \textit{VHLL}
% alvo desenvolvendo-se um adaptador de código nativo para cada objeto sendo
% abstraído.

Our previous work shows handmade native code adapters for KESO
JVM \cite{Ludwich:IADIS_AC:2011}.
In the present work we have used the EBG to generate the same adapter for KESO
JVM and also an adapter for the LuaVM.
In the case of \lua~the native code adapter is a C code using the FFI of \lua~
that is ready to be used.
In the case of KESO JVM the native code adapter is a \emph{weavelet}
class, which is used during the KESO generation process, generating the final
native code adapter.

In order to test the ME component we wrote an application with mimics the
behavior of an H.264 encoder, it provides the component with pictures, get from
the component the ME results (motion vectors and motion cost), and checks if
the results are correct.
The Figure \ref{progjava:dmec_java_app} shows the Java version of the
application, and the Figure \ref{proglua:dmec_lua_app} shows the same
application written in \lua.

\progjava{dmec_java_app}{ME Java application.}

\proglua{dmec_lua_app}{ME Lua application.}

% \multprogjavatwoh{dmec_java_app}{dmec_lua_app}{ME Java and Lua applications.}


\subsection{Temperature Sensing}
The third case study is an application for temperature sensing.
This is a distributed application, composed by a \emph{Sensor} node which
measures the temperature and sends the obtained measures to a \emph{Sink} node
which receive the temperature values and process them.
The communication between the nodes is performed by radio on the context of a
\emph{Wireless Sensor Network} (WSN).

The Figure \ref{progjava:sensor_app} shows the application executed by the
Sensor node, and the Figure \ref{progjava:sink_app} shows the application
executed by the Sink node.
Both applications are written in \java.
We have generated native code adapters for the temperature sensor
mediator (\emph{Temperature\_Sensor}), and for the network interface card
(\emph{NIC}) mediator which abstracts the radio used for communicating between
the nodes.

\progjava{sensor_app}{Sensor application.}
\progjava{sink_app}{Sink application.}

The target FFIs were KESO FFI and NanoVM FFI.
The application remains the same for both virtual machines.
The application which runs on KESO JVM was deployed in the AVR8 (8 bit)
architecture, and the application which runs on NanoVM was deployed in the ARM7
architecture (32 bit).
The platform used was the \emph{EPOS Mote} (version AVR8 and ARM7).
\emph{EPOS Mote} is an open source and open hardware mote for WSN which runs
the EPOS system \cite{EposMoteSite}.

In order to evaluate the performance of the generated native code adapters,
we have measured the time overhead for accessing the hardware mediators used by
the temperature sensing application.
Table \ref{tab:tempsensing_app_time_overhead_keso_epos_avr8} shows the
results for the AVR8 architecture using the KESO FFI.
Table \ref{tab:tempsensing_app_time_overhead_nanovm_epos_arm7} shows the
results for the ARM7 architecture using the NanoVM FFI.
For the temperature sensor device was measured the time of the \emph{sample}
method which samples the temperature from the environment.
For the NIC has measured the time for send an arbitrary message.
%  
\begin{table}[t]
\begin{center}
\begin{tabular}{|c|c|c|c|}
\hline
\textbf{Method} & \textbf{Total ($\mu s$)} &
\textbf{Device ($\mu s$)} &
\textbf{Overhead (\%)}\\
\hline
\emph{Temp::sample} & 334.22 & 330 & 1.26 \\
\hline
\emph{NIC::send} & 8586.22 & 8580 & 0.072 \\
\hline
\end{tabular}
\caption{Time overhead generated by the native code adapters.
Architecture AVR8, KESO FFI.}
\label{tab:tempsensing_app_time_overhead_keso_epos_avr8}
\end{center}
\end{table}
% 
\begin{table}[t]
\begin{center}
\begin{tabular}{|c|c|c|c|}
\hline
\textbf{Method} & \textbf{Total ($\mu s$)} &
\textbf{Device ($\mu s$)} &
\textbf{Overhead (\%)}\\
\hline
\emph{Temp::sample} & 946.04 & 942 & 0.43 \\
\hline
\emph{NIC::send} & 958.39 & 950 & 0.87 \\
\hline
\end{tabular}
\caption{Time overhead generated by the native code adapters.
Architecture ARM7, NanoVM FFI.}
\label{tab:tempsensing_app_time_overhead_nanovm_epos_arm7}
\end{center}
\end{table}
% 
%
The ``Total'' time and the ``Device'' time from the two tables are not
comparable, since they are obtained from distinct platforms.
However, the overhead value is, and it is less than 1.3\% for all devices.
The time overhead generated by NanoVM should be bigger than in KESO JVM because
NanoVM has an overhead for interpreting the \java~bytecode.
KESO JVM does not have this kind of overhead since all bytecode is translated
to C and then to native code.
However, this is not seen comparing the tables because the bytecode
interpretation overhead was not measured for NanoVM.
The measure overhead from NanoVM comes specifically from the native method,
which uses ``push'' and ``pop'' operations to interact with NanoVM stack for
obtaining the method's arguments and for returning the method's results.
\input{tex/dmec_app}
% % ------------------------------------------------------------------------------
% NOTA TODO: Revisar / Reescrever esta subseção: Serial Communication
\subsection{Serial Communication}
% NOTA:  Acho que é interessante citar os meus papers anteriores IADIS AC 2011.
% Algo como, resultados iniciais mostrados para KESO em \cite...
The first case study evaluated as a synthetic application which uses the
\emph{Universal Asynchronous Receiver Transmitter} (UART) hardware device for
serial communication.
Such application, was evaluated in \java, using the KESO FFI, for the
architectures IA32, AVR8, and PPC32.
The Figure \ref{progjava:uart_app} shows the \java~source of the application,
which uses the UART hardware mediator in order to write characters on a serial
device.

\progjava{uart_app}{UART application.}

The \java~ class \emph{UART}, generated as one of the outputs of EBG, is the
\java~counterpart of the UART hardware mediator and has only native methods
without implementation.
The other output of EBG is the native code adapter already tailored to be
integrated with the KESO JVM.
This integration process is show in details in our previous work
\cite{Ludwich:IADIS_AC:2011}.
At such work the UART application using our approach is compared to the a
equivalent application written using the \emph{Java Standard Edition} platform,
which uses the \emph{Java Native Interface} (JNI) as FFI, running on Linux.
Our approach is around 38X faster than using JNI and Linux.

% ------------------------------------------------------------------------------
% % ------------------------------------------------------------------------------
% NOTA TODO: Revisar / Reescrever esta subseção: Temperature Sensing
\subsection{Temperature Sensing}
The third case study is an application for temperature sensing.
This is a distributed application, composed by a \emph{Sensor} node which
measures the temperature and sends the obtained measures to a \emph{Sink} node
which receive the temperature values and process them.
The communication between the nodes is performed by radio on the context of a
\emph{Wireless Sensor Network} (WSN).

The Figure \ref{progjava:sensor_app} shows the application executed by the
Sensor node, and the Figure \ref{progjava:sink_app} shows the application
executed by the Sink node.
Both applications are written in \java.
We have generated native code adapters for the temperature sensor
mediator (\emph{Temperature\_Sensor}), and for the network interface card
(\emph{NIC}) mediator which abstracts the radio used for communicating between
the nodes.

\progjava{sensor_app}{Sensor application.}
\progjava{sink_app}{Sink application.}

The target FFIs were KESO FFI and NanoVM FFI.
The application remains the same for both virtual machines.
The application which runs on KESO JVM was deployed in the AVR8 (8 bit)
architecture, and the application which runs on NanoVM was deployed in the ARM7
architecture (32 bit).
The platform used was the \emph{EPOS Mote} (version AVR8 and ARM7).
\emph{EPOS Mote} is an open source and open hardware mote for WSN which runs
the EPOS system \cite{EposMoteSite}.

In order to evaluate the performance of the generated native code adapters,
we have measured the time overhead for accessing the hardware mediators used by
the temperature sensing application.
Table \ref{tab:tempsensing_app_time_overhead_keso_epos_avr8} shows the
results for the AVR8 architecture using the KESO FFI.
Table \ref{tab:tempsensing_app_time_overhead_nanovm_epos_arm7} shows the
results for the ARM7 architecture using the NanoVM FFI.
For the temperature sensor device was measured the time of the \emph{sample}
method which samples the temperature from the environment.
For the NIC has measured the time for send an arbitrary message.
%
\begin{table*}[t]
\begin{center}
\begin{tabular}{|c|c|c|c|}
\hline
\textbf{Method} & \textbf{Total ($\mu s$)} &
\textbf{Device ($\mu s$)} &
\textbf{Overhead (\%)}\\
\hline
\emph{Temp::sample} & 334.22 & 330 & 1.26 \\
\hline
\emph{NIC::send} & 8586.22 & 8580 & 0.072 \\
\hline
\end{tabular}
\caption{Time overhead generated by the native code adapters.
Architecture AVR8, KESO FFI.}
\label{tab:tempsensing_app_time_overhead_keso_epos_avr8}
\end{center}
\end{table*}
%
\begin{table*}[t]
\begin{center}
\begin{tabular}{|c|c|c|c|}
\hline
\textbf{Method} & \textbf{Total ($\mu s$)} &
\textbf{Device ($\mu s$)} &
\textbf{Overhead (\%)}\\
\hline
\emph{Temp::sample} & 946.04 & 942 & 0.43 \\
\hline
\emph{NIC::send} & 958.39 & 950 & 0.87 \\
\hline
\end{tabular}
\caption{Time overhead generated by the native code adapters.
Architecture ARM7, NanoVM FFI.}
\label{tab:tempsensing_app_time_overhead_nanovm_epos_arm7}
\end{center}
\end{table*}
%
%
The ``Total'' time and the ``Device'' time from the two tables are not
comparable, since they are obtained from distinct platforms.
However, the overhead value is, and it is less than 1.3\% for all devices.
The time overhead generated by NanoVM should be bigger than in KESO JVM because
NanoVM has an overhead for interpreting the \java~bytecode.
KESO JVM does not have this kind of overhead since all bytecode is translated
to C and then to native code.
However, this is not seen comparing the tables because the bytecode
interpretation overhead was not measured for NanoVM.
The measure overhead from NanoVM comes specifically from the native method,
which uses ``push'' and ``pop'' operations to interact with NanoVM stack for
obtaining the method's arguments and for returning the method's results.

% ------------------------------------------------------------------------------


% ------------------------------------------------------------------------------
% Section <Discussion>
\input{tex/conclusions}

% ------------------------------------------------------------------------------
% References
\bibliographystyle{elsarticle-num}
\bibliography{hw,os,pl,mm,se}


\end{document}

%------------------------------------------------------------------------------

