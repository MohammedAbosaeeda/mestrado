\documentclass[10pt, conference, compsocconf]{IEEEtran}
\usepackage{times}

\usepackage[english]{babel}

\usepackage[utf8]{inputenc}

\newcommand{\epos}{\textsc{Epos}}
\newcommand{\emote}{\textsc{EposMote}}
\newcommand{\lisha}{\textsc{Lisha}}

\begin{document}

\title{Minicurso II: Redes de Sensores Sem Fio\\
(Tutorial 2: Wireless Sensor Networks)}

\author{\IEEEauthorblockN{Arliones Hoeller Jr. and Antônio Augusto Fröhlich}
\IEEEauthorblockA{Software/Hardware Integration Lab\\
Federal University of Santa Catarina\\
Florianópolis, Brazil\\
\{arliones,guto\}@lisha.ufsc.br}
}

\maketitle

\thispagestyle{empty}

\begin{abstract}
Wireless sensor networks and other pervasive systems are getting common place in
our everyday life. This technology made it possible the deployment of a new
class of applications encompassing a set of new challenges. To cope with these
new requirements developments in the last decade enhanced the level of
integration of processing (e.g., CPU) and communication (e.g., radio)
components. Also, new approaches emerged to address specific requirements for
routing and medium access control (MAC) in the network. In contrast with what
has been done in the past, these new approaches had now to deal with severe
limitations in terms of processing power, data transfer power, and, the most
important, energy consumption. New operating systems emerged to target these
systems. These operating systems abstract domain-specific functionalities such
as sensing and communication, speeding up the application development process.
This tutorial will present the fundamentals behind the development of wireless
sensor network applications. Special focus will be given to
application-dependent system parameters. The practical experience obtained in
the development of \epos~\cite{Froehlich:2001} and \emote~\cite{EPOSMote:2010}
projects will be shared, including some hands-on exercises.
\end{abstract}

\begin{IEEEkeywords}
wireless sensor networks, operating system, sensing, medium access control
\end{IEEEkeywords}


\section{Identificação}

\begin{itemize}
  \item \textbf{Título do minicurso:}\\
        Redes de sensores sem fio
  \item \textbf{Autores:}\\
        Arliones Hoeller Jr. e Antônio Augusto Fröhlich
  \item \textbf{Apresentadores:}\\
        Arliones Hoeller Jr. e Antônio Augusto Fröhlich
\end{itemize}

\section{Dados gerais}

Redes de sensores sem fio e outros sistemas pervasivos são cada vez mais comuns
em nosso dia-a-dia. Esta tecnologia traz, junto das novas possibilidades de
aplicações, um grande conjunto de desafios, o que deu margem ao surgimento de
novas soluções de integração de componentes de processamento (CPU) e rádio,
novos algoritmos de roteamento, variadas implementações de controle de acesso
ao meio (MAC), sempre atendendo às rígidas restrições em termos de capacidade
de processamento, potência de transmissão e, de modo especial, consumo de
energia. Para tratar tudo isso, uma série de novos sistemas operacionais focou
no domínio de redes de sensores sem fio, abstraindo as funcionalidades de
sensoriamento e comunicação, e agilizando o processo de desenvolvimento de
aplicações. Este minicurso visa apresentar as principais características das
redes de sensores sem fio, focando em temas importantes a serem considerados
durante o desenvolvimento de aplicações, utilizando-se para isso dos trabalhos
desenvolvidos para implementação do \epos~\cite{Froehlich:2001} e do
\emote~\cite{EPOSMote:2010}.

O minicurso reúne informações relevantes das tecnologias envolvidas na aquisição
de dados e comunicação em redes de sensores sem fio, analisando estas
tecnologias e apresentando como estas foram implementadas em um módulo de
sensores sem fio, o \emote, e um sistema operacional, o \epos, desenvolvidos
pelo \lisha~(Laboratório de Integração Software/Hardware da UFSC). O \epos~é um
sistema operacional com mais de 10 anos de constante desenvolvimento. O grupo
que desenvolve o EPOS tem trabalhado com redes de sensores sem fio desde 2003, e
os resultados alcançados estão apresentados em mais de 20 publicações
relacionadas ao sistema~\cite{LISHAPubs:2010}. As informações apresentadas neste
curso devem permitir a especificação e dimensionamento de redes de sensores
sem fio para aplicações específicas.

Deste modo, o objetivo do minicurso é dotar os participantes dos conceitos
envolvidos na especificação de uma rede de sensores sem fio, incluindo
características dos módulos de sensoriamento, da(s) rede(s) de comunicação e dos
sistemas operacionais existentes.

São objetivos específicos do curso apresentar os seguintes tópicos: redes
de sensores sem fio e os principais problemas associados; arquitetura padrão de
um módulo de sensoriamento; tipos de sensores (analógicos/digitais) e aquisição
de dados; comunicação sem fio; controle de acesso ao meio (MAC) e topologias;
sistemas operacionais para redes de sensores sem fio; atividade prática
utilizando módulos de sensoriamento sem fio.


\section{Conteúdo do minicurso}
\label{sec:conteudo}

Esta seção apresenta a estrutura do minicurso.

\begin{itemize}
  \item \textbf{Redes de sensores sem fio: } Inicialmente é realizada uma  
  introdução de conceitos relacionados a redes de sensores sem fio, apresentando
  exemplos de aplicações e revisando os problemas tratados pelas tecnologias que
  serão discutidas durante o restante do curso.

  \item \textbf{Módulos de sensoriamento: } Em seguida será abordada a
  arquitetura dos módulos de sensoriamento. Será discutida a arquitetura básica
  (sensores + processador + transceptor)~\cite{Pottie:2000,Barr:2002}, assim
  como abordagens de integração destes componentes. Também serão discutidos os
  requisitos de um módulo de sensoriamento (dimensões, consumo de energia,
  modularidade, adaptabilidade do canal de comunicação).

  \item \textbf{Sensores e aquisição de dados: } Sensores são dispositivos que
  medem uma quantidade de alguma grandeza física e convertem esta medição em
  um sinal que pode ser lido por um observador ou um instrumento. Nesta parte do
  minicurso serão abordados os tipos de sensores (mecânicos, elétricos,
  eletrônicos, óticos), bem como os modelos de interfaceamento destes sensores
  com processadores (i.e., sensores analógicos e digitais).

  \item \textbf{Controle de acesso ao meio (MAC) e topologias: } Em seguida será
  feita uma revisão de mecanismos de controle de acesso ao meio (TDMA, FDMA,
  CDMA, CSMA) e a apresentação de exemplos de protocolos MAC utilizados em redes
  de sensores sem fio (B-MAC, S-MAC, T-MAC, Z-MAC, C-MAC, IEEE 802.15.4).

  \item \textbf{Sistemas Operacionais para RSSF: } Finalmente, será realizada
  uma revisão sobre características necessárias aos sistemas operacionais de
  redes de sensores sem fio (gestão de recursos, portabilidade,
  configurabilidade, multitarefa, comunicação). Também serão apresentados
  sistemas operacionais existentes (\epos, \textsc{TinyOS}, \textsc{SOS},
  \textsc{Contiki}, \textsc{MantisOS}).

  \item \textbf{Exercícios práticos: } O minicurso incluirá atividades práticas
  envolvendo o sistema operacional \epos~\cite{Froehlich:2001} em uma plataforma
  de sensoriamento desenvolvida pelo \lisha, o \emote~\cite{EPOSMote:2010}.
\end{itemize}

\section{Material de referência}

Dentre referências bibliográficas apresentadas abaixo há um texto incluindo os
tópicos apresentados na seção~\ref{sec:conteudo}~\cite{Hoeller:WSCAD:2010}. Este
texto descreve em detalhes os tópicos abordados no minicurso e apresenta a
solução desenvolvida para cada um destes tópicos no \epos~ou no \emote.

\section*{Agradecimentos}

Este minicurso não seria possível sem os importantes desenvolvimentos sobre a
plataforma \emote~e o sistema operacional \epos~realizados pelas pessoas que
passaram pelo \lisha~nos últimos 10 anos. Os autores reconhecem a importância
destes colaboradores no desenvolvimento deste material.

\nocite{Froehlich:2001}
\nocite{Wiedenhoft:OSR:2008}
\nocite{Wanner:JCC:2008}
\nocite{Hoeller:DIPES:2006}
\nocite{Marcondes:ETFA:2006}
\nocite{Lin:2004}
\nocite{Mottola:2010}

\bibliographystyle{IEEEtran}
\bibliography{lisha,wsn}

\end{document}
