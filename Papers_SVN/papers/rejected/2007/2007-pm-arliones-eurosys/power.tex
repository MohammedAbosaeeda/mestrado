\documentclass[preprint]{sigplanconf}

\usepackage[T1]{fontenc}
\usepackage[english]{babel}   
\usepackage[latin1]{inputenc}  
\usepackage{times}
\usepackage{graphicx}
\usepackage{url}
\usepackage{listings}
\usepackage{subfigure}
\usepackage{cite}


\lstloadlanguages{[ANSI]C++,HTML}
\lstdefinelanguage{XML} {
  keywords={xml,version,DOCTYPE,SYSTEM,EPOSConfig,family,member,name,type,
  feature,trait,class,performance,codesize,energy,power,default,pos,pre}}
\lstdefinestyle{prg} {basicstyle=\small\sffamily, lineskip=-0.2ex}
\lstdefinestyle{prgbox} {basicstyle=\small\sffamily lineskip=-0.2ex}
\lstdefinestyle{inlineprg} {basicstyle=\small\sffamily}

\newcommand{\fig}[4][htbp]{
  \begin{figure}[#1] {\centering{\scalebox{#4}{\includegraphics{fig/#2}}}\par}
    \caption{#3\label{fig:#2}}
  \end{figure}
}

\newcommand{\largefig}[4][htb]{
  \begin{figure*}[#1] {\centering{\scalebox{#4}{\includegraphics{fig/#2}}}\par}
    \caption{#3\label{fig:#2}}
  \end{figure*}
}


\newcommand{\prg}[4][htb]{
  \begin{figure}[#1]
%    \vspace{\parskip}
%    \makebox[\textwidth][c]{
      \lstinputlisting[language=#2,style=prg]{prg/#3.prg} %}
%    \vspace{0.4\parskip}
    \caption{#4\label{prg:#3}}
  \end{figure}
}

%\newcommand{\note}[1]{\marginpar{\footnotesize{#1}}}
\newcommand{\note}[1]{}

\begin{document}

\conferenceinfo{EuroSys '07}{Lisbon}
\copyrightyear{2005}
\copyrightdata{[to be supplied]}

\title{Application-Driven Power Management for Embedded Systems}

\authorinfo{Arliones Stevert Hoeller Junior, Lucas Francisco Wanner and Antnio Augusto Frhlich}
           {Laboratory for Software and Hardware Integration\\P.O.Box 476, 88040900\\Florianpolis - Brazil}
           {\{arliones,lucas,guto\}@lisha.ufsc.br}

\maketitle

\begin{abstract}

  Deeply Embedded Systems usually are simple, battery-powered systems
  with resource limitations. In some situations, their batteries
  lifetime becomes a primordial factor for reliability. Because of
  this, it is very important to handle power consumption of such
  devices in a non-restrictive and low-overhead way. This power
  management cannot restrict the wide variety of different low-power
  modes such devices often feature, thus allowing a wider system
  con\-fi\-gu\-ra\-bi\-li\-ty. However, once in such devices
  processing and memory are often scarce, the power management
  strategy cannot compromise large amounts of system resources. In
  this paper we propose a simplified interface for power management
  of software and hardware components. The approach is based on the
  hierarchical organization of such components in a component-based
  operating system and allows power management of system components
  without the need for costly techniques or strategies. A case study
  including real implementations of system and application is
  presented to evaluate the technique and shows energy saves of
  almost 40\% by just allowing applications to express when certain
  components are not being used.

\end{abstract}

%%%%%%%%%%%%%%%%%%%%%%%%%%%%%%%%%%%%%%%%%%%%%%%%%%%%%%%%%%%%%%%%%%%%%%%%%%%%%%%

\section{Introdu��o}
\label{sec:introducao}
% ------------------------------------------------------------------------------
\section{Introduction}
Very-High Level Languages (VHLL), from which \java~and \lua~are examples, is a
kind of programming language which provide developers with features to improve
their productivity\cite{Wilson:1999}.
Productivity improvement is obtained by using constructions with a higher level
of abstraction enabling the developer to express and validate his ideas in a
short period of time (such as object orientation, domain specific constructions
and APIs), and by features that make the occurrence of programming errors less
often reducing the time spend on program debugging (such as automatic memory
management, memory protection, and exceptions).

During the last ten years several initiatives have been taken in order to
enable the use of VHLLs not only in general propose systems scenario as well in
embedded systems scenario fulfilling the time and resource requirements impose
by such systems.
However, in order to be really useful for embedded systems VHLLs must provide
features for interacting with the environment where the embedded system
is inserted on.
Such interaction is usually implemented by using hardware devices.
Sensors and actuators enable the system to interact with the environment.
Transmitters and receivers are used for communicating with other systems.
Timers and alarms are used to implement real-time operations.

The interaction between VHLLs and hardware devices is performed by using the so
called Foreign Function Interface (FFI).
However, a FFI do not specify by itself how to abstract hardware or how
to organize these abstractions.
This work aims to fulfill this gap, introducing a method to interface hardware
devices and applications written using VHLLs in context of embedded systems.
We propose a method to abstract such hardware devices and we show that the
problem of adapting a hardware device to be used for a VHLL can be faced as an
aspect weaving problem, automatically generating the binding between the device
and the language.

The next sections of this paper are organized in the following way: Section
\ref{sec:relat} reviews how VHLLs interact with hardware devices and how
hardware devices can be abstracted and organized.
Section \ref{sec:proposal} introduces the proposed method for
abstracting hardware devices and shows how the adaptation of a hardware device
for a specific VHLL can be solved as an aspect weaving.
Section \ref{sec:eval} presents our cases study as well the obtained results
on evaluating our proposal according to performance, memory consumption,
portability, support to the developer, and reuse.
Our final considerations are presented in Section \ref{sec:conc}.

% ------------------------------------------------------------------------------


\section{O gerente de consumo de energia proposto}
\label{sec:gerente}
\note{Pol�ticas tradicionais analisam dinamicamente o comportamento do
sistema e da aplica��o. APIs (drivers) r�gidas e
incompletas. Padroniza��o num n�vel muito baixo.}

%%% essa afirma��o das APIs/drivers ficou leviana. o foco de critica deveria
%%% ser o 'nivel' da abstracao.

Power management policies in conventional operating system (e.g. Linux,
Windows) dynamically analise the system's behavior in order to determine
when a hardware component should change its operation to a lower or
higher power consumption mode. The software implementation responsible
for those state migrations often rely on hardware-specific interfaces,
which are exported through rigid, and sometimes incomplete APIs
(application programming interfaces) or device drivers. In these
environments, power managing interfaces are standardized in a very low
abstraction level, closer to the actual hardware than the system's
abstractions.


\note{APM e ACPI s�o ``tentativas'' de padroniza��o. Muito usados, definem
interface entre hardware e software. N�o se adaptam a SE.}

%%% A segunda parte ficou muito esquisita, e desconexa. 
%%% Que tem a ver o servidor e o laptop? Porque n�o
%%% d� pra usar o mesmo procedimento do servidor (desligar qd nao usa)
%%% num SE? Acho que perdeu o foco aqui.

Most general purpose computer hardware devices implement either
\textsc{Apm} (\textit{Advanced Power Management}) or \textsc{Acpi}
(\textit{Advanced Configuration Power Interface}) standard interfaces to
allow power management. Although these standards share little in comon,
their objective is the same: to allow devices to be turned on, off or to
be put in a low power consumption mode for a certain period of time.
These techniques work well in environments such as servers that
frequently do not use certain resources or laptop computers, that may
\emph{suspend to disk} or turn off the system when battery charge is
low. These procedures, however, are hardly ever applicable in embedded
systems.

\note{Diversidade da necessidade das aplica��es em termos de consumo
de energia demanda uma granularidade fina de configura��o de modos de
opera��o. PM em computa��o gen�rica foca CPU. SE tem que focar
perif�ricos.}

%%% n�o � o mesmo objetivo nos dois mundos? economizar, permitindo
%%% operar adequadamente?

Most of the resource in power managing interfaces and techniques is
focused on general purpose computer hardware (e.g. personal computers,
servers, laptop and handheld computers), and many research efforts focus
on managing the consumption of the main microprocessor (CPU), as these
devices are responsible for most of the power consumption in these
systems. In embedded systems, however, processors and microcontrollers
are usually very simples, and consume little power. Most of the power
consumed by these systems comes from peripheral devices. Thus, power
managing for these systems must focus on fine grain techniques that
conserve power from peripheral devices, while allowing the system to
operate properly.


\note{An�lise din�mica n�o pode ser comportada. Considerando que a
maioria dos SE rodam apenas uma aplica��o (computa��o dedicada), o
melhor lugar pra determinar a estrat�gia de ger�ncia de energia � na
pr�pria aplica��o}

%%% o texto original estava s� repetindo o que j� tinha sido
%%% dito antes. mudei um pouco, mas acho que ainda n�o t� 100%.

%%% gerente 'ativo' n�o "soa" melhor do que gerente 'dinamico'?

Embedded systems often have to deal with severe resource restrictions,
from restricted hardware capabilities (e.g. memory, processing power) or
functional requirements (e.g. availability, real-time responsiveness).
Thus, most embedded systems cannot afford the cost of dynamic, active
power managers. Previous research \cite{1,2,3,4} indicates that the most
efficient power managing techniques are the ones that take into
consideration the behavior of the target applications for a given
system.  Considering that most embedded system perform specific tasks,
and run a single application~\cite{seilaquem}, we may conclude that the
best place to define a power managing strategy is in the application
itself.

\note{Sumarizar a proposta.}

%%% t� bem explicado, mas n�o fica claro o objetivo. naqueles
%%% itens da introdu��o acho que t� melhor.

In this paper, we present a software infra-structure that allows
application-driven power management for embedded systems. We provide an
uniform, hardware-independent \textsc{API} (\textit{Application
  Programming Interface}) that allows applications to change operating
modes for every component in an embedded operating system. In order to
ensure correct and deterministic behavior, relations and dependencies
regarding power management for every component are formalized through
Petri Networks. This formalization allows high-level analisys of the
power state migration procedures for every component, and stablishes a
message exchange mechanism in which components coordenate to ensure
consistent power state changes in subsystems (e.g. communication,
processing, sensing), or the system as a whole.


\note{Esta se��o descreve a proposta (API, Redes e propaga��o)}

\note{novamente o nome :-)}

This section describes DSPM, an application-driven, Deterministic Static
Power Manager for embeddded systems. We stablish an \emph{Application
  Programming Interface} (API) implemented by every system component,
that allows changes in the component's power state.  \emph{Migration
  Networks} formalize the changes in operating modes of components or
groups of components (subsystems), and controls components instancies,
allowing the system to know every component that is currently in use and
to propagate systemwide changes in operation modes.

\subsection{Power Managing Interface for Software and Hardware Components}


\note{Foi definida uma API que permite acesso das aplica��es aos
  componentes do sistemas, bem como a troca de mensagens entre os
  componentes internos.}

%%% a historia do 'acordar automaticamente' ficou mal-explicada

In our power management strategy, the application programmer is
expected to specify in his source code, whenever ceitain 
components will not be used. Thus, an uniform API to allow power management 
was defined. This interface allows interaction between the application
and the system, between system components and hardware devices,
and directly between application and hardware. In order to
free the application programmer from having to \emph{wakeup}
components whenever they are needed, the power managing mechanism
abstracted by this interface ensures that components return to
their previous operational states whenever they are used.

\largefig{api}{Power Manager API}{.7}

\note{Figura apresentando modos de acesso, dando exemplo em sistema
hipot�tico}

Figure~\ref{fig:api} presents all these interaction modes in a
hipothetical system instance. The application may access a global
component (\texttt{System}) that has knowledge of every other component
in the system (in this case \texttt{IPC}, \texttt{Processing},
\texttt{Sensing} and their respective underlying components), triggering
a system-wide power mode change. Annother way the application may use
this interface is through subsystems (e.g., \textit{Inter-Process
  Communication} (\texttt{IPC}), \texttt{Processing}, \texttt{Sensing}). In this
way, messages are propagated only to the components used in the
implementation of each subsystem. The application may also acess the
hardware directly, using the API available in the device drivers, such
as \textit{Network Interface Card} (\texttt{NIC}), \texttt{CPU},
\texttt{Thermistor}. The API is also used between the system's
components, as the message exchanges between \texttt{System} and the
three subsystems in the figure illustrates.




\note{Portabilidade e facilidade de desenvolvimento da aplica��o.\\
Simplicidade da interface -> facilidade de uso.\\
Modos universais -> evita consulta a manuais de HW.}

In order to attain application portability, and to facilitate
application development, the power managing interface was defined with a
minimal set of methods and universal operating modes with unified
semantics thoughout the system. Portability comes from the fact that the
application doesn't need to implement specific procedures for each
device in order to change its operating mode.  These procedures are
abstracted by the API. Easiness of use comes from the fact that the
application programmer doesn't need to analyse specific hardware manuals
in order to indentify available operating modes, the procedures to
change those modes, and the consequences of these changes.


% Contudo, a API ainda fornece acesso aos componentes de hardware, n�o
% impedindo que o programador de aplica��o gerencie o dispositivo
% diretamente se desejar.


\note{Composi��o da API:\\
- M�todos de interface 'set' e 'get'.\\
- Rela��o de modos de opera��o universais.}

Two methods are defined in the API: one to change the operating mode,
and another to identify the current mode. In addition to these methods,
the API includes a list of modes available to each component. This list
does not have a fixed size, as each component must enumerate in it every
operating mode available. Low-power hardware components often present a
wide range of operating modes. Allowing every mode to be used increses
system configurability, but may increase application complexity and
compromise portability. In order to deal with this issue, a set of high
level universal operations was defined: \texttt{FULL}, \texttt{LIGHT},
\texttt{STANDBY} and \texttt{OFF}.  These modes free the programmer from
having to know details regarding the modes available hardware components
in the system. However, these modes may be extended as necessary. It is
up to the application programmer to associate universal modes to
specific modes available to hardware devices.




\note{Sem�ntica dos modos de opera��o:\\
- FULL: Alto consumo, todas funcionalidades, m�ximo desempenho.\\
- LIGHT: Menor consumo, maioria
das funcionalidades (documenta��o), desempenho degradado.}

When the device is operating at full capacity, it is in the
\texttt{FULL} mode. In this operation mode, the system configures the
device to operate providing its service in the most efficient manner
possible, including all its functionalities, but at full power
consumption. The \texttt{LIGHT} mode puts the device in an operating
mode where it offers most of its functionalities, but consumes less
power than the full mode and, very likely degrades its performance.
Examples of this mode include devices that allow operation in different
voltage supplies or frequencies (\textsc{DVFS} - \textit{Dynamic Voltage
  and Frequency Scaling}). The migration from the \texttt{LIGHT} mode to
the \texttt{FULL} mode is fast, and usually does not imply in
considerable delay for the application.

\note{- STANDBY: Baix�ssimo consumo, quase nenhuma funcionalidade, parado.\\
  - OFF: Nenhum (ou m�nimo) consumo, nenhuma funcionalidade, parado (RESET).}

In the \texttt{STANDBY} and \texttt{OFF} modes, the device stops
operating.  When in \texttt{STANDBY}, however, the device is ready to
continue operating when necessary, and is able to continue its operation
from the point before it was stoped. An exemple of such a mode is the
``sleep'' modes of a processor. Altough in this mode the device is
stopped, it still consumes a small amount of power. This power is
required to keep the device's internal memory and registers alive until
it is restarted.  In the \texttt{OFF}, however, the device is turned
off, and loses its internal configuration. When a component migrates
from an \texttt{OFF} state to another state, it is reset.

%\note{- Extens�o: Comunica��o: modos SEND\_ONLY e RECV\_ONLY.\\
%- Outros: definidos especificamente -  portabilidade comprometida.}

%\input{tbl/modos}

\fig[t]{manager}{UML Diagram for the Power Management aspect}{0.55}



\note{Ger�nia de energia -> Propriedade n�o funcional.\\
Adaptador de Cen�rio para gerenciamento de energia:\\
- M�todos set e get para power.\\
- Vari�vel de estado.}

In addition to the functional requirements, the API should be easily
mantainable and appliable to existing systems, As power management is a
non-functional requirement for operating systems~\cite{Lohmann:2004},
our power management API was modeled as an aspect~\cite{Kiczales:1997},
and may thus be isolated from the rest of the system.
Figure~\ref{fig:manager} presents an UML diagram for this aspect,
representing also its dependancies to other system components.







\note{Compatibilidade dos componentes com o adaptador: Fornecer a API.}


\fig{general_net_behaviour}{Generalized Migration Network Behavior}{.45}

In order for a power management strategy to be properly abstracted as an
aspect, this strategy must be scenario-independent, and appliable to any
component. There is no generic method to implement the migration between
different operating modes, as these procedures are dependant to the
particular characteristics of different devices. Thus, each system
component must implement a method to allow changes in its power state.
This method must be private, and innacessible from the application and
other components whenever the power management aspect has not been
applied to the system.


\subsection{Operation mode migration networks}
\label{sc:migration_nets}

\note{Formaliza��o das migra��es entre modos de opera��o.\\
Redes de Petri devido ao mapeamento envento->condi��es e �
representatividade gr�fica e alg�brica:\\
- Transi��es -> A��es.\\
- Lugares -> Ativadores de a��es.\\}



% a explicacao de petri nets ficou horrivel, mas n�o sei como melhorar.

In order to map coherent conectivity between different abstraction
levels in the system, a formal operating mode migration network was
defined. In this section, we describe this formal mechanism, which was
defined through Petri networks. These networks feature clear graphical
representation, and a wide range of mathematical analisys
models~\cite{Peterson:1977}. These models allow proof of liveness and
reachability of desirable states, as well unreachability of incorrect
states.

\note{Migra��es generalizadas. Comportamento da rede
generalizada. Figura simplificada. Rede completa em anexo.}

Altough the procedures to migrate power states are specific to each
component (both software and hardware), the control and dispatch of
these migrations may be expressed in a generic form.  In order to allow
that, a network of mode migrations that specifies the transitions
between different operating modes was formalized.
Figura~\ref{fig:general_net_behaviour} presents a simplified overview of this
network, illustrating the migration of a component from the OFF to the FULL
mode. As illustrated in the figure, there are places associated with the
existing operating modes (FULL and OFF). A resource in these places marks
the component's current operating mode.



\largefig{hierarchical_net}{Hieralquical Petri Network}{0.6}


\note{Descrever sequ�ncia de disparo de transi��es.}

The \texttt{Atomic\_Execution} place is responsible for ensuring that
different mode change operatiions do not execute simultaniously. For
that, this palce is always initialized with one resource. This resource
enables the transactions that enable changes in operating mode. The
moment this transaction is triggered (through a function call to the
power management API), the transactions that would start different
migrations are disabled, as the resource in the
\texttt{Atomic\_Execution} place is consumed. Additionallym a new
resource inserted into the \texttt{Triggering\_FULL} place enables the
transactions that remove the resources that marks the component's
current operating mode (OFF). As the component in the example is in the
OFF state, only the \texttt{OFF\_TO\_FULL} transaction is enabled.  When
this transaction is triggered, the resource that marked the \texttt{OFF}
place is conumed, and three resources are inserted into the
\texttt{FULL\_Enable} place. This anables the \texttt{Enter\_FULL}
transaction, that is responsible for executing the operations that
actually change the component's power mode. After this transaction is
triggered, two resources are inserted into the \texttt{FULL} place,
anebling the \texttt{FULL\_Entered} transaction, which finalizes the
process, consuming the final resource in the \texttt{FULL\_Enable}
place, and inserting one resource back into the
\texttt{Atomic\_Execution} place. The entire process results with a
resource removed from the \texttt{OFF} place and inserted into the
\texttt{FULL} place. In order to avoid deadlocks when transactions that
result in the component's current operating mode, a \texttt{Recurrence}
transaction was inserted into the model. This transaction returns the
resource removed from the \texttt{Atomic\_Execution} place in case of
recurrency.


\note{Provas matem�ticas.}

This Petri network was analysed through tradditional Petri net tools,
and was found to be deadlock free, and to have finite reachability.:


\note{Vivacidade -> deadlock free.}


\note{Hierarquia -> Redes de Petri Hier�rquicas.}

The generalized network represents the transitions of operating mode
from a high level perspective, where the particular characteristics
involved in the transition of each component are not specified. However,
a refinement process is required in order to allow the inferrence of the
migration proderes from this network model. This refinement explores the
hierarquical characteristic of Petri nets, which allows an entire
network to be replaced by a place or transaction in order to model a
higher level abstraction and, on the other hand, allows places and
transactions to be replaced with sub-networks in order to provide a
refined, more detailed model. Figure~\ref{fig:hierarchical_net} presents
the notation for this representation. In this example, the higher
abstraction network \texttt{P0} abstracts the \texttt{A} sub-network,
and the \texttt{T2} abstracts the \texttt{B} sub-network.


\note{Substitui��o utilizando subrede para chamada 'Enter'. Exemplo.}

In order to refine the migration procedures responsible for migrating
the operating mode, the \texttt{Enter} transactions are replaced by
sub-networks that implement the migration procedures in higher detail.
Figure~\ref{fig:mac_full} presents the sub-network that implements the
migration of the \texttt{B-MAC} component to the FULL operating mode. In
order to form the migration network for this component, this subnetwork
replaces the \texttt{Enter\_FULL} transaction in the general migration
network. This subnetwork also presents transactions that abstract the
triggering of transactions that change the operating mode of other
components.


\fig[t]{mac_full}{Sub-network implementing the migration procedures for
  the \texttt{B-MAC} component.}{.6}

% \subsubsection{Exemplos}

% Sensor->ADC

% Communicator->...->NIC ::>> Novos problemas aqui.


%\subsubsection{Interpreta��o est�tica das redes de migra��o}

% Meta-programa (Aspecto) implementa o comportamento da rede
% generalizada e utiliza a API para implementar as transi��es 'Enter'


\subsection{Message Propagation}

\note{Quanto mais componentes, mais complexo o sistema, mais complexo
gerenciar o consumo de energia destes componentes individualmente.}

As the complexity of embedded application increases, more system
components are used. Thus, the control of individual components' power
consumption by the application may be inpracticable. For example,
figure~\ref{prg:app_complexa} presents a hipothetical \emph{power-aware}
application. The application implements a remote monitoring module, that
periodically samples a pressure sensor, and sends the value read through
a GPRS modem. Figure~\ref{prg:app_complexa}(a) illustrates the
complexity resulting from controlling individual components. In this
example, the application must stop the TCP/IP communication stack prior
to turning off a modem, i.e. all pening data must be sent before the
communication may be stopped. After the modem is turned off, the
application turns off the serial ports (UART) used to communicate with
the modem. Similar complexities are present in almost every subsystem.
Abstracting these details enhances the usability of the power management
API, as figures~\ref{prg:app_complexa}(b) and~\ref{prg:app_complexa}(c)
present.


\begin{figure*}[t]
\begin{center}
\begin{footnotesize}

\lstset{language=c++,frame=lrtb}
\lstset{basicstyle=\ttfamily}
\lstset{commentstyle=\textit}
%
\subfigure[Controlando todos componentes]{
  \begin{minipage}{7cm}
    \lstinputlisting{prg/app_complexa01.cc}
  \end{minipage}
}
\subfigure[Controlando subsistemas]{
  \begin{minipage}{7cm}
    \lstinputlisting{prg/app_complexa02.cc}
  \end{minipage}
}
\subfigure[Controlando todo o sistema]{
  \begin{minipage}{7cm}
    \begin{center}
    \lstinputlisting{prg/app_complexa03.cc}
    \end{center}
  \end{minipage}
}


\caption{Aplica��es hipot�ticas com ger�ncia do consumo de energia dirigido pela aplica��o}
\label{prg:app_complexa}
\end{footnotesize}
\end{center}
\end{figure*}




% Contudo, sistemas orientados a objeto trazem um fator complicante
% para esta proposta, j� que, devido a recursos que permitem
% configurabilidade aos componentes (e.g., recursos de programa��o
% gen�rica e polimorfismo), n�o � poss�vel saber de antem�o exatamente
% quais componentes est�o sendo utilizados. As se��es seguintes
% apresentam a proposta para a solu��o deste problema.


%\subsubsection{Propaga��o hier�rquica de mensagens}

\note{O que � necess�rio?\\
  Estabelecer mecanismo de ``comunica��o'' entre componentes para
  garantir o correto desligamento dos subsistemas.
}

In order for a subsystem to be deactivated or migrated to low-power
operating modes in an efficient manner, it is necessary to ensure that
the software and hardware artifacts first finalize operations currently
executing, or adapt to the new operating parameters. It is also necessary
to ensure that these subsystems operate correctly after returning to 
their funcional operating modes. Thus, a mechanism and a policy for 
interaction between components must be established.


\note{Como fazer?
  Inferir m�todos de troca de modo de opera��o a partir das redes.
}

Given the presented API and migration networks, it is simple to infer
the migration procedures for each subsystem. The interaction mechanism
is thus formed by message exchanges through the API.  The policy may be
derived from the migration networks for each subsystem. This policy
forms the correct sequence for the migration of each component.
Figure~\ref{fig:mac_full},presents transactions that trigger migrations
in the networks of other components (e.g.,
\texttt{Radio.Trigger\_FULL}). These transactions are the points in
which messages are exchanged between components.


\note{Gera��o autom�tica dos m�todos de migra��o � realizada atrav�s
  da interpreta��o est�tica das redes.}

A partir da interpreta��o destas redes � poss�vel a montagem, em tempo
de compila��o, dos m�todos que garantir�o a sequ�ncia correta de
execu��o dos procedimentos de migra��o.  No exemplo da
Figura~\ref{fig:mac_full} pode-se observar a conex�o de tr�s outras
redes de migra��o � rede do \texttt{B-MAC} (\texttt{Timer},
\texttt{SPI} e \texttt{Radio}).  O \texttt{B-MAC} � uma implementa��o
em software de um MAC (Media Access Control) para um m�dulo de rede de
sensores sem fio~\cite{Mica2}.  Neste dispositivo, a comunica��o entre
o processador e o r�dio � realizada atrav�s de um barramento serial
(SPI).  Neste exemplo, espera-se que a aplica��o utilize a API do
componente \texttt{B-MAC} como interface do subsistema de comunica��o.
Ao executar um comando ``\texttt{MAC.power(OFF)}'', por exemplo, o
desligamento do subsistema de comunica��o deve iniciar pelo
desligamento do pr�prio \texttt{B-MAC}, que deve esvaziar seus buffers
de envio e desligar o \texttt{Timer} que utiliza para recep��es antes
de requisitar que o r�dio se desligue.  Como as redes de migra��o
foram organizadas de modo hier�rquico, o resultado final da gera��o do
procedimento de migra��o do subsistema de comunica��o seria um
procedimento algoritmico como o representado na
Figura~\ref{prg:mignet_mac.cc}.

\prg{C++}{mignet_mac.cc}{M�todos de migra��o de modo de opera��o derivados
da redes de migra��o}

% \fig{mignet_mac}{Rede de Migra��o do modo de opera��o STANDBY do \texttt{MAC}}{1}


\begin{description}
\item{\bf{Propaga��o para todo o sistema}}
\end{description}

\note{Listas de inst�ncias para acessar todos os componentes.}
A��es de ger�ncia do consumo de energia do sistema como um todo s�o
tradadas por um componente global do sistema (\texttt{System}). Este
componente cont�m refer�ncias para todos os subsistemas em uso pela
aplica��o. Ent�o, se uma aplica��o deseja alterar o modo de opera��o
do sistema inteiro, isto pode ser feito acessando a API deste
componente, que propagar� este pedido para todos os subsistemas. Esta
lista deve ser montada em tempo de execu��o atrav�s do aspecto de
ger�ncia de energia, que utilizar� as chamadas de constru��o e
destrui��o de componentes para, respectivamente, incluir e remover
refer�ncias a inst�ncias de componentes desta lista. Quando a API de
ger�ncia do consumo de energia do sistema � acessada pela aplica��o, o
sistema realiza uma varredura pela lista de inst�ncias que possui,
disparando chamadas �s APIs dos componentes que registrou.


\subsubsection{Compartilhamento de recursos}

\note{Hardware compartilhado. Esclarecer problema. Exemplo ADC c/ figura.}
O compartilhamento de recursos � uma caracter�stica de sistemas
computacionais que precisa ser tratada nesta proposta. Problemas podem
ocorrer na migra��o de modos de opera��o quando componentes de alto
n�vel compartilham, o mesmo componente de hardware. Por exemplo, uma
aplica��o que utiliza dois sensores que compartilham o mesmo conversor
anal�gico-digital (ADC) multiplexado n�o pode ter o ADC desligado
devido � solicita��o de um dos sensores se o outro sensor ainda o est�
utilizando.

\largefig{sensor}{ADC sendo compartilhado por dois sensores.}{.6}

\note{Descrever estrutura de contadores.}
Para resolver este problema, adotou-se um mecanismo de
\emph{contadores de uso}. Cada componente compartilhado possui
contadores que indicam quantos componentes solicitam cada modo de
opera��o. Sempre que uma chamada � realizada � API, o contador
referente ao estado atual do componente � decrementado, e o contador
referente ao estado pretendido � incrementado. A migra��o solicitada �
realizada sempre que a maioria absoluta das refer�ncias estiverem
contabilizadas em um �nico contador.

\note{Estudar o problema do roteamento na rede.}

\section{Implementation}
\label{sec:implementacao}
\input{implementacao}

\section{Case study}
\label{sec:estudosdecaso}
\input{estudosdecaso}

\section{Related Work}
\label{sc:related}
\section{Related Work}
\label{sc:relacionados}


%\textsc{Grace-OS}~\cite{Yuan:2004} � um sistema operacional eficiente em termos de energia para aplica��es m�veis de multim�dia. Esse sistema usa t�cnicas de adapta��es multi-camadas para garantir \qos{} em sistemas com \emph{software} e \emph{hardware} adaptativos. \textsc{Grace-OS} combina escalonamento de tempo real com mecanismos de DVS para dinamicamente gerenciar o consumo de energia. Ele foi implementado sobre o sistema operacional \textsc{Linux} e suporta apenas tarefas \emph{soft real-time}. \textsc{GRUB-PA}~\cite{Scordino:2004} �, de certa forma, semelhante ao \textsc{Grace-OS}. A principal diferen�a � que \textsc{GRUB-PA} suporta tanto tarefas \emph{soft real-time} quanto tarefas \emph{hard real-time}.

\textsc{Grace-OS}~\cite{Yuan:2004} is an energy-efficient operating system for
mobile multimedia applications. This system uses a cross-layer adaptation
technique to guarantee \qos{} on systems with adaptive software and hardware.
It combines real-time scheduling with DVS mechanisms to dynamically manage
energy consumption. It was implemented over the \textsc{Linux} operating
system and it only supports soft real-time tasks.
\textsc{GRUB-PA}~\cite{Scordino:2004} is somehow similar to \textsc{Grace-OS}.
The main difference is \textsc{GRUB-PA} supports both soft and hard real-time
tasks.

%Niu~\cite{Niu:2005} prop�s minimizar a anergia consumida para sistemas \emph{soft real-time} enquanto garante requisitos de \qos{}. Esse objetivo � alcan�ado atrav�s de um algoritmo de escalonamento h�brido (est�tico/din�mico) que utiliza \textsc{DVS} e atrav�s de t�cnicas de particionamento do conjunto de tarefas em tarefas obrigat�rias e em tarefas opcionais. Nesse trabalho, os requisitos de \qos{} s�o qualificados pela restri��o \textit{(m,k)}, a qual especifica que tarefas devem atender no m�nimo \textit{m} \emph{deadlines} em qualquer \textit{k} libera��es de tarefas consecutivas. Em um trabalho semelhante, Harada~\cite{Harada:2006} prop�s resolver o compromisso entre a maximiza��o dos n�veis de \qos{} e a minimiza��o do consumo de energia. Nesse trabalho, cada tarefa � dividida em parte obrigat�ria e em parte opcional, e � realizada a aloca��o de ciclos e freq��ncia do processador com garantias de \qos{}.

Niu~\cite{Niu:2005} proposed to minimize energy consumed by soft real-time
systems while guaranteeing \qos{} requirements. This goal is achieved by a hybrid
static/dynamic scheduling algorithm that it uses DVS mechanisms and it
partitions the set of tasks in mandatory and optional tasks. In this work, the
\qos{} requirements are qualified by \textit{(m,k)} constraints which it
specifies that tasks must meet at least \textit{m} deadlines in any \textit{k} 
consecutive task releases. In a similar work, Harada~\cite{Harada:2006} proposed
to resolve the trade-off between QoS maximization and energy consumption
minimization. It uses an allocation of processor cycles and frequency with
\qos{} guarantees and it divides each task into mandatory and optional parts.

%Outras pesquisas exploram um balanceamento entre \qos{} das aplica��es e consumo de energia atrav�s de adapta��es nas aplica��es visando atender o tempo especificado pelo usu�rio. Um sistema que utiliza essa t�cnica � \textsc{Odyssey}~\cite{Flinn:1999}. \textsc{Odyssey} realiza o monitoramento da energia fornecida e da energia necess�ria para executar as tarefas. Com essas informa��es o monitor � capaz de selecionar o estado correto entre economia de energia e qualidade da aplica��o. Esse trabalho tamb�m demonstra como as aplica��es podem dinamicamente alterar seus comportamentos (``fidelidade'' dos dados) com o objetivo de economizar energia.

Other projects explore trade-off between application's \qos{} and energy
consumption through adaptations in the applications aiming to meet the time
specified by application. \textsc{Odyssey}~\cite{Flinn:1999} uses that idea.
It monitors the energy budget and with this information it can select the 
correct state between energy saving and quality of application. This work 
also demonstrates how the applications can dynamically change their behavior 
(``fidelity'' of the data) with the goal of saving energy.

%\textsc{ECOSystem}~\cite{Zeng:2002} � outro sistema operacional que suporta aplica��es adaptativas. Esse sistema � baseado em uma ``moeda'' corrente que as aplica��es utilizam para ``pagar'' (alocar) e utilizar recursos do sistema (CPU, disco, rede), chamada \textit{currentcy}. O sistema distribui \textit{currentcies} periodicamente para as tarefas de acordo com uma equa��o que define uma velocidade de descarga que a bateria pode assumir para for�ar o sistema a durar um per�odo de tempo definido. Isso faz com que as aplica��es adaptem as execu��es de acordo com seus \textit{currentcies}. Esse modelo unifica o c�lculo de energia sobre os diferentes dispositivos de \emph{hardware} e proporciona uma satisfat�ria aloca��o de energia entre as aplica��es.

\textsc{ECOSystem}~\cite{Zeng:2002} is another operating system that supports
application adaptation. This system is based in a ``currency'' that the
applications use to allocate (``to pay'') system resources 
(e.g., access to memory,
network or disks), called \textit{currentcy}. The system distributes
\textit{currentcies} periodically to tasks accordingly to an equation that
defines the discharge rate that the system battery can assume to force the
system to last for a defined period of time. This allows applications to adapt
their execution based on their \textit{currentcy} balance. This model unifies
the calculation of energy on the various hardware devices and it provides a
satisfactory energy allocation among the applications.



\section{Conclusion}
\label{sec:conclusao}
\section{Conclusion}
\label{sec:Conclusion}
In this paper, we have shown how to construct a development environment for embedded applications based on specific hardware/software requirements and introduce the automatic exchange of configuration parameters as one anatomic part of fully automated debug.

The integrated development environment provides independence of the physical target platform for development and test. Its an important step, since some embedded systems may not be able to store the extra data needed to support debug. The impact of enable debug information in code size and in the execution time of the real-world application was more than 80\%. Also, developers no longer need to spend time understanding a new development platform whenever some characteristic of the embedded system changes.

The automatic exchange was evaluated using two kinds of test. The fully automated test works with no prior information of the application, but it was possible to generate valid configurations, that could be tested as alternative solutions. In partial automated test all generated configurations were valid and the report was useful to discovery that some parameter values were better then others.

In this sense, was possible to realize that even a small part of the complete automated solution produce answers to help developers find and fix a bug. With only a hundred tries was possible to find error/restriction in the code. Thus, as future work we can integrate the automatic exchange script with a tool that has artificial intelligence, in order to achieve a conscious exchange of type parameters.



%\appendix

%\section{Rede de migra��es de modos de opera��o gereralizada}
%\label{apx:general_net}
%\begin{figure} {\centering{\scalebox{0.55}{\rotatebox{270}{\includegraphics{fig/general_net}}}}\par}
%\end{figure}

\acks

Authors would like to thank Augusto Born de Oliveira, Hugo Marcondes,
Rafael Cancian and, specially, Prof. Jlio Zseremetta from Federal
University of Santa Catarina for very helpful discussion.

\bibliographystyle{plain}
\bibliography{power}

\end{document}
