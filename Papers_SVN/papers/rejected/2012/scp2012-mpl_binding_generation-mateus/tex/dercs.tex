% ------------------------------------------------------------------------------
\section{DERCS} \label{dercs}
% + Intro
Our method uses an intermediate representation to describe hardware mediators and
FFI aspects.
This section describes the meta-model we have used and
Section \ref{sec:proposal} describes the method we have proposed.

% NOTE: Maybe move
% + Why we have used DERCS
% and
% + How we have used DERCS
% To here.

% + DERCS def
Model-Driven Engineering (MDE) proposes the development of complete
computational systems from high level specifications which are transformed,
in one of more phases, in order to generate the final system. % cite MDE
In such context, the Distributed Embedded Real-Time Compact Specification
(DERCS) is a specification/meta-model used to represent platform independent
models.
Such models, according to the
Aspect-oriented Model-Driven Engineering for Real-Time systems (AMoDE-RT)
methodology, are used together with the description of the target platform and
mapping rules to generate the final system, which encompass software and hardware
elements \cite{Wehrmeister:2009}.

% + DERCS and AMoDE-RT
According to the AMoDE-RT methodology,
a DERCS model is generated from UML's class and sequence diagrams, and
from diagrams that specify non-functional elements as aspects.
Therefore, the DERCS meta-model defines
structural elements, behavioral elements, and aspect-oriented elements,
encompassing distinct visions into a single model.

% + DERCS's Structural and Behavioral Elements
% Figures \ref{fig:structural_elements} and \ref{fig:behavioral_elements}
% show, respectively, the structural and behavioral elements defined by DERCS.
Among the structural elements of DERCS are classes, attributes,
methods, and parameters.
The behavioral elements determine the behavior of a method, detailing which
messages such method can send to objects, which are its
local variables, actions, etc.
The structural and behavioral elements of DERCS have semantics similar to the
semantics of object-oriented programming languages.
Details concerning structural and behavioral elements of DERCS can be found at
\cite{Wehrmeister:2009}.

% \figtwocolumn{1}{structural_elements}{DERCS structural elements. Adapted from \cite{Wehrmeister:2009}.}

% \figtwocolumn{1}{behavioral_elements}{DERCS behavioral elements. Adapted from \cite{Wehrmeister:2009}.}

% + Aspect-Oriented Elements
Figure \ref{fig:ao_elements} shows the aspect-oriented elements
defined by DERCS.
Among the aspect-oriented elements are aspects, pointcuts,
structural adaptations, and behavioral adaptations.
Those elements correspond to the concepts of Aspect-Oriented Programming (AOP),
as defined by \cite{Kiczales:1997} whereas
aspect adaptations (\emph{AspectAdaptation})
represent the concept of \emph{advices}.
An aspect adaptation can be structural (\emph{StructuralAdaptation}) or
behavioral (\emph{BehavioralAdaptation}).
Structural adaptations modify the structure of the elements of the model, for
example, adding methods to a class or parameters to a method.
On the other hand, behavioral adaptations modify the behavior of the elements of
the model, for example, executing tasks before or after the execution of a
method behavior, or completely changing a method's behavior.

\figtwocolumn{1}{ao_elements}{DERCS aspect-oriented elements. Adapted from \cite{Wehrmeister:2009}.}

% + For what DERCS was used
The aspect-oriented elements of DERCS meta-model can be used to modify the
interface of a component in order to adapt such component according to what is
required by its client.
As next section shows, we have explored this feature of DERCS in order to
adapt a hardware mediator interface for what is expected by a specify FFI.

% \clearpage

% ------------------------------------------------------------------------------
