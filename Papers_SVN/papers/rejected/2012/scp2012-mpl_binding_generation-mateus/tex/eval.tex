% ------------------------------------------------------------------------------
\section{Case study} \label{sec:eval}
% This section presents the evaluation of the proposal of binding generation to
% interface hardware devices and embedded MPL,
% introduced at Section \ref{sec:proposal}.
In order to evaluated our proposal, we have implemented it in a case study
encompassing \emph{Motion Estimation} (ME) for digital video encoding.
The chosen component to be interfaced with the MPLs \java~and \lua~
performs ME for H.264 video encoding.
ME is a significant stage for H.264 encoding (it consumes around 90\%
of the total time of the encoding process \cite{XiangLi:2004}) so it is a adequate
candidate to evaluate the performance of the binding code generated by using
our approach.

Such component for ME computation was developed by the authors
for the Brazilian
project \emph{Rede H.264}, which aims to develop standards and products for the
Brazilian Digital Television \cite{RedeH264:2009}.
One of the goals of the project \emph{Rede H.264} is the integration between its
components.
That is the case of the set top boxes, where interactive user applications
written in \java~using the Ginga-J \cite{Ginga:2011} middleware are integrated
with
encoders and decoders written in C, C++, and in hardware description languages.

We have evaluated our proposal according to the
reuse of binding code classes among distinct MPLs and FFIs,
showing that the functional part of a binding code can be reused at distinct
MPLs and FFIs without modifications.
In order to demonstrate the feasibility of the generated binding code
we also we have evaluated them for performance, and memory consumption.



% ------------------------------------------------------------------------------
