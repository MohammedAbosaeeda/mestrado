% ------------------------------------------------------------------------------
\section{Conclusion} \label{concl}
% + Conclusion
% Data partition is feasible (quality) and advantageous (speedup) strategy for
% ME optimization.
% DMEC modeling and implementation allows such ME optimization while achieving
% portability between distinct architectures and machines, including a
% dedicated hardware.
% 
In this article, we present the design and the implementation of DMEC, a component
for distributed motion estimation.
We have shown that, by using a careful design process based on domain engineering,
it is feasible to develop a component for ME while keeping the same interfaces for
all its implementations. Also, our strategy of ME distribution based on picture partitioning proved
to be effective. The visual interdependence between partitions is not significant enough to
influence the encoding quality, and allows for a speedup since it enables the simultaneous
processing of each picture partition.
To demonstrate the flexibility of the proposed interfaces, we have implemented DMEC for the
Cell BE, Multicore IA32, and also as dedicated hardware. We also have evaluated DMEC according to
encoding time and PSNR, demonstrating an optimized version of ME which keeps the quality of the
generated bitstream.




% The evaluation of the proposed optimizations in the JM H.264 Reference Encoder
% assess ME quality and good performance gains. It also demonstrates the best 
% combination between subsampling and truncation. 
% The combination of 4:1 subsampling and 2 LSB truncation presents a quality loss
% of less than 0.5dB for all considered sequences and an average speedup of 2.64.




% ------------------------------------------------------------------------------
