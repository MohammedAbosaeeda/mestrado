\documentclass[english]{article}
\usepackage[T1]{fontenc}
\usepackage[latin1]{inputenc}
\usepackage{babel}
\usepackage{times}
\usepackage{latex8}
\usepackage{graphics}

%\twocolumn
%\pagestyle{plain}
%\setlength{\textheight}{8.875in}
%\setlength{\textwidth}{6.875in}
%\setlength{\columnsep}{0.3125in}
%\setlength{\oddsidemargin}{-0.1in}
%\setlength{\evensidemargin}{-0.1in}
%\setlength{\topmargin}{1in}
%\setlength{\headheight}{-0.1in}
%\setlength{\headsep}{-0.1in}
%\setlength{\parindent}{1pc}

\onecolumn
\pagestyle{plain}
\linespread{1.4}
\setlength{\textheight}{8.8in}
\setlength{\textwidth}{6.5in}
\setlength{\columnsep}{0.3125in}
\setlength{\oddsidemargin}{-0.1in}
\setlength{\evensidemargin}{-0.1in}
\setlength{\topmargin}{1in}
\setlength{\headheight}{-0.1in}
\setlength{\headsep}{-0.1in}
\setlength{\footskip}{.5in}
\setlength{\parindent}{1pc}
\fontsize{14}{16pt}

\font\tenhv  = phvb at 10pt
\font\elvbf  = ptmb scaled 1100

% Hyphenation
\hyphenation{per-form-ance}

% Thesis
\newcommand{\aosd}{application-oriented system design}
\newcommand{\Aosd}{Application-oriented system design}
\newcommand{\AOSD}{Application-Oriented System Design}
\newcommand{\aosds}{application-oriented system designs}
\newcommand{\aoos}{application-oriented operating system}
\newcommand{\aooss}{application-oriented operating systems}
\newcommand{\aos}{application-oriented system}
\newcommand{\aoss}{application-oriented systems}
\newcommand{\aodd}{application-oriented domain decomposition}
\newcommand{\aod}{application-oriented decomposition}
\newcommand{\aocf}{application-oriented component framework}

% Software Engineering Methods
\newcommand{\fbd}{family-based design}
\newcommand{\Fbd}{Family-based design}
\newcommand{\FBD}{Family-Based Design}
\newcommand{\isd}{incremental system design}
\newcommand{\Isd}{Incremental system design}
\newcommand{\ISD}{Incremental System Design}
\newcommand{\oop}{object-oriented programming}
\newcommand{\ood}{object-oriented design}
\newcommand{\Ood}{Object-oriented design}
\newcommand{\OOD}{Object-Oriented Design}
\newcommand{\cbd}{collaboration-based design}
\newcommand{\Cbd}{Collaboration-based design}
\newcommand{\CBD}{Collaboration-Based Design}
\newcommand{\sop}{subject-oriented programming}
\newcommand{\Sop}{Subject-oriented programming}
\newcommand{\SOP}{Subject-Oriented Programming}
\newcommand{\aop}{aspect-oriented programming}
\newcommand{\Aop}{Aspect-oriented programming}
\newcommand{\AOP}{Aspect-Oriented Programming}
\newcommand{\dpat}{design patterns}
\newcommand{\Dpat}{Design patterns}
\newcommand{\DPat}{Design Patterns}

% Systems
\newcommand{\epos}{\tm{Epos}}
\newcommand{\aboelha}{\tm{Aboelha}}
\newcommand{\pure}{\tm{Pure}}
\newcommand{\peace}{\tm{Peace}}
\newcommand{\flux}{\tm{Flux}}
\newcommand{\harmony}{\tm{Harmony}}
\newcommand{\ax}{\tm{AX}}
\newcommand{\mach}{\tm{Mach}}
\newcommand{\amoeba}{\tm{Amoeba}}
\newcommand{\mars}{\tm{Mars}}
\newcommand{\vkernel}{\tm{V-Kernel}}
\newcommand{\chorus}{\tm{Chorus}}
\newcommand{\synthesis}{\tm{Synthesis}}
\newcommand{\vino}{\tm{Vino}}
\newcommand{\spin}{\tm{Spin}}
\newcommand{\choices}{\tm{Choices}}
\newcommand{\genesys}{\tm{GeneSys}}
\newcommand{\ethos}{\tm{Ethos}}
\newcommand{\apertos}{\tm{Apertos}}
\newcommand{\xkernel}{\tm{X-Kernel}}
\newcommand{\thesys}{\tm{The}}
\newcommand{\multics}{\tm{Multics}}
\newcommand{\unix}{\tm{Unix}}
\newcommand{\sysv}{\tm{Unix System V}}
\newcommand{\bsd}{\tm{Berkeley System Distribution}~(BSD) of \tm{Unix}}
\newcommand{\linux}{\tm{Linux}}
\newcommand{\posix}{\tm{Posix}}
\newcommand{\windows}{\tm{Windows}}
\newcommand{\dos}{\tm{Dos}}
\newcommand{\mpi}{\tm{MPI}}

% Programming Languages
\newcommand{\cpp}{\tm{C++}}
\newcommand{\clang}{\tm{C}}
\newcommand{\cwc}{\tm{C with Classes}}
\newcommand{\modular}{\tm{Modula/R}}
\newcommand{\modulatwo}{\tm{Modula/2}}
\newcommand{\modulathree}{\tm{Modula/3}}
\newcommand{\oberon}{\tm{Oberon}}
\newcommand{\fortran}{\tm{Fortran}}
\newcommand{\autoconf}{\tm{GNU Autoconf}}
\newcommand{\imake}{\tm{X11 Imake}}
\newcommand{\uml}{\tm{UML}}
\newcommand{\corba}{\tm{Corba}}
\newcommand{\java}{\tm{Java}}

% Hardware
\newcommand{\snow}{\tm{Snow}}
\newcommand{\suprenum}{\tm{Suprenum}}
\newcommand{\manna}{\tm{Manna}}
\newcommand{\mosaic}{\tm{Mosaic C}}
\newcommand{\atomic}{\tm{Atomic}}
\newcommand{\myrinet}{\tm{Myrinet}}
\newcommand{\ethernet}{\tm{Fast Ethernet}}
\newcommand{\pci}{\tm{PCI}}
\newcommand{\isa}{\tm{ISA}}
\newcommand{\scsi}{\tm{SCSI}}
\newcommand{\pentium}{i\tm{x86}}
\newcommand{\intel}{Intel i\tm{x86}}
\newcommand{\psix}{\tm{P6}}
\newcommand{\lanai}{\tm{LANai}}

% Others
\newcommand{\internet}{\tm{Internet}}
\newcommand{\mkernel}{$\mu$-kernel}

%%% Local Variables: 
%%% mode: latex
%%% TeX-master: "thesis"
%%% End: 


\newcommand{\tm}[1]{\textsc{#1}}
\newcommand{\sym}[1]{\texttt{#1}}
\newcommand{\refsec}[1]{[section~\ref{sec:#1}]}

\newcommand{\fig}[3][htbp]{
  \begin{figure}[#1] {\centering\scalebox{0.8}{\includegraphics{fig/#2}}\par}
    \caption{#3\label{fig:#2}}
  \end{figure}
}

\title{A Strategy to Develop Component Frameworks\\
  for Parallel Computing}
%  \thanks{The research work described in this article has been partially
%  supported by Fraunhofer FIRST.}}

%\author{
%  Ant�nio Augusto Fr�hlich\\
%  UFSC/CTC/LISHA\\
%  PO Box 476\\
%  88049-900 Florian�polis - SC, Brazil\\
%  \texttt{guto@lisha.ufsc.br}\\
%  \texttt{http://www.lisha.ufsc.br/$\sim$guto}}

%\author{
%  Ant�nio Augusto Fr�hlich\\
%  Fraunhofer FIRST\\
%  kekul�stra�e 7\\
%  12489 Berlin, Germany\\
%  \texttt{guto@first.fhg.de}\\
%  \texttt{http://www.first.fhg.de/$\sim$guto}}

\date{}

\bibliographystyle{latex8}

\begin{document}

\maketitle
\thispagestyle{empty}

\begin{abstract}
  
  Properly bridging a parallel architecture to a given class of parallel
  applications is a defying task that calls for custom designed
  \emph{run-time support systems}. Only in this way would applications
  get minimal-overhead system services delivered through
  application-oriented interfaces. Designing and implementing a new
  run-time support system for each class of parallel applications,
  however, would be impracticable in terms of costs. Notwithstanding,
  software component engineering could help to meet a compromise with
  applications by means of tailorable run-time systems.  This paper
  describes a strategy to develop component frameworks for parallel
  computing based on the \AOSD\ method proposed by
  Fr�hlich~\cite{Froehlich:2001}.  These frameworks are the substrata to
  build tailorable run-time support systems.

  \paragraph{Keywords:} application-oriented system design, 
  component frameworks, aspect-oriented programming, static
  metaprogramming, generative programming.

\end{abstract}

%\clearpage

%\section{Introduction}

%The advent of software engineering techniques that support the
%construction of systems as arrangements of software components might
%incorrectly lead one to believe that, by adopting such techniques, it
%would be possible to construct run-time support systems with a high
%degree of customizability. In particular, parallel applications could
%benefit from statically configurable run-time support systems, thus
%avoiding the high price of dynamic reconfiguration.

%Nevertheless, a repository of components and a mechanism to arrange them
%together may not suffice to comply with the requirements of distributed
%applications.  Ordinary run-time support systems frequently fail to
%deliver the expected services, or the expected service
%quality~\cite{Anderson:1992a, Maheshwari:1994}.  Slicing one such a
%system in a set of components will certainly not improve the case for
%applications.

% The method explores commonality and
%variability analysis during domain decomposition to model families of
%highly reusable, adaptable, application-ready abstractions, isolating
%scenario aspects and capturing fragments of reusable system
%architectures.  The refinement of design entities identified during
%domain analysis is subsequently approached, considering peculiarities in
%regard to the organization of abstractions in families, the conciliation
%of family members under a common interface, the modeling of scenario
%aspects that can be transparently applied, and representation of
%architectural aspects in component frameworks.  Afterwards,
%considerations about the implementation of \aosds\ are stated,
%emphasizing implementations in the \cpp\ programming language.


%%%%%%%%%%%%%%%%%%%%%%%%%%%%%%%%%%%%%%%%%%%%%%%%%%%%%%%%%%%%%%%%%%%%%%%%%%%%%%
\Section{Introduction}


% Suffocated by standards {Pike:2000}
Historically, operating systems have been constructed aiming at
abstracting physical resources in a way that is convenient to the
hardware, not to applications.  Undoubtedly, the monolithic structure of
early operating systems contributed to this scenario, for it must have
been very difficult, if not impracticable, to customize such systems in
order to accomplish the demands of particular applications. The notion
that applications have to be adapted to the operating system was so
established. Since then, a succession of standardizations has been
freezing application program interfaces~(API), thus helping to
consolidate the situation.  Consequently, contemporary operating systems
are suffocated by thick layers of standards that prevent internal
improvements from reaching applications~\cite{Pike:2000}.

% Out of date with modern software engineering
Besides failing to accompany the natural evolution of applications, many
operating systems also fail to keep updated as regards software
engineering. Perhaps this is also a consequence of extreme
standardization, whereas there is little room for novel software
engineering concepts in the constrained realm of operating systems.
Astonishingly, this is a very complex software realm, that spans from
hardware to applications, and would greatly profit from modern software
engineering techniques. In reality, however, the obsolescence of the
techniques deployed in some systems comes out to
impact applications. % as it is the case with \windows\ infamous
%crashes~\cite{Kabay:2000}.

% Conceived for hardware + standards, not for applications
Even modern systems that support customization have difficulties to
match up with application requirements, mainly because they usually
target the design of configurable features, the heart of any
customization strategy, on standard compliance and on hardware aspects,
and do not adequately address application requirements.  Hence, an
application programmer may be invited to select features such as \posix\ 
or \textsc{TCP/IP} compliance, or to select drivers for a certain
hardware device, but seldom will have the chance to select a parallel
object infrastructure. Deploying a general-purpose operating system to
support parallel applications is likely to result in a situation where
applications get uncountable services that are not needed, but still
have to implement much of what is needed. This also leads to the
phenomenon that transforms standard APIs, such as \mpi, in middleware
layers.  A properly constructed run-time support system could deliver
its services under a variety of APIs, eliminating such middleware
layers.

% How can CBSE fail?
Building a system as an aggregate of reusable \emph{software components}
has the potentiality to considerably improve the case for applications.
Nevertheless, component-based software engineering is just a means to
construct systems that can be customized to fulfill the demands of
particular applications.  Inadequately modeled components, or inadequate
mechanisms to select and combine components, may render the extra effort
of building reusable software components unproductive. The goal of
application-driven customization can only be achieved if the system as a
whole is designed considering the fulfillment of application
requirements.

% Customizability = user-driven configuration
Furthermore, the way customization is typically carried out in
component-based systems makes it difficult to pair with application
requirements.  As a rule, customization in these systems is delegated to
end users, which are assisted by some sort of tool in selecting and
combining components to produce an executable system.  In this case,
successfully customizing the system becomes conditioned to the knowledge
the user has about it. Hence, user-driven customization is entangled in
the balance of component granularity:

\begin{itemize}
\item If components are \emph{coarse-grained}, the chance of an ordinary
  user, i.e., a user without deep knowledge about the system, to
  successfully conduct customizations grows, but the probability that
  components will meet application requirements decreases
  proportionally.
\item If components are \emph{fine-grained}, the chance that the system
  will match application requirement grows, but it is likely that users
  will not be able to understand the peculiarities of such a large
  collection of components, probably missing the most adequate
  configurations.
\end{itemize}

% The Linux case
Improvements in user-driven configuration have been pursued by enabling
components to be selected indirectly. The \linux\ system, for instance,
utilizes a mechanism to select kernel components through the features
they implement.  Instead of pointing out which components will be
included in the system, users can select the desired system features.
Features, in turn, are interrelated by dependencies and mapped into
components.  Nevertheless, even though \linux\ kernel components are
coarse-grained (they are mainly device drivers and subsystems) and will
seldom satisfy the specific requirements of individual applications,
selecting features from a list with approximately 1450
options\footnote{The number of \linux\ configurable kernel features has
  been estimated by executing the following command in a system based on
  kernel version 2.4.18: \sym{grep CONFIG /usr/src/linux/.config | wc
    -l}.} is a sordid activity. A mechanism that allows applications
themselves to guide the configuration process would be more appropriate.

% SO only implements what is needed and delivers as requested
%   Scales with architecture AND APPLICATIONS
%   Automatic configuration and generation 
Notwithstanding, software engineering seems to be mature enough to
produce run-time support systems that, besides scaling with the
hardware, also scale with applications; that deliver all the
functionality required by applications in a form that is convenient for
them; and that deduce application requirements to automatically
configure itself. Many of the related issues have already been addressed
in the context of all-purpose computing by \emph{reflective systems}. In
order to comply with the requirements of high-performance parallel
applications, the subject is approached in this paper from the
perspective of statically configurable component-based systems. The
paper elaborates on the \emph{\AOSD} method~\cite{Froehlich:2001} to
delineate \emph{a strategy to develop component frameworks for parallel
  computing}.


%%%%%%%%%%%%%%%%%%%%%%%%%%%%%%%%%%%%%%%%%%%%%%%%%%%%%%%%%%%%%%%%%%%%%%%%%%%%%%
\Section{Domain Engineering}
%   + Domain Analysis
%     - The operating system realm as seen by applications
%     - Canonical partitioning X Families
%     - Families of abstractions
%     - Scenario independence
%     - Scenario adapters
%     - Reusable architectures
%     - Case study

% - OS -> conventions -> old -> standards -> solidified
% - {Pike:2000}
Although no design can go further than its perception of the
corresponding problem domain, domain engineering and run-time support
systems are subjects that seldom come together. The fact that the
operating system domain is basically made of conventions, many of which
established long ago in projects such as \thesys~\cite{Dijkstra:1968}
and \multics~\cite{Organick:1972}, seems to have fastened it to a
``canonical'' partitioning. This partitioning includes abstractions such
as \emph{process}, \emph{file}, and \emph{semaphore}, and is taken
``as-is'' by most designers. Indeed, it is now consolidated by standards
on one side and by the hardware on the other, leaving very little room
for new interpretations.

% - New perspectives (active objects)
% - Not as middleware!
Notwithstanding this, revisiting the problem domain during the design of
a new operating system would probably reveal abstractions that are
better tuned with contemporary applications. For example, the triple
\texttt{(process, file, socket)} could be replaced by \emph{persistent
  communicating active objects}. Actually, most run-time platforms
feature this perspective of the operating system domain through a
middleware layer such as \corba~\cite{corba} and \java~\cite{java}.
However, the middleware approach goes the opposite direction of
application-orientation, whereas it further generalizes an already
generic system.

% - Traditional abstractions -> families
% - Analysis => vocabulary
% - Design => abstractions
% - Domain Analysis => all applications in a domain
Nevertheless, even if one endures domain analysis knowing that
decomposition will have to be carried out respecting the boundaries
dictated by standards, programming languages, and hardware, there is at
least one important reason to do it: to avoid the monolithic
representation of abstractions. If an \aoos\ is to be the output of
design, capturing application-specific perspectives of each abstraction
and modeling them as independently deployable units, as suggested by
\emph{\sop}~\cite{Harrison:1993}, is far more adequate than the
monolithic approach. After all, the product of domain engineering is not
a single system, but a collection of reusable software artifacts that
model domain entities and can be used to build several systems.


\SubSection{Application-Oriented Domain Decomposition\label{sec:aodd}}

% AODD And OOD
An application-oriented decomposition of the problem domain can be
obtained, in principle, following the guidelines of
\emph{object-oriented decomposition}~\cite{Booch:1994}. However, some
subtle yet important differences must be considered.  First,
object-oriented decomposition gathers objects with similar behavior in
class hierarchies by applying variability analysis to identify how one
entity specializes the other.  Besides leading to the infamous ``fragile
base class'' problem~\cite{Mikhajlov:1998}, this policy assumes that
specializations of an abstraction (i.e. \emph{subclasses}) are only
deployed in presence of their more generic versions (i.e.
\emph{superclasses}).

Applying variability analysis in the sense of
\emph{\fbd}~\cite{parnas:1976} to produce independently deployable
abstractions, modeled as members of a family, can avoid this restriction
and improve on application-orientation.  Certainly, some family members
will still be modeled as specializations of others, as in
\emph{\isd}~\cite{habermann:1976}, but this is no longer an imperative
rule. For example, instead of modeling connection-oriented as a
specialization of connectionless communication (or vice-versa), what
would misuse a network that natively operates in the opposite mode, one
could model both as autonomous members of a family.

% Scenario-independence
A second important difference between application-oriented and
object-oriented decomposition concerns environmental dependencies.
Variability analysis, as carried out in object-oriented decomposition,
does not emphasizes the differentiation of variations that belong to the
essence of an abstraction from those that emanate from execution
scenarios being considered for it. Abstractions that incorporate
environmental dependencies have a smaller chance of being reused in new
scenarios, and, given that an \aos\ will be confronted with a new
scenario virtually every time a new application is defined, allowing
such dependencies could severely hamper the system.

Nevertheless, one can reduce such dependencies by applying the key
concept of \emph{\aop}~\cite{Kiczales:1997}, i.e.  aspect separation, to
the decomposition process.  By doing so, one can tell variations that
will shape new family members from those that will yield scenario
aspects.  
%\footnote{The representation of scenario aspects will be
%  discussed later in section~\ref{sec:sa}, for now it is only important
%  to avoid modeling unnecessary family members.}.
For example, instead of modeling a new member for a family of
communication mechanisms that is able to operate in the presence of
multiple threads, one could model multithreading as a scenario aspect
that, when activated, would lock the communication mechanism (or some of
its operations) in a critical section.

% - Why it gets scenario-dependent?
The phenomenon of mixing scenario aspects and abstractions seems to
happen spontaneously in most other design methods, thus learning to
avoid it might require some practice. Perhaps the most critical point is
the fact that systems are often conceived with an implementation
platform in mind, which is often better understood than the
corresponding problem domain.  In principle, there is nothing wrong in
studying the target platform in details before designing a system,
actually it might considerably save time, but designers tend to
misrepresent abstractions while considering how they will be implemented
in the chosen platform. In an \aosd, this knowledge about implementation
details should be driven to identify and isolate scenario aspects. In
general, aspects such as identification, sharing, synchronization,
remote invocation, authentication, access control, encryption,
profiling, and debugging can be represented as scenario aspects.

% Configurable features
Building families of scenario-independent abstractions and identifying
scenario aspects are the main activities in \aodd, but certainly not the
only ones. The primary strategy to add functionality to a family of
abstractions is the definition of new members, but sometimes it is
desirable to extend the behavior of all members at once.  Specializing
each member would double the cardinality of the family.  \Aosd\ deals
with cases like this by modeling the extended functionality as a
\emph{configurable feature}.  Just like scenario aspects, configurable
features modify the behavior of all members of a family when activated,
but, unlike those, are not transparent.  One could say that scenario
aspects have ``push'' semantics, while configurable features have
``pull''.

A configurable feature encapsulates common data structures and
algorithms that are useful to implement a family's feature, but leave
the actual implementation up to each family member.  Abstractions are
free to reuse, extend, or override what is provided in a configurable
feature, but are requested to behave accordingly when the feature is
enabled.

\fig[t]{aod}{An overview of \aodd.}

The case for configurable features can be illustrated with a family of
networks and features such as multicasting, in-order delivery, and error
detection.  If new family members were to be modeled for each such a
feature, a family of 10 networks subjected to 10 features could grow up
to $10^{10}$ members.  Modeling this kind of feature as a scenario
aspect is usually not possible either, since its implementation would
have to be specialized to consider particular network architectures.

% Inter-family relationships
Another relevant issue to be considered during domain decomposition is
how abstractions of different families interact. Capturing ad-hoc
relationships between families during design can be useful to model
reusable software architectures, helping to solve one of the biggest
problems in component-based software engineering: how to tell correct,
meaningful component compositions from unusable ones. A reusable
architecture avoids this question by only allowing predefined
compositions to be carried out. For example, one could determine that
the members of a family of process abstractions must use the family of
memory to load code and data, avoiding an erroneous composition with
members of the file family.  In \aosd, reusable architectures are
captured in component frameworks that define how abstractions of
distinct families interact.  Although such frameworks are defined much
later in the design process, taking note of ad-hoc relationships during
domain decomposition can considerably ease that activity.
 
% Overview
An overview of \aodd\ is presented in figure~\ref{fig:aod}. In summary,
it is a multiparadigm domain engineering method that promotes the
construction of \aoss\ by decomposing the corresponding domain in
families of reusable, scenario-independent abstractions and the
respective scenario aspects. Reusable system architectures are
envisioned by the identification of inter-family relationships that will
later build component frameworks.


%%%%%%%%%%%%%%%%%%%%%%%%%%%%%%%%%%%%%%%%%%%%%%%%%%%%%%%%%%%%%%%%%%%%%%%%%%%%%%%
\Section{Component Frameworks: capturing reusable system architectures\label{sec:aof}}

% - What is a reusable system architecture?
Along with the specification of abstraction families and scenario
aspects, an \aosd\ delivers specifications of reusable system
architectures, which define how abstractions can be arranged together in
a functioning system. Reusable system architectures are usually defined
considering the domain knowledge acquired during the design of previous
systems.  After having developed some systems, or some versions of a
system, for a certain problem domain, developers begin to agree on how
to implement the abstractions that build up the domain; how they
interact with each other, with the environment, and with applications;
and how the implied non-functional requirements can be accomplished.
Such an expertise can be captured in an architectural specification to
be reused in upcoming systems.

% - Why to reuse an architecure?
Capturing reusable system architectures in a component-based system is
fundamental, since a pile of components, by itself, is nothing but a
pile of components. A component-based system is only achieved when
components can be arranged together in an assemblage of predictable
behavior. In \aosd, reusable architectures begin to be modeled yet
during domain decomposition with the identification of relationships
between families of abstractions. These relationships are enriched by
scenario constraints during the specification of scenario aspects and
serve as input for this phase, which aims at producing a detailed
specification of reusable system architectures in the form of component
frameworks.

% - Component frameworks
An \emph{\aocf} captures a reusable architecture by specifying the
families of abstractions that take part in a certain kind of system, as
well as rules that guide their interaction. Systems produced by
component frameworks, when compared to arbitrary arrangements of
components, are less prone to misbehavior, since only compositions that
have been envisioned by system architects are allowed.  Although
component frameworks are not the unique alternative to capture reusable
architectures---among others, \emph{aspect programs}, \emph{subjects},
and \emph{collaborations} could also be used for this purpose---they fit
perfectly with application-orientation's notion of isolating scenario
aspects from abstractions by means of scenario
adapters~\cite{Froehlich:sci:2000}.

% - AOSD framework
An \aocf\ could be defined as a collection of interrelated scenario
adapters as shown in figure~\ref{fig:aof}. Each scenario adapter would
set up a ``socket'' for components of the corresponding family. Plugging
components into the framework would be accomplished by binding the
inflated interface of every used family to the desired family member.
The way scenario adapters are arranged in the framework would define the
basic architecture of resultant systems, while architectural elements
that do not concern components could be hard-coded in the framework.

\fig{aof}{An \aocf.}

A component framework defined in terms of scenario adapters would also
present advantages concerning \emph{system-wide properties}, which could
be modeled as scenario aspects to be enforced on components by the
respective scenario adapters~\cite{Szyperski:1998}. Moreover, a
component framework of this kind does not require complex code
manipulation tools to generate a system instance.  After all, the
representation of a component framework as a socket board to which
components can be plugged is well understood and accepted by users.

% - example 
Figure~\ref{fig:comm_fmk} shows a schematic representation of a
component framework that embodies a plausible system architecture for
the domain of high-performance communication in clusters of
workstations.  It illustrates the relationships between three families
of abstractions: \sym{Communication End-Point}, \sym{Communication
  Strategy}, and \sym{Network}. Firstly, it shows a mutual dependency
between the families of strategies and end-points, i.e., by selecting a
certain strategy, one automatically selects the corresponding end-point,
and vice-versa\footnote{In this model, \sym{Connection} is the end-point
  for the \sym{Stream} strategy, \sym{Port} and \sym{Mailbox} for
  \sym{Datagram}, \sym{Active Message Handler} for \sym{Active Message},
  \sym{Asynchronous Remote Memory Segment} for \sym{Asynchornous Remote
    Memory Read/Write/Copy}, and \sym{Dsitributed Shared Memory Segment}
  is the end-point for the \sym{Distributed Shared Memory} strategy.}.
It also shows that a network is \emph{used} by the members of the
communication strategy family. The respective sets of components are
also shown to illustrate the ``select-and-plug'' organization of the
framework.

\fig{comm_fmk}{A component framework for the domain of high-performance
  communication.}

Arbitrary compositions handled in the real of this component framework
could be adapted to scenarios such as \emph{thread safe},
\emph{buffered}, and \emph{reliable} simply by selecting the proper
scenario aspects.


%%%%%%%%%%%%%%%%%%%%%%%%%%%%%%%%%%%%%%%%%%%%%%%%%%%%%%%%%%%%%%%%%%%%%%%%%%%%%%%
\Section{Case Study: \epos\ component framework}

The \epos\ system was born in 1997 at the Research Institute for
Computer Architecture and Software Engineering~(FIRST) of the German
National Research Center for Information Technology~(GMD) as a project
to experiment with the concepts and mechanisms of \aosd.  Indeed, \epos\
and \aosd\ cannot be disassociated, since both have been evolving
together from the very beginning.

The acronym \epos\ stands for \emph{Embedded Parallel Operating System}.
It was coined considering the main research goal established for the
early system: to embed the operating system in a parallel application.
The strategy followed to achieve this goal consisted in modeling the
corresponding domain entities as a set of reusable and adaptable
components, and developing a mechanism to allow parallel applications to
easily specify constraints to guide the arrangement of these components
in a functioning system. As the system evolved, it became clear that
this strategy could be applied to a broader universe of applications.
Concerning design and organization, \epos\ is inherently tied with
dedicated computing and static configurability, but whether a platform
is dedicated to an application temporarily (like in traditional parallel
systems) or permanently (like in most embedded systems) does not play a
significant role.  Hence, \emph{Embedded Parallel Operating System} can
also be interpreted as a system that targets both embedded and parallel
applications.

\epos\ owes much of its philosophy to the \peace\ 
system~\cite{Preikschat:1994a}, from which it inherited the notion that
``generic'' and ``optimal'' are adjectives that cannot be simultaneously
applied to the same operating system, besides a rich perception of \fbd.
\epos\ implementation for the \intel\ architecture reuses some of the
strategies adopted in \aboelha~\cite{Froehlich:intersymp:1996}, a former
research operating system developed by the author. However, the design
of \aboelha\ did not promote reuse, so these strategies had to be
completely remodeled for \epos.

\epos\ component framework is realized by a \emph{static metaprogram}
and a set of \emph{composition rules}. The metaprogram is responsible
for adapting system abstractions to the selected execution scenario and
arranging them together during the compilation of an
application-oriented version of \epos. Rules coordinate the operation of
the metaprogram, specifying constraints and dependencies for the
composition of system abstractions. Composition rules are not encoded in
the metaprogram, but specified externally. They are interpreted by
composition tools in order to adjust the parameters of the metaprogram.

The separation of composition rules from the framework metaprogram
allows a single framework to yield a variety of software architectures.
Indeed, one could say that \epos\ has many frameworks, each
corresponding to the execution of the metaprogram with a different set
of arguments. Moreover, the use of static metaprogramming to compose
system abstractions does not incur in run-time overhead, thus yielding
composites whose performance is directly derived from their parts.


\SubSection{Framework Metaprogram}

\epos\ \emph{component framework metaprogram} is executed in the course
of a system instance compilation, adapting selected abstractions to
coexist with each other and with applications in the designated
execution scenario.  During this process, scenario-independent
abstractions have their original properties preserved, so that internal
compositions can be carried out before scenario adaptation.  This is
accomplished having the framework metaprogram to import
scenario-independent abstractions in one namespace and export the
corresponding scenario-adapted versions in another.

For example, the cascaded aggregation of \sym{Communication End-Point},
\sym{Communication Stra\-tegy}, and \sym{Network} illustrated in the
previous section would take place at the scenario-independent level.
The resultant composite would later be adapted to the selected scenario
as a whole. 

\fig{epos_fmk}{A top-view of \epos\ component framework metaprogram.}

Figure~\ref{fig:epos_fmk} shows a top-view diagram of the component
framework static metaprogram. Each of the elements represented in the
figure will be subsequently described. A class diagram representing the
\sym{Handle} framework element is depicted in
figure~\ref{fig:epos_fmk_handle}.  Like most other elements,
parameterized class \sym{Handle} takes a system abstraction (class of
system objects) as parameter.  When instantiated, it acts as a
``handle'' for the supplied abstraction, realizing its interface in
order that invocations of its methods are forwarded to \sym{Stub}.
Hence, system objects are manipulated by applications via their
``handles''.

\sym{Handle} provides additional operations to check if a system object
was successfully created\footnote{The use of \cpp\ exceptions as a
  mechanism to signalize system object creation failure was avoided
  because it would make difficult the integration of \epos\ with
  applications written in other programming languages.} and to obtain
its id.  Besides, when the \sym{Shared} scenario aspect is enabled,
\sym{Handle} provides the extra constructors used to designate sharing
in that scenario. The aggregation relationship between \sym{Handle} and
\sym{Stub} enables the system to enforce allocation via a system
allocator (instead of a programming language one), thus allowing for the
\sym{Allocated} scenario aspect\footnote{The allocator used with each
  abstraction is selected by \sym{Adapter} after consulting
  \sym{Traits<Abs>::allocated}.}.

\fig{epos_fmk_handle}{\epos\ framework: the \sym{Handle} element.}

The \sym{Stub} framework element is depicted in
figure~\ref{fig:epos_fmk_stub}. This parameterized class is responsible
for bridging \sym{Handle} either with the abstraction's scenario adapter
or with its proxy.  It declares two formal parameters: an abstraction
and a boolean flag that designates whether the abstraction is local or
remote to the address space of the calling process.  By default,
\sym{Stub} inherits the abstraction's scenario adapter, but it has a
specialization, namely \sym{Stub<Abs, true>}, that inherits the
abstraction's proxy.  Therefore, making \sym{Traits<Abstraction>::remote
  = false} causes \sym{Handle} to take the scenario adapter as the
\sym{Stub}, while making it \sym{true} causes \sym{Handle} to take the
proxy.

\fig{epos_fmk_stub}{\epos\ framework: the \sym{Stub} element.}

The \sym{Proxy} framework element is deployed in a remote invocation
scenario.  \sym{Proxy} realizes the interface of the abstraction it
represents, forwarding method invocations to its \sym{Agent} (see
figure~\ref{fig:epos_fmk_agent}).  Each instance of \sym{Proxy} has a
private ROI message, which is initialized in such a way that forthcoming
invocations only need to push parameters into the message.  Moreover,
because \sym{Proxy} is metaprogrammed, parameters are pushed directly
into the message, without being pushed into the stack first. \sym{Proxy}
operations invoke method \sym{invoke}\footnote{The semantics of the
  \sym{invoke} method varies according to the selected configuration.
  In some cases, it causes the application process to ``trap'' into the
  kernel, in others, it directly accesses a communicator to perform a
  message exchange.} to perform a message exchange with \sym{Agent},
which, likewise \sym{Handle} for a local scenario, forwards invocations
to the abstraction's \sym{Adapter}.

%\fig{epos_fmk_proxy}{\epos\ framework: \sym{Proxy} element.}
\fig{epos_fmk_agent}{\epos\ framework: \sym{Proxy} and \sym{Agent}
  elements.}

The \sym{Adapter} framework element is depicted in the
figure~\ref{fig:epos_fmk_adapter}. This parameterized class realizes a
scenario adapter for the abstraction it takes as parameter, adapting its
instances to perform in the selected scenario.  Adaptations are carried
out by wrapping the operations defined by the abstraction within the
\sym{enter} and \sym{leave} scenario primitives, and also by enforcing a
scenario-specific semantics for creating, sharing, and destroying its
instances. The role of \sym{Adapter} in \epos\ framework is to apply the
primitives supplied by \sym{Scenario} to abstractions, without making
assumptions about the scenario aspects represented in these primitives.
In this way, \sym{Adapter} is able to enforce any combination of
scenario aspects.

\fig{epos_fmk_adapter}{\epos\ framework: the \sym{Adapter} element.}

\fig{epos_fmk_scenario}{\epos\ framework: the \sym{Scenario}
  element.}

The execution scenario for \epos\ abstractions is ultimately shaped by
the \sym{Scenario} framework element depicted in
figure~\ref{fig:epos_fmk_scenario}. Each instance of parameterized class
\sym{Scenario} delivers scenario primitives that are specific to the
abstraction supplied as parameter.  Firstly, it incorporates the
selected \sym{Id} aspect, which is common to all abstractions in a
scenario; then it consults the abstraction's \sym{Traits} to determine
which aspects apply to it, aggregating the corresponding scenario
aspects. The strategy to cancel an aggregation is similar to the one
used with \sym{Stub}, i.e. a parameterized class that inherits the
selected aspect by default, but is specialized to inherit nothing in
case the aspect is not selected for the abstraction.

Besides designating which scenario aspects apply to each abstraction,
the parameterized class \sym{Traits} maintains a comprehensive
compile-time description of abstractions that is used by the metaprogram
whenever an abstraction-specific element has to be configured.


\SubSection{Composition Rules\label{sec:epos_fmk_rules}}

\epos\ component framework metaprogram is able to adapt and assemble
selected components to produce an \aoos. However, though the metaprogram
knows about particular characteristics of each system abstraction from
its \emph{traits}, it does not know of relationships between
abstractions and hence cannot guarantee the consistency of the
composites it produces. In order to generate a meaningful instance of
\epos, the metaprogram must be invoked with a coherent parameter
configuration.

Therefore, the operation of the framework metaprogram is coordinated by
a set of \emph{composition rules} that express elements of reusable
system architecture captured during design.  A consistent instance of
\epos\ comprises system abstractions, scenario aspects, hardware
mediators, configurable features, and non-functional requirements.
Composition rules specify dependencies and constraints on such elements,
so that invalid configurations can be detected and rejected.
Nevertheless, guarantying that a composite of \epos\ elements is
``correct'' would depend on the formal specification and validation of
each element, what is outside the scope of this research.

Sometimes, composition rules are implicitly expressed during the
implementation of components.  For example, by referring to the
\sym{Datagram} channel, the \sym{Port} communicator implicitly specifies
a dependency rule that requires \sym{Datagram} to be included in the
configuration whenever \sym{Port} is deployed.  However, most
composition rules, especially those designating constraints on combining
abstractions, can only be expressed externally. For instance, the rule
that expresses the inability of the \sym{Flat} address space to support
the \sym{Mutual} task abstraction must be explicitly written.

In order to support the external specification of composition rules,
\epos\ elements are tagged with a \emph{configuration key}. When a key
is asserted, the corresponding element is included in the configuration.
Elements that are organized in families are selected by assigning a
member's key to the family's key, causing the family's inflated
interface to be bound to the designated realization. This mechanism
implements the selective realize relationships modeled during design.
For example, writing \sym{Synchronizer := Semaphore} causes the inflated
interface of the \sym{Synchronizer} family of abstractions to be bound
to member \sym{Semaphore} and writing \sym{Id := Capability} binds the
\sym{Id} scenario aspect to \sym{Capability}.  Elements that do not
belong to families have their keys asserted accordingly. For example,
writing \sym{Busy\_Waiting := True} enables the \sym{Busy\_Waiting}
configurable feature if the \sym{CPU\_Scheduler} abstraction.

Composition rules are thus defined associating pre- and postconditions
to configuration keys. For instance, the following rule for the
\sym{Task} family of abstractions requires the \sym{Paged} address space
to be selected before the \sym{Mutal} task can be selected:
\begin{center}
\sym{Mutual $\Rightarrow$ pre: Address\_Space = Paged}
\end{center}
\noindent Alternatively, this constraint could be expressed as a
composition rule for the \sym{Address\_Space} family that selects the
\sym{Exclusive} task whenever the \sym{Flat} address space is selected:
\begin{center}
  \sym{Flat $\Rightarrow$ pos: Task := Exclusive}
\end{center}

Composition rules are intended to be automatically processed by
configuration tools. Hence, it is fundamental to keep them free of
cycles.  The following rule, though understandable for a human, could
bring a tool to deadlock:
\begin{center}
  \sym{A1 $\Rightarrow$ pre: B = B1\\B1 $\Rightarrow$ pre: A = A1}
\end{center}

%????

In order to ensure that \epos\ composition rules build a direct
acyclic graph, the following directives were observed:

\begin{itemize}
\item Configuration keys are \emph{totally ordered} according to an
  arbitrary criterion;
\item Preconditions are restricted to \emph{expressions} involving only
  \emph{preceding} keys;
\item Postconditions are restricted to \emph{assignments} involving only
  \emph{succeeding} keys.
\end{itemize}

Along with the \emph{traits} of each abstraction, composition rules
control the process of tailoring \epos\ to a particular application. The
set of configuration keys selected by the user is validated and refined
by means of composition rules, yielding a set of elements that are
subsequently assembled by the framework metaprogram consulting the
traits of abstractions. % A tool that is able to perform automatic
%tailoring will be described later in section~\ref{sec:epos_conf}.


%%%%%%%%%%%%%%%%%%%%%%%%%%%%%%%%%%%%%%%%%%%%%%%%%%%%%%%%%%%%%%%%%%%%%%%%%%%%%%%
\Section{Conclusions}

Historically, applications have been adapted to the operating system,
adhering to standardized application program interfaces that covey
uncountable useless services (for each individual application), and yet
fail to deliver much of what is necessary.  An application-oriented
operating system ruptures with this notion, implementing services that
emanate from application requirements and delivering them as a set of
configurable components that can be assembled to produce
application-tailored system instances.

In this paper, a strategy to deploy the \aosd\ method in order to build
component frameworks for parallel computing has been proposed.
Frameworks produced under this strategy consist of intertwined scenario
adapters that build a socket board for software components. The case
study on \epos\ component framework illustrates the strategy, supplying
run-time system designers with valuable guidelines to meet the parallel
application challenge.


%Such \aooss\ can be constructed deploying the \aosd\ method %, which is
%%outlined in figure~\ref{fig:aosd}, 
%to engineer the envisioned domain. In this way, the domain is decomposed
%in abstractions that model application-specific perspectives of each
%entity.  These abstractions are gathered in families according to what
%they have in common, with variability analysis being deployed to
%identify subsequent members of each family. During this process,
%variations that belong to the essence of abstractions are separated from
%those that emanate from execution scenarios, the former shaping family
%members, and the latter yielding scenario aspects.  This separation
%improves on reusability, for scenario-independent abstractions can be
%reused in a larger variety of
%scenarios. % Moreover, the complexity of individual abstractions is
%%decreased, thus enhancing software quality in general.


\bibliography{guto,se,os}

\end{document}
