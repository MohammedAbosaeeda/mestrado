% ------------------------------------------------------------------------------
\section{Conclusões}
\label{sec:discussion}
% Aplicações para sistemas embarcados usualmente necessitam interagir com
% diversos tipos de dispositivos de hardware como sensores, atuadores, 
% transmissores, receptores e \emph{timers}.
%
% Interface de função estrangeira é o mecanismo adotado por Java para superar as
% limitações da linguagem e permitir acesso direto a memória e a dispositivos de
% hardware. 
% Entretanto, como mostrado na seção \ref{sec:related_work}, as principais FFIs 
% Java não são eficientes em termos de consumo de recursos, ou possuem limitações
% de projeto que dificultam o desenvolvimento de novas interfaces entre Java e os
% dispositivos de hardware.
%
Neste artigo, apresentou-se um meio de realizar a interface entre componentes 
de hardware e aplicações Java para sistemas embarcados. 
Isto foi obtido utilizando-se a interface de função estrangeira da JVM KESO e o
EPOS.

% O EPOS permite o desenvolvimento de aplicações portáveis, independentes de
% especificidades de máquina. 
% Isto é conseguido utilizando-se o conceito de mediadores de hardware, os quais
% sustentam um contrato de interface entre abstrações de sistemas e a máquina.

% O JVM KESO compila o bytecode de uma aplicação Java em código C e gera as partes
% da JVM necessárias pela aplicação. 
% A FFI da KESO também utiliza esta abordagem estática, gerando o código C 
% especificado nas classes \emph{Weavelet}. 
% Então o código C gerado pelo compilador KESO e pela FFI da KESO são compilados
% em conjunto em código nativo, utilizando-se um compilador C padrão.

Nós avaliamos nossa abordagem em termos de desempenho, consumo de memória e 
portabilidade.
Para a aplicação utilizando o mediador de hardware da UART o sobrecusto 
de tempo obtido foi menos de 0.04 \% do tempo total de execução da aplicação e
nossa solução é 38 vezes mais rápido do que a JNI da Sun.
O consumo de memória para tal aplicação foi de 33KB, incluindo todo o suporte
de ambiente de execução, o qual é adequado para diversos sistemas embarcados.
Utilizando o EPOS nós obtivemos portabilidade para várias plataformas e 
utilizando o conceito de componentes híbridos podemos utilizar os mesmos 
adaptadores de código nativo tanto para componentes implementados em 
hardware como implementados em software.

Visando avaliar nossa abordagem em uma aplicação real, nós escrevemos
adaptadores de código nativo para um componente o qual realizada estimativa de
movimento para codificação de vídeo H.264.

% ------------------------------------------------------------------------------

