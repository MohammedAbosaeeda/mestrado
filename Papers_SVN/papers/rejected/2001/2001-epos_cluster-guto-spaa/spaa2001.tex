\documentclass[english,10pt]{article}
\usepackage[T1]{fontenc}
\usepackage[latin1]{inputenc}
\usepackage{babel}
\usepackage[letterpaper,right=2.5cm,left=2.5cm,top=2.5cm,bottom=1.9cm,noheadfoot]{geometry}
\usepackage[dvips]{graphics}

\newcommand{\fig}[2]{
  \begin{figure}[htb]
  {\centering\scalebox{.8}{\includegraphics{#1.eps}}\par}
  \caption{#2\label{fig:#1}}
  \end{figure}
}

\pagestyle{empty}

\begin{document}

\title{On Application-Oriented Parallel Run-time Systems}

\author{
  Ant�nio Augusto Fr�hlich\\
  GMD-FIRST\\
  Kekul�stra�e 7\\
  12489 Berlin, Germany\\
  guto@first.gmd.de \and
  Wolfgang Schr�der-Preikschat\\
  University of Magdeburg\\
  Universit�tsplatz 2\\
  39106 Magdeburg, Germany\\
  wosch@ivs.cs.uni-magdeburg.de }

\date{}

\maketitle
\thispagestyle{empty}

\section*{\centering Extended Abstract for SPAA Revue}

% - Cluster computing X communication
The parallel computing community has been using clusters of commodity
workstations as an alternative to expensive massively parallel
processors (MPP) for several years by now. While MPPs can rely on custom
hardware to achieve high performance, their development follows a slow
pace, mainly due to the limited production scale. Clusters, on the other
hand, benefit from the frenetic pace with which the workstation market
evolves. As a matter of fact, when an MPP comes to the market, it is
very likely that processor, memory, and interconnection systems with
similar features will already be available at the commodity market.
Hence, it seems evident that both technologies are going to merge soon.
 
Nevertheless, when we compare the performance of parallel applications
running on clusters and on MPPs, the figures show a different scenario:
clusters are still quite behind their expensive relatives~\cite{npb}. We
believe the origin of this shortcoming to be in the software environment
normally used in clusters. While MPPs rely on custom software,
especially developed to support parallel applications, clusters often
apply the ``commodity'' principle to the software also.  Commodity
workstation software, however, has not been designed to support parallel
computing.  Indeed, every parallel application has particular run-time
demands that will never be adequately covered by all-purpose operating
systems, for their generality prevents most optimizations.  This
contradiction between generic and optimal is consequently discussed
in~\cite{Preikschat:1994}.

In the context of project \textsc{Epos} (\emph{Embedded Parallel
  Operating System})~\cite{Froehlich:ehpc:1999}, we developed a novel
operating system design method that is able to produce run-time support
systems that can be customized to closely match the requirements of
particular parallel applications. Because of its strong compromise with
applications, we named this method \emph{Application-Oriented System
  Design} (AOSD). The design of an application-oriented operating system
begins with the decomposition of the operating system domain in
abstractions that are natural to application programmers. This is
exactly the decomposition strategy promoted by \emph{Object-Oriented
  Design}~\cite{Booch:1994} and may sound obvious to application
designers, but most system designers simply neglect the problem domain
and let implementation details, such as target architecture, programming
languages, and standardized interfaces, guide the design
process~\cite{Pike:2000}. Application programmers, not seldom, get
run-time systems that barely resemble the corresponding domain.

The next step is to model software components that properly represent
the abstractions from the decomposed domain. Extensive components, that
encapsulate all perspectives of an abstraction in a single entity, are
not an alternative, since we want components to closely match the
requirements of particular applications. A more adequate approach would
be to apply the commonality and variability analysis of
\emph{Family-Based Design}~\cite{Parnas:1976} to yield a family of
abstractions, with each member capturing a significant variation and
shaping a component.  Nonetheless, this approach has the inconvenient of
generating a high number of components, thus considerably increasing
complexity.  We handle this drawback by exporting all members of a
family through a single \emph{inflated interface}.  In a system designed
accordingly, adequate members of each required family could be
automatically select by a tool that performs syntactical analysis of the
corresponding application's source code. Moreover, application
programmers should feel comfortable to use these inflated interfaces,
for they emanate directly from the problem domain.

Another important factor to be considered while modeling abstractions is
scenario independence. When a designer realizes that an abstraction has
to to be specialized in order to join a given scenario, for instance a
\emph{mailbox} abstraction in a \emph{multithreaded} scenario, he has to
choose between modeling a new family member and capturing this scenario
dependency in a separate construct.  As remarked by
\emph{Aspect-Oriented Programming}~\cite{Kiczales:1997}, allowing
abstractions to incorporate scenario dependencies reduces their degree
of reusability and produces an explosion of scenario-dependent
components. Therefore, an application-oriented design should try to
avoid it, only allowing those variations that are inherent to the family
to shape new members.  The resulting \emph{scenario-independent
  abstractions} shall be reusable in a larger variety of scenarios, some
of them unknown at the time they were modeled. Scenario specificities,
in turn, can be captured in constructs like the \emph{scenario adapters}
described in~\cite{Froehlich:sci:2000}. Because scenario adapters share
the semantics of collaborations in \emph{Collaboration-Based
  Design}~\cite{Reenskaug:1992}, one could say that an abstraction
collaborates in a scenario.

In order to produced a run-time support system, the components resulting
from the process described above have to assembled together correctly
and efficiently. By correct assembly we mean preserving the individual
semantics of each component in the presence of others and under the
constraints of an execution scenario.  By efficient assembly we mean
preserving their individual efficiency in the resulting composite. One
possibility to produce the desired compositions is to capture a reusable
system architecture in a \emph{component framework}. A framework enables
system designers to predefine the relationships between abstractions and
therefore can prevent misbehaved compositions.  Furthermore, a framework
defined in terms of scenario adapters can achieve a high degree of
adaptability. Efficient composition can be accomplished if the framework
uses \emph{Generative Programming} techniques~\cite{Czarnecki:2000},
such as \emph{static metaprogramming}.  Since static metaprograms are
executed at compile-time, a statically metaprogrammed framework can
avoid most of the overhead typical of traditional object-oriented
frameworks.  It is also important to notice that, though component
composition would take place at compile-time, nothing would prevent
components from using dynamic reconfiguration mechanisms to internally
adapt themselves.

\fig{aosd}{An overview of Application-Oriented System Design.}

In brief, \emph{Application-Oriented System Design}
(figure~\ref{fig:aosd}) is a multiparadigm design method that supports
the construction of customizable run-time support systems by decomposing
the system domain in families of reusable, scenario-independent
abstractions and the corresponding scenario adapters. Reusable system
architectures are modeled as component frameworks that can guide the
compilation of the target system. Application programmers interact with
the system through inflated interfaces, without having to know details
about the organization of families or scenarios.

We are applying Application-Oriented System Design in project
\textsc{Epos}, which aims at delivering automatically generated run-time
support systems for dedicated applications (parallel and embedded).
\textsc{Epos} consists of a collection of components, a component
framework, and a set of tools that support application programmers in
generating run-time support systems for their applications. It is
currently being implemented for a cluster of PCs interconnected with
32-bits \textsc{Myrinet}. A preliminary evaluation of this
system~\cite{Froehlich:hpcn:2000} revealed unprecedented performance
figures, reaching a bandwidth of 119 MB/s in a machine limited to
approximately 130 MB/s. This results corroborate the guidelines of
\emph{Application-Oriented System Design}, as well as \textsc{Epos}
design decisions. Nevertheless, \textsc{Epos} is a long term project
that aims at delivering application-oriented run-time systems to a large
universe of applications. Thus, several system abstractions, scenario
adapters, and tools are still to be implemented or improved.

\bibliographystyle{abbrv}
\bibliography{se,os,network,cluster,parallel,guto}

\end{document}

%%% Local Variables: 
%%% mode: latex
%%% TeX-master: t
%%% End: 
