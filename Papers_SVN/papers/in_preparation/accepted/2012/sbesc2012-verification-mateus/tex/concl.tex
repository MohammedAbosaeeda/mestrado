% ------------------------------------------------------------------------------
\section{Conclusion} \label{concl}
% + Conclusion Remember the need of computational system to be correct.
% States that using our approach one can verify correctness of embedded system
% components at System Level Design.
%
In this paper, we have introduced an approach to verify
functional correctness properties, and safety properties
of embedded system components formally and at System-Level.
The functional correctness properties of a component are expressed by its
contract containing class invariants, and preconditions and postconditions for
each public method.
The safety properties are automatically generated by the CBMC model checker,
used in this work.
The component target of verification is instrumented with its contract using
the \emph{Verified Scenario}, and the resulting component is verified by CBMC.

We have chosen the C++ language for implementing the component and
for specifying its contract thus, no special knowledge about formal methods is
required from the component designer.

Although we have chosen C++ and CBMC, we believe that the general ideas of our
approach can easily be adapted for other languages and model checkers.
In the case of using SystemC, the verified scenario can be directly applied,
since SystemC is implemented as a C++ library.
In the case of SpecC, the verified scenario can be applied by
using aggregation
instead of inheritance
in order to compose
the verification aspects with the \emph{VerifiedScenario},
and the \emph{VerifiedScenario} with the \emph{ScenarioAdapter}.
% since SpecC is based in C and does not support inheritance.
Then, any model checker supporting such languages and supporting
user-defined assertions can be used.

In order to evaluate out approach we have formally verified the scheduler
component of the EPOS embedded operating system.
A software version of such scheduler can be obtained using the GCC compiler and,
using HLS tools, a hardware version can be synthesized from the same scheduler
description.

In this paper, we have presented the verification
of the scheduler independently of the object been schedule.
As future directions for this work we planning on verifying
thread abstractions of the operating system thus,
verifying the whole task scheduling of the system.

% ------------------------------------------------------------------------------
