\documentclass[a4paper,english,11pt]{article}
\usepackage[T1]{fontenc}
\usepackage[latin1]{inputenc}
\usepackage{babel}
\usepackage{times}
\usepackage{graphics}
\usepackage[margin=3cm]{geometry}

\title{EPOS: Paving the Path for Parallel Applications}

\author{
  Ant�nio Augusto Fr�hlich\\
  GMD FIRST\\
  kekul�stra�e 7\\
  12489 Berlin, Germany\\
  \texttt{guto@first.fhg.de}\\
  \texttt{http://www.first.fhg.de/$\sim$guto}
}

\date{}

\begin{document}

\maketitle
\thispagestyle{empty}

\begin{abstract}

Every time more applications demand performance levels that can only be
achieved by parallelization. In order to properly support them, new
operating systems and tools are to be conceived. Our experiences
developing runtime support systems for parallel applications convinced
us that adjectives such as "generic" and "all purpose" do not fit
together with "high performance", whereas different parallel
applications have quite different requirements regarding the operating
system. Even apparently flexible designs, like micro-kernel based
operating systems, may imply in waste of resources that, otherwise,
could be used by applications.  The promotion of configurability has
been properly addressed by the PURE operating system. PURE is designed
as a collection of configurable classes that can be seen as building
blocks to assemble application-oriented operating systems. Approaches
like this, although doing much for performance, reusability and
maintainability, usually are not enough to support application
programmers, since the number and the complexity of available building
blocks grows quickly with the system evolution. In such a context,
selecting and configuring the proper building blocks becomes a nightmare
and yields a gap between that what the operating system offers and that
what the applications expect.

EPOS aims to deliver, whenever possible automatically, a customized
runtime support system for each parallel application. In order to
achieve this, EPOS introduces three main concepts: 1 - adaptable,
scenario independent system abstractions that result from composing PURE
building blocks into application-ready abstractions. These abstractions
are designed to be as much independent from the execution scenario as
possible. 2 - Scenario adapters that adapt existing system abstractions
to a given execution scenario, for instance, by making an existing
thread abstraction ready to run in a SMP configuration. 3 - Inflated
interfaces that export the system abstraction repository by gathering
several different implementations of each system abstraction in a
single, well-known interface.


An application designed and implemented following the guidelines behind
these concepts can be submitted to a tool that will proceed syntactical
and data flow analysis to extract an operating system blueprint. This
blueprint is then refined by dependency analysis against information
about the execution scenario acquired from the user via visual tools;
and then submitted to another tool that will generate the
application-oriented operating system. With this approach, EPOS shall
diminish the gap that usually separates operating systems from parallel
applications.
\end{abstract}

\end{document}
