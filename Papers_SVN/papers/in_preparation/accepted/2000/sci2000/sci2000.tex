\documentclass[letterpaper,twocolumn,10pt,english]{article}
\usepackage[T1]{fontenc}
\usepackage[latin1]{inputenc}
\usepackage[tmargin=1.5cm,noheadfoot,textwidth=17.7cm,textheight=23.9cm]{geometry}
\usepackage{babel}
\usepackage{times}
\usepackage[pdftex]{graphics}
\usepackage{listings}

\setlength{\parindent}{0em}
\setlength{\parskip}{2ex}
\setlength{\columnsep}{1.5cm}
%\setlength{\footnotesep}{2ex}

\lstset{language=C++,basicstyle=\normalsize,keywordstyle=\ttfamily\textbf,flexiblecolumns=true,labelstep=0}

\newcommand{\tild}{\raisebox{-.75ex}{\~{}}}

\newcommand{\putfig}[5]{
  \begin{figure}[htbp]
  {\centering\resizebox*{#2}{#3}{\includegraphics{fig/#1.pdf}}\par}
  \caption{#5\label{#4}}
  \end{figure}
}

\newcommand{\putprog}[5]{
  \begin{figure}[htbp]
  {\centering\fbox{\lstinputlisting[first=#2,last=#3]{#1}}\par}
  \caption{#5\label{#4}}
  \end{figure}
}

\newcommand{\prog}[3]{
  \mbox{\lstinputlisting[first=#2,last=#3]{#1}}
%  \vspace{-2ex}
}

\renewcommand{\thesection}{\arabic{section}.\hspace{-0.75em}}
\renewcommand{\thesubsection}{
  \arabic{section}.\arabic{subsection}.\hspace{-0.75em}
}

\begin{document}
\pagestyle{empty}

\title{Scenario Adapters: Efficiently Adapting Components\thanks{This
    research has been partially supported by the Federal University of
    Santa Catarina, by CAPES Foundation grant no. BEX 1083/96-1, by
    Deutsche Forschungsgemeinschaft grant no. SCHR 603/1-2, and by the
    Bundesministerium f�r Bildung und Forschung grant no. 01 IS 903
    D2.}}

\author{
  Ant�nio Augusto Fr�hlich\\[-0.5ex]
  \small{GMD-FIRST}\\[-0.5ex]
  \small{Kekul�stra�e 7, 12489 Berlin, Germany}\\[-0.5ex]
  \small\texttt{guto@first.gmd.de}\\[1ex]
  \small{and}\\[1ex]
  Wolfgang Schr�der-Preikschat\\[-0.5ex]
  \small{University of Magdeburg}\\[-0.5ex]
  \small{Universit�tsplatz 2, 39106 Magdeburg, Germany}\\[-0.5ex]
  \small\texttt{wosch@ivs.cs.uni-magdeburg.de}
}   

\date{}
\maketitle
\thispagestyle{empty}
 
\sloppy

\section*{\protect\centering Abstract}

In this paper we consider the utilization of com\-po\-nent-ba\-sed
software engineering techniques for the development of adaptable systems
compromised with performance. The \textsc{Scenario Adapter} construct is
proposed as an effective means to achieve this goal. It can be used to
adapt scenario-independent abstractions to an execution scenario known
at compile-time. The efficiency of \textsc{Scenario Adap\-ters} has been
demonstrated in the Project \textsc{Epos}, which aims to deliver,
whenever possible automatically, a tailored run-time support system for
each application.

Keywords: component-based software engineering, static adaptation,
template metaprogramming.


\section{\protect\centering Introduction}

Object-orientation together with component-based software engineering is
enabling the long dreamed production of software as an assemblage of
ordinary components. Similarly to other industries, software developers
can nowadays reuse components, reducing costs and accelerating
production. Moreover, in comparison to other assembly lines, for
instance to the largely acclaimed automotive industry, component-based
software engineering shows an expressive advantage: software components,
besides being reused, can easily be adapted to match particular
requirements. This, for the car industry, would mean having a single
engine that could be adapted either to propel a limousine or a small
city car.

Conceiving a system as an assemblage of adaptable components, however,
brings about new challenges. To begin with, few software engineers would
dare to suggest a definition for the term software component. Module,
class, class category, object file, and server are just some of the
constructs commonly referred to as component. Perhaps a dictionary
definition, such as the one from the Oxford English Dictionary, which
defines a component as ``any of the parts of which something is made'',
would be more adequate. Nevertheless, the lack of a clear definition for
the term does not seem to prevent this technology from been successfully
used. Other open questions have a more restrictive impact. For instance:
What is a good size for a component? How can a component be adapted? How
can it be glued together with other components? How to grant that a
component composition yields a system that matches the requisites?
Questions like these have to be answered in order to make
component-based software engineering really effective.

The main concern of this paper is the adoption of component-based
software engineering techniques to construct efficiently adaptable
systems. The paper begins with some considerations about adaptable
components, after what the \textsc{Scenario Adapter} construct is
described in details. Scenario adapters can be used to efficiently adapt
an existing component to join a specific execution scenario. Next a case
study about the use of scenario adapters in the project \textsc{Epos} is
presented.


\section{\protect\centering Adaptable Abstractions}

Despite the controversy about what a component is (or should be), in
this paper we will assume any plausible abstraction in the system's
application domain to be a component. Considering an operating system,
abstractions such as \texttt{thread} and \texttt{mailbox} could be our
components, be them classes, servers, or any other kind of ``part''. If
an abstraction can be adjusted to satisfy the requisites of several
execution scenarios, we say it is an adaptable abstraction. Based on
these assumptions, a strategy to implement efficiently adaptable
abstractions is described next.


\subsection{Scenario Independence}

The advantage of a system in which abstractions are independent from the
execution scenario they run in is obvious: it can be adapted to join
many scenarios. Whenever the system is requested to join a new scenario,
some sort of adapter has to be implemented, but the basic abstractions
remain untouched. Achieving this goal, however, is not a trivial matter,
since abstractions tend to embed several scenario peculiarities.

The pollution of abstractions with scenario specific aspects happens
naturally along the abstraction life-cycle. During the analysis phase,
whenever the application domain is successfully partitioned,
abstractions are assigned clear responsibilities that are completely
independent from execution scenarios. These responsibilities are
eventually translated into behavior and structure specifications during
the design phase. If carried out properly, this phase can still yield
scenario-independent abstractions, with dependencies isolated in proper
constructs. However, the migration to the implementation phase tends to
extend each abstraction's specification to satisfy the predicted
execution scenarios, thus breaking down scenario-independence.

As an example, consider a \texttt{mailbox} abstraction in an operating
system. From the analysis phase, we could have got that a
\texttt{mailbox} is the abstraction responsible for supporting
\mbox{$n$-to-$n$}, bidirectional communication among active objects.
During design, this specification would probably have been extended to
accommodate operations such as the sending and receiving of messages, as
well as relations to constructs that will enable \texttt{mailbox}
identification and location, message buffering, operation
synchronization, etc. However, it would not be unusual if, along the
implementation phase, questions regarding possible underlying networks,
buffer management, security, and many others, had distorted the
\texttt{mailbox} abstraction, transforming it into a scenario-specific
abstraction.
 
This phenomenon has several implications on software quality metrics.
First, it impacts reusability, since abstractions that are polluted with
execution scenario details are seldom reusable in other scenarios. Very
often different versions of an abstraction have to be implemented, one
for each execution scenario. This impacts maintainability and increases
the complexity of system configuration, since there are now several
realizations for each abstraction. Therefore, it is very important to
keep abstractions as independent from execution scenarios as possible.

Aspect-independence is also the main appeal behind \emph{Aspect-Oriented
  Programming} (AOP)~\cite{Kiczales:ecoop:1997}. However, although AOP
suggests means to adapt aspect-independent abstractions according to an
aspect program, AOP itself does not enforce a design policy that yields
aspect-independent abstractions.

\subsection{Component Granularity}

When talking about efficient adaptation of components, granularity
becomes an important matter, since it directly impacts performance,
configurability and maintainability. On the one hand, a system made up
of a large amount of fine components will certainly achieve better
performance than one made up of a couple of coarse components, because
each coarse component brings along functionality that will not always be
used. The component functionality that is not used often turns into
overhead for the applications ~\cite{Parnas:1976,Cordsen:iwooos:1991}.
On the other hand, a set of fine grain components is harder to configure
and to maintain.

We do not propose components to have this or that size, but it is
certainly more difficult to efficiently adapt a large component.
Besides, extremely large components usually result from poor application
domain partitioning during system analysis.


\section{\protect\centering Scenario Adapters}

Several alternatives to achieve scenario-independence have been proposed
by the software engineering community, ranging from simple
implementation constructs, to special languages and complete
methodologies. Nevertheless, as adaptability is considered one of the
``noble'' qualities a software can have, paying a high price for it,
specially in terms of performance, is usually acceptable. This condition
often prevents adaptation techniques from being used in high performance
computing system. Next we describe a low-overhead construct that can be
used to efficiently adapt a scenario-independent abstraction to a given
execution scenario: the \textsc{Scenario Adapter}.

The basic structure of a \textsc{Scenario Adapter} is depicted in figure
\ref{fig:adapter}. At a first glance, it may resemble the
\textsc{Adapter} design pattern from \cite{Gamma:1995}. There are,
however, important differences. While the \textsc{Adapter} pattern
suggests a polymorphic implementation, in which \texttt{Implementor} and
\texttt{Scenario} are only bound at run-time, the \textsc{Scenario
  Adapter} is designed to be bound at compile-time.

\putfig{scenario_adapter}{.99\columnwidth}{!}{fig:adapter}{The
  basic organization of a \textsc{Scenario Adapter}.}
 
The special care for static binding arises from the fact that many
abstractions in a system have a single \texttt{Implementor}. In such
cases, there is no sense in paying the high price of polymorphism.
Moreover, even if an abstraction have several \texttt{Implementors},
many low-level scenarios are mutually exclusive. For example, a well
configured \texttt{thread} should exist either in the
\texttt{multiprocessor} or in the \texttt{uniprocessor} scenario, since
ordinary computers have either one or more than one processors when the
system starts up and this situation is not expected to change during
execution. In some other cases, abstractions refrain from changing
execution scenarios at run-time just for the sake of performance. In
this way, statically binding \textsc{Scenario Adapters} usually does not
imply in flexibility loss and yields quite better performing systems.

Each of the elements in figure \ref{fig:adapter} will now be described
in details. For the sample code, we select the C++ language because it
is the most accepted object-oriented language in the system software
community. Care has been taken to consider language characteristics and
not compiler-dependent aspects. Nevertheless, in order to be sure that
the compiler is not misinterpreting some language definitions or
generating inefficient code\footnote{Some compilers generate code (and
  the respective calls) for empty methods, including constructors.},
some code generation checks may be convenient. We proceeded such checks
for the GNU C++ compiler (egcs-1.1.2) with positive results, i.e.,
\textsc{Scenario Adapters} did not incur in any overhead over
scenario-independent abstractions.


\paragraph{\texttt\textbf{Interface}:}
exports the abstraction functionality and defines what an
\texttt{Implementor} has to implement. Although a true interface
declaration is missing in C++, it is possible to achieve a similar
effect with a tricky class declaration. First, declaring the constructor
\texttt{protected} avoids class instantiation and defining it to be
empty cancels any influence over the realizations. Second, declaring all
member functions to be private forces derived classes to implement them.
The realize relationship between \texttt{Interface} and
\texttt{Implementor} is then implemented via inheritance \footnote{Note
  that implementing a realize relationship via inheritance, for all the
  compilers we had access to, has the undesirable side-effect of
  enlarging the resulting object by the size of an integer. This
  side-effect results from the C++ language definition that two pointers
  to two distinct objects may never have the same value, what is usually
  granted by having the compiler to assign a minimum size of 1 to any
  object. However, as the instantiation of the class \texttt{Interface}
  is prevented by declaring its constructor \texttt{protected}, having a
  zero size assigned to it would never break any language definition.}.

An interface declared in this fashion is not as semantically strong as a
pure abstract base class, since the compiler will not complain if an
\texttt{Implementor} fails to implement an operation until it is
instantiated. Nevertheless, it has the advantage of avoiding virtual
function calls and run-time type information.

Sample code:\\[2ex]
\indent\prog{prg/scenario_adapter.h}{9}{14}

\paragraph{\texttt\textbf{Implementor}:}
realizes the \texttt{Interface} in a sce\-na\-rio-independent fashion.
There may be several \texttt{Implementors} for each \texttt{Interface}.

Sample code:\\[2ex]
\indent\prog{prg/scenario_adapter.h}{16}{20}

\paragraph{\texttt\textbf{Scenario}:}
gathers aspects that are common to all abstractions running in a
scenario. Since all abstractions inherit the \texttt{Scenario} via the
\texttt{Adapter}, it can also be used to decorate abstractions with
scenario specific constructs. This could be useful, for example, to tag
all abstractions with an authentication key in a secure scenario. A
scenario has at least two methods: \texttt{enter} and \texttt{leave}.
They are invoked by the \texttt{Adapter} respectively before and after
each abstraction's method invocation. In the previous example about a
secure scenario, \texttt{enter} would be responsible for authenticating
all operations, while \texttt{leave} would probably be empty.

It is also usual for a \texttt{Scenario} to redefine common system
methods with scenario-optimized versions, so that \texttt{Abstractions}
can transparently access them. For example, a \texttt{Scenario} may
redefine the \texttt{operator new} in order to optimize the memory
allocation according to the current execution conditions.

\noindent Sample code:\\[2ex]
\noindent\prog{prg/scenario_adapter.h}{1}{7}

\paragraph{\texttt\textbf{Adapter}:}
adapts an \texttt{Implementor} to to join a \texttt{Scenario}. It is
implemented as a parametrized class (\texttt{template}) that inherits
from the \texttt{Implementor} given as parameter. It defines all member
functions declared in the \texttt{Interface} in such a way that
\texttt{Implementor} operations are wrapped between the
\texttt{enter}/\texttt{leave} pair. The \texttt{Adapter} is also the
ideal place to carry out operations such as tracing, profiling, and
\emph{Remote Object Invocation}.

Sample code:\\[2ex]
\indent\prog{prg/scenario_adapter.h}{22}{33}

\paragraph{\texttt\textbf{Abstraction}:}
is the construct that will be instantiated by the system clients. We say
that an abstraction has an \texttt{Interface} and one or more
\texttt{Implementors}, and that it is adapted to a \texttt{Scenario} via
an \texttt{Adapter}. It is implemented by the instantiation of the
parametrized \texttt{Adapter} with an \texttt{Implementor}.

Sample code:\\[2ex]
\indent\prog{prg/scenario_adapter.h}{35}{35}


\section{\protect\centering Discussion}

The \textsc{Scenario Adapter} mechanism described above has proved to
produce very efficient code. If the operations in \texttt{Scenario} and
\texttt{Adapter} are declared \texttt{inline}, the result of an
abstraction method invocation will be a direct call to the corresponding
\texttt{Implementor}'s operation surrounded by the \texttt{enter} and
\texttt{leave} primitives. Therefore, not only virtual function calls
are avoided, but function calls at all. Another advantage of these
statically metaprogrammed scenario adapters is that they can easily be
optimized by the compiler: segments of the metaprogram that are not used
are not included in the output.

A collection of \textsc{Scenario Adapters} can be arranged to form a
statically metaprogrammed framework~\cite{Veldhuizen:1995}. Such a
framework would define scenario-independent relationships among
abstractions, letting open ``holes'' where \textsc{Implementors} can be
plugged in.


\section{\protect\centering The \textsc{Epos} System}

The project \textsc{Epos}~\cite{Froehlich:sbac:1999} aims at the
construction of highly adaptable run-time system to support parallel
computing on distributed memory machines, particularly clusters of
workstations. In order to deliver each application a tailored operating
system, \textsc{Epos} takes on \textsc{Pure}~\cite{Schoen:98} building
blocks to implement a set of scenario-independent system abstractions
that can be adapted to a given execution scenario with the aid of
\textsc{Scenario Adapters}. These abstractions are collected in a
repository and are exported to the application programmers via
\textsc{Inflated Interfaces}. This strategy, besides drastically
reducing the number of exported abstractions, enables programmers to
easily express their application's requirements regarding the operating
system.

An application designed and implemented according to the strategy
proposed by \textsc{Epos} can be submitted to a tool that will proceed
syntactical analysis to extract a blueprint for the operating system to
be generated. The blueprint is then refined by dependency analysis
against information about the execution scenario acquired from the user
via visual tools. The outcome of this process is a set of keys that will
support the compilation of an application-oriented operating system.

\textsc{Scenario Adapters} are used in \textsc{Epos} to define a
metaprogrammed framework. System abstractions such as tasks, threads,
communicators, synchronizers, memory, and peripherals have their
interrelations defined in the framework, and can be arranged in systems
for the following scenarios: kernel, library, local, remote, protected,
single/multi-task, single/multi-thread. Several application-oriented
systems have already been generated and evaluated, corroborating the
efficiency of \textsc{Scenario Adapters}.


\section{\protect\centering Conclusion}

In this paper we considered the utilization of com\-po\-nent-based
software engineering for the development of adaptable systems
compromised with performance. The \textsc{Scenario Adapter} construct
was proposed as an effective alternative to achieve this goal. The
adoption of \textsc{Scenario Adapters} in the Project \textsc{Epos}
helped to demonstrate the potentiality of this construct to support
parallel applications running on cluster of workstations. Its use in
deeply embedded system is now being considered.

The successful use of \textsc{Scenario Adapters}, however, is a small
achievement when one considers the amount of problems that are still to
be solved in order to make component-based software engineering a
leading software development strategy.


\nocite{Coplien:1999}

\bibliographystyle{plain}
\bibliography{cluster_computing,operating_systems,communication,processors,parallel_machines,software_engineering,guto}

\end{document}
