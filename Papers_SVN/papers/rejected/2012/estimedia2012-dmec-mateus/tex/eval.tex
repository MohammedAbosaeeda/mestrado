% ------------------------------------------------------------------------------
\section{Practical Experiments} \label{eval}
% + Practical Experiments
% First shows speedup and quality results for DMEC
% using 1 (without partitioning) to 6 worker threads.
% Show that speedup is high and quality is kept acceptable.
% 
% Then, show speedup and quality results for DMEC integrated to JM and compares
%  to the original JM (and, if possible, to other works).
% 

% + JM
% + Dizer como realizamos os experimentos. E/ou quais as variáves observadas:
%   Evaluate the component in isolation DMEC to show its speedup.
%    And how it scales.
%   Evaluate the component in JM to show sppeedup and PSNR.
We have evaluated DMEC in two stages.
First, in order to verify how DMEC's performance scales from 
one to six \emph{Workers} instances, we have evaluated all DMEC implementations 
in a test case.
The test case application mimics the behavior of an H.264 encoder: it provides 
DMEC with pictures, obtain the ME results 
(motion vectors and motion cost), and checks if the results are correct.
Secondly, in order to assess DMEC influence on the final video quality, we have
evaluated all DMEC implementations in the
JM H.264 Reference Encoder~\cite{site:jm}.
The PSNR degradation is computed as the absolute PNSR difference between the
original encoder and the optimized ones.

% P: Dizer pq focamos em luma
For inter macroblock modes in H.264 (i.e. modes related to the ME),
the motion cost for chrominance components derives from the motion cost for 
luminance components~\cite{1101854}. 
Consequently the PSNR for chrominance components derives from the PSNR for 
luminance components. 
For this reason, in this paper we focus on the PSNR variation of the luminance 
component.

Figure \ref{fig:dmec-speedup_workers} show the speedup of DMEC in there
test case application with a different number of \emph{Workers} instances.
For such test, we have used an arbitrary set of pictures with a resolution of
1080p (Full-HD).
The speedup is normalized to one \emph{Worker} instance (speedup of 1X).

% TODO
% \textit{Comments about results in: Cell BE, Muticore IA32, and HW.}
It is worth to mention that for each number of \emph{Worker} instances
a different partition mode was used, according to
Figure \ref{fig:picture_partition}.
For one \emph{Worker} instance we have used the ``Single Partition'',
for two \emph{Worker} instances we have used the ``2x1'' partition and so on,
up to the ``2x3'' partition mode (used for six \emph{Worker} instances).

Besides the additional performance obtained by using a higher number of \emph{Worker} instances,
the partition mode also has influence on the speedup.
The reason of such influence is that, during the partitioning process, the dimensions of the
search window shrinks, thus reducing the area of the picture searched for
similarities.

\fig{.45}{dmec-speedup_workers}{Time performance scalability of DMEC}

% Sobre RD curves
% Figures XXX show the speedup of DMEC while tested already integrated to JM.
% The obtained values are compared to the ones obtained while using the original
% JM, without DMEC.
In order to evaluate in details the behavior of DMEC for distinct values 
of encoding bit-rates, we have used the BD-PSNR (Bjøntegaard Delta PSNR) metric
using the following values of QP (Quantization Parameter): 16,20,24,28; as 
described in \cite{gisle_bjntegaard_calculation_2001}.
It is important to evaluate quality (PSNR) for distinct bit-rates to test 
whether the approach can be used in distinct scenarios of application.
Figure \ref{fig:crowd-bitrate_psnr} shows the rate-distortions (RD) curves using the
original JM encoder and the optimized encoder using DMEC.
The video sequence used for this curves was \texttt{Crowd Run}, a 1080p sequence with  a
high ammount of motion.
Lower values of bit-rate are obtained for higher values of QP since by using
higher values for QP more data is discarded, thus increasing the
compression ratio. 
The two curves very near from each other indicates that the DMEC
presents a good rate-distortion performance for all the evaluated bit-rates.

\fig{.45}{crowd-bitrate_psnr}{RD curve of a 1080p video sequence}

We have evaluated also the speedup obtained in the
% encoding time
ME run time
while using
DMEC for the same QP values we used for BD-PSNR.
Figure \ref{fig:crowd-bitrate_speedup} shows the obtained values while using
6 \emph{Worker} instances.
For Muticore IA32, a speedup of around 9 times is obtained for all
bit-rate values.
For Cell BE this value is about 2 times.
A small speedup for the Cell BE, while compared to Multicore IA32 and the dedicated hardware, is due
to the memory transferences (picture samples and ME results) which is performed using
the DMA requisitions of Cell BE.


\fig{.45}{crowd-bitrate_speedup}{Speedup vs bit-rate of a 1080p sequence}
%
% \multfigtwov{.65}{bd_psnr}{bd_speedup} {bd} {RD curve (a) and speedup vs bit-rate (b) of 1080p sequence}

% Discussion


% Falar do paralelismo / particionamento de dados
% Qualidade ficou boa mesmo particionando e desempenho aumentou: speedup ~ 70%
%The strategy of ME distribution based on picture partitioning has been shown 
%effective.
% We have obtained a speedup higher than XXX\% without loosing quality.
%Data partitioning is effective because the visual interdependence between
%partitions is not significant to influence on the encoding quality, and allows
%for a speedup because it enables the simultaneous processing of each picture
%partition.

% - Falar da comunicação
% - Necessidade via espaço de endereçamentos diferentes
% A arquitetura Cell Broad Band demonstrou-se uma arquitetura interessante para o
% processamento paralelo de vídeo, pois possui unidades funcionais dedicadas 
% (i.e. SPEs) para processamento de dados. 
% A principal dificuldade encontrada em trabalhar-se com o Cell foi a capacidade 
% limitada da memória local das SPEs.
% Outra dificuldade foi lidar com as transferências de memória entre SPE e PPE. 
% Isto em fato foi superado pelas estratégias que desenvolvemos de baferização e 
% também com a utilização do Element Interconect Bus (EIB) do Cell que realiza 
% DMAs com altas taxas de transferências.
% 
% - Solução 1: Buffer de preditores, contribuiu bastante
% A estratégia de armazenamento de preditores nas SPEs foi significativa no 
% aumento do desempenho, pois vetores de movimentos necessários para o cálculo da
% ME não precisam ser consultados na memória principal. É coerente a decisão de 
% manter uma cópia local destes vetores, pois todos os vetores que a ME irá 
% precisar foram calculados pela partição em questão e por nenhuma outra.


% ------------------------------------------------------------------------------
