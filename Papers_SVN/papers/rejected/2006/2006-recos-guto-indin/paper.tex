%%%%%%%%%%%%%%%%%%%%%%%%%%%%%%%%%%%%%%%%%%%%%%%%%%%%%%%%%%%%%%%%%%%%%%%%%%%%%%%
\documentclass[a4paper,times,10pt,twocolumn]{article} 

\usepackage[latin1]{inputenc}
\usepackage[english]{babel}
\usepackage[T1]{fontenc}

\usepackage{graphicx}
\usepackage{url}

\usepackage{latex8}
\usepackage{times}
\pagestyle{empty}

\newcommand{\fig}[3][htbp]{
  \begin{figure}[#1] {\centering\scalebox{1}{\includegraphics{fig/#2}}\par}
    \caption{#3\label{fig:#2}}
  \end{figure}
}

\title{On Operating Systems for Reconfigurable Computing}


\author{
  %Ant�nio Augusto Fr�hlich and Wolfgang Schr�der-Preikschat\\
  Ant�nio Augusto Fr�hlich\\
  Laboratory for Software and Hardware Integration\\
  Federal University of Santa Catarina\\
  PO Box 476 -- 88049-900 -- Florian�polis, SC, Brazil\\
  guto@lisha.ufsc.br\\
%  {\texttt{http://www.lisha.ufsc.br/}}
} 

%\author{Antonio Augusto Frohlich and Wolfgang Schr�der-Preikschat}
%\author{Antonio Augusto Frohlich}

\date{}

\begin{document}

\maketitle

\begin{abstract}

  Dynamically reconfigurable hardware architectures are the substratum
  of \textbf{reconfigurable computing}. As such architectures allow for
  the reprogramming of some hardware building blocks while others
  continue to operate normally, they bring about much of the flexibility
  usually associated with software also to the hardware realm.
  Nonetheless, in order to achieve such flexibility: a complete
  hardware/software infrastructure is necessary.

  In this essay, we discuss several issues concerning operating systems
  for reconfigurable computing, aiming at identifying concepts and
  mechanisms of traditional operating systems that can be reused or
  adapted to this novel context, and also proposing answers for some
  questions that cannot be satisfactorily answered based on current
  technology. 


\paragraph{Keywords:} reconfigurable computing, hardware-software
co-design, morphware, operating systems.

\end{abstract}

%%%%%%%%%%%%%%%%%%%%%%%%%%%%%%%%%%%%%%%%%%%%%%%%%%%%%%%%%%%%%%%%%%%%%%%%%%%%%%%
\section{Introduction\label{sec:intro}}

% Make the case for reconfigurable computing
The forthcoming of dynamically reconfigurable hardware architectures is
motivating several research projects and is coining a new term:
\textbf{reconfigurable computing}. In currently available
Field-Pro\-gramma\-ble Gate Arrays (FPGAs), elementary functional units
physically implemented in hardware can be grouped together, or
programmed, to shape more sophisticated components, whose purpose is
dictated by particular system needs. Some FPGAs allow for partial
reprogramming of specific blocks, while others retain their original
program, thus leading to the notion of "reconfigurable computing", in
which the computer as a whole, software and hardware, can be dynamically
adapted according to varying application requirements during its
operation time~\cite{Compton:2002, Hartenstein:2004, Morphware:2004}.
More recent architectures, such as those based Magnetoresistive Random
Access Memory (MRAM), forecast an even broader scenario for
reconfiguration as they allow for dynamic modification in the wiring of
elementary functional units, taking reconfiguration to the level of
individual gates~\cite{Koch:2005}.

% State the problem: operating system support
% - Application-driven and Instruction-level are just parts of the problem
% - OS-level abstractions can improve on it: well defined building-blocks for ReComp
The main benefit associated to reconfigurable computing techniques is
the possibility of reusing hardware and software components for multiple
purposes, eliminating undesirable resource replications and allowing the
system to cope with requirements that were not initially taken in
consideration. Eliminating replicated components directly improves
metrics such as size and power consumption, while increasing reusability
and flexibility directly affects non-recurring engineering costs.
Nonetheless, in order to achieve these benefits, it is not enough that
designers base their projects on FPGAs that support partial
reconfiguration: a complete infrastructure, at both sides software and
hardware, is necessary.

% - It went wrong once (Apertos), will it work now?
A dynamic reconfiguration support system able to identify "which",
"when" and "how" hardware and software components must be reconfigured
in order to adapt a computing system to particular application demands
is still far away. Indeed, some researchers predict that the
computational cost of such infrastructure is more likely to overwhelm
the benefits associated with the technology. This prediction is perhaps
motivated by the history of dynamically reconfigurable operating systems
of the 80s and 90s, like \textsc{Apertos}~\cite{Yokote:1992} and
\textsc{Ethos}~\cite{Szyperski:1992}, whose reconfiguration
infrastructure incurred in very high run-time overhead, preventing them
from reaching the market despite all claimed advances.  Notwithstanding,
the actual scenario for reconfigurable computing is more hardware-bound
then that of the 90s and is bringing about new opportunities that must
be investigated from a more contemporary perspective.

% State the solution: reuse previous knowledge
% - Basic infrastructure for reconfiguration
% - Plus services that resemble resource management
%   - Traditional OS resources: CPU time, memory, disk, etc
%   - ReComp resources: traditional plus chip area, energy, performance
% - What we need is a component architecture capable of embracing all this
In this essay, we discuss several issues concerning operating systems
for reconfigurable computing, aiming at identifying concepts and
mechanisms of traditional operating systems that can be reused or
adapted to this novel context, and also proposing answers for some
questions that cannot be satisfactorily answered based on current
technology. First, the three main scenarios for reconfigurable computing
are introduced: application-driven, instruction-level, and OS-supported.
Subsequently, two reasonings are presented: how an operating system for
reconfigurable computing could look like, and how a component
architecture for reconfigurable computing could be organized. Finally,
the perspectives for the upcoming of a real operating system for
reconfigurable computing are discussed.


%%%%%%%%%%%%%%%%%%%%%%%%%%%%%%%%%%%%%%%%%%%%%%%%%%%%%%%%%%%%%%%%%%%%%%%%%%%%%%%
\section{Reconfigurable Computing\label{sec:recomp}}

% Application-driven reconfigurable computing
The inherent flexibility of reconfigurable computing has the
potentiality to sustain a new generation of electronic devices that are
able to self-modify themselves according to user's needs. A cell phone,
for instance, could be reconfigured to perform the functions of a PDA,
MP3 player, digital camera, navigation system, game pad, among others.
Note that this is a scenario completely different from nowadays'
multipurpose gadgets, which require specific circuitry to be integrated
for each function they are supposed to perform. A reconfigurable cell
phone would instead reprogram its hardware and software building blocks
on-the-fly, just as the user selects a different function. If the gadget
were to perform as an MP3 player, reprogramming would give rise to a
specific audio decoder. While performing a game pad, reprogramming would
probably build a kind of graphics processing unit (GPU). In this
example, reconfiguration is directly controlled by applications, which
explicitly initiate the reprogramming of building blocks as
needed~\cite{Agarwal:1999}. Therefore, we call this kind of
reconfiguration \emph{application-driven}.

% Instruction-level reconfigurable computing
Another reconfigurable computing scenario that has been consistently
explored is the implicitly reconfiguration of hardware components
without direct intervention by applications. This perspective of
reconfigurable computing is based on the constant monitoring of hardware
operational conditions, so as to initiate a pre-programmed
reconfiguration whenever the associated conditions are observed. A good
example of this kind of reconfiguration is a processor that is able to
instantiate additional functional units as it detects overload
situations. For instance, an application that runs into a heavy
floating-point operations cycle would induce the processor to
instantiate an additional floating-point unit (FPU) along with the
structures needed to operate both units in parallel~\cite{Taylor:2004}.
We will refer to this kind of reconfiguration as \emph{instruction-level}.

% OS-supported reconfigurable computing
A third scenario for reconfiguration lays between the previous two, at
the \emph{operating system} level. At the one hand, application-driven
reconfigurations are only possible for very specific applications,
designed and implemented for an specific platform, for it presupposes
the application itself must know how to reconfigure the system.
Instruction-level, on the other hand, usually limits reconfiguration to
small functional units, missing the notion of macro-units such as
complete subsystems. A properly designed operating system could bring
some of the reconfiguration autonomy of instruction-level to the
granularity of application-driven reconfiguration, thus yielding a more
effective scenario for reconfigurable computing. The key issues in that
design will be discussed next.


%%%%%%%%%%%%%%%%%%%%%%%%%%%%%%%%%%%%%%%%%%%%%%%%%%%%%%%%%%%%%%%%%%%%%%%%%%%%%%%
\section{Operating Systems for Reconfigurable Computing\label{sec:os}}

% What can drive reconfigurations:
% - Functional requirements (e.g. advanced algorithms)
% - Non-functional requirements (e.g. performance, energy consumption, real-time)
% - Parallel -> determinism ???
An operating system designed to support reconfigurable computing will
have to handle the dynamic reconfiguration of software and hardware
components on-demand. Reconfiguration itself can be driven either by
functional or non-functional requirements of specific applications. For
instance, if an application needs a certain functionality that is not
yet available in the system, reconfiguration procedures will be carried
out in order to add software and hardware building blocks to the system
that together will deliver the required functionality. Independently of
how long it takes to reconfigure the system, or how much energy the
reconfiguration process consumes, reconfiguration driven by
\emph{functional requirements} must be done, otherwise the application
will not be able to run.

Besides specific functional requirements, \emph{non-functional
  requirements} can also drive reconfigurations and are often not
restricted to individual applications. For instance, performance, energy
consumption and real-time operation are requirements that must be met
for the system as a whole and usually drive reconfigurations that affect
several applications. In order to improve the performance of a system as
a whole, some critical operations, or even complete algorithms, could be
migrated to hardware, where parallelism can be more promptly explored.
In order to save energy, some hardware components could be reconfigured
to operate at lower clock rates, or perhaps could be replaced by more
elementary ones. Real-time could be improved, for instance, by the
replacement of sequential data structures in software, such as process
lists, by associative memory in hardware, thus increasing determinism
and reducing operational jitter.

% Demystify ReCOS: resources are resources
% - Reasoning about things that can be reused/adapted from todays OOSS
% - Resource management for ReComp: functionality, area, energy, performance
In principle, the scenario for an operating system that aims at
supporting reconfigurable computing might seem very different from that
of ordinary operating systems, with functional and non-functional
requirements changing along with the system execution and igniting
complex reconfiguration procedures.  Nonetheless, even ordinary
operating systems usually implement sophisticated resource management
strategies: CPU time, memory, disk, network and a variety of resources
are consistently shared among processes and users. If we assume
functional units, time, energy, and silicon area to be resources just
like memory and disk area, then it should be possible to deploy many of
the concepts, mechanisms, and algorithms of traditional operating
systems in the realm of reconfigurable computing.

% - Static model: functionality, execution-time, energy, area
% - Dynamic model: event counters(not only for energy management, but
% also for garbage collection of components no longer in use)
In addition to an interface, which would be accounted for the functional
properties, components could also be tagged, during fabrication, with
execution overhead, energy consumption and silicon area estimates.
Combined, these informations could sustain some old, yet effective,
resource management strategies. For instance, if the system becomes
overloaded, reconfiguration could be started in order to replace active
components by others with compatible interfaces, but better performance.
If this leads the system to a higher energy consumption configuration,
the previous configuration could be restored as soon as the load is set
back to normality. Estimates defined at development-time, however,
usually fail to remain meaningful as the system undergo severe
operational conditions.

Event counters, currently used to estimate energy
consumption~\cite{Bellosa:2000}, can be used to complement the static
management model. Knowing how intensively a functional unit has been
used in a given time interval can provide the resource manager with
valuable information about possible reconfiguration targets. For
instance, if an active component is being used only sporadically, and
there is another functionally equivalent component that consumes less
resources (e.g. silicon area), then it might be the case to replace it.
The same counters could be used to trigger garbage collection
operations, removing components that have not been used for a given
time.

% Infrastructure for ReComp: replugging of components
Although many technological difficulties are still to be overcame in
order to make reconfigurable computing really attractive, from the
operating system perspective, there is much knowhow that can be promptly
reused. Indeed, we believe that the biggest challenge for an operating
system that aims at supporting reconfigurable computing is to provide a
dynamically repluggable software-hardware component infrastructure. On
top of that infrastructure, a fully fledged operating system could be
build mostly on currently available technology.


%%%%%%%%%%%%%%%%%%%%%%%%%%%%%%%%%%%%%%%%%%%%%%%%%%%%%%%%%%%%%%%%%%%%%%%%%%%%%%%
\section{A Software-Hardware Architecture for Reconfigurable Computing\label{sec:arch}}

As stated in the previous section, an operating system for
reconfigurable computing seems to depend mostly on an adequate
software-hardware component infrastructure. At first sight, such an
infrastructure could actually be built on available mechanisms: from the
hardware side, strategies to map components to defined FPGA blocks, thus
enabling them to be reconfigured without interfering in other
components; from the software side, mechanisms such as dynamically
loadable modules and shared libraries, which give similar answers.
Nonetheless, reconfigurable computing presupposes not only loading and
unloading components---what on its own can be a complex task when both
hardware and software elements are involved, but also replacing them
during system operation.  Issues such as state preservation,
compatibility of interfaces across software and hardware domains,
limited resources for the infrastructure itself must be addressed while
defining a \emph{software-hardware architecture for reconfigurable
  computing}.

\subsection{Components\label{sec:components}}

% Close the focus
The discussion about what components are---and what they are not---can
be tracked to the early days of computing. It is not the goal of this
essay to get into that more philosophical discussion, but to identify
aspects of components that are relevant to the development of an
operating system for reconfigurable computing.

% Granularity of components: AOSD
Perhaps the most significant aspect of a reconfigurable computing
component is the relationship to its clients; that is, its interface. If
the component architecture enables the system to track the use of
elementary hardware components up to the application program, then the
universe for reconfiguration will be much broader then otherwise.  By
exporting high-level abstractions, the system can more easily keep track
of ``what'' applications are doing at any time, thus enabling the
reconfiguration of larger blocks. Consider a system configured to
support a music playing application, for instance. Initially, the system
might have been configured to include a custom decoder in hardware,
although software and DSP versions of the same component were also
available. Now suppose the application must go on-line through a radio
network that demands complex signal filtering. It would be impossible
for a system to replace the custom decoder by the DSP-based one---thus
giving rise to the infrastructure necessary to do signal filtering---had
it missed the information that the application was using a hardware
component (i.e.  the decoder) for an specific purpose (i.e. decode an
audio stream). Nevertheless, reconfiguration would be straightforward if
the system exported a \texttt{decoder} abstraction.

Similarly, replacing a software by hardware is usually far more
effective in the context of high-level components with well-defined
interfaces. If the application in the example above were executing the
decoder software without assistance of the operating system, it would be
extremely hard for the reconfiguration infrastructure to detect that the
series of repetitive operations overloading the CPU were indeed related
to an abstraction that is available in hardware, and thus trigger a
reconfiguration. Furthermore, a redirection mechanism would have to be
build around the application to intercept function calls and suppress
them in detriment of hardware-equivalent operations.

% Pre-compiled software and hardware components
% Automatic partitioning in languages such as SystemC are yet far away
Another important issue regarding software-hardware components concerns
implementation languages and tools. In principle, describing
software-hardware components in a single language such as
\textsc{SystemC}~\cite{Panda:2001} would be ideal, since partitioning of
software and hardware could be done, in theory, at any point. In
practice, those languages are yet in their early days and manual
intervention is often need in order to enable the translation of
high-level constructs, thus supporting synthesis~\cite{Schulz-Key:2004,
  Seppo:2001}.  An alternative is to have software components
implemented in mature programming languages and hardware components
implemented in mature hardware description languages. A meta description
of such components can latter reconcile them in the sense of the
component architecture: clients do not get to know whether to component
is software, hardware, or a combination of both~\cite{Azevedo:2005,
  Tondello:AICCSA:2005}.

\subsection{Component Replugging\label{sec:replug}}

% Component plugs and component outlets
The primordial condition for component replugging are well-defined
interfaces that are able to sustain some sort of plug/outlet
relationship between components. The main difficulty of achieving that
for reconfigurable computing is that sometimes the outlet is software
and the plug is hardware (or vice-versa). From the software perspective,
an interface is defined by a set of method signatures and, sometimes,
behavioral constrains. From the hardware perspective, an interface is
defined by a set of signals (or wires) and their operational conditions
(including timing). Plugging such distinct artifacts together demands
some sort of ``adapter''.

Hardware mediators, defined in the context of Application-Driven System
Design~\cite{Froehlich:2001}, solve the problem of hardware-software
plugging by deploying static metaprogramming techniques to wrap the
hardware with a software compatible interface, thus giving origin to a
\emph{virtual software/hardware interface}. This interface has already
been consistently discussed by the authors~\cite{Polpeta:ETFA:2005}.

% Component state save/restore
Another major issue pertaining the replugging of components is state
maintenance. When a component is replaced, its successor must have its
state initialized accordingly. Saving the state of the old component
simply by copying the memory region that contain its attributes
(software) or dumping its registers (hardware) will probably not be
enough to support the initialization of the new one, since the internal
representation of information is particular to each component. State
saving and restoring must some how be done through the interface of
components, which are the only sustainable compromise among them. An
interface method such as \texttt{shutdown\_and\_give\_me\_your\_state}
is a straightforward beginning, but the returned state will sometimes
have to be reinterpreted before being used to initialize the new
component. This reinterpretation can be carried out by components
themselves of by some sort of ``component broker''. Anyway, if
components are not fruit of a consistent hardware/software co-design
methodology, state incompatibility might render reconfiguration
impractical.

% Component mapping to silicon
Besides proper interfaces, adapters, and mechanisms to save and restore
state, component replugging is only feasible if the reconfiguration
infrastructure is able to track logical components down to
reprogrammable hardware blocks. If the silicon area formerly used by a
component that has been shutdown is to be used by a new one, then
precise information about that area (i.e. which cells/slices were
occupied by the component) must be kept at run-time. Furthermore, the
hardware fabric must be so that reprogramming some blocks does not
disturb the operation of others. This issue has been extensively
addressed by the hardware community~\cite{Dutt:1999, Noguera:2004,
  Huebner:2004, Robertson:2004, Colavin:2003} and could be regarded by
the operating system mostly like a resource allocation problem similar
to keeping track of allocated blocks in a disk or frames in a memory,
just that some blocks hare particular characteristics (e.g. \emph{block
  RAM}).

% Introduce extension, contraction and replacement
A component architecture build around these premises should be able to
support the tree basic system reconfiguration operations: extension,
contraction, and replacement.

\subsubsection{Extension\label{sec:replug-ext}}

% Demand loading
System extension happens when a new component must be added to it in
order to fulfill the current set of requirements defined by
applications, be it functional or not. In essence, extension can be done
based on \emph{demand loading} strategies adopted in traditional
operating systems to dynamically load large libraries: objects are first
loaded when they get used. Typical strategies consist in registering
component interfaces within the system, but leaving their
implementations unbound. A first attempt to access the component causes
a fault that triggers the loading and binding of the missing
component~\cite{}. A similar strategy could be used to dynamically load
software and hardware components into the system.

\subsubsection{Contraction\label{sec:replug-cont}}

% Garbage collection
System contraction happens when a component that is no longer in use
gets removed in order to free resources. The main motivation to system
contraction from the perspective of software is to increase the amount
of available resources for eventual extensions.  From a more generic
perspective, however, non-renewable resources must also be taken into
consideration. For instance, keeping an unused component operational
will very likely consume energy that cannot be recovered. In this sense,
tracking information about which components are actually in use becomes
crucial for a reconfigurable system.

Contraction can be directly started by applications when they invoke
\texttt{destructor} or \texttt{finalize} methods. Additionally, event
counters traditionally used for energy management~\cite{Bellosa:2000}
can be deployed to identify components that have not been used for
``long'' periods of time. Defining ``long'' becomes the issue here, for
there is no guarantee that a component that has not been used for some
time will never be used again. Contracting the system to short latter
expand it back to the same configuration is certainly something to be
avoided.  Fortunately, once more there is extensive research in the
context of cache and main memory management to be reused. The concept of
\emph{workset} and the related approximation algorithms are good
candidates to drive the \emph{garbage collection} of system
components~\cite{Silberschatz:1998}.

\subsubsection{Replacement\label{sec:replug-rep}}

% Context save, contraction, extension
Component replacement becomes straightforward if the architecture
presents an appropriate hardware/ software interface and the
reconfiguration infrastructure supports state saving, area mapping,
extension, and contraction mechanisms. Perhaps the most obscure issue
regarding component replacement today is the management of the knowledge
needed to decide which components must be replaced, which components
must replace them, and when replacing must take place. Similar issues
pervade the reflective systems literature and can be regarded as a
research field on its own.


\subsection{Component Repository\label{sec:repository}}

% Components usually cannot be pre-loaed into RAM, for u-controllers
% simply don't have enough of that
Dynamic component replugging presuppose the existence of previously
validated components. At load-time, such components usually take the
form of binary files that, along with formally described interfaces,
build a repository. In non-embedded systems, component repositories are
normally stored in a file system or data bank. This, however, is mostly
inadequate for embedded systems, since most of them do not feature a
mass storage unit. Pre-loading all component into main memory is usually
not possible too, for this is often a limited resource.

%\subsubsection{Local\label{sec:local}}

% Flash with a repository of pre-compiled software and hardware
% components
For stand-alone embedded systems, devices such as flash memory are
currently the most viable alternative to store the component repository.
During the building of the system, all components that are plausible to
be used are loaded into the flash, along with the operating system
\emph{nucleus}\footnote{The term ``kernel'' was avoided here not to
  induce the reader to assume that a reconfigurable computing
  infrastructure will be similar in size to a traditional operating
  system kernel.  Based on previous
  experiments~\cite{Polpeta:ETFA:2005}, we believe that it can be rather
  small.} that implements the basic reconfiguration infrastructure.

%\subsubsection{Remote\label{sec:remote}}

% Dynamic fetch over the net
For connected systems, downloading components from a remote repository
might also be an alternative. Depending on the coupling level, even some
of the reconfiguration infrastructure can be moved out of the embedded
system into a server. This could, for instance, be the case for a
machine or an automobile. Reconfiguration from a remote repository
requires additional communication mechanisms and main memory (to store
components during download), but brings along a powerful \emph{upgrade}
mechanism almost for free: by replacing components in the server's
repository and triggering a reconfiguration, the system gets upgraded
with having to be stopped.


%%%%%%%%%%%%%%%%%%%%%%%%%%%%%%%%%%%%%%%%%%%%%%%%%%%%%%%%%%%%%%%%%%%%%%%%%%%%%%%
\section{Perspectives\label{sec:conclusion}}

% A lot of hardware, but few software.
The recent achievements of the hardware community with regard to
reconfigurable computing are very promising.  However, the same is not
yet true for the software by-side . At the one hand, it is natural that
researchers in the field were first challenged by the hardware aspects
of reconfigurable computing, but, at the other hand, the absence of
specific operating systems and even of adequate hardware/software
component architectures might be a signal that the software aspects of
reconfigurable computing are being neglected or that the scientific
community intends, one more time, to deploy the ``commodity'' principle
and adopt an all-purpose operating system for the first generations of
reconfigurable computers.

% PCA has a software architecture, but seems to distort facts to make it cool
We believe that the consolidation of reconfigurable computing depends,
among other things, of a proper component architecture to sustain the
design of an reconfigurable system as a whole.  The Polymorphous
Computing Architecture project defines a component
architecture~\cite{Richards:2003}, but misses a hardware/software
co-design strategy. Moreover, many concepts seems to be closer to
nowadays web-services than to real embedded systems.

% Much can be built on existing technology
Nonetheless, we believe that a component architecture for reconfigurable
computing can be built mostly on existing concepts. The most significant
of them have been identified in this essay and pertain to an
infrastructure for component replugging, an strategy for resource
management, and a component architecture. We believe that the
combination of these elements with the existing hardware-based
reconfiguration engines could yield a more application-oriented
perspective for reconfigurable computing systems.

%%%%%%%%%%%%%%%%%%%%%%%%%%%%%%%%%%%%%%%%%%%%%%%%%%%%%%%%%%%%%%%%%%%%%%%%%%%%%%%
\bibliographystyle{latex8}
\bibliography{recos,guto,os}

\end{document}

%%% Local Variables: 
%%% mode: latex
%%% TeX-master: t
%%% End: 
