\section{Sensores e aquisição de dados}
\label{sec:sensores}

%introdução
Um sensor é um dispositivo que responde a estímulos físicos (e.g., luz,
temperatura, pressão, campo magnético ou movimento) e transmite um impulso
resultante. Um sensor normalmente interage com um sistema digital, provendo
informações sobre o mundo analógico. Essa informação pode ser fornecida através
de um sinal analógico, que é convertido para valores digitais através de um
conversor analógico-digital externo ao sensor, ou através de uma interface
digital que converte internamente os sinais analógicos. Interfaces de dados de
sensores podem variar amplamente mesmo dentro da mesma classe de
sensores~\cite{Wanner:MSC:2006}.

%tipos de sensores
\subsection{Tipos de sensores}

Devido ao pequeno porte e às suas limitações em termos de disponibilidade de
energia, módulos de \rssf normalmente empregam sensores eletrônicos ou óticos
com relativa eficiência energética. É o caso de, por exemplo, termistores para
medir temperatura, foto-didos para medir luminosidade ou acelerômetros
para medir deslocamento.

De maneira a possibilitar um melhor entendimento dos desafios envolvidos no
projeto e implementação de um sistema de aquisição de dados de sensores, esta
seção apresenta alguns dispositivos sensores usados em nodos de sensor
contemporâneos. Esta não é uma lista exaustiva, mas deve prover um caso geral de
uso de dispositivos sensores em \rssf.

\subsubsection{Termistores}

Um termistor é um resistor cuja resistência varia com mudanças de temperatura. A
equação de Steinhart-Hart é uma aproximação de terceira ordem amplamente
utilizada para determinar a curva de resposta de um termistor:
\begin{equation}
\label{eq:steinhart_hart}
T = \frac{1}{a + b.ln{R_t} + c.(ln{R_t})^3}
\end{equation}
onde $a$, $b$, e $c$ são parâmetros Steinhart-Hart específicos para cada
termistor, $T$ é a temperatura Kelvin, e $R_t$ é a resistência em Ohms
apresentada pelo termistor na temperatura atual. Um termistor normalmente é
ligado a um conversor analógico-digital através de um circuito divisor de tensão
simples. A estimativa de temperatura baseada em leituras do ADC pode depender do
cálculo em tempo de execução de funções de aproximação complexas (e.g., a
equação Steinhart-Hart), ou pode fazer uso de tabelas de conversão previamente
calculadas. Termistores diferentes podem ter constantes de tempo e precisão
diferentes, bem como diferentes constantes de dissipação de energia.

\subsubsection{Sensores digitais de temperatura}

A família SHT1x~\cite{Sensirion:SHT1x:2005} de sensores de umidade e temperatura provê
um exemplo de sensores digitais de temperatura usados em \rssf. O sensor é
fabricado pela Sensirion, e provê leituras digitais calibradas por uma interface
SPI de dois fios. Um microcontrolador pode e ler dados do sensor enviando um
comando específico solicitando uma medida temperatura para o sensor pela
interface SPI. Ao completar da leitura, o sensor envia um sinal de dados
prontos. O microcontrolador pode, então, ler os dados acompanhados de um código
CRC para validação. Depois dessa transmissão, o sensor entra em modo inativo. Os
coeficientes de calibração são programados na memória interna do sensor, e são
usados internamente durante as leituras para calibrar os sinais lidos. Dados
sensoriais providos pelo SHT1x podem ser convertidos para valores de temperatura
através de uma e função linear. Um registrador de status provê uma interface de
detecção de baixa tensão, configura a resolução da leitura (por exemplo, 8 bits,
16 bits) e controla um aquecedor interno.

\subsubsection{Foto-resistores e foto-diodos}

Um foto-resistor é um componente eletrônico cuja resistência diminui com o
aumento da intensidade de luz incidente. A maioria dos foto-resistores são
implementados através de células de sulfeto de cádmio, que usam a habilidade
desse material de variar sua resistência (por exemplo, apresentando $2k\Omega$
em condições de baixa luminosidade e $500\Omega$ quando exposto à luz). Essas
células também são capazes de reagir a uma ampla gama de frequências, incluindo
infravermelho, luz visível ultravioleta. Como no caso dos termistores,
foto-resistores normalmente fazem interface com um conversor analógico-digital
através de um circuito divisor de tensão.

Um foto-diodo é um semicondutor que responde a estímulo óptico. Foto-diodos
operam pela absorção de fótons que geram uma variação na corrente que flui
através deles. Um circuito RC (resistor-capacitor) auxiliar tem sua fonte de
corrente afetada por este sensor, o que permite a detecção da presença ou
ausência de quantidades diminutas de luz através de medidas de tensão no
resistor desse circuito auxiliar.

\subsubsection{Sensores digitais de luminosidade}

O sensor TSL2550~\cite{TAOS:TSL:2005}, fabricado pela TAOS, provê um exemplo de
sensor digital de luminosidade usado em \rssf. Ele combina dois foto-diodos e um
conversor analógico-digital de 12 bits num circuito integrado para prover
medidas de luz com uma sensitividade parecida com a do olho humano. Um dos
foto-diodos é sensível a luz visível e infravermelha, enquanto o segundo
foto-diodo é sensível primariamente a luz infravermelha.

Dados de sensoriamento são lidos através de uma interface de dois fios SMBus. Um
registrador de controle gere o dispositivo, e dois registradores de ADC
armazenam os dados de para leitura. As saídas dos dois canais ADC podem ser
usadas em uma função linear para se obter um valor que aproxima a resposta do
olho humano na unidade comumente utilizada de Lux.

\subsubsection{Magnetômetros}

Magnetômetros são sensores capazes de medir força e/ou direção de campos
magnéticos. Magnetômetros podem ser divididos em magnetômetros escalares, que
medem a força total do campo magnético ao qual eles são expostos, e
magnetômetros de vetor, que têm a capacidade de medir a componente do campo
magnético em determinada direção. Um conjunto de magnetômetros de vetor podem
ser combinados para permitir a definição de força, declinação e inclinação de um
campo magnético. A família Honeywell HMC100x~\cite{Honeywell:HMC:2005} de sensores
magnetoresistivos são dispositivos de pontes resistivas simples que requerem
apenas uma fonte de tensão para medir campos a magnéticos. Esses dispositivos
são capazes de medir qualquer campo magnético ambiente ou aplicado em um eixo
sensível. A tensão de saída do sensor é linear em relação ao campo magnético
aplicado.

\subsubsection{Acelerômetros}

Acelerômetros são usados para medir mudanças na velocidade. Na sua forma mais
simples, um acelerômetro é composto por uma massa suspensa e um dispositivo
sensível à deflexão. A família Analog Devices
ADXL345BCCZ-RL7~\cite{Analog:ADXL345:2009} provê um exemplo de acelerômetros com
2 eixos e baixo consumo de energia. O sensor provê tanto saídas analógicas
quanto digitais. O valor de saída do sensor é linear em relação à aceleração
aplicada, e calibração para baixas gravidades podem usar o campo gravitacional
da Terra como referência.

%aquisição de dados
\subsection{Calibração de sensores}

Dados lidos diretamente de sensores precisam ser calibrados para garantir um
determinado nível de qualidade. Imprecisões nas medições podem surgir por
diversas razões~\cite{Albertazzi:2008}:
\begin{itemize}
    \item \textbf{Definição do mensurando:} diz respeito ao conhecimento
    disponível acerca da grandeza física que é objeto de estudo.
    \item \textbf{Procedimento de medição:}  diz respeito aos procedimentos
    utilizados para realizar a medição. Neste caso se aplica, por exemplo, a
    precisão dos modelos matemáticos sendo utilizados para a medição de uma
    determinada grandeza.
    \item \textbf{Condições ambientais:} diz respeito a condições que podem
    interferir no resultado de uma medição, como por exemplo, interferência
    eletromagnética ou de pressão atmosférica sobre o sistema de medição.
    \item \textbf{Sistema de medição:} diz respeito a interferência gerada no
    resultado de uma medição pelos equipamentos utilizados no processo. Esta
    interferência pode vir, por exemplo, da incerteza do sensor utilizado ou
    erros de arredondamento na conversão analógico-digital.
    \item \textbf{Operador:} diz respeito a erros que possam ser ocasionados
    pela ação humana quando o procedimento de medição depende da ação de um
    operador. Normalmente, este fator tem pouco efeito em aplicações de \rssf,
    já que estas aplicações tendem a operar de modo autônomo.
\end{itemize}

Estes fatores agem sobre o sistema formado pelo mensurando (objeto de estudo) e
pelo sistema de medição que, no caso das \rssf, é formado pelo sensor escolhido,
pelo conversor analógico-digital ou pelo microcontrolador no caso de um sensor
digital. O que resulta destas interferências é uma variação no valor indicado.
Para tornar as leituras mais confiáveis é necessário, de algum modo, compensar o
efeito destas interferências.

As interferências geradas por estes agentes no processo de medição podem ser
agrupados de modo a formar duas variáveis de erro: o erro sistemático e o erro
aleatório. O erro sistemático é aquela parcela do erro que se mantém constante
entre várias medições. Para permitir uma correção das medidas, é possível
definir uma estimativa do erro sistemático chamada de \emph{Tendência}. A
\emph{Tendência} pode ser determinada através de medições sucessivas de um
mensurando com valores conhecidos. Por exemplo, para calibrar um sensor de
temperatura, pode-se realizar leituras sucessivas deste sensor em um ambiente
com temperatura constante conhecida. A \emph{Tendência} é dada, neste caso, pela
diferença entre a média das leituras realizadas e o valor do mensurando. Este
procedimento é conhecido calibração do sensor.

Não há meio de se estimar o valor exato do erro aleatório, mas é possível,
através de tratamento estatísticos de uma série de amostras realizadas sobre um
mensurando com valores conhecidos, determinar a repetitividade do sistema de
medição e, consequentemente, seu erro máximo. Do ponto de vista de \rssf é
importante analisar o erro máximo do sistema definido para verificar sua
adequação à aplicação que se pretende desenvolver. Outra característica
importante do erro aleatório é que a soma dos erros aleatórios de medidas
sucessivas de um mesmo mensurando tende a se anular no infinito. Logo, a média
de sucessivas medições pode diminuir o efeito deste erro, gerando resultados
mais confiáveis.

%modelos de interfaceamento destes sensores com processadores
% \subsection{Interfaces de senores}
% interfaces analógicas e digitais

%fusão de dados
%\subsection{Fusão de dados}

\subsection{Sensores no \emote}

Como apresentado na Seção~\ref{sec:emote_arch}, o \emote apresenta uma interface
padrão para um módulo de entrada e saída que deve ser utilizada para conectar os
sensores a serem utilizados pela aplicação em questão. No atual estágio de
desenvolvimento, o \emote conta com a \emph{startup board}, um módulo de entrada
e saída que, além de componentes como interface USB, botões e LEDs, conta com um
sensor de temperatura do tipo termistor, modelo
ERT-J1VG103FA~\cite{Panasonic:ERTJ:2004}, e um acelerômetro digital de 3 eixos,
modelo ADXL345BCCZ-RL7~\cite{Analog:ADXL345:2009}. A fim de exemplificar o
processo de calibração de um sensor, vamos aqui demonstrar como foi realizada a
calibração do sensor de temperatura da \emph{startup board} do \emote.

O mediador\footnote{Mediador é o artefato de software que realiza interface do
sistema operacional com o hardware no Projeto \epos} do sensor de temperatura foi
implementado utilizando a Equação de Steinhart-Hart~(\ref{eq:steinhart_hart}).
Os valores dos parâmetros de Steinhard-Hart podem ser
calculados~\cite{Steinhart:1968} a partir de dados fornecidos pelo fabricante
do termistor, e são substituídos na Equação~\ref{eq:steinhart_hart} de modo a
montar o modelo matemático do sensor em uso:
\begin{eqnarray}
%T = \frac{1}{a + b.ln{R} + c.(ln{R})^3}\\
\label{eq:stnhrt_a}a = 1,0750492 \times 10^{-3}\\
\label{eq:stnhrt_b}b = 0.27028218 \times 10^{-3}\\
\label{eq:stnhrt_c}c = 0.14524838 \times 10^{-6}
\end{eqnarray}

Para efetuar este cálculo, é necessário conhecer a resistência apresentada pelo
termistor. Como o termistor está conectado ao ADC por um circuito divisor de
tensão a resistência pode ser calculada da seguinte forma:
\begin{eqnarray}
V_s = I.(R+R_t)\\
V_o = I.R\\
com\;I\;constante\;no\;divisor,\;logo,\; \frac{V_s}{R+R_t} =
\frac{V_o}{R}\\ \label{eq:r_t_value}isolando\;R_t,\; R_t = R.\left ( \frac{V_s}{V_o} - 1 \right)
\end{eqnarray}
onde $V_s$ é a tensão de entrada do circuito de divisão de tensão, $I$ é a
corrente que passa pelas resistências (igual tanto sobre o resistor do divisor
de tensão quanto sobre o termistor), $R$ é a resistência conhecida do resistor
do divisor de tensão, $V_o$ é a tensão de saída do divisor, que é lido pelo
conversor analógico-digital, e $R_t$ é o valor atual da resistência do
termistor, que se deseja obter.

Ao final, substituindo o valor da resistência atual do termistor definido pela
Equação~(\ref{eq:r_t_value}), e os parâmetros de Steinhart-Hart definidos
em~(\ref{eq:stnhrt_a}),~(\ref{eq:stnhrt_b}) e~(\ref{eq:stnhrt_c}) na
Equação~(\ref{eq:steinhart_hart}) (Steinhart-Hart), e ainda, sabendo que o valor
do resistor utilizado no circuito divisor de tensão é de $10\;k\Omega$,
podemos chegar à equação para o cálculo da temperatura amostrada pelo termistor
estudado.
% Primeiro, vamos simplificar o termo que calcula o logaritmo natural da
% resistência do termistor:
% \begin{eqnarray}
% ln\;R_t = ln \left [ R \times \left ( \frac{V_s}{V_o} - 1 \right ) \right ]\\
% ln\;R_t =\;ln\;R + ln \left( \frac{V_s}{V_o} - 1 \right)\\
% com\;R\;conhecido,\;logo,\;ln\;R_t\;=\;ln\;10^4 + ln \left( \frac{V_s}{V_o} - 1
% \right)\\ ln\;R_t =\;2,302585093 + ln \left( \frac{V_s}{V_o} - 1 \right)\\
% considerando,\; L_{R_t} = ln\;R_t\\
% \nonumber temos,\\
% T^{-1} = \left[
%   1,0750492 +
%   (0.27028218\,\times\,L_{R_t}) +
%   \left(0.14524838\times10^{-3} \times L_{R_t}^3\right)
% \right]\times10^{-3}
% T = \left\{
% \frac{1}{
% a +
% (b \times L_{R_t}) +
% [c \times (L_{R_t})^3]
% }
% \right\}
% \end{eqnarray}

A calibração do sensor de temperatura foi realizado por comparação com outro
sensor de temperatura, modelo SHT-11~\cite{Sensirion:SHT1x:2005}, previamente
calibrado. Ambos os sensores foram conectados a um \emote, e os dois \emote
foram colocados em uma caixa hermeticamente fechada. Os \emote utilizados
tiveram seus timers configurados igualmente, e foram conectados a uma mesma
fonte de alimentação. Eles foram programados para realizar 1000 medidas de
temperatura espaçadas de 10 segundos cada e armazená-las em sua memória interna.
Ao final das medições, as leituras foram enviadas por rádio para uma
estação-base, de onde a tendência do sistema foi extraída. A correção, definida
como o inverso da tendência ($C\,=\,-Td$), foi então inserida no sistema,
ficando a equação final de temperatura do termistor assim definida:
\begin{equation}
T = \left\{
\frac{1}{
a +
(b \times ln\,R_t) +
[c \times (ln\,R_t)^3]
}
\right\}
+ C
\end{equation}
com a indicação $T$ expressa em Kelvin. A implementação deste modelo, na
linguagem \textsc{C++}, para o sistema operacional \epos, é apresentada na
Figura~\ref{prg:steinhart_hart_epos}.

É importante destacar que as arquiteturas utilizadas no \emote, \textsc{Avr} e
\textsc{Arm7-TDMI-S}, não apresentam unidade de ponto flutuante (FPU), logo, o
sistema carece de software adicional para executar as operações em ponto fixo,
chegando a resultados semelhantes, porém com muito mais demanda de
processamento. Por isso, é comum utilizar, em algumas aplicações, versões de
mediadores para o termistor que utilizem tabelas pré-calculadas de conversão das
leituras do ADC para um valor equivalente de temperatura, ao custo, neste caso,
de um consumo extra de espaço de armazenamento. De qualquer modo, o processo de
calibração descrito acima continua sendo necessário, neste caso, para a
construção da tabela que ficará armazenada na memória do nodo.

\prg{C}{steinhart_hart_epos}{Código para implementação do termistor calibrado
em C++ para o \epos.}

\subsection{Abstrações de sensores no \epos}
\label{sec:epos_sensing}

Do ponto de vista do programador da aplicação, abstrair questões específicas
como o procedimento de calibração de sensores descrito acima é de grande
importância. Neste contexto, o \epos fornece suporte de sensoriamento às
aplicações através de uma interface de software/hardware que abstrai famílias de
sensores de forma uniforme~\cite{Wanner:ETFA:2006}. O sistema define classes de
dispositivos baseado em sua finalidade (e.g. medir aceleração ou temperatura), e
estabelece um substrato comum para cada classe. Para cada dispositivo são
armazenadas propriedades e parâmetros operacionais, de maneira similar ao TEDS
(\textit{Transducer Electronic Data Sheet}) do padrão IEEE 1451. Uma camada fina
de software adapta dispositivos individuais (e.g., a converte leituras de ADC em
valores contextualizados, aplica as correções) para adequá-lo às características
mínimas da sua classe de sensores. Desta forma, um termistor simples é exportado
para a aplicação exatamente do mesmo modo que um sensor de temperatura digital
complexo~\cite{Wanner:JCC:2008}.

\fig{sensor_classes}{Diagrama de classes das abstrações de
sensoriamento do \epos.}{scale=.1}

A Figura~\ref{fig:sensor_classes} apresenta um diagrama de classes com as
abstrações de sensoriamento do \epos. No subsistema de sensoriamento do \epos,
métodos comuns a todos dispositivos de sensoriamento são definidos pela
interface \texttt{Sensor\_Common}. O método \texttt{get()} provê leituras para
um único sensor em um único canal (i.e. habilita o dispositivo, espera os dados
estarem disponíveis, lê o sensor, desabilita o dispositivo e retorna a leitura
convertida em unidades físicas previamente determinadas). Os métodos
\texttt{enable()}, \texttt{disable()}, \texttt{data\_ready()} e
\texttt{get\_raw()} permitem que o sistema operacional e as aplicações realizem
controle de grão fino sobre leituras de sensores (e.g., realizar leituras
sequenciais, obter dados não convertidos de sensores). O método
\texttt{convert(int v)} pode ser utilizado para converter valores não
processados de sensores em unidades científicas ou de engenharia. O método
\texttt{calibrate()} executa calibragem específica para cada sensor.

Cada família de sensor pode especializar a interface \texttt{Sensor\_Common}
para abstrair adequadamente características específicas da família. A família
\texttt{Magnetometer} pode adicionar, por exemplo, métodos para realizar
leituras em diferentes eixos de sensibilidade. A família \texttt{Thermistor},
por outro lado, provavelmente não precisará estender a interface comum. Cada
família também define uma estrutura \texttt{Descriptor} específica, que define
campos como precisão, dados para calibração e unidades físicas. Cada dispositivo
sensor implementa uma das interfaces definidas, e preenche a estrutura
\texttt{Descriptor} da família com valores específicos do sensor. Valores padrão
de configuração para cada dispositivo (e.g., frequência, ganho, etc.) são
armazenados em uma estrutura de \emph{traits de configuração}.

Sempre que o sistema operacional ou uma aplicação precisam fazer referência a um
dispositivo de sensoriamento, estes podem utilizar um \emph{dispositivo
específico} e realizar operações específicas do dispositivo, ou utilizar a
\emph{classe do dispositivo}, e restringir-se às operações definidas por aquela
classe. Uma realização utilizando metaprogramação estática da classe do
dispositivo agrega os dispositivos disponíveis em uma configuração do sistema.

\tab{sensing_footprint}{Tamanhos de código e dados de componentes de
sensoriamento (em bytes).}

\tab{sampling_rates}{Taxas de amostragem dos sistemas de sensoriamento (em Hz).}

A Tabela~\ref{tab:sensing_footprint} apresenta características de tamanho de
código e dados do subsistema de sensoriamento do \epos em comparação com as estruturas
equivalentes dos sistemas TinyOS e MANTIS OS. A Tabela~\ref{tab:sampling_rates}
mostra as taxas máximas de amostragem possíveis no \epos, também em comparação
com os outros sistemas. O baixo sobrecusto e alta taxa de amostragem no \epos
são resultado direto do projeto do sistema, que minimiza dependências entre
componentes de sensoriamento e o resto do sistema. No \epos, um componente que
abstrai um sensor analógico normalmente depende apenas do conversor
analógico-digital da plataforma e do seu subsistema de I/O, que são abstraídos
por operadores metaprogramados \textit{inline}. Estes mecanismos envolvidos no
desenvolvimento do \epos serão melhores descritos na Seção~\ref{sec:epos}.
