% ------------------------------------------------------------------------------
\section{Strategies for ME Optimization} \label{sota}
ME is a technique employed to explore the
similarity between neighboring pictures in a video sequence.
Figure \ref{fig:motion_estimation} illustrates the ME process for the neighboring
pictures \emph{A} and \emph{B}.
By searching for similarities between these two pictures, it is possible to determine
which blocks from picture \emph{A} are also found in picture \emph{B}.
Such displacement of picture blocks is encoded by \emph{motion vectors}
(represented by the small arrows in the bottom side of
Figure \ref{fig:motion_estimation}). 
There are several strategies to optimize the execution time of ME:
fast-search algorithms, macroblock subsampling, sample truncation, 
multi-resolution ME, subsampled motion-field estimation, 
and parallel and hardware implementations of algorithms.

\fig{.4}{motion_estimation}{Motion Estimation}


% P3 ... Pn-1
% Um paragrafo para cada estratégia dizendo:
% + how these solutions match/ contribute to Goals { g1, g2, .., gn} and Features {f1 , f2 , ..., fn}
% + And what set of features F is still missing? (GAP)
%
% P3 - fast-search
\emph{Fast-search algorithms} are BMAs that look only in 
specific positions of the search window
\cite{SunNingning:TSS:2009, ShipingZhu:DS:2009}.
The search window defines the region of the reference frame that is scanned for a macroblock
partition similar to the current one. 
Only the motion vectors that correspond to the match with the lowest \emph{motion cost} 
are chosen. 
The main drawback of this approach is that, since some positions of the search 
window are discarded, it is possible to find suboptimal motion vectors.

% P4 macroblock subsampling and sample truncation
Two other strategies to optimize ME during block-matching are \textit{macroblock 
subsampling} and \textit{sample truncation}. 
Macroblock subsampling takes into consideration only a macroblock partition 
(i.e. some samples of a macroblock) while the matching for a 
position of the search window is being performed. 
Sample truncation is performed by ignoring the least significant bits of a 
sample. 
These strategies have been used separately
\cite{liu:sub:1993} (subsampling),
\cite{ChiaChunLin:PMRME:2007} (truncation), and
integrated \cite{Ludwich:WebMedia:2011} (subsampling and truncation).

% P5 multi-resolution ME
\emph{Multi-resolution ME} is the strategy in which the motion 
vectors are computed for distinct resolutions of the same frame. 
Motion vectors computed in a more coarse level can be successively refined until
the finest level (higher resolution). 
If the search is performed sequentially as in 
\cite{ChiaChunLin:FastAlgPlusArch:2006}, the time of ME can be increased due to
the dependencies between distinct levels. 
On the other hand, if the search is executed in parallel for each resolution 
level, as in \cite{ChiaChunLin:PMRME:2007}, hardware functional units need to be
replicated. 
A similar technique is \emph{subsampled motion-field estimation} 
\cite{liu:sub:1993}, which is based on the assumption that motion vectors of neighboring 
blocks tends to be similar. Thus, for each block, only a set of motion 
vectors (a motion-field) is computed, while the others are interpolated.

% P7 | P6 parallel and hardware implementations of algorithms
Other strategies for optimizing motion estimation are based on finding 
parallelism in ME stages, especially in the block-matching algorithms, in order
to execute them simultaneously. 
These parallel strategies commonly have been the base for dedicated hardware implementations. 
The \emph{Sum of Absolute Differences} (SAD) is a metric of error used in 
block-matching algorithms. This technique is frequently parallelized using functional 
units in hardware \cite{ChiaChunLin:PMRME:2007,HoyoungChang:HW:2009}, 
Hardware implementations of shared buffers for frame data are also
common \cite{HoyoungChang:HW:2009, HaibingYin:HW:2010}.

% ------------------------------------------------------------------------------
