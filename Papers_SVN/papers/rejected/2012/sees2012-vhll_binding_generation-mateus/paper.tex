\documentclass[10pt, conference, compsocconf]{IEEEtran} % SEES 2012
\usepackage{balance} % SEES 2012

\usepackage[english]{babel}	% for multilingual support

\usepackage[utf8]{inputenc} % for use utf8
\usepackage{graphicx}

\usepackage[caption=false]{subfig} % Para usar duas ou mais figuras como uma só.

\usepackage{algorithmic}
\usepackage{algorithm}


% ------------------------------------------------------------------------------

% Command to use code as figure ------------------------------------------------
\usepackage{listings}
\lstset{keywordstyle=\bfseries, flexiblecolumns=true}
\lstloadlanguages{[ANSI]C++,HTML}
\lstdefinestyle{prg} {basicstyle=\small\sffamily, lineskip=-0.2ex, showspaces=false}

% header C
\newcommand{\headerc}[3][tbp]{
 \begin{figure}[#1]
     \lstinputlisting[language=C,style=prg]{fig/#2.h}
   \caption{#3\label{headerc:#2}}
 \end{figure}
}

% C
\newcommand{\progc}[3][tbp]{
 \begin{figure}[#1]
     \lstinputlisting[language=C,style=prg]{fig/#2.c}
   \caption{#3\label{progc:#2}}
 \end{figure}
}

% header C++
\newcommand{\headercpp}[3][tbp]{
 \begin{figure}[#1]
     \lstinputlisting[language=C++,style=prg]{fig/#2.hh}
   \caption{#3\label{headercpp:#2}}
 \end{figure}
}


% C++
\newcommand{\progcpp}[3][tbp]{
 \begin{figure}[#1]
     \lstinputlisting[language=C++,style=prg]{fig/#2.cc}
   \caption{#3\label{progcpp:#2}}
 \end{figure}
}

% Java
\newcommand{\progjava}[3][tbp]{
 \begin{figure}[#1]
     \lstinputlisting[language=Java,style=prg]{fig/#2.java}
   \caption{#3\label{progjava:#2}}
 \end{figure}
}


% Para colocar 2 programas como um só - dispostos horizontalmente
\newcommand{\multprogjavatwoh}[5][htbp]{
\begin{figure*}[#1]
  \centering

\subfloat[]{\label{fig:#2}\lstinputlisting[language=Java,style=prg
]{fig/#2.java}}

\subfloat[]{\label{fig:#3}\lstinputlisting[language=Pascal,
style=prg]{fig/#3.lua}}
  \caption{#5}
  \label{fig:#4}
\end{figure*}
}
% e.g.
%\multfigtwoh{fig_plot_time_orig}{fig_plot_time_mod}
%{fig_plot_time_all}
%{Original (a) and modified (b) benchmarks execution time comparison.}


% Lua
\newcommand{\proglua}[3][tbp]{
 \begin{figure}[#1]
     \lstinputlisting[language=Pascal,style=prg]{fig/#2.lua}
   \caption{#3\label{proglua:#2}}
 \end{figure}
}

% KCL
\newcommand{\progkcl}[3][tbp]{
 \begin{figure}[#1]
     \lstinputlisting[language=C++,style=prg]{fig/#2.kcl}
   \caption{#3\label{progkcl:#2}}
 \end{figure}
}

% OCL
\newcommand{\progocl}[3][tbp]{
 \begin{figure}[#1]
     \lstinputlisting[language=C++,style=prg]{fig/#2.ocl}
   \caption{#3\label{progocl:#2}}
 \end{figure}
}

% OIL
\newcommand{\progoil}[3][tbp]{
 \begin{figure}[#1]
     \lstinputlisting[language=C++,style=prg]{fig/#2.oil}
   \caption{#3\label{progoil:#2}}
 \end{figure}
}

% ------------------------------------------------------------------------------

% Commands to insert figures ---------------------------------------------------
\newcommand{\figu}[4][ht]{
  \begin{figure}[#1] {\centering\scalebox{#2}{\includegraphics{fig/#3}}\par}
    \caption{#4\label{fig:#3}}
  \end{figure}
}

\newcommand{\fig}[4][ht]{
  \begin{figure}[#1] {\centering\scalebox{#2}{\includegraphics{fig/#3}}\par}
    \caption{#4\label{fig:#3}}
  \end{figure}
}
% fig usage:
% \fig{<scale>}{<file>}{<caption>}
% e.g.: \fig{.4}{uml/uml_comportamental_dia}{Diagramas comportamentais da UML}
% The figure label will be "fig:" plus <file>.
% The figure file must lie in the "fig" directory.

\newcommand{\figtwocolumn}[4][ht]{
  \begin{figure*}[#1] {\centering\scalebox{#2}{\includegraphics{fig/#3}}\par}
    \caption{#4\label{fig:#3}}
  \end{figure*}
}

\newcommand{\figb}[4][hb]{
  \begin{figure}[#1] {\centering\scalebox{#2}{\includegraphics{fig/#3}}\par}
    \caption{#4\label{fig:#3}}
  \end{figure}
}

% Para colocar 2 figuras como uma só - dispostas horizontalmente
\newcommand{\multfigtwoh}[6][htbp]{
\begin{figure*}[#1]
  \centering
  \subfloat[]{\label{fig:#3}\scalebox{#2}{\includegraphics{fig/#3}}}
  \subfloat[]{\label{fig:#4}\scalebox{#2}{\includegraphics{fig/#4}}}
  \caption{#6}
  \label{fig:#5}
\end{figure*}
}
% e.g.
%\multfigtwoh{.65}{fig_plot_time_orig}{fig_plot_time_mod}
%{fig_plot_time_all}
%{Original (a) and modified (b) benchmarks execution time comparison.}

% Para colocar 2 figuras como uma só - dispostas verticalmente
\newcommand{\multfigtwov}[6][htbp]{
\begin{figure}[#1]
  \centering
  \subfloat[]{\label{fig:#3}\scalebox{#2}{\includegraphics{fig/#3}}}\\
  \subfloat[]{\label{fig:#4}\scalebox{#2}{\includegraphics{fig/#4}}}
  \caption{#6}
  \label{fig:#5}
\end{figure}
}
% e.g.
%\multfigtwov{.35}{fig_epos_mem_framework}{fig_epos_mem_framework_spm}
%{fig_epos_mem_framework_all}
%{EPOS memory mapping before (a) and after (b) using the new framework}


% Alguns outros comandos:
\newcommand{\lua}{\textsc{Lua}}
\newcommand{\java}{\textsc{Java}}
\newcommand{\python}{\textsc{Python}}
\newcommand{\ruby}{\textsc{Ruby}}


% ------------------------------------------------------------------------------

\begin{document}

% \title{VHLL binding generation as an AOP problem}
% \title{Efficient generation of wrappers for hardware devices in the embedded
% system scenario}
\title{VHLL binding generation for the embedded system scenario}


\author{\IEEEauthorblockN{Mateus Krepsky Ludwich and Antônio Augusto Fröhlich}
\IEEEauthorblockA{Laboratory for Software and Hardware Integration (LISHA)\\
Federal University of Santa Catarina (UFSC)\\
Florianópolis, Brazil\\
\{mateus,guto\}@lisha.ufsc.br}
}

\maketitle

\begin{abstract}
Programming Languages have a main role in computational systems development.
Among them, the so called Very-High Level Languages (VHLL), from which \java~and
\lua~are examples, provide developers with features to improve their
productivity.

Several initiatives have been taken on the last ten years in order to enable the
use of VHLLs not only for general propose systems as well for embedded
systems fulfilling the time and resource usage requirements impose by these
systems.
However, in order to be really useful in embedded systems VHLLs must provide
features for interacting with the environment where the embedded system
is inserted on.
Such interaction is usually implemented by using hardware devices, such as
sensors and actuators, transmitters and receivers, and timers and alarms.

This paper presents a method to abstract hardware devices in order to be used by
applications written using VHLL in the embedded systems scenario.
Hardware mediators are used to abstract and to organize hardware devices
in a suitable manner for embedded systems fulfilling time and resource
consumption requirements.
By isolating hardware mediators from the specificities a VHLL the problem of
adapting a hardware device to work with a new VHLL can be faced as an aspect
weaving problem.
The proposed method is evaluated on the VHLLs \java~and \lua~among three
cases study encompassing serial communication, video encoding, and temperature
sensing.
The obtained results corroborate the suitability of the proposed method on the
requirements of performance, memory consumption, reuse, and portability.
\end{abstract}

\begin{IEEEkeywords}
Binding Generation, Embedded Systems, Foreign Function Interface

\end{IEEEkeywords}

\IEEEpeerreviewmaketitle

%-------------------------------------------------------------------------------
% Section <Introduction>
% ------------------------------------------------------------------------------
\section{Introduction} \label{intro}
% + Introduction
% 
% The very first letter is a 2 line initial drop letter followed
% by the rest of the first word in caps.
%
% form to use if the first word consists of a single letter:
% \IEEEPARstart{A}{demo} file is ....
%
% form to use if you need the single drop letter followed by
% normal text (unknown if ever used by IEEE):
% \IEEEPARstart{A}{}demo file is ....
%
% Some journals put the first two words in caps:
% \IEEEPARstart{T}{his demo} file is ....
%
% Here we have the typical use of a "T" for an initial drop letter
% and "HIS" in caps to complete the first word.
% \IEEEPARstart{T}{his} demo file is intended.

\IEEEPARstart{E}{nergy} consumption is a determining factor when designing wireless sensor networks.
As a consequence, battery lifetime is a limitation on the development of such systems.
Therefore, the idea of extracting energy from the environment has become attractive.
Looking to the energy consumption problem, the intelligent usage of the stored energy contributes to extend the sensor nodes' longevity.
Consequently, energy schedulers have been developed in order to adequately assess the energy consumption and adapt the system accordingly to the available amount of energy.
The purpose of this work is to adapt a solar energy harvesting circuit to supply energy to low power wireless platforms, i.e., those that operate under $50~mW$.
Simultaneously, we aim at improving the performance of the energy-aware task scheduler in wireless sensor network systems by providing fine-grained battery and environmental monitoring.

Among a number of energy sources that have been studied so far, solar has proved to be one of the most effective~\cite{Roundy:2003}.
The solar energy conversion through photovoltaic (PV) cells is better performed at an optimum operating voltage.
Operating a solar panel on this voltage results in transferring to the system the maximum amount of power available.
In this context, \emph{maximum power point tracker circuits} (MPPT) have been proposed.
The drawback is that MPPT circuitry may introduce losses to a solar harvesting system.
Concerning low-power applications, it may be more energy efficient to have a good matching between the solar panel and the energy storage unit~\cite{Raghunathan:2005}.
This well matched system is than able to work close to the maximum power point with less power loss.

In this work, an evaluation of the proposed harvesting circuit is performed in order to show improvements on an energy-aware task scheduler~\cite{Hoeller:SMC:2011}.
It is shown that the combination of the proposed circuit with the cited scheduler not only extended the longevity of the wireless sensor network, but also improved system quality.

The paper is organized as follows:
Section~\ref{fund} presents the fundamentals of solar energy harvesting and energy-aware task scheduler.
Section~\ref{design} discusses the design of the harvesting circuit under the perspective of low power wireless platforms.
Section~\ref{case} presents the evaluation of the harvesting circuit and a case study showing the improvements on system quality.
Finally, section~\ref{concl} closes the paper.

% ------------------------------------------------------------------------------


% ------------------------------------------------------------------------------
% Section <Java requirements for embedded systems>
% -----------------------------------------------------------------------------
\section{Trabalhos relacionados}
\label{sec:related_work}
% Nosso trabalho aborda a questão de como prover acesso a dispositivos de hardware
% ao Java e como prover este acesso de uma forma bem estruturada levando em 
% consideração todos os requisitos do cenário de sistemas embarcados.
%
A linguagem de programação Java é desprovida do conceito de \emph{ponteiro}, 
presente em linguagens como C e C++. 
O endereço das \emph{variáveis de referência}, utilizadas para acessar objetos Java,
é conhecido apenas pela JVM, a qual trata de todos os acessos à
memória. Como a maioria dos dispositivos de hardware são mapeados em endereços de
memória, acessá-los diretamente é um problema para a linguagem Java. 
FFI é a abordagem
utilizada por Java para superar esta limitação uma vez que ela permite ao Java
utilizar construções, como ponteiros C/C++, para acessar diretamente dispositivos
de hardware.
FFIs também tem sido utilizadas por plataformas Java na reutilização de código
escrito em outras linguagens de programação como C e C++ e para embarcar JVMs em 
aplicações nativas permitindo as mesmas acessar funcionalidades 
Java \cite{Liang:1999}.
%(\cite{Liang:1999}, \cite{1288968}).

\emph{Java Native Interface} (JNI) é a principal FFI Java, a qual é utilizada na
plataforma \emph{Java Standard Edition} \cite{Liang:1999}. 
Na JNI, a interface entre código nativo e Java é realizada durante o tempo de 
execução do programa. Isto significa que, durante a execução de um programa, 
a JVM procura e carrega a implementação dos métodos marcados como nativos 
(métodos que possuem a palavra reservada \emph{native} em suas assinaturas).
Usualmente a implementação dos métodos nativos é armazenada em uma biblioteca 
ligada dinamicamente.
Este mecanismo de busca e carga de métodos aumenta a necessidade de memória em
tempo de execução e o tamanho da JVM. Por esta razão eles são evitados em 
sistemas embarcados.

% NOTA: Não estou certo destas limitações da KNI, apesar de estarem na 
% especificação da mesma.
A plataforma \emph{Java Micro Edition} (JME) utiliza uma FFI ``leve'',
chamada de \emph{K Native Interface} (KNI) \cite{_k_2002}. 
A KNI não carrega métodos nativos dinamicamente na JVM, evitando o sobrecusto 
de memória da JNI. 
Na KNI a interface entre Java e código nativo é realizada estaticamente, 
durante o tempo de compilação. 
Entretanto decisões de projeto da KNI impõem algumas limitações. 
A KNI proíbe a criação de objetos Java (exceto de strings) a partir do código 
nativo. 
% É proibida também a chamada de métodos Java, a partir do código nativo. 
Além disto, na KNI os únicos métodos nativos que podem ser invocados são aqueles 
pré-compilados na JVM. 
Não há uma Interface de Programação de Aplicações 
(API - Application Programming Interface) em nível Java para invocar outros 
métodos nativos. % cuidado com esta afirmação. Frase original: There is no Java-level API to invoke others native methods.
Como consequência, é difícil de criar novos controladores de dispositivos de 
hardware utilizando-se a KNI.

% NOTA: Os último argumento, de chamar Java a partir do C... é fraco
A FFI da KESO, utilizada neste trabalho, foca em sistemas embarcados. 
Assim como a KNI, a FFI da KESO não realiza carga dinâmica de métodos nativos.
Entretanto, diferentemente da KNI, a FFI da KESO provê aos programadores uma 
API em nível Java para criação de novas interfaces com código nativo. 
Também não exite problema do código nativo chamar código Java, uma vez que 
KESO e a FFI da KESO geram código C.

O tarefa de escrita de adaptadores para código nativo pode ser facilitada de
duas maneiras, por APIs de alto nível e por ferramentas geradoras. As APIs de
alto nível fornecem métodos específicos para auxiliar na criação desses adaptadores,
enquanto as ferramentas geradoras podem gerar parte de adaptadores ou adaptadores
completos a partir de análise de código nativo ou a partir de uma especificação
em mais alto nível.
SWIG e a biblioteca de função estrangeira de Python \emph{ctypeslib} são exemplos
de ferramentas que geram adaptadores a partir de arquivos \emph{headers} C/C++
como entrada. O primeiro suporta diversas linguagens como saída, como por exemplo
Python, D e Java. O segundo foca em programas Python 
\cite{swig-site},\cite{ctypeslib-site}.
Ravit et al. apresenta uma ferramenta que tem como objetivo prover funcionalidades
da linguagem de mais alto nível (tuplas, por exemplo) para serem utilizadas no 
código dos adaptadores. A ferramenta proposta por Ravit et al. gera adaptadores
Python a partir de código escrito em C e descrições de interface, as quais 
contêm, dentre outras, informações sobre funções e seus respectivos parâmetros
\cite{Ravitch:2009:AGL:1542476.1542516}.
Outras soluções, como a linguagem \emph{Jeannie}, misturam código C e Java em 
um único programa a partir do qual geram adaptadores JNI 
automaticamente \cite{1297030}.
A FFI da KESO, utilizada neste trabalho, provê uma API baseada em aspectos para
ajudar na criação de adaptadores. É possível especificar quais pontos do programa
Java serão afetados pela criação dos adaptadores, assim como qual código deve
ser gerado para cada ponto do programa Java a ser afetado.

% The binding code can be checked for correctness and bug detection. Tools such 
% as J-BEAM and Ilea perform bug checking based on the source code, using static
% analysis techniques\cite{jbeam:2008}, \cite{Ilea:2007}.
% Lee et al. deal with bug detection dynamically, when the FFI code is been 
% used \cite{Lee:2010:JSD:1809028.1806601}.
% Their tool Jinn synthesizes dynamic bug detectors for FFIs from Finite State 
% Machines whose encode FFI constrains that should be tested. The FFI targeted 
% for the Java language is the JNI which contains hundreds os API calls. 
% Although not aborted by this paper, similar bug detectors can be adapted to 
% check bindings generated using the KESO FFI.


% -----------------------------------------------------------------------------



% ------------------------------------------------------------------------------
% Section <proposal>
\section{Building a Trustful Infrastructure for Future Internet}
\label{sec:solution}
The Internet architecture demonstrate inefficiency and problems in several and large areas, such as mobility, real-time applications,
failures (e.g. equipment, software bugs, and configuration mistakes), and especially in pervasive security problems \cite{Rexford:2010}.
Moreover, the Internet lacks effective solutions in terms of scalability and sustainability, 
consuming much more energy and hindering the management of countless sensor devices that are so important for several applications in the Future Internet.
Hence, we propose the use of a stack of communication protocols (UDP@NDN@C-MAC), in the scope of the EPOSMote project,
designed specifically to guarantee a trustful communication
%Our solution also includes EPOSMote II, an embedded platform. Thus, 
while still compromised with the low utilization of resources (processing, memory, power and communication bandwidth).
%and the use of EPOSMote II which is an embedded platform and represents a typical Future Internet device.

\subsection{EPOSMote}
The EPOSMote is an open hardware project~\cite{eposmote}. Initially it aimed at 
the development of a wireless sensor network module, and focused on environment 
monitoring. Its first version, the EPOSMote I, features an 8-bit AVR microcontroller, 
IEEE 802.15.4 communication capability and a small set of sensors.

As the project evolved a second version arose, with the objective of delivering a 
hardware platform to allow research on energy harvesting, biointegration, and 
MEMS-based sensors. The EPOSMote II focus on modularization, and thus is composed 
by interchangeable modules for each function.

Figure \ref{emote2-block_diagram} shows an overview of the EPOSMote II architecture.
Its hardware is designed as a layer architecture composed by a main module,
a sensoring module, and a power module. The main module is responsible for processing
and communication. It is based on the Freescale MC13224V microcontroller~\cite{mc13224v}, which possess 
a 32-bit ARM7 core, an IEEE 802.15.4-compliant transceiver, 128kB of flash memory, 80kB of ROM memory
and 96kB of RAM memory. We have developed a startup sensoring module, which contains some sensors  
(temperature and accelerometer), leds, switches, and a micro USB (that can also be used as power supply). 
Figure \ref{emote2-mc13224v-pictures-real_white_background} shows the development kit which is slightly 
larger than a R\$1 coin, on the left the sensoring module, and on the right the main module.

\fig{.45}{emote2-block_diagram}{Architectural overview of EPOSMote II.}

\fig{.07}{emote2-mc13224v-pictures-real_white_background}{EPOSMote II SDK side-by-side with a R\$1 coin.}

\subsection{C-MAC}
C-MAC is a highly configurable MAC protocol for WSNs realized as a framework of
medium access control strategies that can be combined to produce
application-specific protocols~\cite{steiner:2010}. It enables application
programmers to configure several communication parameters (e.g.  synchronization,
contention, error detection, acknowledgment, packing, etc) to adjust the protocol
to the specific needs of their applications. The framework was implemented in C++ 
using static metaprogramming techniques (e.g. templates, inline functions, and 
inline assembly), thus ensuring that configurability does not come at expense of 
performance or code size. The main C-MAC configuration points include:

\textbf{Physical layer configuration:} These are the configuration points defined
by the underlying transceiver (e.g. frequency, transmit power, date rate).

\textbf{Synchronization and organization:} Provides mechanisms to send or receive
synchronization data to organize the network and synchronize the nodes duty
cycle.

\textbf{Collision-avoidance mechanism:} Defines the contention mechanisms used to
avoid collisions. May be comprised of a carrier sense algorithm (e.g. CSMA-CA),
the exchange of contention packets (\emph{Request to Send} and \emph{Clear to
Send}), or a combination of both.

\textbf{Acknowledgment mechanism:} The exchange of \emph{ack} packets to
determine if the transmission was successful, including preamble acknowledgements.

\textbf{Error handling and security:} Determine which mechanisms will be used to
ensure the consistency of data (e.g. CRC check) and the data security.

The Future Internet will be composed by a wide range of both applications and devices, 
each with its own requirements and available resources. Through C-MAC configurability we
can provide the most adequate MAC functionalities for each case, instead of providing a 
general non-optimal solution for all of them.

\subsection{NDN}
Communication in NDN is impelled by the data consumers.
Nodes that are interested in a content transmit \emph{Interest} packets, which contains the name of the requested data. %selector, nonce
Every node that receives the \emph{Interest} and have the requested data can respond with a \emph{Data} packet that follows back the path from which the \emph{Interest} came. %content name, signature, signed info, data
It is important to notice that one \emph{Data} satisfies one \emph{Interest}, thus ensuring flow balance in the network.
Since the content being exchanged is identified by its name, all nodes interested in the same content can share transmissions (considering a broadcast medium, which is the case for most Future Internet devices).

NDN packet forwarding engine has three main data structures: the FIB (Forwarding Information Base), which is used to forward \emph{Interest} packets to potential sources; 
the ContentStore, which is a buffer memory used to maximize the sharing of packets; 
and the PIT (Pending Interest Table), which is used to keep track of \emph{Interest} packets so that \emph{Data} packets can be sent to its requester(s).

When a node receives an \emph{Interest} packet it searches for its content name, looking for a match primarily at the ContentStore, then the PIT, and ultimately at the FIB.
If there is a match at the ContentStore, it is sent and the \emph{Interest} discarded.
Otherwise, if there is a match at the PIT, the set of requesting interfaces for that data is updated, and the \emph{Interest} discarded (at this point an \emph{Interest} in this data has already been sent).
Otherwise, if there is a match at the FIB, the \emph{Interest} is sent towards the data, and a new PIT entry is created. 
In case there is no match for the \emph{Interest} then it is discarded.

As for the \emph{Data} packet they simply follow the chain of PIT entries back to the original requester(s).
When a node receives a \emph{Data} packet it searches for its content name. 
If there is a ContentStore match, then the \emph{Data} is a duplicate and is discarded.
%A FIB match means there are no matching PIT entries, so the \emph{Data} is unsolicited and it is discarded.
In case of a PIT match, the data is validated, added to the ContentStore, and sent to the set of requesting interfaces from the corresponding PIT entry.

In NDN the name in every packet is bound to its content with a signature.
This enables data integrity and provenance, allowing consumers to trust the data they receive regardless of how the data came to them.
To provide content protection and access control NDN uses encryption.
The encryption of content or names is transparent to the network, since to NDN it is all just named binary data.
%The signature algorithm used may be selected by the content publisher, 
%and chosen to meet performance requirements such as latency or computational cost of signature generation or verification.
Nevertheless, NDN does not mandate any particular key distribution scheme, signature, or encryption algorithm.

\subsection{UDP}
The User Datagram Protocol has been chosen for its simplicity. Its simple transmission 
model avoids unnecessary overhead, since it does not handle reliability, ordering, 
and data integrity, leaving these characteristics to be treated in other layers if necessary, which is a 
perfect blend with the rest of our protocol stack.


% ------------------------------------------------------------------------------
% Section <Evaluation>
% ------------------------------------------------------------------------------
\section{Practical Experiments} \label{eval}
% + Practical Experiments
% First shows speedup and quality results for DMEC
% using 1 (without partitioning) to 6 worker threads.
% Show that speedup is high and quality is kept acceptable.
% 
% Then, show speedup and quality results for DMEC integrated to JM and compares
%  to the original JM (and, if possible, to other works).
% 

% + JM
% + Dizer como realizamos os experimentos. E/ou quais as variáves observadas:
%   Evaluate the component in isolation DMEC to show its speedup.
%    And how it scales.
%   Evaluate the component in JM to show sppeedup and PSNR.
We have evaluated DMEC in two stages.
First, in order to verify how DMEC's performance scales from 
one to six \emph{Workers} instances, we have evaluated all DMEC implementations 
in a test case.
The test case application mimics the behavior of an H.264 encoder: it provides 
DMEC with pictures, obtain the ME results 
(motion vectors and motion cost), and checks if the results are correct.
Secondly, in order to assess DMEC influence on the final video quality, we have
evaluated all DMEC implementations in the
JM H.264 Reference Encoder~\cite{site:jm}.
The PSNR degradation is computed as the absolute PNSR difference between the
original encoder and the optimized ones.

% P: Dizer pq focamos em luma
For inter macroblock modes in H.264 (i.e. modes related to the ME),
the motion cost for chrominance components derives from the motion cost for 
luminance components~\cite{1101854}. 
Consequently the PSNR for chrominance components derives from the PSNR for 
luminance components. 
For this reason, in this paper we focus on the PSNR variation of the luminance 
component.

Figure \ref{fig:dmec-speedup_workers} show the speedup of DMEC in there
test case application with a different number of \emph{Workers} instances.
For such test, we have used an arbitrary set of pictures with a resolution of
1080p (Full-HD).
The speedup is normalized to one \emph{Worker} instance (speedup of 1X).

% TODO
% \textit{Comments about results in: Cell BE, Muticore IA32, and HW.}
It is worth to mention that for each number of \emph{Worker} instances
a different partition mode was used, according to
Figure \ref{fig:picture_partition}.
For one \emph{Worker} instance we have used the ``Single Partition'',
for two \emph{Worker} instances we have used the ``2x1'' partition and so on,
up to the ``2x3'' partition mode (used for six \emph{Worker} instances).

Besides the additional performance obtained by using a higher number of \emph{Worker} instances,
the partition mode also has influence on the speedup.
The reason of such influence is that, during the partitioning process, the dimensions of the
search window shrinks, thus reducing the area of the picture searched for
similarities.

\fig{.45}{dmec-speedup_workers}{Time performance scalability of DMEC}

% Sobre RD curves
% Figures XXX show the speedup of DMEC while tested already integrated to JM.
% The obtained values are compared to the ones obtained while using the original
% JM, without DMEC.
In order to evaluate in details the behavior of DMEC for distinct values 
of encoding bit-rates, we have used the BD-PSNR (Bjøntegaard Delta PSNR) metric
using the following values of QP (Quantization Parameter): 16,20,24,28; as 
described in \cite{gisle_bjntegaard_calculation_2001}.
It is important to evaluate quality (PSNR) for distinct bit-rates to test 
whether the approach can be used in distinct scenarios of application.
Figure \ref{fig:crowd-bitrate_psnr} shows the rate-distortions (RD) curves using the
original JM encoder and the optimized encoder using DMEC.
The video sequence used for this curves was \texttt{Crowd Run}, a 1080p sequence with  a
high ammount of motion.
Lower values of bit-rate are obtained for higher values of QP since by using
higher values for QP more data is discarded, thus increasing the
compression ratio. 
The two curves very near from each other indicates that the DMEC
presents a good rate-distortion performance for all the evaluated bit-rates.

\fig{.45}{crowd-bitrate_psnr}{RD curve of a 1080p video sequence}

We have evaluated also the speedup obtained in the
% encoding time
ME run time
while using
DMEC for the same QP values we used for BD-PSNR.
Figure \ref{fig:crowd-bitrate_speedup} shows the obtained values while using
6 \emph{Worker} instances.
For Muticore IA32, a speedup of around 9 times is obtained for all
bit-rate values.
For Cell BE this value is about 2 times.
A small speedup for the Cell BE, while compared to Multicore IA32 and the dedicated hardware, is due
to the memory transferences (picture samples and ME results) which is performed using
the DMA requisitions of Cell BE.


\fig{.45}{crowd-bitrate_speedup}{Speedup vs bit-rate of a 1080p sequence}
%
% \multfigtwov{.65}{bd_psnr}{bd_speedup} {bd} {RD curve (a) and speedup vs bit-rate (b) of 1080p sequence}

% Discussion


% Falar do paralelismo / particionamento de dados
% Qualidade ficou boa mesmo particionando e desempenho aumentou: speedup ~ 70%
%The strategy of ME distribution based on picture partitioning has been shown 
%effective.
% We have obtained a speedup higher than XXX\% without loosing quality.
%Data partitioning is effective because the visual interdependence between
%partitions is not significant to influence on the encoding quality, and allows
%for a speedup because it enables the simultaneous processing of each picture
%partition.

% - Falar da comunicação
% - Necessidade via espaço de endereçamentos diferentes
% A arquitetura Cell Broad Band demonstrou-se uma arquitetura interessante para o
% processamento paralelo de vídeo, pois possui unidades funcionais dedicadas 
% (i.e. SPEs) para processamento de dados. 
% A principal dificuldade encontrada em trabalhar-se com o Cell foi a capacidade 
% limitada da memória local das SPEs.
% Outra dificuldade foi lidar com as transferências de memória entre SPE e PPE. 
% Isto em fato foi superado pelas estratégias que desenvolvemos de baferização e 
% também com a utilização do Element Interconect Bus (EIB) do Cell que realiza 
% DMAs com altas taxas de transferências.
% 
% - Solução 1: Buffer de preditores, contribuiu bastante
% A estratégia de armazenamento de preditores nas SPEs foi significativa no 
% aumento do desempenho, pois vetores de movimentos necessários para o cálculo da
% ME não precisam ser consultados na memória principal. É coerente a decisão de 
% manter uma cópia local destes vetores, pois todos os vetores que a ME irá 
% precisar foram calculados pela partição em questão e por nenhuma outra.


% ------------------------------------------------------------------------------


% ------------------------------------------------------------------------------
% Section <Discussion>
% ------------------------------------------------------------------------------
\section{Conclusões}
\label{sec:discussion}
% Aplicações para sistemas embarcados usualmente necessitam interagir com
% diversos tipos de dispositivos de hardware como sensores, atuadores, 
% transmissores, receptores e \emph{timers}.
%
% Interface de função estrangeira é o mecanismo adotado por Java para superar as
% limitações da linguagem e permitir acesso direto a memória e a dispositivos de
% hardware. 
% Entretanto, como mostrado na seção \ref{sec:related_work}, as principais FFIs 
% Java não são eficientes em termos de consumo de recursos, ou possuem limitações
% de projeto que dificultam o desenvolvimento de novas interfaces entre Java e os
% dispositivos de hardware.
%
Neste artigo, apresentou-se um meio de realizar a interface entre componentes 
de hardware e aplicações Java para sistemas embarcados. 
Isto foi obtido utilizando-se a interface de função estrangeira da JVM KESO e o
EPOS.

% O EPOS permite o desenvolvimento de aplicações portáveis, independentes de
% especificidades de máquina. 
% Isto é conseguido utilizando-se o conceito de mediadores de hardware, os quais
% sustentam um contrato de interface entre abstrações de sistemas e a máquina.

% O JVM KESO compila o bytecode de uma aplicação Java em código C e gera as partes
% da JVM necessárias pela aplicação. 
% A FFI da KESO também utiliza esta abordagem estática, gerando o código C 
% especificado nas classes \emph{Weavelet}. 
% Então o código C gerado pelo compilador KESO e pela FFI da KESO são compilados
% em conjunto em código nativo, utilizando-se um compilador C padrão.

Nós avaliamos nossa abordagem em termos de desempenho, consumo de memória e 
portabilidade.
Para a aplicação utilizando o mediador de hardware da UART o sobrecusto 
de tempo obtido foi menos de 0.04 \% do tempo total de execução da aplicação e
nossa solução é 38 vezes mais rápido do que a JNI da Sun.
O consumo de memória para tal aplicação foi de 33KB, incluindo todo o suporte
de ambiente de execução, o qual é adequado para diversos sistemas embarcados.
Utilizando o EPOS nós obtivemos portabilidade para várias plataformas e 
utilizando o conceito de componentes híbridos podemos utilizar os mesmos 
adaptadores de código nativo tanto para componentes implementados em 
hardware como implementados em software.

Visando avaliar nossa abordagem em uma aplicação real, nós escrevemos
adaptadores de código nativo para um componente o qual realizada estimativa de
movimento para codificação de vídeo H.264.

% ------------------------------------------------------------------------------



% ------------------------------------------------------------------------------
% trigger a \newpage just before the given reference
% number - used to balance the columns on the last page
% adjust value as needed - may need to be readjusted if
% the document is modified later
%\IEEEtriggeratref{8}
% The "triggered" command can be changed if desired:
%\IEEEtriggercmd{\enlargethispage{-5in}}

% Better way for balancing the last page:

\balance

% References
\bibliographystyle{IEEEtran}
\bibliography{hw,os,pl,mm}


\end{document}

%------------------------------------------------------------------------------

